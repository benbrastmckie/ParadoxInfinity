
\documentclass[12pt,a4paper]{article}


%Spacing Packages
\usepackage{fullpage}
\usepackage{a4wide}
%\usepackage{setspace}
%\usepackage{endnotes} \let\footnote=\endnote %Remember to update below

% Bibliography Packages
%\usepackage{linquiry}
\usepackage{natbib}

%Other Packages
\usepackage{amssymb}
\usepackage{amsmath}
\usepackage{euscript}
%\usepackage{latexsym}
%\usepackage{amsfonts}

\usepackage[lf]{venturis} %% lf option gives lining figures as default; 
			  %% remove option to get oldstyle figures as default
\usepackage[T1]{fontenc}


\usepackage{enumerate}

%Diagram packages
%\usepackage{bar}
%\usepackage{curves}
%\usepackage{pst-plot}

%Tree packages
%\usepackage{ecltree}
%\usepackage{eclbip}
%\usepackage{eepic}
%\usepackage{epic}


\begin{document}

\begin{center} {\large 24.118x -- Paradox and Infinity \\ \vspace{1mm}}
 {\large Problem Set 6: Answer Key \\ \vspace{1mm}}
 
\end{center}
\vspace{3mm}


\subsection*{Problems:}


\begin{enumerate}

\item \begin{enumerate}[(a)]
	\item Yes, trivially, the empty set is well ordered by $\in$. The empty set is, in fact, our friend the ordinal $0_o$.
	\item \begin{itemize}
		\item $\mathcal{P}(\varnothing)=\{\varnothing\}$
		\item $\mathcal{P}(\mathcal{P}(\varnothing))=\{\  \varnothing, \{\varnothing\}\  \}$
		\item $\mathcal{P}(\mathcal{P}(\mathcal{P}(\varnothing)))=\{\ \  \varnothing,\ \{\varnothing\}, \ \{\{\varnothing\}\}, \ \{\,\varnothing,\{\varnothing\} \,\}\ \ \}$
		\end{itemize}
	$\mathcal{P}(\mathcal{P}(\mathcal{P}(\varnothing)))$ is \emph{not} well-ordered by $\in$, as, for instance, $\varnothing$ is not a member of $\{\{\varnothing\}\}$, or vice-versa.
	\item $$\mathcal{P}(\mathcal{P}(\mathcal{P}(\varnothing))-\{\varnothing\} = \{\ \ \{\varnothing\}, \ \{\{\varnothing\}\}, \ \{\,\varnothing,\{\varnothing\} \,\}\ \ \}$$ This set is \emph{not} well-ordered by $\in$ as $\{\{\varnothing\}\}$ is not a member of $\{\,\varnothing,\{\varnothing\}\, \}$, or vice-versa.
	\item $$\mathcal{P}(\mathcal{P}(\mathcal{P}(\varnothing))-\{\{\varnothing\}\} = \{\  \varnothing, \ \{\{\varnothing\}\}, \ \{\,\varnothing,\{\varnothing\} \,\}\ \ \}$$ This set is \emph{not} well-ordered by $\in$, as, for instance, $\varnothing$ is not a member of $\{\{\varnothing\}\}$, or vice-versa.
	\item $$\mathcal{P}(\mathcal{P}(\mathcal{P}(\varnothing))-\{\{\{\varnothing\}\}\} = \{\  \varnothing, \ \{\varnothing\}, \ \{\,\varnothing,\{\varnothing\} \,\}\ \ \}$$ This set is, indeed, well-ordered by $\in$. In fact, it is our friend the ordinal $3_o$.
	\end{enumerate}
		
\item \begin{enumerate}[(a)]
	\item For all $\alpha$, $\alpha\notin\varnothing$. Hence, for all $\alpha$, it is not the case that $\alpha<_o\varnothing$. Hence there is no ordinal $\alpha$ such that $\alpha<_o\varnothing$.
	\item $\omega$ is the set $\{0_o, 1_o, 2_o,\ldots\}$. All of $0_o, 1_o, 2_o,\ldots$ are finite. So there is no infinite ordinal in $\omega$. So there is no infinite ordinal $\alpha$ such that $\alpha<_o\omega$.
	\end{enumerate}

\item \begin{enumerate}[(a)]
	\item Here is a bijection from $\omega'$ to $\omega$: \begin{itemize}
		\item For all $\alpha<\omega$, map $\alpha$ to $\alpha+1_o$; 
		\item map $\omega$ to $0_o$.
		\end{itemize}
		You should convince yourself that $f$ is both an injection and a surjection, and, hence, a bijection.
	\item Consider an arbitrary bijection $f$ from $\omega$ to $\omega'$. Consider the $\alpha\in\omega$ such that $f(\alpha)=\omega$. For all $\alpha\in\omega, \alpha<_o\omega$. But for all $\alpha\in\omega$, there is some $\beta\in\omega$ such that $\alpha<_o\beta$. Whereas, whatever $f(\beta)$ is, it's got to be such that $f(\beta)<_of(\alpha)=\omega$. So $f$ is not such that $\alpha<_o\beta$ iff $f(\alpha)<_of(\beta)$. So $\omega$ and $\omega'$, as ordered by $<_o$, are not isomorphic, and do not have the same well-order type.
	\end{enumerate}

\item \begin{enumerate}
	\item $\alpha+1_o=a'$ --- true.
	
	 $\alpha + 1_o$ is one step up from $\alpha$ in the ordinal hierarchy; you can see that by considering that the well-ordering represented by $\alpha+1_o$ is the next well-ordering from the ordering represented by $\alpha$. $\alpha'$ is \emph{also} the next step up in the ordinal hierarchy from $\alpha$; you can see that by recalling that the rule for making the next ordinal is to pool together all the previous ordinals, which is exactly what you do to make $\alpha'$.
	
	\item $1_o + \omega = \omega + 1_o$ --- false. The left hand side equals $\omega$, which is not equal to $\omega+1_o$.
	
	\item $1_o +3_o = 3_o +1_o$ --- true. In general, ordinal addition is commutative for finite ordinals.
	\item $(\omega+2_o)+\omega <_o (\omega+\omega)+3_o$ --- true. The left hand side equals $\omega+\omega$, which is indeed earlier in the ordering of ordinals than $(\omega+\omega)+3_o$
	\item $1_o\times3_o = 3_o\times1_o$ --- true. In general, ordinal multiplication is commutative for finite ordinals.
	\item $3_o\times\omega = (\omega + \omega) + \omega$ --- false. The left hand side equals $\omega$, which is not equal to $(\omega + \omega) + \omega$.
	\item $\omega\times3_o =\omega+(\omega+\omega)$ --- true. Both sides equal $(\omega+\omega)+\omega$.
	\item $(\omega\times2_o)+\omega <_o (\omega\times\omega)+2_o$ --- true. The left hand side equals $\omega\times3_o$, which is much earlier in the ordering of ordinals than $(\omega\times\omega)+2_o$.
	\item $\omega\times\omega <_o \omega\times (2_o \times\omega)$ --- false. The left and right hand sides are equal.
	\item $\omega\times(\omega+\omega)=(\omega\times\omega)+(\omega\times\omega)$ --- true. Both sides represent an $\omega$-sequence of $\omega$-sequences, followed by another $\omega$-sequence of $\omega$-sequences. In general, ordinals are what is called ``distributive on the left'' --- that is, in general, $\alpha\times(\beta+\gamma)=(\alpha\times\beta )+ (\alpha\times\gamma)$. They are \emph{not}, however, distributive on the right, in general --- that is, it is not always the case that  $(\alpha+\beta)\times\gamma = (\alpha\times\gamma) + (\beta\times\gamma)$.
	\end{enumerate}

\item  \begin{enumerate}
	\item $(\omega\times 2_o)+\omega$ --- one $\omega$-sequence, followed by another, followed by a third.
	\item $\omega+(\omega\times 2_o)$ --- same as above.
	\item $\omega\times\omega$ --- an $\omega$-sequence, followed by a second, followed by a third, and so on, with one $\omega$ sequence for every natural number. That is: an $\omega$-sequence of $\omega$-sequences.
	\item $(\omega\times\omega)\times\omega$ --- an $\omega$-sequence of the things described in the previous answer. That is: an $\omega$-sequence of $\omega$-sequences of $\omega$-sequences.
	\item $(\omega\times\omega)\times(\omega\times\omega)$ --- an $\omega$-sequence of the things described in the previous answer. That is: an $\omega$-sequence of $\omega$-sequences of $\omega$-sequences of $\omega$-sequences.
	\end{enumerate}

\end{enumerate}

\end{document}

