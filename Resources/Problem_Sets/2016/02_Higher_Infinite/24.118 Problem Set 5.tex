


\documentclass[12pt,a4paper]{article}


%Spacing Packages
\usepackage{fullpage}
\usepackage{a4wide}

%Other Packages
\usepackage{amssymb}
\usepackage{amsmath}
\usepackage{euscript}
\usepackage{enumerate}

\usepackage[lf]{venturis} %% lf option gives lining figures as default; 
			  %% remove option to get oldstyle figures as default
\usepackage[T1]{fontenc}



\begin{document}

\begin{quote}

\begin{center} {\large 24.118 -- Paradox and Infinity \\ \vspace{1mm}}
 {\large Problem Set 5: Orderings and the Higher Infinite\\ \vspace{1mm}}
 
\end{center}
\vspace{3mm}

\noindent How these problems will be graded:

\begin{itemize} 

\item You will be graded both on the basis of whether your answers are correct and on the basis of whether they are properly justified. 
\end{itemize} 

You can assume anything in the course notes. There is no word limit.


\end{quote}

\subsection*{Problems:}


\begin{enumerate}

\item Definitions: 
\begin{itemize}
\item $\varnothing$ is the empty set. If $A$ and $B$ are sets, $A-B$ is the set whose members are the elements of $A$ that are not also in $B$ (so, for instance $\{1, 2\} - \{1\} = \{2\}$). Note that $\{1, \{1\}\} - \{1\}$ is $\{\{1\}\}$, \emph{not} $\{1\}$.
\item $\mathcal{P}(A)$ is the power set of $A$ (i.e. the set of all subsets of $A$).
\end{itemize}

Say of each of the following sets whether it is well-ordered by $\in$. (1 point each)
\begin{enumerate}[(a)]
\item $\varnothing$
\item $\mathcal{P}(\mathcal{P}(\mathcal{P}(\varnothing)))$
\item $\mathcal{P}(\mathcal{P}(\mathcal{P}(\varnothing))) - \{\varnothing\}$
\item $\mathcal{P}(\mathcal{P}(\mathcal{P}(\varnothing))) - \{\{\varnothing\}\}$
\item $\mathcal{P}(\mathcal{P}(\mathcal{P}(\varnothing))) - \{\{\{\varnothing\}\}\}$
\end{enumerate}

\item Recall that that the ordinals are well-ordered, and that $\alpha<_o\beta$ iff $\alpha\in\beta$.
\begin{enumerate}[(a)]
\item Show that there is no ordinal $\alpha$ such that $\alpha<_o\varnothing$. (2 points)
\item Show that there is no infinite ordinal (i.e., no ordinal with infinitely many members) $\alpha$ such that $\alpha<_o\omega$. (2 points)
\end{enumerate}

\item  If $\alpha$ is an ordinal, say that $\alpha'=\alpha \cup \{\alpha\}$. (For example, $\omega' = \omega \cup \{\omega\} = \{0_o, 1_o,\ldots,\omega\}$.)
\begin{enumerate}[(a)]
\item Show that there is a bijection between $\omega$ and $\omega'$. (1 point)
\item Show that $\omega$ and $\omega'$, as ordered by $<_o$, do not have the same well-order type --- that is, the orderings are \emph{not} isomorphic. (2 points)
\end{enumerate}

\item Below, ``+'' and ``$\times$'' always refer to ordinal addition and ordinal multiplication.

Which of the following are true? (1 point each)
\begin{enumerate}[(a)]
\item $\alpha + 1_o =  \alpha'$
\item $1_o + \omega = \omega + 1_o$
\item $1_o + 3_o = 3_o + 1_o$
\item $(\omega + 2_o) + \omega <_o (\omega + \omega) + 3_o$
\item $1_o \times 3_o = 3_o\times 1_o$ 
\item $3_o\times\omega = (\omega + \omega) + \omega$
\item $\omega\times 3_o = \omega + (\omega + \omega)$
\item $(\omega\times 2_o) + \omega <_o (\omega\times\omega) + 2_o$
\item $\omega\times\omega <_o  \omega\times (2_o\times\omega)$
\item $\omega\times (\omega + \omega) = (\omega\times \omega) + (\omega\times \omega)$
\end{enumerate}

\item \textbf{Extra Credit}

Explain in words what the well-ordering represented by each of the following ordinals looks like. (1 point each)
\begin{enumerate}[(a)]
\item $(\omega\times 2_o) + \omega$
\item $\omega + (\omega\times 2_o)$
\item $\omega\times\omega$ 
\item $(\omega\times\omega)\times\omega$
\item $(\omega\times\omega)\times(\omega\times\omega)$
\end{enumerate}

\end{enumerate}


\end{document}





