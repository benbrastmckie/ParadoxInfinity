


\documentclass[12pt,a4paper]{article}


%Spacing Packages
\usepackage{fullpage}
\usepackage{a4wide}

%Other Packages
\usepackage{amssymb}
\usepackage{amsmath}
\usepackage{euscript}

\usepackage[lf]{venturis} %% lf option gives lining figures as default; 
			  %% remove option to get oldstyle figures as default
\usepackage[T1]{fontenc}

\begin{document}

%\section{}
%\subsection{}
\begin{quote}
\begin{center} {\large 24.118x -- Paradox and Infinity \\ \vspace{1mm}}
 {\large Problem Set 2: Newcomb's Problem \\ \vspace{1mm}}
 
 \end{center}
\vspace{3mm}

\noindent How these problems will be graded:
\begin{itemize} 

\item Assessment will be based on both whether you give the correct answer and on the \emph{reasons} you give in support of your answers. (Note that not every question has a single correct answer.) Even if it is unclear whether your answer is correct, it can be clear whether or not the reasons you have given in support of your answer are good ones. 

\item  \emph{No answer may consist of more than 150 words}. Words after the first 150 will be ignored.


\end{itemize} 

These two constraints are often in competition: it may sometimes seem to you that you can't argue for your answer properly in 150 words or less. Learning to deal with this problem is a skill you will acquire with practice. The ability to distill what is essential about a point in a few words requires clear thinking, and it is clear thinking that we are after.

\end{quote}

\subsubsection*{Problems:}
\subsubsection*{Evidential Decision Theory vs. Causal Decision Theory}

The following example is due to Caspar Hare.

Imagine that the universe is divided in two along a plane. As far as anyone can tell, the universe is completely symmetrical across this plane. Everything that happens on one side appears to be an exact mirror of everything that happens on the other side; there are so far no observed divergences from this pattern.

The plane alternates between being opaque and transparent. Scientists have established that when the plane is opaque, there are no causal interactions between events on different sides of the plane; when it is transparent, you can see through it, but otherwise there are no causal interactions.

The plane is a great tourist attraction. People go up to the plane to peer into the other side, and invariably see their symmetrical twin peering back at them. A favourite trick is to wait until the plane turns opaque, reveal some crazy prop or hold some ridiculous pose, and then when the plane turns transparent find your twin holding exactly the same crazy prop or holding the same ridiculous pose.

Now, suppose you are a tourist at the plane, having fun with your symmetry twin, when the plane turns opaque. A mysterious but credible stranger comes up to you and offers you a deal. ``Here is a thousand dollars,'' she says.``That's yours to keep, if you want. But guess what; my symmetrical twin is right now giving your twin, on the other side of the opaque plane, a thousand dollars. And if, when the plane turns transparent, it turns out that your twin has burned her thousand dollars, I will give you a million dollars. By the way, here is a lighter."

Suppose the only thing you want to do is make the most money possible out of this situation.

\begin{enumerate}
\item What should you do, according to an evidential decision theorist? Explain.
\item What should you do, according to a causal decision theorist? Explain.
\end{enumerate}

\subsubsection*{Indicative and Counterfactual Conditionals}
In many cases, indicative and counterfactual conditionals go together: If ``If I start my homework on Sunday, I'll be done by Monday" is true, then ``If I had started my homework on Sunday, I would have been done by Monday" is true (or will soon become true). That is, if the indicative conditional is true, then the corresponding counterfactual conditional is true.

This is not the case in the Newcomb scenario. The indicative conditional is, ``If you one-box, you'll almost certainly find a million dollars in the large box." True. Now supposed you two-box, and as expected, the large box is empty. The counterfactual conditional ``If you had one-boxed, you almost certainly would have found a million dollars in the large box" is clearly false, since the box is empty, and it was empty from the start. 

\begin{enumerate}
\setcounter{enumi}{2}
\item Give an example of a scenario and two corresponding conditionals (one indicative, one counterfactual) for which it seems intuitive that if the indicative conditional is true, the counterfactual conditional is also true. (The homework scenario and conditionals given above would be one such example; give another.)
\item Give an example of a scenario and two corresponding conditionals (one indicative, one counterfactual) for which it seems intuitive that if the indicative conditional is true, the counterfactual conditional is false. (The Newcomb scenario and conditionals are one such example; give another.)
\end{enumerate}


\subsubsection*{Tickle Defense}

\begin{enumerate}
\setcounter{enumi}{4}
\item Construct a tickle-defence for the Newcomb case. Be clear on what screens off what.
\end{enumerate}

\subsubsection*{Prisoner's Dilemma}

Jones and you committed a crime, and are under arrest. The police don't have much evidence against you, so if both of you keep quiet you'll each be charged with a relatively minor offence, and spend only 10 days in prison. But it is common knowledge between you and Jones that there is an offer on the table. If either of you agrees to sign a statement accusing the other, the defector will be allowed to leave scot-free, and the accused will be charged with a felony offence, and sentenced to 10,000 days in prison. Unfortunately, there is a catch. Should Jones and you both choose to defect, you will both be charged with felony offences, but given the lesser sentence of 9,000 days in prison because of your cooperation with the police.

\begin{center}
\begin{tabular}{r ||c| c}
& \textbf{Jones Keep Quiet} & \textbf{Jones Defects}\\ \hline \hline
\textbf{You Keep Quiet} & 10 days in prison & 10,000 days in prison\\ \hline
\textbf{You Defect} & 0 days in prison & 9,000 days in prison\\
\end{tabular}
\end{center}

Jones and you have been placed in separate interrogation rooms, and are not allowed to communicate with each other. So neither of you can do anything to affect the decision of the other. 

Jones and you will never see each other again, regardless of what you decide to do. And your action will have no consequences beyond those that have already been mentioned. (You need not worry about Jones exacting revenge, or about feeling guilt, or about acquiring a bad reputation, or anything like that.) You don't care at all about Jones -- she's a jerk. All that matters to you, as you make your decision, is minimising your amount of time in prison.

\begin{enumerate}
\setcounter{enumi}{5}
\item What should you do --- keep quiet or defect? Why? Be sure to reference principles of decision making that we have talked about in class.
\end{enumerate}

Consider the following twist on the case: Jones is your clone. The two of you are genetically identical, and have grown up in very similar environments. So there is a very high chance that the two of you will go through similar trains of thought, and end up making the same decision.
\begin{enumerate}
\setcounter{enumi}{6}
\item Does this make a difference to what you should do? Why, or why not? Be sure to reference principles of decision making that we have talked about in class.
\end{enumerate}


\end{document}  