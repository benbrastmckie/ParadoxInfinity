\documentclass[11pt]{article}

\usepackage{amsmath,amsthm,amsfonts,amssymb,amscd}

\usepackage[margin=1.5in,headsep=.5in]{geometry}

\usepackage{fancyhdr}

\setlength{\headheight}{20pt}

\usepackage[colorlinks]{hyperref} 
\usepackage{cleveref}

\usepackage{enumerate}

\usepackage{enumitem}
\setlist[enumerate]{itemsep=0mm}

\theoremstyle{definition}
\newtheorem{defn}{Definition}
\newtheorem{reg}{Rule}
\newtheorem{exer}{Exercise}
\newtheorem{note}{Note}
\newtheorem*{theorem*}{Theorem}
\newtheorem{theorem}{Theorem}[section]
\newtheorem{corollary}{Corollary}[theorem]
\newtheorem{thm}{Theorem}
\newtheorem{prop}[thm]{Proposition}
\newtheorem{lem}[thm]{Lemma}
\newtheorem{conj}[theorem]{Conjecture}

\pagestyle{fancy}

\newcommand{\counterfactual}{\ensuremath{%
  \Box\kern-1.5pt
  \raise1pt\hbox{$\mathord{\rightarrow}$}}}

\begin{document}

\pagenumbering{gobble}

\lhead{$24.118$ Paradox and Infinity}
\rhead{Recitation $6$: Probability}



\begin{center}
{\LARGE \bf Answers to In-class Exercises}
\end{center}

\smallskip

\section{Trouble in St. Petersburg}

Wandering\footnote{The two puzzles in this section and the next are both taken from Arntzenius, Elga, and Hawthorne, "Bayesianism, infinite decisions, and binding" (2004).} along the Nevsky Prospect, Bill Gates encounters Ivan, a shady character who is casually tossing a coin. 'Hey Biell, you vant some acsion?' Ivan asks. 'Here's the deel. I offer some bets. You take or leaf. That's iet.' Ivan explains that he will toss the coin until it lands tails, and offers Bill an infinite sequence of bets.

\begin{exer}
Suppose the coin is unbiased: the probability that it lands heads/tails on each toss is $1/2$. What is the probability that it first lands tails on toss 4? What it the probability that it never lands tails?
\end{exer}

\begin{proof}[Answer]
Since each coin toss is independent, $Pr(\text{first lands tails on toss 4}) = (\frac{1}{2})^4 = \frac{1}{16}$; $Pr(\text{never lands tails}) = \prod_{n=1}^{\infty}\frac{1}{2} = 0$.

\end{proof}


Here are the bets Ivan offers:
\begin{enumerate}[leftmargin=2 \parindent]
\item[Bet $1$:] You lose $1$ if the coin never lands tails, you gain $3$ if it first lands tails on toss 1. If it first lands tails on some other toss, the bet is void.
\item[Bet $2$:] You lose $4$ if the coin first lands tails on toss 1, you gain $9$ if it first lands tails on toss 2. If it first lands tails on some other toss, the bet is void.
\item[Bet $3$:] You lose $10$ if the coin first lands tails on toss 2, you gain $21$ if it first lands tails on toss 3. If it first lands tails on some other toss, the bet is void.
\item[...]
\item[Bet $n$:] Let $k$ what you will lose if you lose Bet $n-1$, and $m$ be what you will win if you win Bet $n-1$. You lose $2k+2$ if the coin first lands tails on toss $n-1$, you gain $2m+3$ if it first lands tails on toss $n$. If it first lands tails on some other toss, the bet is void.
\item[...]
\end{enumerate}

\begin{exer}
What is the expected utility of Bet $1$? What about Bet $4$? Can you show that each bet is guaranteed to have a positive utility?
\end{exer}

\begin{proof}[Answer]
\begin{align*}
EV(B_1) = (-1) \times 0 + 3 \times \frac{1}{2} = \frac{3}{2} \\
EV(B_3) = (-22) \times \frac{1}{8} + 45 \times \frac{1}{16} = \frac{-11}{4} + \frac{45}{16} = \frac{1}{16}
\end{align*}
Consider Bet $n$. It is easy to see that suppose $k$ is what you will lose if you lose the Bet, then $2k+1$ is what you will gain if you win the Bet. Also, the probability that you will lose is $\frac{1}{2^{n-1}}$, and the probability that you will win is $\frac{1}{2^n}$. Hence
\begin{align*}
EV(B_n) = (-k) \times \frac{1}{2^{n-1}} + (2k+1) \times \frac{1}{2^n} = (-k + (2k+1) \times \frac{1}{2}) \times \frac{1}{2^{n-1}} = \frac{1}{2^n}
\end{align*}

\end{proof}


Since each bet has a positive expectation, Bill takes them all. Ivan says 'Gud. No neet to sro ze coin. Just gif me vun dollar. Sanks.' 

\begin{exer}
Show that Ivan is right in saying that: in particular, show that in each situation, Bill will lose one dollar. 
\end{exer}

\begin{proof}[Answer]
Suppose the coin first lands tail on toss $n$. Then all the other bets except Bet $n$ and Bet $n+1$ will go void. Also, Bill will win Bet $n$ and lose Bet $n+1$. But it is easy to calculate that if the amount of money that Bill will gain on Bet $n$ is $k$, then the amount of money that Bill will lose on Bet $n+1$ is $k+1$. Hence Bill will lose one dollar. Also, if the coin never lands tails, then all the other bets except Bet $1$ will go void, and Bill will lose Bet $1$ and hence lose one dollar.

\end{proof}

\section{Two Tickets to Paradise}

Dutch state lottery officials have decided to hold a special lottery, which works as follows. There is one lottery ticket, $A$, which will be sold to the highest bidder. The day after it is sold, the dollar value of $A$ will be determined by a chance process as follows:

\begin{quote}
$A$ is worth $\$ 1$ with chance $\frac{1}{8}$, $\$ 2$ with chance $\frac{7}{32}$, $\$ 4$ with chance $\frac{21}{128}$, $\$ 8$ with chance $\frac{63}{512}$, etc. The remaining possible dollar values are all and only the remaining powers of 2. For all values $n$ greater or equal to $2$, if the chance of $n$ is $\frac{x}{y}$, the chance of $2n$ is $\frac{3x}{4y}$.
\end{quote}

\begin{exer}
What is the expected utility of $A$?
\end{exer}

\begin{proof}[Answer]
$$EV(A) = 1 \times \frac{1}{8} + 2 \times \frac{7}{32} + 4 \times \frac{21}{128} + ...$$
It is easy to see that each subsequent term is bigger than each previous term, since the probabilities get multiplied by 3/4 each time, and the dollar amounts double each time. So all terms are bigger than $\frac{1}{4}$. So the sum is infinite. (One might prefer to say that the expected amount is ill-defined rather than infinite, since the sum in question does not converge to a finite number.)

\end{proof}

According to Bill Gates, dollars equal utilities. So he is willing to pay any amount for $A$. He does his research and wins the auction for the ticket. However, before the chance process which determines the worth of $A$ has occurred, the lottery officials offer him a new deal:

\begin{quote}
``If you give us ticket $A$ back, and pay us $\$ 0.01$, we will give you a new lottery ticket $B$. The value of $B$ will be determined as follows. After we have determined the value of $A$, we will roll a fair seven-headed die. If the die comes up $1$, $2$ or $3$, then $B$ will be worth double what $A$ is worth. Otherwise $B$ will be worth half what $A$ is worth. The only exception is that if $A$ turns out to be worth $1$, then, for sure, $B$ will be worth $2$. Do you want this deal?'
\end{quote}

\begin{exer}
Suppose $A$ turns out to be worth $\$ 2^n$, what will be the expected value of exchanging $A$ with $B$? Should Bill accept the deal?
\end{exer}

\begin{proof}[Answer]
If $n = 0$, then $EV(B) = 2 - 1.01 = 0.99$. 
If $n > 0$, then
$$EV(B) = \frac{3}{7} \times 2^{n+1} + \frac{4}{7} \times 2^{n-1} -2^n - 0.01 = \frac{2}{7} \times 2^{n-1} - 0.01$$
Bill should accept the deal as the expected value of switching is positive in every possible situation.

\end{proof}

Bill takes the offer and gives them $A$ and $\$ 0.01$. The Dutch officials then offer Bill another deal. They say: 

\begin{quote}
``We have with us a very special ticket. Its value will also be determined by a chance process. Given any possible value of $B$, the chance that this special ticket will be worth twice as much as $B$ is $\frac{3}{7}$, and given any value of $B$ the chance that this special ticket will be worth half what $B$ is worth is $\frac{4}{7}$. The only exception occurs is if $B$ is worth $1$; in that case this ticket is worth $2$. Do you want to give us ticket $B$ and $0.01$ in exchange for this very special ticket?"
\end{quote}



\begin{exer}
What is the expected value of exchanging? Should Bill accept the deal?
\end{exer}

\begin{proof}[Answer]
The reasoning is exactly the same as in the case of the previous deal.

\end{proof}

Bill accepts the deal again and gives the officials $B$ and $\$ 0.01$. To his astonishment the Dutch officials then give him ticket $A$ back, and say that this concludes the deal.

\begin{exer}
Show that the Dutch officials are right: $A$ is indeed such a special ticket.
\end{exer}

\begin{proof}[Answer]
For any $n$,
\begin{align*}
Pr(A = 2^n \land B = 2^{n+1}) & = \frac{1}{8} \times (\frac{3}{4})^n = Pr(A = 2^{n+1} \land B = 2^n) \\
Pr(B = 2^{n+1}) & = Pr(A = 2^{n} \land B = 2^{n+1}) +  Pr(A = 2^{n+2} \land B = 2^{n+1}) \\
& = \frac{1}{8} \times (\frac{3}{4})^n + \frac{1}{8} \times (\frac{3}{4})^{n+1} \\
& = \frac{1}{8} \times ((\frac{3}{4})^n + (\frac{3}{4})^{n+1})
\end{align*}
Hence, by Bayes' Law, 
\begin{align*}
Pr(A = 2^n | B = 2^{n+1}) = \frac{Pr(A = 2^n \land B = 2^{n+1})}{Pr(B = 2^{n+1})} = \frac{\frac{1}{8} \times (\frac{3}{4})^n}{\frac{1}{8} \times ((\frac{3}{4})^n + (\frac{3}{4})^{n+1})} = \frac{4}{7} 
\end{align*}
Similarly, when $n > 0$,
\begin{align*}
Pr(B = 2^n | A = 2^{n+1}) & = \frac{Pr(A = 2^{n+1} \land B = 2^{n})}{Pr(A = 2^{n+1})} = \frac{\frac{1}{8} \times (\frac{3}{4})^n}{\frac{7}{32} \times (\frac{3}{4})^n} = \frac{4}{7} \\
Pr(A = 2^{n+1} | B = 2^n) & = \frac{Pr(B = 2^n | A = 2^{n+1}) \times Pr(A = 2^{n+1})}{(Pr(B = 2^n)} \\
& = \frac{\frac{4}{7} \times\frac{7}{32} \times (\frac{3}{4})^n }{\frac{1}{8} \times ((\frac{3}{4})^{n-1} + (\frac{3}{4})^{n})} = \frac{3}{7}
\end{align*}
Finally, if $B = 1$, then it has to be the case that $A = 2$ and the dice lands on $4$ or above. Hence $Pr(A = 2 | B = 1) = 1$.

\end{proof}

\end{document}