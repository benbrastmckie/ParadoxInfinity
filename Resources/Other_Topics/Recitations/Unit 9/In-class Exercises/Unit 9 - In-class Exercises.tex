\documentclass[11pt]{article}

\usepackage{amsmath,amsthm,amsfonts,amssymb,amscd}

\usepackage[margin=1.5in,headsep=.5in]{geometry}

\usepackage{fancyhdr}

\setlength{\headheight}{20pt}

\usepackage[colorlinks]{hyperref} 
\usepackage{cleveref}

\usepackage{mathrsfs}
\usepackage{mathtools}

\usepackage{enumerate}

\usepackage{enumitem}
\setlist[enumerate]{itemsep=0mm}

\usepackage{placeins}

\theoremstyle{definition}
\newtheorem{defn}{Definition}
\newtheorem{reg}{Rule}
\newtheorem{exer}{Exercise}
\newtheorem{note}{Note}
\newtheorem*{theorem*}{Theorem}
\newtheorem{theorem}{Theorem}[section]
\newtheorem{corollary}{Corollary}[theorem]
\newtheorem{thm}{Theorem}
\newtheorem{prop}[thm]{Proposition}
\newtheorem{lem}[thm]{Lemma}
\newtheorem{conj}[theorem]{Conjecture}
\newtheorem*{wo}{The Well-Ordering Principle}

\pagestyle{fancy}

\newcommand{\counterfactual}{\ensuremath{%
  \Box\kern-1.5pt
  \raise1pt\hbox{$\mathord{\rightarrow}$}}}
  
  \def\Hat{\mkern-3mu\text{\textasciicircum}}

\begin{document}

\pagenumbering{gobble}

\lhead{$24.118$ Paradox and Infinity}
\rhead{Recitation $9$: G\"odel's Theorem}



\begin{center}
{\LARGE \bf In-class Exercises}
\end{center}

\smallskip


The goal of the following exercises is to first show you an alternative proof of G\"odel's First Incompleteness Theorem, using the famous G\"odel sentence, and then show you how G\"odel's Second Incompleteness Theorem is derivable from the First. In the following, we will refer to the formal system of arithmetic as the system $R$.

\section{G\"odel Coding}

\begin{defn}
The arithmetical language, $\mathscr{L}$, consists of the followings symbols: $\neg, \&, \forall, (, ), 0, 1, =, +, \times, \hat{\text{}}, x_0, x_1, x_2, ...$
\end{defn}

For each symbol in $\mathscr{L}$, say $\neg$, we associate with it a unique natural number as its \textit{G\"odel code}, $\ulcorner \neg \urcorner$. The codes for all symbols are as follows: 
\begin{table}[h]
\begin{tabular}{llllllllllllllll}
Symbols    & $\neg$ & $\&$  & $\forall$  & (  &  )  & 0 & 1 & = & + & $\times$  & $\, \, \hat{\text{}}$  &  $x_0$  &  $x_1$  &  $x_2$  & ...         \\
Code    & 1 & 2 & 3 & 4 & 5 & 6 & 7 & 8 & 9 & 10 & 11 & 13 & 17 & 19 & ...
\end{tabular}
\end{table}

The G\"odel number of each numerical variable is a prime number strictly greater than 11; we could continue this list to include countably many more numerical variables.

If $\phi$ is a string of $m$ many symbols, with G\"odel codes $n_1, ..., n_m$, then the G\"odel code of $\phi$, $\ulcorner \phi \urcorner$, equals to $p_1^{n_1} \times p_2^{n_2} \times ...\times p_m^{n_m}$, where $p_1$ is the first prime number (i.e. 2), $p_2$ is the second prime number (i.e. 3), etc.

\begin{exer}
What is the G\"odel code of the sentence $\neg (0=1)$ (it is not the case that 0 = 1)? You can leave it in factored form!
\end{exer}

\section{Translating Meta-Mathematics into Arithmetic}

The point of G\"odel coding is to translate meta-mathematical sentences into purely arithmetical ones.

\begin{exer}
Show that a sentence $\phi$ begins with ``$\neg$" if and only if $\ulcorner \phi \urcorner$ is divisible by $2$ but not by $4$.
\end{exer}

The above example shows how we translate a meta-mathematical predicate - a predicate about which symbol a mathematical sentence begins with - into an arithmetic predicate about divisibility that can be expressed using the language $\mathscr{L}$.

\begin{exer}
Express the predicate ``$x_0$ is divisible by 2 but not by $4$" in $\mathscr{L}$.
\end{exer}

G\"odel showed that virtually all meta-mathematical predicates about the formal system of arithmetic can be translated into predicates definable in $\mathscr{L}$ via G\"odel coding, just like in the above example. Most importantly, the meta-mathematical predicate ``has a proof in the formal system of arithmetic", or in short, ``is provable in $R$", can be translated into a predicate definable in $\mathscr{L}$:

\begin{theorem}\label{Prov}
We can define in $\mathscr{L}$ a predicate $Prov(x_0)$, such that for any sentence $\phi$, $\phi$ is provable in $R$ if and only if $Prov(\ulcorner \phi \urcorner)$ is provable in $R$.
\end{theorem}

Another major key step in G\"odel's proof of the incompleteness theorem is the following famous lemma:

\begin{theorem}[Diagonalization Lemma] Let $F(x_0)$ be a predicate definable in $\mathscr{L}$. Then there exists a sentence $\psi$ in $\mathscr{L}$ such that it is provable in $R$ that $\psi \leftrightarrow F(\ulcorner \psi \urcorner )$\footnote{$\psi \leftrightarrow F(\ulcorner \psi \urcorner)$ is the conjunction of $\psi \rightarrow F(\ulcorner \psi \urcorner)$ and $F(\ulcorner \psi \urcorner) \rightarrow \psi$.}.

\end{theorem}

Metaphorically, the Diagonalization Lemma tells us that for any predicate $F$ in $\mathscr{L}$, there is a sentence that says ``I am $F$". This sentence is usually referred to as the \textit{fixed point} of $F$.

These two theorems are the keystones of G\"odel's results, and they are too hard to be left as exercises. So we will let you use them for free.

\section{The G\"odel' Sentence and First Incompleteness}

The G\"odel sentence is the fixed point of the negation of the provability predicate:

\begin{corollary} \label{GS}
There exists a sentence $\theta$ in $\mathscr{L}$ such that it is provable in $R$ that $\theta \leftrightarrow \neg Prov (\ulcorner \theta \urcorner)$.
\end{corollary}

\begin{defn} \label{ConsistentComplete}
Let $S$ be a formal system. Then $S$ is
\begin{enumerate}
\item \textit{consistent} just in case there is no sentence $\phi$ such that both $\phi$ and $\neg \phi$ are provable in $S$;
\item \textit{complete} just in case for every sentence $\phi$, either $\phi$ or $\neg \phi$ is provable in $S$.
\end{enumerate}
\end{defn}

\begin{theorem}[G\"odel''s First Incompleteness Theorem] If a formal system $S$ is consistent and at least as strong as $R$, then it is not complete.
\end{theorem}

In the lectures we have already seen a proof of this theorem using Turing machines. Now we will try to go through another proof of this theorem using the G\"odel sentence $\theta$. We will focus on the special case when $S$ is precisely $R$.

\begin{exer} \label{Exer}
Show that if $R$ is consistent, then $\theta$ is not provable in $R$.
\end{exer}

\begin{exer}
Show that if $R$ is consistent, then $\neg \theta$ is not provable in $R$.
\end{exer}

\begin{exer}
Show that if $R$ is consistent, then it is not complete.
\end{exer}

\section{The Second Incompleteness Theorem}

G\"odel's Second Incompleteness Theorem follows from the first Incompleteness Theorem. It states that any formal system that as at least as strong as $R$ cannot prove its own consistency. This does not mean that such a system is inconsistent, or that its consistency can never be proved; it only means that to prove the consistency of such a formal system, we need another (stronger) formal system. \\

Recall that the point of G\"odel coding is to translate meta-mathematical sentences into purely arithmetical ones. With the provability predicate, we are now able to translate the meta-mathematical sentence ``$R$ is consistent" into a sentence in the language of arithmetic.

\begin{defn}
$Con(R)$ is the following sentence in the language of arithmetic: $\neg Prov(\ulcorner 0 = 1 \urcorner)$\footnote{I know what you are thinking. As defined above, $Con(R)$ is literally the sentence that says that ``0=1" is not provable in $R$. Why is it equivalent to saying that $R$ is consistent? Well, this is because in classical logic we have the principle of explosion, which is the principle that allows us to infer \textit{any} sentence from a contradiction, which is a sentence of the form $\phi \land \neg \phi$. Now, if $R$ is not consistent, then by Def \ref{ConsistentComplete}, we can prove a contradiction in $R$, and therefore any sentence is provable in $R$, including the sentence that $0=1$. Conversely, if ``$0=1$" is provable in $R$, then since ``$\neg (0 = 1)$" is also provable in $R$ (it quite directly follows from the axioms in $R$ that says $0$ is not a successor and $1$ is the successor of $0$), the contradiction that $0=1 \land \neg (0 = 1)$ is provable in $R$, and hence $R$ is inconsistent.}.

\end{defn}

To prove the Second Incompleteness from the First we need another lemma. It says that the whole proof of Exercise \ref{Exer} can be formalized \textit{within} $R$, using the language of arithmetic, in the following sense ($\theta$ is the G\"odel sentence):

\begin{lem} \label{lemma}
It is provable in $R$ that $Con(R) \rightarrow \neg Prov(\theta)$.
\end{lem}

\begin{theorem}[G\"odel''s Second Incompleteness Theorem]
If a formal system $S$ is consistent and at least as strong as $R$, then $Con(S)$ is not provable in $S$.
\end{theorem}

\begin{exer}
Use Lemma \ref{lemma}, show that if $R$ is consistent, then $Con(R)$ is not provable in $R$.
\end{exer}


\end{document}