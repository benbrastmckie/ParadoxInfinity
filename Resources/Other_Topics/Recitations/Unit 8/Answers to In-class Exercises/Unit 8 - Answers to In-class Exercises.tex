\documentclass[11pt]{article}

\usepackage{amsmath,amsthm,amsfonts,amssymb,amscd}

\usepackage[margin=1.5in,headsep=.5in]{geometry}

\usepackage{fancyhdr}

\setlength{\headheight}{20pt}

\usepackage[colorlinks]{hyperref} 
\usepackage{cleveref}

\usepackage{enumerate}

\usepackage{enumitem}
\setlist[enumerate]{itemsep=0mm}

\theoremstyle{definition}
\newtheorem{defn}{Definition}
\newtheorem{reg}{Rule}
\newtheorem{exer}{Exercise}
\newtheorem{note}{Note}
\newtheorem*{theorem*}{Theorem}
\newtheorem{theorem}{Theorem}[section]
\newtheorem{corollary}{Corollary}[theorem]
\newtheorem{thm}{Theorem}
\newtheorem{prop}[thm]{Proposition}
\newtheorem{lem}[thm]{Lemma}
\newtheorem{conj}[theorem]{Conjecture}
\newtheorem*{wo}{The Well-Ordering Principle}

\pagestyle{fancy}

\newcommand{\counterfactual}{\ensuremath{%
  \Box\kern-1.5pt
  \raise1pt\hbox{$\mathord{\rightarrow}$}}}

\begin{document}

\pagenumbering{gobble}

\lhead{$24.118$ Paradox and Infinity}
\rhead{Recitation $8$: Computability}



\begin{center}
{\LARGE \bf Answers to In-class Exercises}
\end{center}

\smallskip


In the following exercises, we will use $T_n$ to denote the Turing machine with code $n$. Also, we will assume that Turing machines only take natural numbers as inputs and only deliver natural numbers as outputs.

\begin{defn}
A Turing Machine (TM), $T$, is \textit{total} just in case for any input $i \in \mathbb{N}$, $T$ halts on $i$. 
\end{defn}

\begin{exer}
Let $f: \mathbb{N} \times \mathbb{N} \rightarrow \mathbb{N}$ be the following function:
$$
f(n, i) = 
\begin{cases}
j, \, \, \, \, \text{if $n$ codes some total TM $T_n$, and $T_n$ outputs $j$ on the input $i$.} \\
0, \, \, \, \, \text{if otherwise.}
\end{cases}
$$

Show that $f$ is not Turing computable, i.e. it is impossible for there to be a Turing machine that computes $f$.
\end{exer}

\begin{proof}[Answer]
Assume for reductio that there is a Turing machine, $T$, that computes $f$. We construct another Turing machine, $T'$, that uses $T$ as a sub-machine, and derive a contradiction from it.

Let $T'$ be as follows: given an input $k \in \mathbb{N}$, first compute the value of $f(k, k)$ using $T$. Let $x$ be the output of $T$ on $(k, k)$. Then let $x+1$ be the final output of $T'$.

Observe that since $f$ has a value on every input, $T$ is a total Turing machine. And therefore $T'$ is a total Turing machine.

Let $m$ be the code of $T'$. What is the output of $T'$ given $m$? Since $m$ codes a total Turing machine, $f(m, m) = T'(m)$. But then $T'(m) = f(m, m)+1 = T'(m)+1$, and hence we have a contradiction.

\end{proof}

\begin{exer}
Let $f: \mathbb{N} \rightarrow \mathbb{N}$ be the following function:
$$
f(n) = 
\begin{cases}
1, \, \, \, \, \text{if $n$ codes some TM $T_n$, and $T_n$ halts on the input $2$.} \\
0, \, \, \, \, \text{if otherwise.}
\end{cases}
$$

Show that $f$ is not Turing computable, i.e. it is impossible for there to be a Turing machine that computes $f$. 
\end{exer}

\begin{proof}[Answer]
Let $T_n$ be an arbitrary $TM$ coded by $n$. We construct another $TM$, $T'_n$, based on $T_n$, as follows: given any natural number $k \in \mathbb{N}$ as input, $T_n$ will halt if $T_n$ halts on the input $n$, and $T_n$ will go on an infinite loop if $T_n$ does not halt on $n$.

Since $T'_n$ has the same behavior on any input, it will behave like this on the input $2$. In particular, it will halt on the input $2$ iff $T_n$ halts on $n$, and it will loop on $2$ if $T_n$ loops on $n$. 

Since $T'_n$ is a $TM$, it has a code. Let its code be $n'$.

Now consider $f(n')$. By definition, 
$$
f(n') = 
\begin{cases}
1, \, \, \, \, \text{if $T'_n$ halts on the input $2$.} \\
0, \, \, \, \, \text{if $T'_n$ loops on the input $2$.}
\end{cases}
$$

Plugging in the equivalences we just get:
$$
f(n') = 
\begin{cases}
1, \, \, \, \, \text{if $T_n$ halts on the input $n$.} \\
0, \, \, \, \, \text{if $T_n$ loops on the input $n$.}
\end{cases}
$$

But this reduces $f$ to the halting function. And we can use this to show that $f$ is not Turing computable. Assume for reductio that $f$ is computed by some Turing machine $T$. Then we construct an impossible Turing machine $T^I$, using $T$ as a sub-program, as follows: given $n \in \mathbb{N}$ as input, we first calculate the value of $n'$. This is doable as $n'$ is defined explicitly using $n$. Then we run $T$ on $n'$, and if $T$ outputs $1$ on $n'$, we let $T^I$ go on an infinite loop; if $T$ outputs $0$ on $n'$, we let $T^I$ halt.

Since $T^I$ is a $TM$ (on the assumption that $T$ exists), it has a code $e$. And we can calculate the value of $e'$ using $e$. Now, feed $e'$ to $f$. $f(e') = 1$ iff $T$ outputs $1$ on $e'$ iff $T^I$ loops on $e$ iff $f(e') = 0$. Contradiction. 

\end{proof}

\begin{exer}
Let $f: \mathbb{N} \rightarrow \mathbb{N}$ be the following function:
$$
f(n) = 
\begin{cases}
1, \, \, \, \, \text{if Condition 1 is the case.} \\
0, \, \, \, \, \text{if Condition 1 is not the case.}
\end{cases}
$$
where Condition 1 is the following: $n$ codes some TM $T_n$, and for every $i \in \mathbb{N}$, either $T_n$ outputs $i$ on the input $i$, or $T_n$ loops on the input $i$.

Show that $f$ is not Turing computable, i.e. it is impossible for there to be a Turing machine that computes $f$. 
\end{exer}

\begin{proof}[Answer]
Let $T_n$ be an arbitrary $TM$ coded by $n$. We construct another $TM$, $T'_n$, based on $T_n$, as follows: given any natural number $i \in \mathbb{N}$ as input, $T_n$ will output $i+1$ if $T_n$ halts on the input $n$, and $T_n$ will go on an infinite loop if $T_n$ does not halt on $n$.

Since $T'_n$ is a $TM$, it has a code. Let its code be $n'$.

Now consider $f(n')$. By definition, $f(n') = 1$ iff for every $i \in \mathbb{N}$, either $T'_n$ outputs $i$ on the input $i$, or $T'_n$ loops on the input $i$. But we know that if $T'_n$ halts on $i$, it will output $i+1$. Therefore it will never output $i$ on $i$. Hence $f(n') = 1$ iff for every $i \in \mathbb{N}$, $T'_n$ loops on the input $i$.

But $T'_n$ loops on every input $i$ just in case $T_n$ loops on $n$. Hence we have: 
$$
f(n') = 
\begin{cases}
1, \, \, \, \, \text{if $T_n$ halts on the input $n$.} \\
0, \, \, \, \, \text{if $T_n$ loops on the input $n$.}
\end{cases}
$$

But this reduces $f$ to the halting function. Using an argument similar to that in the answer to the previous exercise, we can show that $f$ is not computable, as otherwise we will contradict the fact that the halting function is not computable.

\end{proof}




\end{document}