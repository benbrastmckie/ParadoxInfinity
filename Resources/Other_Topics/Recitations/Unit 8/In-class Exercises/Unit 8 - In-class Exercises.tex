\documentclass[11pt]{article}

\usepackage{amsmath,amsthm,amsfonts,amssymb,amscd}

\usepackage[margin=1.5in,headsep=.5in]{geometry}

\usepackage{fancyhdr}

\setlength{\headheight}{20pt}

\usepackage[colorlinks]{hyperref} 
\usepackage{cleveref}

\usepackage{enumerate}

\usepackage{enumitem}
\setlist[enumerate]{itemsep=0mm}

\theoremstyle{definition}
\newtheorem{defn}{Definition}
\newtheorem{reg}{Rule}
\newtheorem{exer}{Exercise}
\newtheorem{note}{Note}
\newtheorem*{theorem*}{Theorem}
\newtheorem{theorem}{Theorem}[section]
\newtheorem{corollary}{Corollary}[theorem]
\newtheorem{thm}{Theorem}
\newtheorem{prop}[thm]{Proposition}
\newtheorem{lem}[thm]{Lemma}
\newtheorem{conj}[theorem]{Conjecture}
\newtheorem*{wo}{The Well-Ordering Principle}

\pagestyle{fancy}

\newcommand{\counterfactual}{\ensuremath{%
  \Box\kern-1.5pt
  \raise1pt\hbox{$\mathord{\rightarrow}$}}}

\begin{document}

\pagenumbering{gobble}

\lhead{$24.118$ Paradox and Infinity}
\rhead{Recitation $8$: Computability}



\begin{center}
{\LARGE \bf In-class Exercises}
\end{center}

\smallskip


In the following exercises, we will use $T_n$ to denote the Turing machine with code $n$. Also, we will assume that Turing machines only take natural numbers as inputs and only deliver natural numbers as outputs.

\begin{defn}
A Turing Machine (TM), $T$, is \textit{total} just in case for any input $i \in \mathbb{N}$, $T$ halts on $i$. 
\end{defn}

\begin{exer}
Let $f: \mathbb{N} \times \mathbb{N} \rightarrow \mathbb{N}$ be the following function:
$$
f(n, i) = 
\begin{cases}
j, \, \, \, \, \text{if $n$ codes some total TM $T_n$, and $T_n$ outputs $j$ on the input $i$.} \\
0, \, \, \, \, \text{if otherwise.}
\end{cases}
$$

Show that $f$ is not Turing computable, i.e. it is impossible for there to be a Turing machine that computes $f$.
\end{exer}

\begin{exer}
Let $f: \mathbb{N} \rightarrow \mathbb{N}$ be the following function:
$$
f(n) = 
\begin{cases}
1, \, \, \, \, \text{if $n$ codes some TM $T_n$, and $T_n$ halts on the input $2$.} \\
0, \, \, \, \, \text{if otherwise.}
\end{cases}
$$

Show that $f$ is not Turing computable, i.e. it is impossible for there to be a Turing machine that computes $f$. 
\end{exer}

\begin{exer}
Let $f: \mathbb{N} \rightarrow \mathbb{N}$ be the following function:
$$
f(n) = 
\begin{cases}
1, \, \, \, \, \text{if Condition 1 is the case.} \\
0, \, \, \, \, \text{if Condition 1 is not the case.}
\end{cases}
$$
where Condition 1 is the following: $n$ codes some TM $T_n$, and for every $i \in \mathbb{N}$, either $T_n$ outputs $i$ on the input $i$, or $T_n$ loops on the input $i$.

Show that $f$ is not Turing computable, i.e. it is impossible for there to be a Turing machine that computes $f$. 
\end{exer}





\end{document}