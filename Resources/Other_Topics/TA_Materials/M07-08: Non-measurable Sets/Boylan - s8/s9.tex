\documentclass[justified]{tufte-handout} 
\usepackage{amsfonts, amssymb, stmaryrd, natbib, qtree, amsxtra}
\usepackage{linguex, color, setspace, graphicx}
\usepackage{enumitem}
\usepackage{bussproofs}
\usepackage{turnstile}
\usepackage{phaistos}
\usepackage{protosem}
\usepackage{txfonts}
\usepackage{pxfonts}
\usepackage[super]{nth}
\thispagestyle{plain}
\definecolor{darkred}{rgb}{0.7,0,0.2}
\bibpunct{(}{)}{,}{a}{}{,}

\input xy
 \xyoption{all}

%New Symbols
\DeclareSymbolFont{symbolsC}{U}{txsyc}{m}{n}
\DeclareMathSymbol{\strictif}{\mathrel}{symbolsC}{74}
\DeclareMathSymbol{\boxright}{\mathrel}{symbolsC}{128}
\DeclareMathSymbol{\Diamondright}{\mathrel}{symbolsC}{132}
\DeclareMathSymbol{\Diamonddotright}{\mathrel}{symbolsC}{134}
\DeclareMathSymbol{\Diamonddot}{\mathord}{symbolsC}{144}

%New commands
\newcommand{\bitem}{\begin{itemize}}
\newcommand{\eitem}{\end{itemize}}
\newcommand{\lang}{$\langle$}
\newcommand{\rang}{$\rangle$}
\newcommand{\back}{$\setminus$}
\newcommand{\HRule}{\rule{\linewidth}{0.1mm}}
\newcommand{\llm}[2][]{$\llbracket${#2}$\rrbracket^{#1}$}
\newcommand{\ul}{$\ulcorner$}
\newcommand{\ur}{$\urcorner\ $}
\newcommand{\urn}{$\urcorner$}
\newcommand{\sub}[1]{\textsubscript{#1}}
\newcommand{\sups}[1]{\textsuperscript{#1}}
\newtheorem{proposition}{\textbfb{Proposition}}[section]
\newtheorem{definition}[proposition]{\textbf{Definition}}
\newcommand{\bfw}{\begin{fullwidth}}
\newcommand{\efw}{\end{fullwidth}}

\begin{document}

\frenchspacing

\begin{fullwidth}
\noindent\Large Section 9, Non-measurable sets and Banach-Tarski \large \\[.3cm]
\noindent  David Boylan \hfill{4.28.17}

\noindent\HRule
\end{fullwidth}

\section{Borel Sets and Lebesgue Measures}

\begin{itemize}

\item Let $[a,b]$ be a line segment. Its length is $b-a$. 

Can we extend this notion to sets like $[a,b] \cup [c,d]$? Well we can extend the notion of length in an intuitive way by saying the length of  $[a,b] \cup [c,d]$ is simply the length of $[a,b]$ plus the length of $[c,d]$.

In fact, we can extend the notion in a natural way to cover a very large class of sets.

\item A set is a Borel set if it can be formed by taking some family of intervals and performing finitely many applications of complementation or countable union.


\item There's exactly one way of extending the notion of length to cover all Borel sets s.t.

\begin{itemize}

\item Length of line-segments: $\lambda([a,b]) = b-a$


\item Non-negativity: For any $A$ $\lambda(A)$ is either a non-negative real number or $\infty$.

\item Countable additivity: For countable pairwise disjoint $A_1$, $A_2$, ... $\lambda(A_1 \cup A_2\cup ...) = \lambda(A_1)+\lambda(A_2) +...$

\end{itemize}

$\lambda$ is called the Lebesgue measure.

\item Satisfying length of line segments \marginnote{Not all measure functions satisfy length of line segments.} is important. It means that $\lambda$ has the following feature:

\begin{itemize}

\item Uniformity: $\lambda(A) = \lambda(A^c)$ where $c$ is the result of translating $A$ by some real number $c$.

\end{itemize}

\noindent Intuitively this means we can move $A$ around without changing its length.

\item Exercises: 

\begin{itemize}


\item Give an examples of finite, countable infinite and uncountably infinite Borel sets.

\item Are there non-Borel sets?

\item What's  $\lambda(\mathbb{R})$?


\item What's the Lebesgue measure of a finite or countably infinite set? 



\end{itemize}




\end{itemize}


\section{Axiom of Choice and Non-Measurable Sets}


\begin{itemize}

\item A choice set for some set of sets $\bigcup A_i$ is a set containing some $x \in A_i$ for every $A_i \bigcup A_i$.

The axiom of choice says that for every such family of sets, there exists a choice set.

\item Important features of the axiom of choice:

\begin{itemize}

\item allows us to prove that there are non-measurable sets;


\item allows us to prove the Banach-Tarski theorem;

\item proves the well-ordering theorem, the theorem that for any set there exists a well-ordering of that set;

\item equivalent to or useful in proving lots of other interesting results.


\end{itemize}

\noindent This is somewhat weird: the Axiom of Choice leads to some bizarre consequences (i.e. Banach-Tarski); but it's not easy to simply give it up either.


\item A Vitali set is an example of an nonmeasurable set. 

Take [0,1] and say that two elements $a$ and $b$ are in the same orbit iff $a-b \in \mathbb{Q}$.

\item Now wrap [0,1] into a circle. Let $V$ be some Vitali set. 

For $q\in\mathbb{Q}\cap[0,1]$, let $V_q$ be the result of rotating $V$ clockwise by $q$.

\item The $V_q$'s partition [0,1]. (How do we know?)

\item So the $V_q$'s are countably many disjoint sets which partition [0,1] and which are all translations of each other.

\item It follows that $\lambda(V)$ is undefined. Why?
\end{itemize} 






\section{Banach-Tarski}



\begin{itemize}

\item The theorem: We can divide a ball into finitely many pieces and reassemble them to get two balls, each the same size as the original.

\item \emph{Warm-up in $\mathbb{R}^2$}. The \textbf{Cayley Graph}: the set of paths from a center s.t. at each $n\in\mathbb{N}$, you move left/right/up/down (not backwards) a distance of $2^{-n}$.\vspace{.2cm}

We can divide the Cayley graph into two copies of itself $-$ how?\vspace{.2cm}


\item  If we build a Cayley graph \emph{on a sphere}, our paths correspond to performing rotations on the sphere. If we choose the right angle, we can guarantee that the endpoints of no two paths will be the same.


\item This graph will have four `quadrants': the set of points you can reach by going left/right/up/down from the center, respectively. \vspace{.2cm}

Using only translation, we can cut the graph in half and then translate the halves to make a new graph that is identical to the original.\vspace{.2cm}


\item Let an `orbit' for a point be the points reachable from it by this kind of graph. \vspace{.2cm}

Using \emph{the axiom of choice}, choose a `center' for each orbit. \vspace{.2cm}

Then we can cut and rotate all the top quadrants at once, and so on$\dots$\vspace{.2cm}

Thus creating not just two new Cayley graphs, but two new spheres with the same shape as the original. \


\end{itemize}

\end{document}
