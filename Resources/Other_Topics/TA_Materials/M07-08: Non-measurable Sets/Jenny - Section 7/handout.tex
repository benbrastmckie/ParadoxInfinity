\documentclass[justified]{tufte-handout} 
\usepackage{amsfonts, amssymb, stmaryrd, fitch, natbib, qtree}
\usepackage{linguex, color, setspace, graphicx}
\usepackage{enumitem}
\usepackage{bussproofs}
\usepackage{turnstile}
\usepackage[super]{nth}
\thispagestyle{plain}
\definecolor{darkred}{rgb}{0.7,0,0.2}
\bibpunct{(}{)}{,}{a}{}{,}

\input xy
 \xyoption{all}

%New Symbols
\DeclareSymbolFont{symbolsC}{U}{txsyc}{m}{n}
\DeclareMathSymbol{\strictif}{\mathrel}{symbolsC}{74}
\DeclareMathSymbol{\boxright}{\mathrel}{symbolsC}{128}
\DeclareMathSymbol{\Diamondright}{\mathrel}{symbolsC}{132}
\DeclareMathSymbol{\Diamonddotright}{\mathrel}{symbolsC}{134}
\DeclareMathSymbol{\Diamonddot}{\mathord}{symbolsC}{144}
\renewcommand{\labelitemi}{$\triangleright$}
\renewcommand{\labelitemii}{$\circ$}
\renewcommand{\labelitemiii}{$\triangleright$}

%New commands
\newcommand{\bitem}{\begin{itemize}}
\newcommand{\eitem}{\end{itemize}}
\newcommand{\lang}{$\langle$}
\newcommand{\rang}{$\rangle$}
\newcommand{\back}{$\setminus$}
\newcommand{\HRule}{\rule{\linewidth}{0.1mm}}
\newcommand{\llm}[2][]{$\llbracket${#2}$\rrbracket^{#1}$}
\newcommand{\ul}{$\ulcorner$}
\newcommand{\ur}{$\urcorner\ $}
\newcommand{\urn}{$\urcorner$}
\newcommand{\sub}[1]{\textsubscript{#1}}
\newcommand{\sups}[1]{\textsuperscript{#1}}
\newtheorem{proposition}{\textbfb{Proposition}}[section]
\newtheorem{definition}[proposition]{\textbf{Definition}}
\newcommand{\bfw}{\begin{fullwidth}}
\newcommand{\efw}{\end{fullwidth}}

\begin{document}

\begin{fullwidth}
\noindent\LARGE Non-Measurable Sets  \normalsize \\[.3cm]
\noindent  \textsc{24.118 Recitation Section $\bullet$ Matthias Jenny\\  {\texttt{\href{mailto:mjenny@mit.edu}{mjenny@mit.edu}}} $\bullet$ Office:  32-D927 $\bullet$ Hours: Thu 11:30-12:30} \hfill{October 31, 2014}
\noindent\HRule
\end{fullwidth}

\includegraphics[height=4.4cm]{pumpkins.jpg}

\section{Pset 7, problem 1}

\noindent \bfw Describe a random process based on infinite series of coin tosses which selects an element of $[0, 1]$ in such a way that the probability of ending up with an element in $[\frac{1}{2},1]$ is $\frac{1}{3}$.\efw

\noindent \emph{Notes:}  \underline{\hspace{15.82cm}}\\\\\underline{\hspace{16.85cm}}\\

\section{Pset 7, problem 2a}

\noindent \bfw Describe a random process based on infinite series of coin-tosses, which, when attached to the cube-machine, delivers the result that there is a $\frac{1}{2}$ probability that the next cube produced by the machine will have a side-length greater or equal to $\frac{1}{2}$.\efw

\noindent \emph{Notes:}  \underline{\hspace{15.82cm}}\\\\\underline{\hspace{16.85cm}}\\

\section{Pset 7, problem 2b}

\noindent\bfw Describe a random process based on infinite series of coin-tosses, which, when attached to the cube-machine, delivers the result that there is a $\frac{1}{2}$ probability that the next cube produced by the machine will have a side-length greater or equal to $\frac{1}{\sqrt[3]{2}}$. Try to have your procedure use a fair coin.\efw

\noindent \emph{Notes:}  \underline{\hspace{15.82cm}}\\\\\underline{\hspace{16.85cm}}\\

\section{Pset 7, problem 3a}

\noindent\bfw Let $f_0(n) = 0$ for every $n$, and let $O_0$ be $f_0$'s orbit. Describe a function $g_0$ that assigns each member of $O_0$ to a different natural number.\efw

\noindent \emph{Notes:}  \underline{\hspace{15.82cm}}\\\\\underline{\hspace{16.85cm}}\\

\section{Pset 7, problem 3b}

\noindent\bfw Given an arbitrary orbit $O$ and given a function $f$ in $O$, show that there is a function $g$ that assigns each member of $O$ to a different natural number.\efw

\noindent \emph{Notes:}  \underline{\hspace{15.82cm}}\\\\\underline{\hspace{16.85cm}}\\

\section{Pset 7, problem 4}

\noindent\bfw Use a non-standard ordering of the natural numbers to show that there is a measure on sets of natural numbers which is similar to the one above, but which is such that the set of multiples of 2 gets assigned measure $\frac{1}{3}$.\efw

\noindent \emph{Notes:}  \underline{\hspace{15.4cm}}\\\\\underline{\hspace{16.43cm}}\\

\section{The $\mathsf{ZFC}$ axioms\marginnote{{\bf Z}ermelo-{\bf F}raenkel set theory + {\bf C}hoice
}}


\begin{enumerate}
\item Two sets are equal (are the same set) if they have the same elements.\marginnote{Axiom of extensionality}
\item Every non-empty set x contains a member y such that x and y are disjoint sets.\marginnote{Axiom of regularity}
\item Any definable subset of a set exists.\marginnote{Axiom schema of specification}
\item If $x$ and $y$ are sets, then there exists a set which contains $x$ and $y$ as elements.\marginnote{Axiom of pairing}
\item The union of the members of a set of sets exists.\marginnote{Axiom of union}
\item The image of any set under any definable mapping is also a set.\marginnote{Axiom schema of replacement}
\item There exists an infinite set.\marginnote{Axiom of infinity}
\item The power set of every set exists.\marginnote{Axiom of power set}
\item For every set of non-empty sets, there's a choice function that picks a member from each set.\marginnote{Axiom of choice}
\end{enumerate}

\end{document}
