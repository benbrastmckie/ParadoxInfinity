


\documentclass[12pt,a4paper]{article}


%Spacing Packages
\usepackage{fullpage}
\usepackage{a4wide}

%Other Packages
\usepackage{amssymb}
\usepackage{amsmath}
\usepackage{euscript}
\usepackage{changepage}


\usepackage{enumerate}

\usepackage[lf]{venturis} %% lf option gives lining figures as default; 
			  %% remove option to get oldstyle figures as default
\usepackage[T1]{fontenc}

\begin{document}

\begin{quote}

\begin{center} {\large 24.118 -- Paradox and Infinity \\ \vspace{1mm}}
 {\large Problem Set 6: What is Probability? \\ \vspace{1mm}}
 
\end{center}
\vspace{3mm}

\noindent How these problems will be graded:

\begin{itemize} 

\item Assessment will be based on both whether you give the correct answer and on the \emph{reasons} you give in support of your answers. (Note that not every question has a single correct answer.) Even if it is unclear whether your answer is correct, it can be clear whether or not the reasons you have given in support of your answer are good ones. 

\item  \emph{No answer may consist of more than 250 words}. Words after the first 250 will be ignored. Showing your work in a calculation does not count towards the word limit.


\end{itemize} 

These two constraints are often in competition: it may sometimes seem to you that you can't argue for your answer properly in 250 words or less. Learning to deal with this problem is a skill you will acquire with practice. The ability to distill what is essential about a point in a few words requires clear thinking, and it is clear thinking that we are after.


\end{quote} 


\subsection*{Problems:}



\begin{enumerate}

\item Suppose that, if you have credence $x$ in a proposition $p$, written as ``$C(p)=x$'', then 
\begin{itemize}
\item you are willing to buy a bet in which, if $p$ turns out to be true, you win $\$1$ (and nothing happens otherwise) for anything less than or equal to $\$x$.
\item you are willing to sell a bet in which, if $p$ turns out to be true, you give $\$1$ (an nothing happens otherwise) for anything more than or equal to $\$x$.
\end{itemize}
Show that you are vulnerable to being dutch-booked unless, for a set of propositions $p_0, p_1, p_2,\ldots$ (one for every natural number), such that at most one of them is true, $$C(p_0 \vee p_1 \vee p_2 \vee \ldots)= C(p_0) + C(p_1) + \ldots$$

\item Is it plausible that anyone who has credence $x$ in any proposition $p$ will always be willing to buy a bet in which she wins $\$D$, if $p$ is true, for $x\times D$ dollars or less? Describe one of the situations in which you find this least plausible. Do you think it is true even in the situation you describe, or not?

\item Consider the following principle that connects objective and subjective probabilities:
\begin{quote}
The objective probability of an event that occurs at time $t$ is, necessarily, equal to the credence that the event will occur at time $t$ of a perfectly rational agent who knows everything about the universe up to time $t$.
\end{quote}
Note that this is \emph{not} the Principle Principal. Also: you can assume that if this rational agent knows something, she knows all the consequences of it too.

Which of the following accounts of objective probability entail that the above principle is \emph{false}?
	\begin{itemize} 
	\item frequentism
	\item Humeanism plus indeterminism
	\item anti-Humeanism plus indeterminism
	\item determinism plus an objective probability measure over initial conditions
	\end{itemize}
Explain your answers.

\item I'm going to flip a fair coin until it lands heads (or as many times as there are natural numbers, if it never lands heads). The coin is indestructible, so it will last as long as it takes. Consider the following series of bets (there is one bet for every natural number):
\begin{itemize}
\item[Bet 1:] If the coin never lands heads, you pay me \$1. If it first lands heads on flip one, I pay you \$3. Otherwise the bet is void.
\item[Bet 2:] If the coin first lands heads on flip one, you pay me \$4. If it first lands heads on flip two, I pay you \$9. Otherwise the bet is void.
\item[Bet 3:] If the coin first lands heads on flip two, you pay me \$10. If it first lands heads on flip three, I pay you \$21. Otherwise the best is void
\end{itemize}
\hspace{70mm}\vdots
\begin{itemize}
\item[Bet $n$:] If the coin lands heads on flip $n-1$, you pay me $(2\times \text{the loss on bet }(n-1))+\$2$. If the coin lands heads on flip $n$, I pay you $(2\times\text{the gain in bet }(n-1))+\$3$. Otherwise, the best is void.
\end{itemize}
\hspace{70mm}\vdots
\vspace{5mm}

Here is a table showing your payoffs for the first few bets, just to help you get the idea:

\begin{adjustwidth}{-10mm}{}
\begin{tabular}{c|c|c|c|c|c}
& Never Lands Heads & Heads on Flip 1 & Heads on Flip 2 & Heads on Flip 3 & Heads on Flip 4\\ \hline
Probability & $0$ & $\frac{1}{2}$ & $\frac{1}{4}$ & $\frac{1}{8}$ & $\frac{1}{16}$\\ \hline
Bet 1 & Lose \$1 & Gain \$3 & & &\\ \hline
Bet 2 & & Lose \$4 & Gain \$9 & &\\ \hline
Bet 3 & & & Lose \$10 & Gain \$21 &\\ \hline
Bet 4 & & & & Lose \$22 & Gain \$45\\
\end{tabular}
\end{adjustwidth}

\begin{enumerate}
\item What is the expected dollar value for you of Bet 1? Bet 2? Bet 3?
\item If I offer you each of Bet 1, Bet 2 and Bet 3, and you want to maximize expected dollar value, which ones should you take?
\item If I offer you a choice between the first $n$ bets and the first $n+1$ bets, and you want to maximise expected dollar value, which should you take?
\item If I offer you \emph{all} the bets, and you wanted to maximise expected dollar value, should you take all the bets?
\end{enumerate} 

\item I have an offer for you. I have an envelope containing some amount of money in it; you don't know how much. You can have what's in the envelope, or, you can agree to the following gamble: I will flip a coin. If it lands heads, I will give you double what's in the envelope. If it lands tails, I will give you half what's in the envelope.
\begin{enumerate}
\item What maximises expected dollar value: taking the envelope or taking the gamble?
\item Is this situation importantly different from the two-envelope situation? If so, how? If not, lay out the argument for what you should do that applies both to this situation and to the two-envelope situation.
\end{enumerate}


\end{enumerate}


\end{document}




