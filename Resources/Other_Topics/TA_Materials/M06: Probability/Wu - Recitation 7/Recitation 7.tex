\documentclass[12pt]{article}

\usepackage{amsmath,amsfonts,amssymb,amscd}

\usepackage{amsthm}

\usepackage[margin=1.5in,headsep=.5in]{geometry}

\usepackage{fancyhdr}

\setlength{\headheight}{20pt}

\usepackage[colorlinks]{hyperref} 
\usepackage{cleveref}

\usepackage{setspace}
\usepackage{enumitem,linegoal}

\usepackage{amssymb}
\newcommand{\counterfactual}{\ensuremath{%
  \Box\kern-1.5pt
  \raise1pt\hbox{$\mathord{\rightarrow}$}}}

\newtheorem{theo}{Theorem}[section] 

\theoremstyle{definition}
\newtheorem{defin}[theo]{Definition} 
\newtheorem{lema}[theo]{Lemma} 
\newtheorem{cor}[theo]{Corollar}
\newtheorem{prop}[theo]{Proposition}
\newtheorem{exer}{Exercise}

\pagestyle{fancy}

\begin{document}

\pagenumbering{gobble}

\lhead{Xinhe Wu (xinhewu@mit.edu)}
\rhead{$24.118$ Paradox and Infinity $|$ Recitation $7$}

\begin{center}
{\Large \bf Chance: Objective Probability}
\end{center}

\smallskip

\section{The Chance-Credence Connection}

\begin{itemize}
\item (\textbf{The Objective-Subjective Connection}) The objective probability of $A$ at time $t$ is equal to the subjective probability that a perfectly rational agent would assign to $A$, if she had perfect information about events before or at $t$ and no information about events after $t$.
\item Your credence in $p$ is the degree in which you believe in $p$. The credence function of a \textit{fully rational} agent has to (a) be a probability function; (b) update by conditionalization; (c) satisfy Bayes' Law; and (d) perhaps satisfy (some good version of) Principle of Indifference.
\end{itemize}

\begin{exer}
Let $H$ be some proposition about the past. (e.g. It rained in Paris yesterday) Show that according the the above principles, the objective probability of $H$ has to be extreme.
\end{exer}

\section{What is Chance?}

\subsection{Finite Frequentism}

\begin{description}
\item[Finite Frequentism] The objective probability of an outcome $A$ in a finite reference class $B$ is the frequency of actual occurrences of (outcomes of the same type as) $A$ within $B$.
\end{description}

\begin{enumerate}
\item Problems of the single case: unrepeatable events.
\item Problems of the reference class: what should be the reference class?
\end{enumerate}


\subsection{Hypothetical Frequentism}
\begin{description}
\item[Finite Hypothetical Frequentism] The objective probability of an outcome $A$ in a finite reference class $B$ (of hypothetical events) is the frequency of occurrences of (outcomes of the same type as) $A$ within $B$.
\end{description}
\begin{enumerate}
\item Problems of undesirable frequencies
\item Problems of reference class
\end{enumerate}

\begin{description}
\item[Infinite Hypothetical Frequentism] The objective probability of an outcome $A$ in an infinite reference class $B$ (of hypothetical events) is the frequency of occurrences of (outcomes of the same type as) $A$ within $B$.
\end{description}
\begin{enumerate}
\item Problems of undefined frequencies
\item Problems of re-ordering
\item Problems of reference class
\end{enumerate}

\subsection{Rationalism}
\begin{description}
\item[Rationalism] The objective probability of a proposition $A$ \textit{just is} the subjective probability that a perfectly rational agent would assign to $A$, if she had perfect information about events before at $t$ and no information about events after $t$.
\end{description}

\begin{enumerate}
\item Problems from premissivism
\item Problems from explanatory loss
\end{enumerate}

\subsection{The Best-System Account}

\begin{description}
\item[The Best System Account] The objective probability of a proposition $A$ is the objective probability of $A$ under our best theory for the relevant phenomenon.
\end{description}

\begin{itemize}
\item A theory is best when (1) it is `fit' with the actual phenomenon; (2) it delivers an optimal combination of simplicity and strength.
\end{itemize}

\begin{enumerate}
\item Problems from its dependence upon human psychology
\item Problems from its dependence upon human language
\end{enumerate}


\subsection{Primitivism}
\begin{description}
\item[Primitivism] The notion of objective probability is primitive and cannot be further analyzed.
\end{description}

\begin{enumerate}
\item Problems from explanatory loss
\end{enumerate}



\end{document}