\section{Graphes probabilistes }
%<–––––––––––––––––––––––––– graphes probabilistes ––––––––––––––––––––––––––>
\subsection{La macro \tkzcname{grProb} } 
\begin{NewMacroBox}{grProb}{\oarg{local options} \var{left} \var{right} \var{N}\var{S}\var{W}\var{E}}

\begin{tabular}{lll}
Arguments   &   & Définition              \\
  \midrule
  \TAline{Vertex-left} {}{Nom du sommet à gauche}  
  \TAline{Vertex-right} {}{Nom du sommet à droite}  
  \TAline{label N} {}{Étiquette située en haut}  
  \TAline{label S} {}{Étiquette située en bas}   
  \TAline{label W} {}{Étiquette située à gauche} 
  \TAline{label E} {}{Étiquette située à droite} 
 \bottomrule
  \end{tabular}

\medskip
\begin{tabular}{lll}
options & défaut & définition                              \\ 
\midrule
\TOline{unit}      {4cm} {distance entre les sommets      } 
\TOline{LposA}     {180} {angle si label extérieur en A   } 
\TOline{LposB}     {0  }  {angle si label extérieur en B  } 
\TOline{Ldist}     {0cm} {écart entre le node et le label } 
\TOline{LoopDist}  {4cm} {longueur des boucles            }  
\bottomrule
\end{tabular}

\medskip
\emph{Cette macro permet de créer un graphe probabiliste d'ordre 2. }
\end{NewMacroBox}

\subsection{Utilisation de  \tkzcname{grProb} } 

\begin{center}
\begin{tkzexample}[vbox]
\begin{tikzpicture}
  \useasboundingbox (-2.5,-2) rectangle (7.5,2);
  \grProb{A}{B}{NO}{SO}{WE}{EA}
\end{tikzpicture}
\end{tkzexample}
\end{center}  

\begin{tkzexample}[latex=5cm]
\begin{tikzpicture}[scale=.5]
  \useasboundingbox (-2.5,-2) rectangle (5,2);
  \grProb[unit=4]{\Rain}{\Sun}{0,4}{0,3}{0,6}{0,7}
\end{tikzpicture}
\end{tkzexample}
 


                  

\subsection{\tkzcname{grProb} et le style par défaut }
\begin{center}
\begin{tkzexample}[latex=5cm]
\begin{tikzpicture}[scale=.5]
  \useasboundingbox (-2.5,-2) rectangle (5,2);
  \grProb{A}{B}{0,8}{0,6}{0,2}{0,4}
\end{tikzpicture}
\end{tkzexample}
\end{center}

\subsection{\tkzcname{grProb} et le style « Simple »}
\begin{center}
\begin{tkzexample}[latex=5cm]
\begin{tikzpicture}[scale=.5] 
\useasboundingbox (-2.5,-2) rectangle (5,2);
\SetVertexSimple
\grProb[Ldist=0.2cm]{Paris}{Lyon}%
  {\scriptstyle\dfrac{2}{3}}{\scriptstyle\dfrac{3}{4}}%
  {\scriptstyle\dfrac{1}{3}}{\scriptstyle\dfrac{1}{4}}%
\end{tikzpicture}
\end{tkzexample}
\end{center}

\subsection{Utilisation d'un style personnalisé}
\begin{center}
\begin{tkzexample}[vbox]
\begin{tikzpicture}
 \useasboundingbox (-2.5,-2.5) rectangle (7.5,2.5);
 \tikzset{VertexStyle/.style = {shape        = circle, 
                                shading      = ball,
                                ball color    = Orange,
                                minimum size = 20pt,
                                draw,color=white}}
 \tikzset{LabelStyle/.style = {draw,color=orange,fill=white}}
 \tikzset{EdgeStyle/.style = {->, thick,
                              double          = orange, 
                              double distance = 1pt}}

\grProb[Ldist=0.1cm,LposA=0,LposB=180]%
            {Paris}{Lyon}%
            {\scriptstyle\dfrac{2}{3}}{\scriptstyle\dfrac{3}{4}}%
            {\scriptstyle\dfrac{1}{3}}{\scriptstyle\dfrac{1}{4}}%
\end{tikzpicture}
\end{tkzexample}
\end{center}

\vfill
\newpage
\subsection{La macro \tkzcname{grProbThree}}

\begin{NewMacroBox}{grProbThree}{\oarg{local options} \var{right} \var{up}\var{down} \var{rr/ru/rd}\var{uu/ud/ur}\var{dd/dr/du}}

\begin{tabular}{llc}
Arguments   &   & Définition              \\
\midrule
\TAline{Vertex-right} {}{Nom du sommet à droite} 
\TAline{Vertex-up}    {}{Nom du sommet en haut}  
\TAline{Vertex-down}  {}{Nom du sommet en bas}   
\TAline{rr/ru/rd}     {}{arête partant de r vers r etc\dots}  
\TAline{uu/ud/ur}     {}{arête partant de u vers u etc\dots}  
\TAline{dd/dr/du}     {}{arête partant de d vers d etc\dots}  
\bottomrule
\end{tabular}
    
\medskip
\begin{tabular}{llc}
Options & Défaut & Définition                              \\ 
\midrule
\TOline{unit}  {4cm} {distance entre les sommets      }    
\TOline{LposA}     {180} {angle si label extérieur en A   }
\TOline{LposB}     {0  }  {angle si label extérieur en B  }
\TOline{Ldist}     {0cm} {écart entre le node et le label }
\TOline{LoopDist}  {4cm} {longueur des boucles            } 
\bottomrule
\end{tabular}
    
\medskip   
\emph{Cette macro permet de créer un graphe probabiliste d'ordre 3. }
\end{NewMacroBox} 

\subsubsection{Graphe probabiliste d'ordre 3}
\begin{center}
\begin{tkzexample}[latex=7cm]
\begin{tikzpicture}[scale=.75]
 \tikzset{LabelStyle/.style = {draw,fill=white}}
 \grProbThree[unit=4]{\Rain}{\Sun}{\Cloud}
  {0.1/0.3/0.6}{0.2/0.3/0.5}{0.25/0.35/0.4}
\end{tikzpicture}
\end{tkzexample}
\end{center}

\endinput