% !TeX root = ./6-handout.tex

%\item<2-> % reveals second and keeps on page in subsequent frames
%\begin{itemize}[<2->] %does for a whole list of items


\setcounter{section}{5} 

\section{Probability}
%\subsection*{test}

\begin{frame}
%\large

\scriptsize{\tableofcontents}

\end{frame}


\begin{frame}
\frametitle{Liable to forget:}
%\large

\begin{itemize}[<+->]

\item PSet 6 is due this Sunday April 9th, 5pm!

\item[] questions 4--8 are in the Part I quiz component %which i hopefully won't delete this week

\item Hunt's office hours today (4/6): 12--1 and 1--2pm



\item Feel free to join \href{https://piazza.com/mit/spring2023/24118}{Piazza}! 

\item Feel free to join PSet partners!
\item[] Groups will be auto-assigned Thursday 


\end{itemize}
\end{frame}

\subsection{Credences vs. Chances}

\begin{frame}
\frametitle{Credences vs. Chances}
%\large

\begin{itemize}[<+->]

\item \emph{Subjective Probability} (a.k.a. credences): A person's subjective probability in $p$ is the degree to which she is confident in $p$ %(we will often refer to this as an agent's \textit{credence} in $p$)

\item[] Example: Jones's subjective probability that it'll rain tomorrow is 0.3 because she is $30\%$ confident that it'll rain tomorrow.

\item \emphz{Objective Probability} (a.k.a. chances): The objective probability of an event is a feature of the world that does not depend on the beliefs of any particular subject. %(We will often refer to objective probabilities as \textit{chances}). 

\item[] Example: the objective probability that a particle of $^{256}\mbox{Sg}$ will decay in the next 8.9 seconds is 50\%.


\end{itemize}
\end{frame}

\begin{frame}
\frametitle{Credences as a guide to Chances}
%\large

\begin{itemize}[<+->]

\item \emph{The Objective--Subjective Connection}: The objective probability of $A$ at time \textbf{t} equals the subjective probability that a \emphz{perfectly rational} agent would assign to $A$ if she were to have perfect information about the way the world is before $t$ 
\item[] (without using any information about the way the world is \textit{after} $t$)

\item[] We presuppose that a perfectly rational agent is always certain about the objective probabilities at $t$, given full information about how the world is before $t$.


\item In this way, the credences of a perfectly rational agent serve as a guide to determining the chances


\end{itemize}
\end{frame}

\begin{frame}
\frametitle{Actual vs. Rational Credences}
%\large

\begin{itemize}[<+->]

\item Recall key distinction from Josh$_1$'s lecture on Tuesday:

\item \emph{Actual credence} in $p$: how likely one takes $p$ to be, e.g. 0.5 likely

\item \emphz{Rational credence} in $p$: how likely one \emphz{ought} to take $p$ to be

\item What is the meaning and nature of this `\emphz{ought}'? 

% When we say that a credence function ought to satisfy particular constraints, we express acceptance of a set of norms that require obeying those constraints
%So the subjective vs. objective Bayesian are at least in a normative dispute (leaving open whether they are also in a Straightforwardly factual dispute)
% I can know to the students that if they are comfortable positing a primitive normative facts about rationality, then perhaps they don't need something like expressivism. But if they are puzzled about the nature of these non-natural normative facts, then expressivism might be satisfying.



\end{itemize}
\end{frame}

\begin{frame}
\frametitle{Prompt \#14: Chances? Ideally rational?}
%\large

\emph{The Objective--Subjective Connection}: The objective probability of $A$ at time \textbf{t} equals the subjective probability that a \emphz{perfectly rational} agent would assign to $A$ if she were to have perfect information about the way the world is before $t$ 

\begin{enumerate}[<+->]

\item But what are chances (i.e. objective probabilities)?

\item[] e.g. what does it mean to say that the objective chance that a particle of $^{256}\mbox{Sg}$ will decay in 8.9 seconds is 50\%? 

\item And what is \emphz{perfect (or ideal) rationality}? 
\item[] e.g. what does it mean to say that an agent \emphz{ought} to have credence 0.45 in a proposition $p$? 

\end{enumerate}
\end{frame}

\subsection{What are the Chances?}

\begin{frame}
\frametitle{Frequentism}
%\large

\begin{itemize}[<+->]

\item What does it mean to say that the objective chance that a particle of $^{256}\mbox{Sg}$ will decay within 8.9 seconds is 50\%? 

\item According to \emph{frequentism}, it is for 50\% of $^{256}\mbox{Sg}$ particles to decay within 8.9 seconds

\item[] \textit{Limitation}: frequentism doesn't apply to `one-off' events, \\ e.g. Biden has x\% chance of winning 2024 presidential election  

\item \emphz{Hypothetical frequentism}: it is for the following subjunctive conditional to be true: if we were to re-run/simulate the 2024 election many times, Biden would win x\% of times

% if sufficiently many $^{256}\mbox{Sg}$ particle decays were to take place, 50\% of them would decay with 8.9 seconds. 

\end{itemize}
\end{frame}

\begin{frame}
\frametitle{A Problem with Hypothetical Frequentism}
%\large

\begin{itemize}[<+->]

\item Consider an indeterministic coin (i.e. knowledge of the past and laws does not determine how the coin will land)

\item Hypothetical frequentism says that for the coin to have an objective chance of 50\% to land heads, it would land heads 50\% of the time if it were tossed many times. 

\item But if we do 10,000 tosses of a putatively fair coin, do we really expect \textit{exactly} 5000 of them to be heads?

%\item However, this conflicts with a key intuition about objective chance: 

\end{itemize}
\end{frame}

\begin{frame}
\frametitle{Law of Large Numbers}
%\large

\begin{itemize}[<+->]

\item What we expect: If the coin were tossed a sufficiently large number of times, then it would \emph{with very high probability} land Heads approximately 50\% of the time.

\item More precisely: Suppose that events of type $T$ have a probability of $p$ of resulting in outcome $O$. 
\item[] -- Then, for any $\epsilon$, $\delta \in \mathbb{R}$ greater than zero, there is an $n\in\mathbb{N}$ such that the following will be true with \emph{a probability of} at least $1-\epsilon$:

\item[] -- If $m > n$ many events of type $T$ occur, the proportion of them that result in outcome $O$ will be $p \pm \delta$.

\item \textbf{Issue}: we're trying to \textit{define} objective probability! So we can't appeal to `probability' in the definition!

\end{itemize}
\end{frame}

\begin{frame}
\frametitle{Primitivism about Chances}
%\large

\begin{itemize}[<+->]

\item \emphz{Chance Primitivism}: part of a full specification of the world is to list the objective chances of fundamental events

\item These objective chances are \textit{brute}: they simply are the case

\item[] -- There is no further or deeper explanation of these chances

\item The chances of non-fundamental events are determined by the chances of fundamental events plus the laws of nature


\end{itemize}
\end{frame}

\begin{frame}
\frametitle{Reducing chances to ideal credences?}
%\large

\begin{itemize}[<+->]

\item \emph{Chance Rationalism}: there is nothing more to chances than the Objective--Subjective Connection

\item Interpret the OS connection as not merely a guide to the objective chances but as a \textbf{definition} of chance: 

\item[] OS connection: The chance of $A$ at time \textbf{t} \emph{IS} the credence that a perfectly rational agent would assign to $A$, if she had perfect information about the world at times $\leq t$ and did not rely on any information about the world at times $> t$

\item What about cases where the Principle of Indifference leads to puzzles (e.g. \textsc{Mystery Square})? 

\item \emph{Local} Chance Rationalism (localism): Chances are well-defined only when the relevant principles of rationality apply, e.g. the PoI 

\end{itemize}
\end{frame}



\iffalse %***********************************************************************************************

\subsection{Rationality}

\begin{frame}
\frametitle{The Principal Principle}
%\large

\begin{itemize}[<+->]

\item \emphz{The Principal Principle}: a rational agent \textcolor{OGlyallpink}{ought} to set their credence in $p$ to what they take the objective chance of $p$ to be

\item Chance Rationalists define the objective chances in terms of the credences of a \textcolor{OGlyallpink}{perfectly rational agent}
%note that a chance rationalist could be a primitivist about these oughts! 

\item Even a Chance Primitivist needs an account of this \textcolor{OGlyallpink}{ought}, insofar as they endorse PP or other principles of rationality

\item Everyone who distinguishes between actual vs. rational credences needs an account of rationality


\end{itemize}
\end{frame}

\begin{frame}
\frametitle{Primitivism about Rationality}
%\large

\begin{itemize}[<+->]

\item \emphz{Rationality Primitivism}: a full specification of the world involves stating which constraints on agent's credences are (perfectly) rational

\item These constraints are presumably (irreducibly) \textcolor{OGlyallpink}{normative}: \\ they are basic claims of the form $\langle$Agents \textcolor{OGlyallpink}{ought} to $\phi$$\rangle$

\item So Rationality Primitivism is (at least  na\"ively) a kind of ``non-naturalism'': the world comes along with facts that do not reduce to physical states of affairs 

\end{itemize}
\end{frame}

\begin{frame}
\frametitle{Descriptive vs. Non-descriptive Claims}
%\large

\begin{itemize}[<+->]

\item \emphz{Descriptive claim}: purports to mirror or represent the way the world is, i.e. some state of affairs

\item[] e.g. ``There is a chalkboard behind me and in front of you''

\item \emph{Non-descriptive claim}: Performs some non-representational functional role

\item[] e.g. expresses an attitude or command

\item[] e.g. ``don't be late''! 

\end{itemize}
\end{frame}


\begin{frame}
\frametitle{Moral Expressivism}
%\large

\begin{itemize}[<+->]

\item \emph{Moral Expressivism}: moral claims are non-descriptive; specifically they express pro- or con- attitudes toward various actions

\item ``The right thing to do is to hold the door open for people'': expresses an attitude of

\item[]  \textit{being for holding the door open for others} 

\item Alternatively: expresses acceptance of a set of norms that \textit{recommend} (or at least permit) holding the door open for others

\end{itemize}
\end{frame}

\begin{frame}
\frametitle{Expressivism about Rationality (Gibbard 1990)}
%\large

\begin{itemize}[<+->]

\item To judge that an action is rational is to express acceptance of a set of norms that recommend (or at least permit) that action

\item[] -- ``\textcolor{highlightA}{perfectly rational}" $\Rightarrow$ \emph{required} by the norms

\item To judge that an action is \emphz{irrational} is to express acceptance of a set of norms that \textcolor{OGlyallpink}{forbid} that action

\item For every action, a complete set of norms renders it required, recommended, permissible, or forbidden 

\item Judgments of rationality may not be ``straightforwardly factual'' (they need not track or mirror states of affairs)
\item[] -- But they may be factual in a weaker, deflationary sense 

%thinking what you ought to do is thinking what to do

\end{itemize}
\end{frame}

\begin{frame}
\frametitle{Subjective vs. Objective Bayesianism}
%\large

\begin{itemize}[<+->]

\item \emph{Subjective Bayesianism}: an agent is perfectly rational just in case their credence function $C$ satisfies (i) Necessity, \\ (ii) Additivity, and (iii) Update by Conditionalization

\item[] -- A large class of prior credences are equally rational

\item  \emphz{Objective Bayesianism}: an agent is perfectly rational just in case their credence function $C$ satisfies (i), (ii), (iii), AND \\ (iv) \textit{Privileged Priors}: their credences match that of a privileged probability function, conditional on their evidence

\item[] -- There is a uniquely rational set of prior credences

\end{itemize}
\end{frame}

\begin{frame}
\frametitle{Understanding the debate between SB vs. OB}
%\large

\begin{itemize}[<+->]

\item If Rationality Primitivism is correct, then the debate between subjective vs. objective Bayesians is straightforwardly factual: \\ at most one of them can be correct about the way the world is

\item If Rationality Expressivism is correct, then this debate is at least partly normative: SB and OB disagree about what we ought to do to count as being rational 

\item -- we might remain agnostic as to whether or not judgments of rationality also play a descriptive/representational role 

\item Either way, how are we to settle who is right?


\end{itemize}
\end{frame}

% % For each of our constraints on rational credences, we can debate whether they are truly required by rationality. E.g., does ideal rationality genuinely require logical omniscience?

% Could note that logical omniscience might be part of a regulative ideal, notion of ideally rational as a limit of rationality.

\begin{frame}
\frametitle{On the Rationality of \emphz{Dominance}}
%\large

\begin{itemize}[<+->]

\item Proponents and opponents of \emphz{Dominance} are at least in a normative dispute:

\item[] They disagree about the norms we ought to endorse when it comes to making decisions

\item Proponents of \emphz{Dominance} are always in favor of choosing strongly dominant strategies

\item Opponents believe that we ought to allow for exceptions 

\item What could settle who is ultimately right? 

\item[] -- i.e. what settles what we should do?


\end{itemize}
\end{frame}

\subsection{Whence Probability Talk?}
% Looking at parts of Amie Thomasson 2023, a neo-pragmatist approach to modality

\begin{frame}
\frametitle{Clues from Developmental Linguistics}
%\large

\begin{itemize}[<+->]

\item Track progression of acquiring probabilistic language through childhood to (philosophical) adulthood

\item Focus on functional roles of probabilistic language

\item[] -- including interpersonal roles, e.g. trying to regulate the behavior of others

\item[] -- expressing attitudes toward propositional contents


\end{itemize}
\end{frame}

\begin{frame}
\frametitle{Beginnings}
%\large

\begin{itemize}[<+->]

\item Around age 2: ability modals such as verb `can'

\item then deontic modals: should, have to

\item then early uses of epistemic modals (beginning around age 3): might, must be the case

\item Possibility claims come before necessity claims

% Amie Thomasson: ``children don't learn the strength of different modal claims until around age 7 (Papafragou 1998, 8--14)"

\end{itemize}
\end{frame}

\begin{frame}
\frametitle{Grammatical Progression}
%\large

\begin{itemize}[<+->]

\item Initially: auxiliary or semi-auxiliary verbs like might, could, must

\item Next (ages 6--12): objectified modal expressions such as ``it is possible that", ``there is a possibility that"
\item[] these are `grammatical metaphors'

\item Eventually (in the philosopher's room): talk of \textit{possible worlds}, an extreme grammatical metaphor!

\end{itemize}
\end{frame}

\begin{frame}
\frametitle{Modalization}
%\large

\begin{itemize}[<+->]

\item Modals like `might' allow us to \textit{temper} the force of what we say: introduces degrees beyond ``$p$ is the case" or ``$p$ is not the case"

\item Compare ``A meteor killed off the dinosaurs" to \\ ``A meteor \textit{might have} killed off the dinos"

\item[] -- Enables speaker to express a judgment of certainty, likelihood, or frequency toward a proposition

\item[] -- in general, express attitudes toward propositions, rather than merely assert or deny a proposition 

\item Not necessarily indicating a fact, but rather influencing behavior

\end{itemize}
\end{frame}

\begin{frame}
\frametitle{Objectifying Modals}
%\large

\begin{itemize}[<+->]

\item Between ages 6-12: acquire expressions ``it is possible that" and ``there is a possibility that" 

\item Modal adjective forms: is possible, is necessary
% is obligatory

\item[] e.g. ``It might rain" $\Rightarrow$ ``Rain \textit{is possible}"

\item Modal noun forms: a possibility, a necessity
% an obligation

\item[] e.g. ``Rain \textit{is possible}" $\Rightarrow$ ``there is \textit{a possibility} of rain"


\end{itemize}
\end{frame}

\begin{frame}
\frametitle{Grammatical Metaphors}
%\large

\begin{itemize}[<+->]

\item The grammatical shifts from modal auxiliary verb, to modal adjective, and to modal noun are each a `grammatical metaphor'

\item Introduce a modal predicate or modal nominalization to make claims in a particular grammatical form

\item Enables us to reason more effectively about the initial ``might" claims such as ``it might rain"

\item But invites philosophical questions: what is it for something to be possible? What are possibilities? What are chances?

\end{itemize}
\end{frame}

\begin{frame}
\frametitle{A Deflationary Answer}
%\large

\begin{itemize}[<+->]

\item philosophical questions: what is it for something to be possible? What are possibilities? What are chances?

\item If modal predicates and nominalizations are introduced simply to aid expression of attitudes toward propositions and reasoning about these attitudes, then there's not mysterious going on

\item Possibility-talk is a useful way of expressing attitudes toward propositions; grammatically more powerful and flexible than sticking with modal auxiliary verbs 

\item So if you weren't puzzled by ``it might rain", you should be no more puzzled by ``there is a possibility of rain"

\end{itemize}
\end{frame}

\begin{frame}
\frametitle{Credences as a comparative concept}
%\large

\begin{itemize}[<+->]

\item Through modal adjectives and nouns, we can introduce graded comparisons

\item ``Snow is \textit{more possible} than rain tomorrow" or ``the possibility of snow is \textit{greater than} the possibility of rain tomorrow"

\item From these comparisons, we can introduce credences:

\item[] -- ``the possibility of snow is 70\%" 

\item Deflationary view: Simply a more sophisticated expression of an attitude toward the proposition ``it will snow tomorrow"



\end{itemize}
\end{frame}

\begin{frame}
\frametitle{Rival Interpretation}
%\large

\begin{itemize}[<+->]

\item Descriptivism about possibility-talk: interpret possibility-talk as \textit{describing} the epistemic state of an agent or describing somebody of evidence

\item Opposed to interpreting possibility-talk as expressing attitudes toward propositions 

\end{itemize}
\end{frame}




\fi %***********************************************************************************************



















