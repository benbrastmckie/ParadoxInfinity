%\RequirePackage{marginnote}
%\let\marginpar\marginnote
%\let\marginnote\undefined

\documentclass{tufte-handout}

%\geometry{showframe}% for debugging purposes -- displays the margins

\usepackage{amsmath}
\usepackage{tcolorbox}

\newtcolorbox{exbox}{}

% Set up the images/graphics package
\usepackage{graphicx}
\setkeys{Gin}{width=\linewidth,totalheight=\textheight,keepaspectratio}
\graphicspath{{graphics/}}

\title{Credence and The Principle of Indifference %Style\thanks{Inspired by Edward~R. Tufte!}
}
\author[]{J.E. Pearson ("Josh$_1$"); Paradox and Infinity}
\date{Tuesday, 4th April}  % if the \date{} command is left out, the current date will be used

% The following package makes prettier tables.  We're all about the bling!
\usepackage{booktabs}

% The units package provides nice, non-stacked fractions and better spacing
% for units.
\usepackage{units}

% The fancyvrb package lets us customize the formatting of verbatim
% environments.  We use a slightly smaller font.
\usepackage{fancyvrb}
\fvset{fontsize=\normalsize}

% Small sections of multiple columns
\usepackage{multicol}

% Provides paragraphs of dummy text
\usepackage{lipsum}

% These commands are used to pretty-print LaTeX commands
\newcommand{\doccmd}[1]{\texttt{\textbackslash#1}}% command name -- adds backslash automatically
\newcommand{\docopt}[1]{\ensuremath{\langle}\textrm{\textit{#1}}\ensuremath{\rangle}}% optional command argument
\newcommand{\docarg}[1]{\textrm{\textit{#1}}}% (required) command argument
\newenvironment{docspec}{\begin{quote}\noindent}{\end{quote}}% command specification environment
\newcommand{\docenv}[1]{\textsf{#1}}% environment name
\newcommand{\docpkg}[1]{\texttt{#1}}% package name
\newcommand{\doccls}[1]{\texttt{#1}}% document class name
\newcommand{\docclsopt}[1]{\texttt{#1}}% document class option name

\begin{document}
\maketitle

Two kinds of probability:
\begin{itemize}
    \item \textbf{Subjective Probability/Credence} 
        - how likely something is \textit{according to some person}. They are necessarily claims about someone's beliefs.
    \item \textbf{Objective Probability/Chance} %\marginnote{Claims about chance are not necessarily claims about what someone believes.}
       - how likely something is “according to the world”. They are not necessarily about someone's beliefs. 
\end{itemize}

\noindent Chance and credence often come apart:

\begin{quote}\marginnote{\begin{itemize}
    \item Chance that coin lands heads $\approx 100\%$.
    \item Josh$_2$'s credence that coin lands heads $\approx50\%$.
    \item Josh$_1$'s credence that coin lands heads $\approx100\%$.
\end{itemize}}

    \textit{Double-Headed Coin.}  In front of me is a double-headed coin that I am about to flip. However, I have lied to Josh$_2$, who now thinks that the coin is perfectly fair.
\end{quote}

%\begin{itemize}
%    \begin{itemize}
%        \item 
%    \end{itemize}
%\end{itemize}

\noindent We can also distinguish between:

\begin{itemize}
    \item One's \textbf{actual} credence that $A$ - how likely one takes $A$ to be \textit{as a matter of fact}.
    %\begin{itemize}
    %    \item How likely an agent, \textit{as a matter of fact} takes $p$ to be. 
    %\end{itemize}

    \item The \textbf{rational} credence to have that $A$ - how like one \textit{ought} to take $A$ to be.
   % \begin{itemize}
   %     \item How likely an agent \textit{ought} to take $p$ to be. The probability it would be \textit{reasonable} for the agent to assign $p$.
   % \end{itemize}
\end{itemize}

\noindent Again, these can come apart:

\begin{quote}
    \textit{Biased Coin.}\marginnote{\begin{itemize}
        \item Gareth's credence that the coin will land heads when flipped looks rational.
        \item Philipp's is arguably irrational: he should be much more than 50\% confident the next flip will land heads!
    \end{itemize}} In front of Gareth and Philipp is a coin which they know has 1 heads-face and 1 tails-face. Gareth has no other information about the coin. Philipp %\marginnote{We’re grad students, what else are we going to do in our spare time?} 
    flipped the coin 10,000 times yesterday, and it landed heads every time. Both are 50\% confident that it will land heads when flipped. 
\end{quote}

\noindent Important Philosophical Question: \textbf{What constraints must ones credences satisfy to count as rational? }

%\section{Bayesian Epistemology}
\vspace{2mm}

\noindent The following idealization  will make this question more tractable: 

\begin{itemize}
    \item Each agent can be represented by function $C$ - their "credence function" - \marginnote{When I'm 50\% confident it will rain, we write $C(Rain)=0.5$.}which assigns to each proposition a value in the unit interval $[0,1]$.
\end{itemize}

\noindent We arrive at the following, highly influential view... 

\begin{exbox}
\textbf{Subjective Bayesianism}

\vspace{1mm}

An agent with credence function $C$ is rational (if and) only if they satisfy:

\begin{itemize}
    \item \textbf{Necessity:} $C(A)=1$ whenever $A$ is a necessary truth.

    \item\textbf{Additivity:} $C(A\vee B)=C(A)+C(B)$ whenever $A$ and $B$ are incompatible. 
\end{itemize}

\end{exbox}

\noindent \textbf{Subjective Bayesianism} is often also supplemented with:\marginnote{Where, for any $C$, $A$ and $B$ with $C(B)>0$, $C(A\mid B)=\frac{C(A\&B)}{C(B)}.$}

\begin{exbox}
    \textbf{Update by Conditionalization}

    \vspace{1mm}

    %An agent responds to evidence rationally only if $C^{new}(A)=C^{old}(A\mid B)$.
    Let $C^{old}$ represent an agent’s credence before receiving information $B$, and $C^{new}$ represent her credences after receiving information $B$. This agent is rational only if: $C^{new}(A)=C^{old}(A\mid B)$.
\end{exbox}

\noindent Subjective Bayesianism is \textit{extremely} permissive. On the other end of the spectrum we have:

\begin{exbox}
    \textbf{Objective Bayesianism}

    \vspace{1mm}

    There exists some probability function, $Pr$, such that an agent with total information (i.e. evidence) $E$ and credence function $C$ is rational iff $C(A)=Pr(A|E)$ for all propositions $A$. 

\end{exbox}

\noindent That sounds good, but we are left with a huge question: \textit{what is $Pr$??}

%\section{The Principal of Indifference}
\vspace{1mm}

\noindent The following principle, if true, gets us pretty far. Let "$A\approx B$" mean that one has no more reason to believe $A$ than $B$, nor $B$ than $A$:

\begin{exbox}
    \textbf{The Principal of Indifference (TPOI)}

\vspace{1mm}

%If, for agent with credences $C$, $A\approx B$, then that an agent is rational only if: $C(A)=C(B)$.
%If $A\!\approx\!B$ for an agent with credence function $C$, then that an agent is rational only if: $C(A)=C(B)$.
Suppose that $A\!\approx\!B$ for an agent with credence $C$. That agent is rational only if: $C(A)=C(B)$.
    
\end{exbox}

%Plausible: e.g. above, wy ought Gareth be 50% confident that the coin will land heads? It's not like he knows anything about the probabilities! 

\noindent \textit{Problem:} \textbf{TPOI} is inconsistent! 

\begin{quote}
    \textit{Mystery Square.} A mystery square is \marginnote{This example comes from Bas van Fraassen, whose birthday is tomorrow! (See his 1989 "Laws and Symmetry".) He's riffing off a more complicated example from Joseph Bertrand (whose birthday was March 11th). 
    
    It is sometimes called "Bertrand's Paradox" - not to be confused with Russell's paradox, developed by (Bertrand) Russell! } 
    known only to be no more than two feet wide. Apart from this constraint, you have no relevant information concerning its dimensions. What is your credence that it is less than one foot wide? 
\end{quote}

\noindent Notation: $L_1$ says he square has length of less than 1 foot; $L_2$ says the square has length between 1 and 2 feet;  $A_1$ says the square has area less than 1 square foot; ...; $A_4$ says the square has area between 3 and 4 feet.

\vspace{2mm}

\noindent Then we can derive an inconsistency as follows...

\begin{enumerate}
    \item[(1)] $L_1\approx L_2$
    \item[(2)] $A_1\approx A_2\approx A_3\approx A_4$
    \item[(3)] $C(L_1)=0.5$\marginnote{(3) and (4) follow from (1), (2) and \textbf{TOPI}.}
    \item[(4)] $C(A_1)=0.25$
    \item[(5)] $C(L_1)=C(A_1)$... \textbf{Oops!}\marginnote{Since $L_1$ and $A_1$ are equivalent!}
\end{enumerate}


\noindent Most philosophers have thereby given up on \textbf{TPOI}. However, in philosophy, no view ever truly dies...\marginnote{Not even one-boxing!}

\vspace{2mm}

 \noindent Roger White\marginnote{In his "Evidential Symmetry and Mushy Credence". MIT faculty member!} argues that something is suspect about the above argument: (1) and (2) already lead to absurdities \textit{even before} we introduce \textbf{TPOI}.

 \vspace{2mm}

\noindent Note that $\approx$ is plausibly an \textit{equivalence relation}.\marginnote{That is, $\approx$ is reflexive, symmetric, and transitive.} Given this, we can derive the following absurd conclusion:

    \begin{enumerate}
    \item[(1)] $L_1\approx L_2$
    \item[(2)] $A_1\approx A_2\approx A_3\approx A_4$
    \item[(3)] $L_1\approx A_1$\marginnote{3 and 4 follow since these propositions are equivalent.}
    \item[(4)] $L_2\approx(A_2 \vee A_3 \vee A_4 )$
    \item[(5)] $A_2\approx(A_2 \vee A_3 \vee A_4)$\marginnote{By manipulating 1-4 with symmetry and transitivity of $\approx$.}
    \end{enumerate}

\noindent \textit{But that can't be right!}  We clearly have more reason to believe $(A_2 \vee A_3 \vee A_4)$ than the logically stronger $A_2$.

\vspace{2mm}

\noindent So, if (1) and (2) are already suspect, it's unclear that it's \textbf{TOPI} that is the problematic assumption in the first argument. 

\vspace{2mm}

\noindent This is a little baffling... how can (1) or (2) be \textit{false}? Isn't it obvious that we have no reason to prefer $L_1$ over $L_2$, or any $A_n$ over any $A_m$? After all, we always know what our reasons are... don't we?...

%\item Potential upshot:

%\begin{itemize}
%    \item \textbf{TOPI} is true, but one of 1 or 2 false.
%    \begin{itemize}
%        \item But that's crazy! How can we really have more reason to believe $L_1$ over $L_2$ (or vice versa), or one of the $A_n$s over the other?
%        \item Don't we always know what our reasons are?
%    \end{itemize}
%\end{itemize}

\begin{exbox}
    \textbf{Writing Prompt:} Do you find the \textit{Mystery Square} argument against \textbf{TPOI} convincing, or do you like White's reply? If you like White's reply, what can you tell me about the credences you ought to assign $L_1$, $L_2$ and $A_1$-$A_4$? If you don't like White's reply, where does his argument go wrong?
\end{exbox}



%\marginnote{\noindent \textit{Writing Prompt:}  Do you find the mystery square argument against TPOI convincing, or do you like White's reply? If you like White's reply, what credences should you assign in this case? If you don;t like White's reply, where does the argument go wrong?} 

\end{document}



