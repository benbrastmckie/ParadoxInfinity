\documentclass[12pt,a4paper]{article}
\usepackage{agustin}


%Spacing Packages
\usepackage{fullpage}
\usepackage{a4wide}

%Other Packages
\usepackage{amssymb}
\usepackage{amsmath}
\usepackage{euscript}



\usepackage[lf]{venturis} %% lf option gives lining figures as default; 
			  %% remove option to get oldstyle figures as default
\usepackage[T1]{fontenc}


\begin{document}
\begin{quote}

\begin{center} {\large 24.118 --- Paradox and Infinity \\ \vspace{1mm}}
 {\large Final Assignment \\ \vspace{1mm}}
 
\end{center}
\vspace{3mm}

\noindent There are 9 problems, but your grade will be determined by your 8 best answers. There is no word limit unless otherwise specified. (When a word limit
is specified, words after the word limit will be ignored.)

\textbf{You may not talk with anyone, other than a TA or me, Damien Rochford, about the content of this problem set.} You are free to access the class lecture notes, readings and answers to previous problem sets, however.

\end{quote}

\vspace{3mm}


\subsection*{Problems:}

\begin{enumerate}

\item Consider the following argument:
\begin{center}
\begin{tabular}{l}
1. If time-travel is possible, then, for any given person, it is possible to know how that\\ person will act hundreds of years before they act.\\
2. If it is possible to know how someone will act hundreds of years before they act, \\then their actions are not free.\\ \hline
3. If time-travel is possible, then nobody acts freely --- i.e., there is no such things as\\ free will.
\end{tabular}
\end{center}

Question: is this a good argument? If not, object to it (be precise). If so, object to it, then respond to the objection (again, be precise). [Word Limit: 200 words]

\item Remember Tina? She was the time-traveler who saw herself perish in archival footage from the 50s (it's in the time-travel notes, if you want to refresh your memory). Suppose she has gone back in time, and now confronts the two buttons. She has decided: she will press the left button.

Should she go through with her decision, according to evidential decision theory? Should she go through with her decision, according to causal decision theory? Justify your answers.

\item Use Cantor's Theorem to prove that there is no set of all sets.

\item Tristram Shandy is writing the story of his life. After two days, he's finished writing about the first day of his life. After four days, he's finished writing about the first two days of his life. After six days, he's finished writing about the first three days of his life. In general, after $2n$ says, he's finished writing about the first $n$ days of his life.

Tristram Shandy goes on to live a long time. Call the day he starts writing ``Day 0''; the day after that, ``Day 1''; the day after that ``Day 2'', and so on. For every finite $n$, Tristram Shandy is alive on day $n$. But there is no infinite number $\omega$ such that Tristram is still alive on day $\omega$.

Now, suppose that Tristram's notebook outlives him. Will his notebook contain his entire life's story, or not? Justify your answer.

\item An ordinal is \textsl{countable} iff it has a countable cardinality. Consider the set of all countable ordinals (don't worry, there is one). Is it an ordinal? Is it a countable ordinal? Justify your answers.

\item 
Consider again the Principal Principle. In the probability notes I stated it this way:
\begin{quote}
If you know that the objective probability of $p$ is $x$, then your subjective probability that $p$ should be $x$.
\end{quote}
Arguably, this can't be \emph{quite} right, because if you know that the objective probability of $p$ is $x$, and $x\neq 1$, \emph{and} you know that $p$ is in fact true, then your subjective probability that $p$ should be 1, not $x$.

Can you think of a situation (however crazy) in which an agent knows that the objective probability of $p$ is $x$, and $x\neq 1$, and she knows that $p$ is in fact true? Describe it. [Word Limit: 200 words]

\item Describe, in your own words, the pieces of the sphere that are involved in the Banach-Tarski theorem. [Word Limit: 200 words]

\item Design a Turing Machine that does the following: when given as input a string of $n$ ones, followed by a blank, followed by a string of $m$ ones, it produces a string of $n\times m$
ones, and halts. (You may assume that $n$ and $m$ are greater than zero.) 

Answers will only given credit if they can be run on the following simulator:
\begin{center}
http://morphett.info/turing/turing.html
\end{center}

\item Matt's statement of G\"odel's First Incompleteness Theorem in lecture was this (I've slightly paraphrased):
\begin{quote}
No TM that, given empty input, runs forever outputting sentences of $L$, is such that
\begin{itemize}
\item every true sentence of $L$ is outputted after some finite number of steps, and 
\item no false sentence of $L$ is ever outputted.
\end{itemize}
\end{quote}

My statement of G\"odel's First Incompleteness Theorem in the notes is (again slight paraphrase):
\begin{quote}
There is no TM that outputs a ``$1$'' if and only if it is given a true sentence of $L$ as input.
\end{quote}

Show that these two statements of the theorem are equivalent. That is, show that if Matt's formulation is true, then mine is true, and if mine is true, Matt's is true.

\end{enumerate}

\end{document}