\documentclass[a4paper, 11pt]{article} % Font size (can be 10pt, 11pt or 12pt) and paper size (remove a4paper for US letter paper)

\usepackage[protrusion=true,expansion=true]{microtype} % Better typography
\usepackage{graphicx} % Required for including pictures
\usepackage{wrapfig} % Allows in-line images
\usepackage{enumitem} %%Enables control over enumerate and itemize environments
\usepackage{setspace}
\usepackage{amssymb, amsmath, mathrsfs} %%Math packages
\usepackage{stmaryrd}
\usepackage{mathtools}
\usepackage{mathpazo} % Use the Palatino font
\usepackage[T1]{fontenc} % Required for accented characters
\usepackage{array}
\usepackage{bibentry}
\usepackage[round]{natbib} %%Or change 'round' to 'square' for square backers
\usepackage{tikz}

  \def\firstcircle{(140:-1.5cm) circle (1.5cm)}
  \def\secondcircle{(210:1.75cm) circle (2.5cm)}
  \def\thirdcircle{(330:1.75cm) circle (2.5cm)}


\setcitestyle{aysep={}}

\linespread{1.05} % Change line spacing here, Palatino benefits from a slight increase by default

\newcommand{\qed}[0]{$\hfill\Box$} %%Box at end of proofs
\newcommand{\s}[0]{\mathcal{S}} %%Box at end of proofs
\renewcommand{\t}[0]{\mathcal{A}} %%Box at end of proofs
\newcommand{\corner}[1]{\ulcorner#1\urcorner} %%Corner quotes
\newcommand{\tuple}[1]{\langle#1\rangle} %%Angle brackets
\newcommand{\set}[1]{\lbrace#1\rbrace} %%Set brackets
\newcommand{\interpret}[1]{\llbracket#1\rrbracket} %%Double brackets
%\DeclarePairedDelimiter\ceil{\lceil}{\rceil}    
\newcommand{\qpar}[0]{\par\hspace*{.2in}}

\usepackage{calc}
\makeatletter
\newcommand{\labelalign@original@item}{}
\let\labelalign@original@item\item
\newcommand*{\labelalign@envir}{labelalign}
\newlength{\labelalign@totalleftmargin}
\newlength{\labelalign@linewidth}
\newcommand{\labelalign@makelabel}[1]{\llap{#1}}%
\newcommand{\labelalign@item}[1][]{%
  \setlength{\@totalleftmargin}%
       {\labelalign@totalleftmargin+\widthof{\textbf{#1 }}+.25in-\leftmargin}%
  \setlength{\linewidth}
       {\labelalign@linewidth-\widthof{\textbf{#1 }}-.25in+\leftmargin}%
  \par\parshape \@ne \@totalleftmargin \linewidth
  \labelalign@original@item[\textbf{#1}]%
}
\newenvironment{labelalign}
  {\list{}{\setlength{\labelwidth}{0in}%
           \let\makelabel\labelalign@makelabel}%
   \setlength{\labelalign@totalleftmargin}{\@totalleftmargin}%
   \setlength{\labelalign@linewidth}{\linewidth}%
   \renewcommand{\item}{\ifx\@currenvir\labelalign@envir
                           \expandafter\labelalign@item
                        \else
                           \expandafter\labelalign@original@item
                        \fi}}
  {\endlist}
\makeatother


\makeatletter
\renewcommand\@biblabel[1]{\textbf{#1.}} % Change the square brackets for each bibliography item from '[1]' to '1.'
\renewcommand{\@listI}{\itemsep=0pt} % Reduce the space between items in the itemize and enumerate environments and the bibliography

\lineskiplimit=-100pt\relax

\renewcommand{\maketitle}{ % Customize the title - do not edit title and author name here, see the TITLE block below
\begin{flushright} % Right align
{\LARGE\@title} % Increase the font size of the title

\vspace{10pt} % Some vertical space between the title and author name

{\@author} % Author name
\\\@date % Date

\vspace{30pt} % Some vertical space between the author block and abstract
\end{flushright}
}

%----------------------------------------------------------------------------------------
%	TITLE
%----------------------------------------------------------------------------------------

\title{\textbf{Writing Informal Proofs}} % Subtitle

\author{\em Benjamin Brast-McKie} % Institution

\date{\today\vspace{-0pt}} % Date

%----------------------------------------------------------------------------------------

\begin{document}

\maketitle % Print the title section

\thispagestyle{empty}

%----------------------------------------------------------------------------------------


\section*{Proofs}

A proof is an argument which consists of assumptions often called \textit{premises}, inferences from those premises, and a \textit{conclusion}.
All arguments aim to provide reasons to believe the conclusion on the basis of believing the premises, though not all arguments are successful, and some arguments are stronger than others.
Unlike arguments in the law or elsewhere in day-to-day dealings, \textit{proofs} aim to draw an especially strong conclusion between the premises and their conclusion.

\section*{Common Locutions}

 	\vspace{.025in}
 	\begin{labelalign} \small
		\item[\sc Assume:] Use `Assume $X$' when you need to proof something of the form `If $X$, then $Y$'.
			You should then try to make use of your assumption to establish $Y$.
			You may then conclude that if $X$, then $Y$ as needed.
			(Note that this is not the only case in which you might use `Assume'.)
		\item[\sc So:] Use `So' or `It follows that', to indicate to your reader that what you are now saying follows from the line above.
			Of course, if the proof is going to work, your claim must actually follow from the line above.
		\item[\sc Since:] Use `Since', `Having already shown that', `Given that', or `Because' to indicate that you are about to restate something that you have already established on a line prior to the previous line to go on to derive some further claim.
			For instance, we might write `Having already shown that $X$, it follows that $Y$' to indicate that we are drawing on $X$ (perhaps together with the previous line) to conclude that $Y$ is the case.
		% \item[\sc Know:] Use `We know that' to pick up with some claim which occurs prior to the previous line to continue with that .
		\item[\sc Cases:] Use `Consider the following cases' or `Either of the following must then hold' before labelling the different possibilities you want to consider in order.
			You may then proceed case by case, citing the labels that you have provided.
			(This is easier to read and write than using columns to consider cases in parallel.)
		\item[\sc RAA:] Use `Assume for RAA' to indicate that you are assuming the opposite of what you want to show for a \textit{reductio ad absurdum} style proof.
		\item[\sc $\bot$:] After deriving both $P$, and then at some latter point deriving $\neg P$ (or \textit{vice versa}, you may use `But this contradicts $P$' followed by `$\bot$' as a sentence of its own, or just `$\bot$' when the contradiction is clear.
		% You may then use `Thus' as below to restate your desired conclusion.
		\item[\sc But:] Use `But' or `However' to recall some past conclusion which contradicts (perhaps after some further reasoning) the claim just above.
			This is a common devise used in writing RAA proofs.
		\item[\sc Let:] Use `Let $\alpha$ be an arbitrary $F$', or `Let $\alpha\in X$', or etc., to establish a general claim of the form `All $\alpha$ which are $F$ are also $G$'.
			After choosing such an arbitrary $\alpha$ which is $F$ (or in $X$, etc.), you should aim to show that $\alpha$ is $G$ before concluding that every $\alpha$ which is $F$ is also $G$.
		\item[\sc Thus:] Use `Thus', `Therefore', or `We may then conclude that' to mark that your proof is coming to a close with the following claim.
 	\end{labelalign}



\vfill

\bibliographystyle{Phil_Review} %%bib style found in bst folder, in bibtex folder, in texmf folder.
\bibliography{Zotero} %%bib database found in bib folder, in bibtex folder


\end{document}

