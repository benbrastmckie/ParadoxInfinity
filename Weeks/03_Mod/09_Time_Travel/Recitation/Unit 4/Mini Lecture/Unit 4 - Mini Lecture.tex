\documentclass[11pt]{article}

\usepackage{amsmath,amsthm,amsfonts,amssymb,amscd}

\usepackage[margin=1.5in,headsep=.5in]{geometry}

\usepackage{fancyhdr}

\setlength{\headheight}{20pt}

\usepackage[colorlinks]{hyperref} 
\usepackage{cleveref}

\usepackage{enumerate}

\theoremstyle{definition}
\newtheorem{defn}{Definition}
\newtheorem{reg}{Rule}
\newtheorem{exer}{Exercise}
\newtheorem{note}{Note}
\newtheorem*{theorem*}{Theorem}
\newtheorem{theorem}{Theorem}[section]
\newtheorem{corollary}{Corollary}[theorem]
\newtheorem{thm}{Theorem}
\newtheorem{prop}[thm]{Proposition}
\newtheorem{lem}[thm]{Lemma}
\newtheorem{conj}[theorem]{Conjecture}

\pagestyle{fancy}

\begin{document}

\pagenumbering{gobble}

\lhead{$24.118$ Paradox and Infinity }
\rhead{Recitation $4$: Time Travel}



\begin{center}
{\LARGE \bf Free Will and Determinism}
\end{center}

\smallskip

\section*{Determinism}

\textbf{Determinism}: For any time $t$, the complete state of the universe at t and the laws of nature together entail the state of the universe at every later time. \\

\noindent
If the world is deterministic, can we possess free will?

\begin{enumerate}
\item[(A)] Compatibilism: Determinism is compatible with the existence of free will.
\item[(B)] Incompatibilism: Determinism is incompatible with the existence of free will.
\end{enumerate}

\section*{A Classical Argument for Incompatibilism}

\begin{enumerate}
\item If someone acts of her own free will, then she could have done otherwise.
\item If determinism is true, no one can do otherwise than one actually does.
\item Therefore, if determinism is true, no one acts of her own free will.
\end{enumerate}

This argument is clearly valid (the conclusion follows necessarily from the premises). Hence, a compatibilist, to refute this argument, has to deny at least one of the two premises.

\section*{Option One: Denying Premise 2}

One way for a compatibilist to respond to the classical argument is to deny premise 2 by claiming that determinism is compatible with the ability to do otherwise. \\

\noindent
For example, the English philosopher A. J. Ayer\footnote{A. J. Ayer, ``Freedom and Necessity".} argues that an agent could have done otherwise just in case:
\begin{enumerate}[\hspace{1cm}]
\item[(a)] she could have done otherwise if she had willed differently;
\item[(b)] her actions were voluntary and nobody has compelled her to act as she did.
\end{enumerate}

This is compatible with determinism because determinism only states that the future is determined given the actual past and laws; it is consistent with the future being different given a different past, a past in which the agent had willed differently.

\section*{A Counter-Argument from the Incompatibilists}

Consider the following case\footnote{Similar examples can be found in Roderick Chisholm's paper ``Human Freedom and the Self".}:

\begin{quote}
Suppose that Danielle is psychologically incapable of wanting to touch a blond haired dog. Imagine that, on her sixteenth birthday, unaware of her condition, her father brings her two puppies to choose between, one being a blond haired Lab, the other a black haired Lab. He tells Danielle just to pick up whichever of the two she pleases and that he will return the other puppy to the pet store. Danielle happily, and unencumbered, does what she wants and picks up the black Lab.
\end{quote}

Intuitively, Danielle could not have picked up the blond Lab, because of psychological conditions. But the conditional analysis of ``could have done otherwise", like that of Ayer's, suggests that she could have done so, as in the closest possible scenario where she wanted to pick up the blond Lab, she would have do so. Hence, the analysis yields the wrong result.

\section*{Option Two: Denying Premise 1}

Another way for a compatibilist to respond to the classical argument is to reject premise 1 by denying that freedom requires the ability to do otherwise. \\

\noindent
Consider the following case\footnote{This is a close approximation to the examples Harry Frankfurt presented in his highly influential paper ``Alternate Possibilities and Moral Responsibility".}:

\begin{quote}
Jones has resolved to shoot Smith. Black has learned of Jones’s plan and wants Jones to shoot Smith. But Black would prefer that Jones shoot Smith on his own. However, concerned that Jones might waver in his resolve to shoot Smith, Black secretly arranges things so that, if Jones should show any sign at all that he will not shoot Smith (something Black has the resources to detect), Black will be able to manipulate Jones in such a way that Jones will shoot Smith. As things transpire, Jones follows through with his plans and shoots Smith for his own reasons. No one else in any way threatened or coerced Jones, offered Jones a bribe, or even suggested that he shoot Smith. Jones shot Smith under his own steam. Black never intervened.
\end{quote}

Intuitively, Jones shot Smith out of his own free will: he did so on his own and unencumbered. But, given Black’s presence in the scenario, Jones could not have failed to shoot Smith (i.e., he could not have done otherwise). Hence, we have a counterexample to Premise 1.

\end{document}