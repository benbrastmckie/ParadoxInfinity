\documentclass[justified]{tufte-handout} 
\usepackage{amsfonts, amssymb, stmaryrd, fitch, natbib, qtree}
\usepackage{linguex, color, setspace, graphicx}
\usepackage{enumitem}
\usepackage{bussproofs}
\usepackage{turnstile}
\usepackage[super]{nth}
\thispagestyle{plain}
\definecolor{darkred}{rgb}{0.7,0,0.2}
\bibpunct{(}{)}{,}{a}{}{,}

\input xy
 \xyoption{all}

%New Symbols
\DeclareSymbolFont{symbolsC}{U}{txsyc}{m}{n}
\DeclareMathSymbol{\strictif}{\mathrel}{symbolsC}{74}
\DeclareMathSymbol{\boxright}{\mathrel}{symbolsC}{128}
\DeclareMathSymbol{\Diamondright}{\mathrel}{symbolsC}{132}
\DeclareMathSymbol{\Diamonddotright}{\mathrel}{symbolsC}{134}
\DeclareMathSymbol{\Diamonddot}{\mathord}{symbolsC}{144}
\renewcommand{\labelitemi}{$\triangleright$}
\renewcommand{\labelitemii}{$\circ$}
\renewcommand{\labelitemiii}{$\triangleright$}

%New commands
\newcommand{\bitem}{\begin{itemize}}
\newcommand{\eitem}{\end{itemize}}
\newcommand{\lang}{$\langle$}
\newcommand{\rang}{$\rangle$}
\newcommand{\back}{$\setminus$}
\newcommand{\HRule}{\rule{\linewidth}{0.1mm}}
\newcommand{\llm}[2][]{$\llbracket${#2}$\rrbracket^{#1}$}
\newcommand{\ul}{$\ulcorner$}
\newcommand{\ur}{$\urcorner\ $}
\newcommand{\urn}{$\urcorner$}
\newcommand{\sub}[1]{\textsubscript{#1}}
\newcommand{\sups}[1]{\textsuperscript{#1}}
\newtheorem{proposition}{\textbfb{Proposition}}[section]
\newtheorem{definition}[proposition]{\textbf{Definition}}
\newcommand{\bfw}{\begin{fullwidth}}
\newcommand{\efw}{\end{fullwidth}}

\begin{document}

\begin{fullwidth}
\noindent\LARGE Time Travel \& Free Will \normalsize \\[.3cm]
\noindent  \textsc{24.118 Recitation Section $\bullet$ Matthias Jenny\\  {\texttt{\href{mailto:mjenny@mit.edu}{mjenny@mit.edu}}} $\bullet$ Office:  32-D927 $\bullet$ Hours: Thu 11:30-12:30} \hfill{September 12, 2014}
\noindent\HRule
\end{fullwidth}

\section{Necessary and sufficient}

$A$ is a \emph{necessary} condition for $B$ if and only if:  \underline{\hspace{4.3cm}}\\\\\underline{\hspace{11.23cm}}\\

\noindent $A$ is a \emph{sufficient} condition for $B$ if and only if:  \underline{\hspace{4.35cm}}\\\\\underline{\hspace{11.23cm}}\\

\begin{enumerate}
\item Rain is a sufficient condition for the streets being wet.\marginnote{{\large$\square$} True  {\large$\square$} False}
\item Rain is a necessary condition for the streets being wet.\marginnote{{\large$\square$} True  {\large$\square$} False}
\item Completing every problem set is a sufficient condition for passing 24.118.\marginnote{{\large$\square$} True  {\large$\square$} False}
\item Completing every problem set is a necessary condition for passing 24.118.\marginnote{{\large$\square$} True  {\large$\square$} False}
\item Being a featherless biped is a sufficient condition for being human.\marginnote{{\large$\square$} True  {\large$\square$} False}
\item Falsity is a sufficient condition for lack of knowledge.\marginnote{{\large$\square$} True  {\large$\square$} False}
\end{enumerate}

\section{Consistency}

A story is \emph{consistent} if and only if:  \underline{\hspace{6cm}}\\\\\underline{\hspace{11.23cm}}\\\\\underline{\hspace{11.23cm}}

\section{Two stories}

\begin{description}
\item[The Box:] Carefully, \dots I removed the lid [of the box].\marginnote{From Graham Priest (1997): ``Sylvan's Box: A Short Story and Ten Morals'', \emph{Notre Dame Journal Formal Logic} 38:4.} The sunlight streamed through the window into the box, illuminating its contents, or lack of them. For some moments, I could do nothing but gaze, mouth agape. At first, I thought that it must be a trick of the light, but more careful inspection certified that it was no illusion. The box was absolutely empty, but also had something in it. Fixed to its base was a small figurine, carved of wood,\marginnote{Consistent?\\{\large$\square$} Yes. {\large$\square$} No.} Chinese influence, Southeast Asian maybe.
\item[Graham's Story:] \dots and then Graham said, ``The box was absolutely empty, but also had something in it. Fixed to its base was\marginnote{Consistent?\\{\large$\square$} Yes. {\large$\square$} No.}  a small figurine, carved of wood, Chinese influence, Southeast Asian maybe.'' 
\end{description}

\section{Easy A}

Suppose you're an excellent philosophy student. You've taken a lot of philosophy classes, and in the past, you got an A on every single exam, every single problem set, and every single paper. You've achieved this success by working extremely hard on every assignment. You have another philosophy paper due in a week, and it's the kind of assignment at which you've excelled in the past.\\

\noindent \emph{Question:} Do you know \emph{now} that you'll get an A on this paper?\marginnote{{\large$\square$} Yes. {\large$\square$} No.}\\

\noindent \emph{Why? Why not?} \underline{\hspace{8.74cm}}\\\\\underline{\hspace{11.23cm}}\\

\noindent \noindent \emph{Question:} Do you act freely as you go on to work really hard on the assignment?\marginnote{{\large$\square$} Yes. {\large$\square$} No.}\\

\noindent \emph{Why? Why not?} \underline{\hspace{8.74cm}}\\\\\underline{\hspace{11.23cm}}

\section{Chipotle}

Suppose you're standing in line at Chipotle. After careful deliberation, you decide to get a sofritas burrito.\\

\noindent \emph{Question:} Do you know that when you're next in line you'll say that you want a sofritas burrito?\marginnote{{\large$\square$} Yes. {\large$\square$} No.}\\

\noindent \emph{Why? Why not?} \underline{\hspace{8.74cm}}\\\\\underline{\hspace{11.23cm}}

\noindent \emph{Question:} Do you act freely when you say you want a sofritas burrito?\marginnote{{\large$\square$} Yes. {\large$\square$} No.}\\

\noindent \emph{Why? Why not?} \underline{\hspace{8.74cm}}\\\\\underline{\hspace{11.23cm}}



\bibliographystyle{authordate1}
\bibliography{/users/justindkhoo/library/texmf/tex/bib/Philosophy}

\end{document}
