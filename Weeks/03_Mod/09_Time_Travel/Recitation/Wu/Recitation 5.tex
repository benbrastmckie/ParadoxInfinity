\documentclass[12pt]{article}

\usepackage{amsmath,amsfonts,amssymb,amscd}

\usepackage{amsthm}

\usepackage[margin=1.5in,headsep=.5in]{geometry}

\usepackage{fancyhdr}

\setlength{\headheight}{20pt}

\usepackage[colorlinks]{hyperref} 
\usepackage{cleveref}

\usepackage{setspace}
\usepackage{enumitem,linegoal}


\newtheorem{theo}{Theorem}[section] 

\theoremstyle{definition}
\newtheorem{defin}[theo]{Definition} 
\newtheorem{lema}[theo]{Lemma} 
\newtheorem{cor}[theo]{Corollar}
\newtheorem{prop}[theo]{Proposition}
\newtheorem{exer}{Exercise}

\pagestyle{fancy}

\begin{document}

\pagenumbering{gobble}

\lhead{Xinhe Wu (xinhewu@mit.edu)}
\rhead{$24.118$ Paradox and Infinity $|$ Recitation $5$}



\begin{center}
{\Large \bf Time Travel and Free Will}
\end{center}

\smallskip

\section{Questions from the Previous Pset}

Fool has infinitely many dollar bills and has labeled each of them with a different natural number (its `serial number'). One minute before midnight, Fool gives you a dollar bill. Half a minute later, he gives you two dollar bills. Fifteen
seconds later, he gives you four dollar bills. And so forth. (For each $i\geq 0$, Fool gives you $2^i$ dollar bills $2^{-i}$ minutes before midnight.) There is, however, a
catch. Each time you receive money from Fool, you are required to put together all your dollar bills, and burn the one with the lowest serial number. \textbf{Assume that, at midnight, you have every dollar bill that you received from Fool and did not burn.} How much money will you have at midnight?

\section{Control Hypothesis}

\begin{description}
\item[Control Hypothesis] An agent acts freely in doing $X$ if and only if:
\begin{enumerate}
\item she does
X by making a certain decision, and
\item she is in a position to do something other than $X$ by making a different decision.
\end{enumerate} 
\end{description}

\begin{enumerate}
\item[Scenario One] In June 1981, Voldemort stole a crystal ball from an ancient wizard family, which predicts the future flawlessly. Voldemort saw through the crystal ball that he would soon attack an Auror couple - James and Lily Potter, together with their newborn son, Harry Potter, and that Harry would somehow survive and eventually become his lifelong enemy. Voldemort knew that the crystal ball is a perfect fortune teller, and believed what he saw. At the following Halloween, he arrived at Godric's Hollow. After killing the Potter couple, he looked at the crying baby on the floor. `The future has been written anyway.' He said to himself. And on this thought, Voldemort made his decision to cast a killing curse to baby Harry, even though he could just put down the wand and leave.

Question: According to the Control Hypothesis, is Voldemort acting freely in casting a killing curse on Harry?
\item[Scenario Two] Rick, the mad genius scientist, loves to take his good-hearted but fretful grandson Morty to insane interdimensional intergalactic misadventures. On Friday night, Rick asked Morty, again, if he would like to go on a adventurous journey together. Morty hesitated and finally decided to join. Unknown to Morty, however, Ricky has inserted a chip inside Morty's brain to track his deliberations. Had Morty intended to decide otherwise, the chip in his brain would have suddenly interfered with Morty's brain waves and made him decide as Rick wants him to. 

Question: According to the Control Hypothesis, is Morty acting freely in joining the new adventure?
\end{enumerate}

Here are some lessons about the Control Hypothesis from the above stories:
\begin{enumerate}
\item The fact that you are destined to do $X$ (in the actual world) does not entail that you do not act freely in doing $X$.
\item The fact that you know that you will do $X$ does not entail that you do not act freely in doing $X$.
\item The fact that you are not able to do otherwise by making a different decision \textit{does entail} that you do not act freely. This is what seems counter-intuitive about the Control Hypothesis.
\end{enumerate}

\section{The Toy Model}

\begin{description}
\item[Law One] In the absence of collisions, a particle’s velocity remains constant.
\item[Law Two] When two particles collide, they exchange velocities.
\item[Wormhole] We have different lines in the diagram to represent the same spacetime points.
\item[Mirror] A stationary object that reflects particles by inverting their velocity.
\end{description}

\begin{exer}
A particle $X$ is traveling on a `paradoxical' path. Depict a way to restore consistency different from the one introduced in class.
\end{exer}


\end{document}