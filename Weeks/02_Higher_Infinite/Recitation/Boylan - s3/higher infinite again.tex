\documentclass[justified]{tufte-handout} 
\usepackage{amsfonts, amssymb, stmaryrd, natbib, qtree, amsxtra}
\usepackage{linguex, color, setspace, graphicx}
\usepackage{enumitem}
\usepackage{bussproofs}
\usepackage{turnstile}
\usepackage{phaistos}
\usepackage{protosem}
\usepackage{txfonts}
\usepackage{pxfonts}
\usepackage[super]{nth}
\thispagestyle{plain}
\definecolor{darkred}{rgb}{0.7,0,0.2}
\bibpunct{(}{)}{,}{a}{}{,}

\input xy
 \xyoption{all}

%New Symbols
\DeclareSymbolFont{symbolsC}{U}{txsyc}{m}{n}
\DeclareMathSymbol{\strictif}{\mathrel}{symbolsC}{74}
\DeclareMathSymbol{\boxright}{\mathrel}{symbolsC}{128}
\DeclareMathSymbol{\Diamondright}{\mathrel}{symbolsC}{132}
\DeclareMathSymbol{\Diamonddotright}{\mathrel}{symbolsC}{134}
\DeclareMathSymbol{\Diamonddot}{\mathord}{symbolsC}{144}
\renewcommand{\labelitemi}{$\triangleright$}
\renewcommand{\labelitemii}{$\circ$}
\renewcommand{\labelitemiii}{$\triangleright$}

%New commands
\newcommand{\bitem}{\begin{itemize}}
\newcommand{\eitem}{\end{itemize}}
\newcommand{\lang}{$\langle$}
\newcommand{\rang}{$\rangle$}
\newcommand{\back}{$\setminus$}
\newcommand{\HRule}{\rule{\linewidth}{0.1mm}}
\newcommand{\llm}[2][]{$\llbracket${#2}$\rrbracket^{#1}$}
\newcommand{\ul}{$\ulcorner$}
\newcommand{\ur}{$\urcorner\ $}
\newcommand{\urn}{$\urcorner$}
\newcommand{\sub}[1]{\textsubscript{#1}}
\newcommand{\sups}[1]{\textsuperscript{#1}}
\newtheorem{proposition}{\textbfb{Proposition}}[section]
\newtheorem{definition}[proposition]{\textbf{Definition}}
\newcommand{\bfw}{\begin{fullwidth}}
\newcommand{\efw}{\end{fullwidth}}

\begin{document}

\frenchspacing

\begin{fullwidth}
\noindent\Large Section 3, The Higher Infinite Again \large \\[.3cm]
\noindent  David Boylan \hfill{11-12, 66-154}

\noindent\HRule
\end{fullwidth}


\section{Ordinal Arithmetic Again}

\begin{itemize}

\item We started ordinal arithmetic by defining addition and multiplication.


\begin{itemize}


\item Is either associative? 


\item Is either commutative?


\end{itemize}




\item  We can also define exponentiation:


\begin{quote}

$\alpha^0 =0'$

$\alpha^{\beta'} = (\alpha^\beta) \times \beta$

$\alpha^\lambda = \bigcup\{\alpha^\beta: \beta <_O \lambda\}$


\end{quote}


\item Multiplication was defined in terms of addition; exponentiation was defined in terms of multiplication. 

We could also define an operation in terms of exponentiation. How might that go? 

\end{itemize}

\section{Ordinals and Size}

\begin{itemize}

\item Here's an important question we touched on in class. We said that $\omega \neq \omega'$. 

But how big are sets of order type $\omega$ and $\omega'$?

How about sets of order type $\omega^{\omega^{\omega^\omega}}$?

\item This shows two things: 

1. Ordinals are about \emph{shape} not size.

2. It's not that straightforward to use ordinals to find big uncountable sets : even $\omega^{\omega^{\omega^\omega}}$ is countable.



\end{itemize}




\section{Finding Bigger Cardinals}


\begin{itemize}


\item We said that $\mathcal{P}^\infty_0$ is not well-defined. Why was that?

We need another way to construct big sets.


\item Imagine a sequence of ordinals like this: $\langle 0, 0', 0'', .... , \omega, \omega' \rangle$

We can use this sequence to build up larger sets in the following way: 

$\langle \mathbb{N}, P^1(\mathbb{N}), P^2(\mathbb{N}),..., \bigcup\{\mathbb{N}, P^1(\mathbb{N})...\}, \mathcal{P}^1(\bigcup\{\mathbb{N}, P^1(\mathbb{N})) \rangle $


\item We can systematically name sets formed by this procedure.

Say that $\mathcal{B}_{\alpha}$ = 

\begin{itemize}

\item $\mathbb{N}$ if $\alpha = 0$

\item $\mathcal{P}(\mathcal{B}_\beta)$, if $\beta = \alpha'$

\item $\bigcup \{\mathcal{B}_\gamma: \gamma <_O \alpha\}$ if $\alpha$ is a limit ordinal $>_O 0$ 

\end{itemize}


\item Now we have a way of using ordinals to talk about particular uncountable sets. We can see that $\alpha <_O \beta$ iff $|\mathcal{B}_\alpha| < |\mathcal{B}_\beta |$




\item The `$\mathcal{B}_\alpha$' gives us names for various particular uncountable sets. We might also want names for the \emph{cardinalities} of those sets.


We say that $\beth_\alpha = |\mathcal{B}_\alpha|$




\item Exercise: 


\begin{enumerate}
\item Come up with a large cardinal with your group.


\item Order the large cardinals suggested by each group.

\end{enumerate}



\end{itemize}



\section{Paradoxes of Set Theory}


\begin{itemize}


\item Paradox of the barbers: suppose we live on an island where there lives a barber. This barber shaves only those people who do not shave themselves. 

Question: does the barber shave himself? What does this show about the barber?


\item This lighthearted example was used to shed light on a rather deep paradox.






\item Let's start with a natural idea: for any property we can find a set of objects containing all and only the things having that property.


Here's a more rigourous characterisation of this idea: 

\ex.[Ext] $\forall F: \exists S: (x$ if $F$ iff  $x \in S)$


\item Say that a set S is \emph{self-membered} just in case $S\in S$. Can you give some examples of sets which are self-membered?


\item Let's consider the set of sets which are not self-membered i.e. $\{S| S \notin S\}$. (By Ext there should be such a set.) Call it $M$.

Is $M\in M$?

 What do you think has gone wrong here? What has led us to the paradox?


\item Important lesson: Ext is \emph{false}. It's not always true that for any group of objects with some property there's an set containing just those objects. 

Question: well what sets \emph{are} there then? This is a central question in set theory.

\item The Burali-Forti paradox, the one we saw in class, is a very similar paradox.

Assume there is a set of all ordinals. One can show that that would have to be well-ordered and so have a well-order type. Call it $\Omega$.

But then we know by our construction there would have to be a successor to that ordinal, i.e. $\Omega +1$. But we have an ordinal which is bigger than all ordinals! Contradiction.



\item One response to this is that there simply is \emph{no} set of all ordinals. (It is a ``collection" in some sense, but it is not a set.) Another example of a property with no corresponding set.

\end{itemize}




\end{document}









