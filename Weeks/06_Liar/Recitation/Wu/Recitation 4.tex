\documentclass[12pt]{article}

\usepackage{amsmath,amsfonts,amssymb,amscd}

\usepackage{amsthm}

\usepackage[margin=1.5in,headsep=.5in]{geometry}

\usepackage{fancyhdr}

\setlength{\headheight}{20pt}

\usepackage[colorlinks]{hyperref} 
\usepackage{cleveref}

\usepackage{setspace}
\usepackage{enumitem,linegoal}


\newtheorem{theo}{Theorem}[section] 

\theoremstyle{definition}
\newtheorem{defin}[theo]{Definition} 
\newtheorem{lema}[theo]{Lemma} 
\newtheorem{cor}[theo]{Corollar}
\newtheorem{prop}[theo]{Proposition}

\pagestyle{fancy}

\begin{document}

\pagenumbering{gobble}

\lhead{Xinhe Wu (xinhewu@mit.edu)}
\rhead{$24.118$ Paradox and Infinity $|$ Recitation $4$}



\begin{center}
{\Large \bf The Liar Paradox and More}
\end{center}

\smallskip

\section{The Notion of Truth}

\begin{itemize}
\item Philosophy: the quest for truth.
\item The notion of knowledge, necessity, analyticity, etc.
\item Logical consequence is the preservation of truth.
\item Parts of mathematics can be reduced to a formal theory of truth:
\begin{itemize}
\item Second order theories of properties can be reduced to first order theories of truths.
\item Truth is a useful device for measuring the provability strength of a theory.
\item Truth can be used to state very strong reflection principles.
\end{itemize}
\item (?) In computer science, the question of how to implement introspection in artificial intelligence agents is equivalent to the question of truth definitions.

\end{itemize}

\section{The Liar  in a Formal Setting}

\begin{defin}
Let our base language $L$ be the language of arithmetic. Let our base theory be $RA$ (Robinson Arithmetic). 
\end{defin}

\begin{theo} [G\"odel Coding]
A G\"odel coding scheme is definable in $RA$. In particular, for any formula   $\phi \in L$, $\ulcorner \phi \urcorner$ is the G\"odel code denoting $\phi$.
\end{theo}

\begin{theo} [ G\"odel's Fixed Point Lemma]
For any formula $\phi(x)$ definable in $RA$, there is a sentence $\psi \in L$ such that
$$\phi \leftrightarrow \psi (\ulcorner \phi \urcorner).$$
\end{theo}

\begin{defin} (The Material Adequacy of Truth Predicate) 
A predicate $T$ is the \textit{truth predicate} of a theory $S$ only if it satisfies the following scheme: for any sentence $\psi \in L_S$,
$$T(\ulcorner \psi \urcorner) \leftrightarrow \psi.$$
\end{defin}

\begin{theo} [Tarski's Undefinability Theorem] 
The truth predicate of $RA$ is not definable in $RA$.
\end{theo}

\begin{proof}
Suppose otherwise. Let $T$ be the truth predicate definable in $RA$. Therefore $\neg T$ is also definable in $RA$.

By 2.3, there exists a sentence $\lambda \in L$ such that $\lambda \leftrightarrow \neg T (\ulcorner \lambda \urcorner)$. By 2.4, $ T(\ulcorner \lambda \urcorner) \leftrightarrow \lambda$. Hence $ T(\ulcorner \lambda \urcorner) \leftrightarrow \neg T (\ulcorner \lambda \urcorner)$. Contradiction.
\end{proof}

\noindent \textbf{Conflict}: We want a formal definition of truth. But Tarski's result shows that any attempt to define a predicate that satisfies the basic requirement for truth leads to paradox.\\

\noindent \textbf{Tarski's way out}: Separation of object language and meta-language. Forms the basis of modern model theory and formal semantics of mathematics.



\end{document}