\documentclass[11pt]{article}

\usepackage{amsmath,amsthm,amsfonts,amssymb,amscd}

\usepackage[margin=1.5in,headsep=.5in]{geometry}

\usepackage{fancyhdr}

\setlength{\headheight}{20pt}

\usepackage[colorlinks]{hyperref} 
\usepackage{cleveref}

\usepackage{enumerate}

\usepackage{enumitem}
\setlist[enumerate]{itemsep=0mm}

\theoremstyle{definition}
\newtheorem{defn}{Definition}
\newtheorem{reg}{Rule}
\newtheorem{exer}{Exercise}
\newtheorem{note}{Note}
\newtheorem*{theorem*}{Theorem}
\newtheorem{theorem}{Theorem}[section]
\newtheorem{corollary}{Corollary}[theorem]
\newtheorem{thm}{Theorem}
\newtheorem{prop}[thm]{Proposition}
\newtheorem{lem}[thm]{Lemma}
\newtheorem{conj}[theorem]{Conjecture}

\pagestyle{fancy}

\newcommand{\counterfactual}{\ensuremath{%
  \Box\kern-1.5pt
  \raise1pt\hbox{$\mathord{\rightarrow}$}}}

\begin{document}

\pagenumbering{gobble}

\lhead{$24.118$ Paradox and Infinity}
\rhead{Recitation $5$: Newcomb's Problem}



\begin{center}
{\LARGE \bf Answers to In-class Exercises}
\end{center}

\smallskip

\section{The Tickle Defense}

\begin{defn}
Let $H$ be an hypothesis and let $A$, $B$ be two pieces of evidences. We say that $B$ \textit{screens off} the evidence of that $A$ provides for $H$ just in case $p(H | A \land B) = p(H | B)$.
\end{defn}


Suppose I am asked to do the Newcomb experiment: in front of me there are a large box and a small box. I am told that the small box contains a thousand dollars, and the large box either contains a million dollars or is empty. I am offered two choices - either taking both boxes, or only taking the large box. One week before I am asked to do the experiment, a predictor with $99\%$ accuracy predicted my choice. If she predicted that I would one-box, then the large box would contain a million dollars; if she predicted that I would two-box, then the large box would be empty.

Before I make the choice, my best friend, John, who is also in the room, takes a look at the content of the large box, and writes on a note, truthfully, whether the large box contains one million dollar or not.

\begin{exer}
Suppose I look at John's note and it says that the large box is empty. Let $J_E$ denote the piece of evidence that John writes the large box is empty. 
Let $2B$ be that I will two box and $E$ be that the large box is empty.
Does $J_E$ screen off the evidential import of $2B$ on $E$?
\end{exer}



\begin{proof}[Answer]
Yes. Since John's note is truthful, $p(E | J_E) = 1$. Hence $p(E | 2B \land J_E) = 1 = p(E | J_E) $, given that $p(2_B \land J_E) \neq 0$.

\end{proof}

\begin{exer}
Suppose I look at the note and it says the large box is empty. According to Evidential Decision Theory, what is the expected value of $1B$ now? What is the expected value of $2B$?  What if John's note says that the large box is full?
\end{exer}


\begin{proof}[Answer]
$EV(1_B) = v(F \land 1_B \land J_E)\cdot p(F|1_B \land J_E) + v(E \land 1_B \land J_E)\cdot p(E|1_B \land J_E) = 1000000 \cdot 0 + 0 \cdot 1 = 0$. \\
$EV(2_B) = v(F \land 2_B \land J_E)\cdot p(F|2_B \land J_E) + v(E \land 2_B \land J_E)\cdot p(E|2_B \land J_E) = 1001000 \cdot 0 + 1000 \cdot 1 = 1000$.
Hence by EDT, I should two-box. The reasoning is similar when John's note says that the large box if full. The key point is that $J_E$/$J_F$ screens off the evidential import of my choice on the content of the large box.

\end{proof}

\begin{exer}
Suppose now I offered to choose between whether to look at John's note or not, before I choose between whether to two-box or one box. According to Evidential Decision Theory, what is the expected value of me choosing to look? What is the expected value of me choosing not to look? Should I look or not?

\end{exer}


\begin{proof}[Answer]
There are many ways to carry out the calculations. But the key idea is that if I am to follow my evidential-decision-theoretic strategies faithfully, I will two-box given that I look at John's note, by the reasoning of the previous exercise. On the other hand, I will one-box given that I don't look at John's note, by the standard reasoning on the Newcomb's case. (In other words, $p(2B | L) = 1$ and $p(1B | \bar{L}) = 1$.) Therefore, the expected value of looking at John's note will turn out to be the same as the expected value of two-boxing, and the expected value of not looking will turn out to be the same as the expected value of one-boxing. And hence EDT suggests against looking.


\end{proof}

\begin{exer}
According to Causal Decision Theory, does looking at the content of John's note have any impact on my decision to one-box or two-box? Should I look at John's note or not?

\end{exer}


\begin{proof}[Answer]
According to Causal Decision Theory, just as my choice of one-boxing or two-boxing has no causal impact on whether the large box is empty, the content of John's note has no causal impact on whether the large box is empty. Hence, for example, $p(2^B \land J_E \, \, \counterfactual E) = p(E)$. Hence it does no impact on my decision to one-box or two-box. Similarly, it does not matter whether I look at John's note or not.

\end{proof}

\begin{exer}
We've shown that as soon as we know something that screens off the evidential import my choice provides for what's in the box, we should two-box according to EDT. Of course, in the standard Newcomb case, there is no John's note. But could you think of a way of using similar reasoning to get Evidential Decision Theory recommend two-boxing in the standard Newcomb case?

\end{exer}

\begin{proof}[Answer]
One possible view is that in the standard Newcomb case, introspecting my own beliefs and desires works the same as looking at John's note. By introspecting, I can tell whether I will one-box or two-box, and once I have that information, which works as evidence about whether the large box is empty, my actual decision will supply no additional evidence about whether the large box is empty.

\end{proof}


\end{document}