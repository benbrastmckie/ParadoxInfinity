\documentclass[justified]{tufte-handout} 
\usepackage{amsfonts, amssymb, stmaryrd, fitch, natbib, qtree}
\usepackage{linguex, color, setspace, graphicx}
\usepackage{enumitem}
\usepackage{bussproofs}
\usepackage{turnstile}
\usepackage[super]{nth}
\thispagestyle{plain}
\definecolor{darkred}{rgb}{0.7,0,0.2}
\bibpunct{(}{)}{,}{a}{}{,}

\input xy
 \xyoption{all}

%New Symbols
\DeclareSymbolFont{symbolsC}{U}{txsyc}{m}{n}
\DeclareMathSymbol{\strictif}{\mathrel}{symbolsC}{74}
\DeclareMathSymbol{\boxright}{\mathrel}{symbolsC}{128}
\DeclareMathSymbol{\Diamondright}{\mathrel}{symbolsC}{132}
\DeclareMathSymbol{\Diamonddotright}{\mathrel}{symbolsC}{134}
\DeclareMathSymbol{\Diamonddot}{\mathord}{symbolsC}{144}
\renewcommand{\labelitemi}{$\triangleright$}
\renewcommand{\labelitemii}{$\circ$}
\renewcommand{\labelitemiii}{$\triangleright$}

%New commands
\newcommand{\bitem}{\begin{itemize}}
\newcommand{\eitem}{\end{itemize}}
\newcommand{\lang}{$\langle$}
\newcommand{\rang}{$\rangle$}
\newcommand{\back}{$\setminus$}
\newcommand{\HRule}{\rule{\linewidth}{0.1mm}}
\newcommand{\llm}[2][]{$\llbracket${#2}$\rrbracket^{#1}$}
\newcommand{\ul}{$\ulcorner$}
\newcommand{\ur}{$\urcorner\ $}
\newcommand{\urn}{$\urcorner$}
\newcommand{\sub}[1]{\textsubscript{#1}}
\newcommand{\sups}[1]{\textsuperscript{#1}}
\newtheorem{proposition}{\textbfb{Proposition}}[section]
\newtheorem{definition}[proposition]{\textbf{Definition}}
\newcommand{\bfw}{\begin{fullwidth}}
\newcommand{\efw}{\end{fullwidth}}

\begin{document}

\begin{fullwidth}
\noindent\LARGE To Infinity and Beyond  \normalsize \\[.3cm]
\noindent  \textsc{24.118 Recitation Section $\bullet$ Matthias Jenny\\  {\texttt{\href{mailto:mjenny@mit.edu}{mjenny@mit.edu}}} $\bullet$ Office:  32-D927 $\bullet$ Hours: Thu 11:30-12:30} \hfill{October 3, 2014}
\noindent\HRule
\end{fullwidth}

\marginnote{\\\\\\\includegraphics[height=7.69cm]{buzz.jpg}}
\section{Some definitions}

\noindent A function $f$ from set $A$ to set $B$ is \emph{one-one}, or \emph{injective}, iff: \underline{\hspace{2.62cm}}\\\\\underline{\hspace{11.48cm}}\\\\\underline{\hspace{11.48cm}}\\

\noindent $f$ is \emph{onto}, or \emph{subjectve}, iff: \underline{\hspace{7.67cm}}\\\\\underline{\hspace{11.48cm}}\\

\noindent $f$ is a \emph{one-one correspondence}, or \emph{bijective}, iff: \underline{\hspace{4.84cm}}\\\\\underline{\hspace{11.48cm}}

\section{Warm up}

\begin{enumerate}
\item In our proof of $|\mathbb{N}|\neq|\mathbb{R}|$, why can't we just add the evil twin to the end or the beginning of the list? \end{enumerate}
\noindent Notes:  \underline{\hspace{15.75cm}}\\\\\underline{\hspace{16.88cm}}\\

\begin{enumerate}
\setcounter{enumi}{1}
\item What is the cardinality of the set $[0,\frac{1}{2}]$?\end{enumerate}
\noindent Notes:  \underline{\hspace{15.75cm}}\\\\\underline{\hspace{16.88cm}}\\

\begin{enumerate}
\setcounter{enumi}{2}
\item To prove that $|[0,1)|=|[0,1]|$, can't we just take the 1 in $|[0,1]|$ and ``move'' it to the front? \end{enumerate}
\noindent Notes:  \underline{\hspace{15.75cm}}\\\\\underline{\hspace{16.88cm}}\\


\section{Pset 4, problem 1}

\begin{quote}
Show that there is a 1-1 correspondence between the whole numbers ($\dots$ -3, -2, -1, 0, 1, 2, 3, $\dots$) and the prime numbers.
\end{quote}

\begin{description}
\item[Step 1:] Show that there are infinitely many prime numbers.\\\\\\\\\\\\\\\\\
\item[Step 2:] Show that there is a one-one correspondence between the natural numbers and the prime numbers.\\\\\\\\\\\\\\\\\
\end{description}


\section{Pset 4, problem 4}

\begin{quote}
Show that there can be no 1-1 correspondence between the natural numbers and the set of all functions from the natural numbers to the set $\{a, b, c, d, e\}$.
\end{quote}

\noindent \emph{Hint:} Assume for \emph{reductio} that there is a one-one correspondence $f$ between the natural numbers and the relevant set of functions.





\end{document}
