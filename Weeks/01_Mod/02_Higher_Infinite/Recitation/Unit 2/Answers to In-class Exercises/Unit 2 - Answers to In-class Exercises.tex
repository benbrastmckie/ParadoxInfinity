\documentclass[11pt]{article}

\usepackage{amsmath,amsthm,amsfonts,amssymb,amscd}

\usepackage[margin=1.5in,headsep=.5in]{geometry}

\usepackage{fancyhdr}

\setlength{\headheight}{20pt}

\usepackage[colorlinks]{hyperref} 
\usepackage{cleveref}


\theoremstyle{definition}
\newtheorem{defn}{Definition}
\newtheorem{reg}{Rule}
\newtheorem{exer}{Exercise}
\newtheorem{note}{Note}
\newtheorem*{theorem*}{Theorem}
\newtheorem{theorem}{Theorem}[section]
\newtheorem{corollary}{Corollary}[theorem]
\newtheorem{thm}{Theorem}
\newtheorem{prop}[thm]{Proposition}
\newtheorem{lem}[thm]{Lemma}
\newtheorem{conj}[theorem]{Conjecture}

\pagestyle{fancy}

\begin{document}

\pagenumbering{gobble}

\lhead{$24.118$ Paradox and Infinity }
\rhead{Recitation $2$: The Higher Infinite}



\begin{center}
{\LARGE \bf Answers to In-class Exercises}
\end{center}

\smallskip

\section{Definitions}

\begin{defn}
A set $x$ is \textit{transitive} just in case any member $y$ of $x$ is a subset of $x$, i.e. for any $y \in x$, if $z \in y$, then $z \in x$.
\end{defn}

\begin{defn}
A set $x$ is \textit{connected under $\in$} just in case for any $y, z \in x$, either $y \in z$, or $y = z$, or $z \in y$.
\end{defn}

\begin{defn}
A set $x$ is an \textit{ordinal} just in case $x$ is both transitive and connected under $\in$.
\end{defn}

\begin{defn}
A set $x$ is a \textit{finite ordinal} just in case $x$ is both finite and an ordinal.
\end{defn}

\section{Exercises}

\begin{exer}
Show that any ordinal $x$ is totally ordered by $\in$, i.e. the ordering $\in$ on $x$ is asymmetric, transitive and connected. (Hint: Axiom of Foundation.)
\end{exer}

\begin{proof}[Answer]
$x$ is already assumed to be connected under $\in$. So we just need to show asymmetry and transitivity. Let $a, b, c \in x$. 

For asymmetry, we need to show that if $a \in b$, then $b \notin a$. Suppose not: suppose $a \in b$ and $b \in a$. Then consider the set $\{a, b\}$. It will be a counterexample to Axiom of Foundation, since neither $a$ nor $b$ is disjoint from $\{a, b\}$.

For transitivity. Assume that $a \in b$ and $b \in c$. We need to show that $a \in c$. Since $x$ is connected under $\in$ and $a, c \in x$, either $a \in c$, or $a = c$, or $c \in a$. We will show that both of the other two options lead to a contradiction. First, suppose $a = c$. Then we have $a \in b$ and $b \in a$, which contradicts Axiom of Foundation by the above reasoning. Second, suppose $c \in a$. Then we have $a \in b$, $b \in c$, and $c \in a$. But then the set $\{a, b, c\}$ will become a counterexample to Axiom of Foundation. Hence $a \in c$.

\end{proof}

\begin{exer}
Show that any ordinal $x$ is well-ordered by $\in$. 
\end{exer}

\begin{proof}[Answer]
Let $y \subseteq x$ be non-empty. We need to show that $y$ has a $\in$-least member, that is, a member $u$ such that for any $z \in y$ other than $u$, $u \in z$.

Since $y$ is non-empty, by Axiom of Foundation, there is some member $v \in y$ such that $v \cap y = \emptyset$. We will show that this $v$ is precisely the $\in$-least member of $y$. Let $z \in y$ and $z \neq v$. Since $x$ is connected under $\in$ and $y \subseteq x$, either $z \in v$ or $v \in z$. But it cannot be that $z \in v$ as otherwise $z \in v \cap y$, making the latter non-empty Hence $v \in z$.
 
\end{proof}

\begin{exer}
Let $x$ be an ordinal and $y \in x$. Show that $y$ is an ordinal.
\end{exer}

\begin{proof}[Answer]
We need to show that $y$ is transitive and connected under $\in$. Since $x$ is transitive, $y \subseteq x$. And since $x$ is connected under $\in$, $y$ is connected under $\in$. Next assume $z \in y$ and $u \in z$, we need to show that $u \in y$. Since $x$ is transitive, $u \in x$. And since $x$ is connected under $\in$, either $u \in y$, or $u = y$, or $y \in u$. But both of the latter options contradict Axiom of Foundation, by reasoning similar to that of above. 

\end{proof}

\begin{exer}
Show that $\emptyset$ is an ordinal. And if $x$ is an ordinal, $x \cup \{x\}$ is an ordinal.
\end{exer}

\begin{proof}[Answer]
$\emptyset$ is trivially an ordinal. Let $x$ be an ordinal. Let $y \in x \cup \{x\}$, then either $y = x$ or $y \in x$. If $y \in x$, then $y \subseteq x \subseteq x \cup \{x\}$, as $x$ is transitive. If $y = x$, then $y \subseteq x \cup \{x\}$. Therefore $x \cup \{x\}$ is transitive. Let $y \neq z \in x \cup \{x\}$. Either $y, z \in x$, and hence $y \in z$ or $z \in y$, as $x$ is connected under $\in$, or (WLOG) $y \in x$ and $z = x$, and hence $y \in z$. Therefore $x \cup \{x\}$ is connected under $\in$.

\end{proof}

\begin{exer}
Let $x, y$ be ordinals such that $x \subseteq y$ and $x \neq y$. Show that $x \in y$.
\end{exer}

\begin{proof}[Answer]
Let $z = \{ u \in y \, \, | \, \, u \notin x\}$. Since $x \neq y$, $z$ is not empty. By Axiom of Foundation, there is some $w \in z$ such that $w \cap z  = \emptyset$. We will show that $w = x$.

Suppose $v \in w$. Then $v \in y$, as $y$ is transitive, but $v \neq z$, and hence $v \in x$.

Suppose $v \in x$. Then $v \in y$, as $y$ is transitive. Since $w \in y$ and $y$ is connected under $\in$, either $v \in w$, or $v = w$, or $w \in v$. But either $v = w$ or $w \in v$ entails that $w \in x$, as $x$ is transitive, which contradicts $w \notin x$, as $w \in z$. Hence $v \in w$. 

Hence $x$ and $w$ have exactly the same members. By Axiom of Extensionality, they are the same.

\end{proof}

\begin{exer}
Let $x, y$ be ordinals. Show that either $x \in y$, or $x = y$, or $y \in x$. (Hint: consider $x \cap y$ and use Exercise 5.)

\end{exer}

\begin{proof}[Answer]
Consider $z = x \cap y$. It is easy to show that $z$ is both transitive and connected under $\in$, as both $x$ and $y$ are transitive and connected under $\in$. Hence $z$ is itself an ordinal.

By Exercise 5, to show what we want we just need to show that $x \subseteq y$ or $y \subseteq x$. Note that $z = x$ iff $x \subseteq y$ and $z = y$ iff $y \subseteq x$. Hence we just need to show that either $z = x$ or $z = y$.

Assume for reductio that $z \neq x$ and $z \neq y$. Then by Exercise 5, $z \in x$ and $z \in y$. Hence $z \in x \cap y = z$, contradicting Axiom of Foundation. Hence either  $z = x$ or $z = y$.

\end{proof}

\begin{exer}
Let $\omega = \{ x \, \, | \, \, x \, \, \text{is a finite ordinal} \}$. Show that $\omega$ is an ordinal.
\end{exer}

\begin{proof}[Answer]
Since $\omega$ only contains ordinals, it is connected under $\in$ by Exercise 6.

Let $x \in y \in \omega$. Then by definition, $y$ is a finite ordinal. Therefore $y$ is transitive and $x \subseteq y$. Hence $x$ is finite. Also, by Exercise 3, since $x \in y$ and $y$ is an ordinal, $x$ is an ordinal. Hence $x$ is a finite ordinal. $x \in \omega$. Hence $\omega$ is transitive.

\end{proof}

\end{document}