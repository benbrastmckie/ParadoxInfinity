\documentclass[justified]{tufte-handout} 
\usepackage{amsfonts, amssymb, stmaryrd, natbib, qtree, amsxtra}
\usepackage{linguex, color, setspace, graphicx}
\usepackage{enumitem}
\usepackage{bussproofs}
\usepackage{turnstile}
\usepackage{phaistos}
\usepackage{protosem}
\usepackage{txfonts}
\usepackage{pxfonts}
\usepackage[super]{nth}
\thispagestyle{plain}
\definecolor{darkred}{rgb}{0.7,0,0.2}
\bibpunct{(}{)}{,}{a}{}{,}

\input xy
 \xyoption{all}

%New Symbols
\DeclareSymbolFont{symbolsC}{U}{txsyc}{m}{n}
\DeclareMathSymbol{\strictif}{\mathrel}{symbolsC}{74}
\DeclareMathSymbol{\boxright}{\mathrel}{symbolsC}{128}
\DeclareMathSymbol{\Diamondright}{\mathrel}{symbolsC}{132}
\DeclareMathSymbol{\Diamonddotright}{\mathrel}{symbolsC}{134}
\DeclareMathSymbol{\Diamonddot}{\mathord}{symbolsC}{144}
\renewcommand{\labelitemi}{$\triangleright$}
\renewcommand{\labelitemii}{$\circ$}
\renewcommand{\labelitemiii}{$\triangleright$}

%New commands
\newcommand{\bitem}{\begin{itemize}}
\newcommand{\eitem}{\end{itemize}}
\newcommand{\lang}{$\langle$}
\newcommand{\rang}{$\rangle$}
\newcommand{\back}{$\setminus$}
\newcommand{\HRule}{\rule{\linewidth}{0.1mm}}
\newcommand{\llm}[2][]{$\llbracket${#2}$\rrbracket^{#1}$}
\newcommand{\ul}{$\ulcorner$}
\newcommand{\ur}{$\urcorner\ $}
\newcommand{\urn}{$\urcorner$}
\newcommand{\sub}[1]{\textsubscript{#1}}
\newcommand{\sups}[1]{\textsuperscript{#1}}
\newtheorem{proposition}{\textbfb{Proposition}}[section]
\newtheorem{definition}[proposition]{\textbf{Definition}}
\newcommand{\bfw}{\begin{fullwidth}}
\newcommand{\efw}{\end{fullwidth}}

\begin{document}

\frenchspacing

\begin{fullwidth}
\noindent\Large Section 2, The Higher Infinite \large \\[.3cm]
\noindent  David Boylan \hfill{11-12, 66-154}

\noindent\HRule
\end{fullwidth}


\section{Background}

\begin{itemize}


\item Quiz answers:

\begin{enumerate}

\item The set of natural numbers is well ordered.

\item The rationals are not well-ordered. Why not?

\item The reals in [0,1] are well ordered. Why not?

\end{enumerate}

\item Given an example of a set $S$ and a relation $R$, where $R$ well orders $S$. Why does it meet each of the conditions for a well-ordering?


\item We'll say that $\langle S,<_1\rangle$ and $\langle S,<_2\rangle$ are of the same well-order type whenever they are isomorphic. But when are they \emph{isomorphic}?

  

\item Which of the following pairs of well-orderings have the same well-order type?

\begin{itemize}

 \item The set $\{7, 2, 13, 12\}$ ordered by $<_N$ and the set $\{412, 708, 20081\}$ ordered by $<_N$,
where $<N$ is the standard ordering of the natural numbers.
 
\item The natural numbers under $<_N$, and natural numbers greater than $17$ under $<_N$.


\end{itemize}

\end{itemize}

\section{Isomorphisms and Ordinals}

\begin{itemize}


\item Let's dwell on the notion of an isomorphism. Isomorphisms capture structure: if two mathematical objects are isomorphic, then intuitively they have the same structure.

\item In fact, since isomorphisms are an equivalence relation, i.e.: 


\begin{itemize}

\item A is always isomorphic to A (reflexive);

\item if there is an isomorphism from A to B, then there's an isomorphism from B to A;


\item if there's an isomorphism from A to B and an isomorphism from A to C, then there's an isomorphism from A to C.


\end{itemize}


 This means that we can divide up the set of objects we are interested in into \emph{equivalences classes}.

\item This means we can pick a representative $R$ for each equivalence class and we will know that if something is isomorphic to $R$ it belongs in that equivalence class.



\item Now we can see that if we have a set plus a well-ordering on it, that is a kind of structure. We can then divide well-ordered sets up into equivalence classes.


\item This helps us answer the question of what work the ordinals are supposed to be doing. The point of the ordinals is that they give us a way of constructing a representative member for each equivalence class.

\item In fact every well-ordered set is isomorphic to some ordinal ordinal. (Sadly we won't be proving this in this class.)


\end{itemize}


\section{Constructing the Ordinals}


\begin{itemize}


\item \textbf{Ordinal Principle.} At each stage, introduce a new ordinal, which is the set of all ordinals
that have been introduced at previous stages of the process.

\item In notation: $\alpha' = \alpha \cup \{\alpha\}$


\item This gives us two kinds of ordinals: 

$\alpha$ is a successor ordinal iff there's a $\beta$ such that $\alpha =\beta'$

$\alpha$ is a limit ordinal iff it is not a successor ordinal. 

(What kind of ordinal is 0?)


\item Let $\alpha <_O\beta$ say that $\alpha$ is introduced before $\beta$. Since our stages of construction are well-ordered, $<_O$ is a well-ordering.



\item This gives us the nice result that $\alpha <_O \beta$ iff $\alpha \in \beta$



\item The ordinals are well-ordered: every ordinal is well-ordered by $\in$.



\item Ordinals are set-transitive: if $\beta \in\gamma$ and $\alpha\in\beta$ then $\alpha \in\beta$. Why?


\item Indeed the official definition of an ordinal is that an ordinal is a set that is well-ordered by $\in$ and set-transitive.

\end{itemize}


\section{Ordinal Arithmetic} 


\begin{itemize}


\item Intuitively, adding $\alpha$ to $\beta$ is playing the $\beta$ sequence after the $\alpha$ sequence.


\item Here is a more rigourous definition: 

$\alpha + 0 = \alpha$

$\alpha + \beta'= (\alpha + \beta)'$

$\alpha + \lambda = \bigcup\{\alpha +\beta: \beta <_O \lambda\}$

\item Intuitively, multiplying  $\alpha$ by $\beta$ is to replace every element of $\beta$ with a sequence of type $\alpha$.

\item A more rigourous definition: 

$\alpha \times  0= 0$

$\alpha \times \beta'=(\alpha \times \beta) + \alpha$

$\alpha \times \lambda = \bigcup \{\alpha\times\beta: \beta<_O \lambda\}$





\item Some questions for you: 

\begin{enumerate}

\item Is addition commutative i.e. does $\alpha +\beta = \beta +\alpha$ ? Why or why not?


\item Is multiplication commutative i.e. does $\alpha\times\beta=\beta\times\alpha$? Why or why not? 

\item Is $(\omega + \omega) + \omega = \omega + (\omega + \omega)$?


\item Is $(\omega +0''')\times 0'' = (\omega \times 0'') + (0''' \times 0'')?$ 


\end{enumerate}


\end{itemize}





\end{document}

\section{Paradoxes of Set Theory}


\begin{itemize}


\item Paradox of the barbers: suppose we live on an island where there lives a barber. This barber shaves only those people who do not shave themselves. 

Question: does the barber shave himself? What does this show about the barber?


\item This lighthearted example was used to shed light on a rather deep paradox.






\item comprehension axiom


\item What do you think has gone wrong here? What has led us to the paradox?



\end{itemize}








