\documentclass[justified]{tufte-handout} 
\usepackage{amsfonts, amssymb, stmaryrd, fitch, natbib, qtree}
\usepackage{linguex, color, setspace, graphicx}
\usepackage{enumitem}
\usepackage{bussproofs}
\usepackage{turnstile}
\usepackage[super]{nth}
\thispagestyle{plain}
\definecolor{darkred}{rgb}{0.7,0,0.2}
\bibpunct{(}{)}{,}{a}{}{,}

\input xy
 \xyoption{all}

%New Symbols
\DeclareSymbolFont{symbolsC}{U}{txsyc}{m}{n}
\DeclareMathSymbol{\strictif}{\mathrel}{symbolsC}{74}
\DeclareMathSymbol{\boxright}{\mathrel}{symbolsC}{128}
\DeclareMathSymbol{\Diamondright}{\mathrel}{symbolsC}{132}
\DeclareMathSymbol{\Diamonddotright}{\mathrel}{symbolsC}{134}
\DeclareMathSymbol{\Diamonddot}{\mathord}{symbolsC}{144}
\renewcommand{\labelitemi}{$\triangleright$}
\renewcommand{\labelitemii}{$\circ$}
\renewcommand{\labelitemiii}{$\triangleright$}

%New commands
\newcommand{\bitem}{\begin{itemize}}
\newcommand{\eitem}{\end{itemize}}
\newcommand{\lang}{$\langle$}
\newcommand{\rang}{$\rangle$}
\newcommand{\back}{$\setminus$}
\newcommand{\HRule}{\rule{\linewidth}{0.1mm}}
\newcommand{\llm}[2][]{$\llbracket${#2}$\rrbracket^{#1}$}
\newcommand{\ul}{$\ulcorner$}
\newcommand{\ur}{$\urcorner\ $}
\newcommand{\urn}{$\urcorner$}
\newcommand{\sub}[1]{\textsubscript{#1}}
\newcommand{\sups}[1]{\textsuperscript{#1}}
\newtheorem{proposition}{\textbfb{Proposition}}[section]
\newtheorem{definition}[proposition]{\textbf{Definition}}
\newcommand{\bfw}{\begin{fullwidth}}
\newcommand{\efw}{\end{fullwidth}}

\begin{document}

\begin{fullwidth}
\noindent\LARGE Ordinals  \normalsize \\[.3cm]
\noindent  \textsc{24.118 Recitation Section $\bullet$ Matthias Jenny\\  {\texttt{\href{mailto:mjenny@mit.edu}{mjenny@mit.edu}}} $\bullet$ Office:  32-D927 $\bullet$ Hours: Thu 11:30-12:30} \hfill{October 10, 2014}
\noindent\HRule
\end{fullwidth}

\section{Some definitions}

\noindent A relation $R$ on a set $A$ is \emph{anti-symmetrical} iff: \underline{\hspace{9.95cm}}\\\\\underline{\hspace{16.88cm}}\\\\\underline{\hspace{16.88cm}}\\

\noindent A relation $R$ on a set $A$ is \emph{transitive} iff: \underline{\hspace{11cm}}\\\\\underline{\hspace{16.88cm}}\\\\\underline{\hspace{16.88cm}}\\

\noindent A relation $R$ on a set $A$ is \emph{total} iff: \underline{\hspace{11.7cm}}\\\\\underline{\hspace{16.88cm}}\\

\noindent A relation $R$ on a set $A$ is a \emph{linear order} (or \emph{total order}) iff: \underline{\hspace{8.3cm}}\\\\\underline{\hspace{16.88cm}}\\

\noindent A relation $R$ on a set $A$ is a \emph{well order} (or \emph{total order}) iff: \underline{\hspace{8.5cm}}\\\\\underline{\hspace{16.88cm}}\\

\noindent A set $A$ is \emph{transitive} iff: \underline{\hspace{13.3cm}}\\\\\underline{\hspace{16.88cm}}\\

\noindent A set $A$ is an \emph{ordinal} iff: \underline{\hspace{13.2cm}}\\\\\underline{\hspace{16.88cm}}\\

\noindent For two ordinals $\alpha$ and $\beta$, $\alpha<_O\beta$ iff \underline{\hspace{11.2cm}}\\\\\underline{\hspace{16.88cm}}\\

\section{Pset 5, problem 4}

\begin{enumerate}[label=\roman*.]
\item $\alpha+1_O=\alpha\cup\{\alpha\}$\marginnote{{\large$\square$} True  {\large$\square$} False}\\

\noindent \emph{Notes:}  \underline{\hspace{15.4cm}}\\

\item $1_O+3_O=3_O+1_O$\marginnote{{\large$\square$} True  {\large$\square$} False}\\

\noindent \emph{Notes:}  \underline{\hspace{15.4cm}}\\

\item $1_O\times 3_O=3_O\times 1_O$\marginnote{{\large$\square$} True  {\large$\square$} False}\\

\noindent \emph{Notes:}  \underline{\hspace{15.4cm}}\\

\item $\omega\times 3_O=\omega+(\omega+\omega)$\marginnote{{\large$\square$} True  {\large$\square$} False}\\

\noindent \emph{Notes:}  \underline{\hspace{15.4cm}}\\

\item $\omega\times\omega<_O\omega\times(2_O\times\omega)$\marginnote{{\large$\square$} True  {\large$\square$} False}\\

\noindent \emph{Notes:}  \underline{\hspace{15.4cm}}\\

\item $1_O+\omega=\omega+1_O$\marginnote{{\large$\square$} True  {\large$\square$} False}\\

\noindent \emph{Notes:}  \underline{\hspace{15.4cm}}\\

\item $(\omega+2_O)+\omega<_O(\omega+\omega)+3_O$\marginnote{{\large$\square$} True  {\large$\square$} False}\\

\noindent \emph{Notes:}  \underline{\hspace{15.4cm}}\\

\item $3_O\times\omega=(\omega+\omega)+\omega$\marginnote{{\large$\square$} True  {\large$\square$} False}\\

\noindent \emph{Notes:}  \underline{\hspace{15.4cm}}\\

\item $(\omega\times 2_O)+\omega<_O(\omega+\omega)+2_O$\marginnote{{\large$\square$} True  {\large$\square$} False}\\

\noindent \emph{Notes:}  \underline{\hspace{15.4cm}}\\

\item $\omega\times(\omega+\omega)=(\omega\times\omega)+(\omega\times\omega)$\marginnote{{\large$\square$} True  {\large$\square$} False}\\

\noindent \emph{Notes:}  \underline{\hspace{15.4cm}}\\
\end{enumerate}




\end{document}
