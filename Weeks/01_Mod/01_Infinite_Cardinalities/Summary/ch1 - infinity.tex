
\documentclass[12pt]{extarticle}
\usepackage{summary-intro}




\newcommand{\re}{\rotatebox[origin=c]{90}{$=$}}



\begin{document}

\sumintro{Infinite Cardinals}{Spring 2023}



\vspace{-15mm}

\section{Basic Definitions}

\begin{description}
\item[Cardinality] $|A|$ is the ``size'' of set $A$ (allegedly)

\item[Bijection Principle]
$|A| = |B|$ iff there is a bijection from $A$ to $B$.

\item[Injection Principle]
$|A| \leq |B|$ iff there is an injection $A$ to $B$.

\textit{Aside}: for the purposes of the PSets (where we are \#normcore), we'll treat cardinality as `size,' but in lecture we will occasionally call this terminological choice into question. 

\end{description}



%\vspace{10mm}
\section{Extended Definitions}



\begin{center}
\begin{tabular}{ccc} \hline
\textbf{Notation} &  \textbf{How it's defined} & \textbf{Informal notion}  \\ \hline
$|A| = |B|$ &  {\small bijection from $A$ to $B$} & {\small just as many members in $A$ as in $B$}  \\ \hline
 $|A| \leq |B|$ & {\small injection from $A$ to $B$} & {\small at most~as many members in $A$ as in $B$}\\
$|A| < |B|$ & {\small $|A| \leq~|B|$ and $|A| \neq |B|$} & {\small fewer members in $A$ than in $B$} \\ \hline
$|A| \geq |B|$ &  {\small $|B| \leq |A|$} & {\small at least as many members in $A$ as in $B$} \\
$|A| > |B|$ & {\small $|A| \geq |B|$ and $|A| \neq |B|$} & {\small more members in $A$ than in $B$}\\
\hline
\end{tabular}
\end{center}


\vspace{1mm}
\section{Properties of $\leq$}


\begin{description}\label{gloss:partial-order}

\item[Reflexivity]
$|A|\leq |A|$

\item[Anti-symmetry] If $|A|\leq
|B|$ and $|B|\leq |A|$, then $|A|= |B|$


\item[Transitivity] If $|A|\leq
|B|$ and
$|B|\leq
|C|$, then
$|A|\leq |C|$
\item[Totality]\hspace{-2.2mm}\footnote{One can only prove Totality if one assumes a controversial set-theoretic axiom: the Axiom of Choice. We'll come across this axiom again. Stay tuned!} 
For any sets $A$ and $B$, either
$|A|\leq |B|$ or $|B|\leq |A|$
\end{description}

\section{Infinity v. Countability}

\begin{description}
\item[$\bm A$ is infinite:]  $|\mathbb{N}| \leq |A|$

\item[$\bm A$ is countable:] $|A| \leq |\mathbb{N}|$
\end{description}



\section{A Little Lemma}


\begin{description}
\item[No Countable Difference Principle:] 
\(|I| = |I \cup C|\), for $I$ infinite and $C$ countable.

\end{description}



\section{Sets with the same size as $\mathbb{R}$}


\begin{center}
\begin{tabular}{ccc} \hline
\textbf{Set} & \textbf{Also known as} & \textbf{Members} \\ \hline
{\small $[0,1)$}  & {\small unit interval (half closed)} & \begin{tabular}{@{}c@{}}{\small real numbers larger or equal} \\ {\small to 0 but smaller than 1}\end{tabular} \\ \hline
{\small $(0,1)$}  & {\small unit interval (open)} & \begin{tabular}{@{}c@{}}{\small real numbers larger} \\ {\small than 0 but smaller than 1}\end{tabular} \\ \hline
{\small $[0,1]$}  & {\small unit interval (closed)} & \begin{tabular}{@{}c@{}}{\small real numbers larger or equal}  \\ {\small to 0 but smaller or equal to 1}\end{tabular} \\ \hline 
{\small $[0,a]$}  & {\small arbitrarily sized interval} & \begin{tabular}{@{}c@{}}{\small real numbers larger or equal to $0$}  \\ {\small  but smaller or equal to $a$ ($a>0$)}\end{tabular} \\ \hline 
{\small $[0,1] \times [0,1]$}  & {\small unit square} & \begin{tabular}{@{}c@{}}{\small pairs of real numbers larger or} \\ {\small equal  to 0 but smaller or equal to 1}\end{tabular} \\ \hline
{\small $\underbrace{[0,1] \times \dots \times [0,1]}_{\text{$n$ times}}$}  & {\small $n$-dimensional hypercube} & \begin{tabular}{@{}c@{}}{\small $n$-tuples of real numbers larger or }  \\ {\small equal to 0 but smaller or equal to 1}\end{tabular} \\ \hline
{\small $\mathbb{R}$}  & {\small real line} & \begin{tabular}{@{}c@{}}{\small real numbers}\end{tabular} \\ \hline
{\small $\powerset(\mathbb{N})$}  & {\small powerset of $\mathbb{N}$} & \begin{tabular}{@{}c@{}}{\small sets of natural numbers}\end{tabular} \\ \hline
\end{tabular}
\label{gloss:interv-1}
\end{center}

\section{Summary of Cardinality Comparisons}

\vspace{3mm}

\[
\begin{array}{ccccccccc}
|\mathbb{N}| &<  &|\powerset (\mathbb{N})| &< &|\powerset(\powerset (\mathbb{N}))| &< &|\powerset(\powerset(\powerset (\mathbb{N})))| &< &\dots \\
\re & \ & \re \\
|\mathbb{Z}| & \ & |\mathbb{R}|\\
\re & \ & \re \\
|\mathbb{Q}| & \ & |[0,1]|\\
\ & \ & \re \\
\ & \ & |\underbrace{[0,1] \times \ldots \times [0,1]}_{\text{$n$ times}}|\\
\end{array}
\]



\end{document}