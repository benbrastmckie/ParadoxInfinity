\documentclass[11pt]{article}

\usepackage{amsmath,amsthm,amsfonts,amssymb,amscd}

\usepackage[margin=1.5in,headsep=.5in]{geometry}

\usepackage{fancyhdr}

\setlength{\headheight}{20pt}

\usepackage[colorlinks]{hyperref} 
\usepackage{cleveref}

\usepackage{enumerate}

\theoremstyle{definition}
\newtheorem{defn}{Definition}
\newtheorem{reg}{Rule}
\newtheorem{exer}{Exercise}
\newtheorem{note}{Note}
\newtheorem*{theorem*}{Theorem}
\newtheorem{theorem}{Theorem}[section]
\newtheorem{corollary}{Corollary}[theorem]
\newtheorem{thm}{Theorem}
\newtheorem{prop}[thm]{Proposition}
\newtheorem{lem}[thm]{Lemma}
\newtheorem{conj}[theorem]{Conjecture}

\pagestyle{fancy}

\begin{document}

\pagenumbering{gobble}

\lhead{$24.118$ Paradox and Infinity }
\rhead{Recitation $3$: Omega-Sequence Paradoxes}



\begin{center}
{\LARGE \bf In-class Exercises}
\end{center}

\smallskip

\section{Yablo's Button}

\begin{exer}
Suppose\footnote{This paradox is from Andrew Bacon, ``A paradox for supertask decision makers".} time has no beginning and there is no first day. More precisely, suppose that time forms a reverse $\omega$-sequence: $..., d_4, d_3, d_2, d_1, d_0$. On each day, a man (who has always existed) must choose whether or not to press \textit{Yablo's Button}. Yablo's Button is designed such that it dispenses a chocolate on it's first pressing (let's assume that the man loves chocolate), and a painful zapping on subsequent pressings (let's say the button has some mechanism that records whether it has been pressed before). 

If the man is rational on day $n$ he would behave as follows:
\begin{enumerate}[\indent(1)]
\item Press the button if it has not been pressed on any earlier day.
\item Leave the button alone if it has been pressed on an earlier day.
\end{enumerate}

Show that the man cannot be rational on every day: in particular, there has to be some day when he violates one of the rules.
\end{exer}

\section{The Knave}

\begin{exer}
Jack the Knave is a liar. Before today, all the sentences Jack has ever uttered have been lies (let's just assume a lie is a sentence that is false.) Today, Jack uttered:
$$\text{Every sentence I have uttered is a lie, including this one.}$$
Is the sentence Jack uttered today true? Or is it a lie?
\end{exer}

\section{Yablo Once Again}

\begin{exer}
Consider the following infinite sequence of sentences:
\begin{itemize}
    \item [] $A_1$: For some $k > 1$,  $A_k$ is not true.
    \item [] $A_2$: For some $k > 2$,  $A_k$ is not true.
    \item [] ...
    \item [] $A_n$: For some $k > n$, $A_k$ is not true.
    \item [] ...
\end{itemize}
Show that for every $n$, $A_n$ is paradoxical.
\end{exer}

\begin{exer}
Consider the following infinite sequence of sentences:
\begin{itemize}
    \item [] $B_1$: There are infinitely many $k > 1$ such that $B_k$ is not true.
    \item [] $B_2$: There are infinitely many $k > 2$ such that $B_k$ is not true.
    \item [] ...
    \item [] $B_n$: There are infinitely many $k > n$ such that $B_k$ is not true.
    \item [] ...
\end{itemize}
Show that for every $n$, $B_n$ is paradoxical.
\end{exer}


\end{document}