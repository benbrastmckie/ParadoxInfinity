\documentclass[justified]{tufte-handout} 
\usepackage{amsfonts, amssymb, stmaryrd, natbib, qtree, amsxtra}
\usepackage{linguex, color, setspace, graphicx}
\usepackage{enumitem}
\usepackage{bussproofs}
\usepackage{turnstile}
\usepackage{phaistos}
\usepackage{protosem}
\usepackage{txfonts}
\usepackage{pxfonts}
\usepackage[super]{nth}
\thispagestyle{plain}
\definecolor{darkred}{rgb}{0.7,0,0.2}
\bibpunct{(}{)}{,}{a}{}{,}

\input xy
 \xyoption{all}

%New Symbols
\DeclareSymbolFont{symbolsC}{U}{txsyc}{m}{n}
\DeclareMathSymbol{\strictif}{\mathrel}{symbolsC}{74}
\DeclareMathSymbol{\boxright}{\mathrel}{symbolsC}{128}
\DeclareMathSymbol{\Diamondright}{\mathrel}{symbolsC}{132}
\DeclareMathSymbol{\Diamonddotright}{\mathrel}{symbolsC}{134}
\DeclareMathSymbol{\Diamonddot}{\mathord}{symbolsC}{144}


%New commands
\newcommand{\bitem}{\begin{itemize}}
\newcommand{\eitem}{\end{itemize}}
\newcommand{\lang}{$\langle$}
\newcommand{\rang}{$\rangle$}
\newcommand{\back}{$\setminus$}
\newcommand{\HRule}{\rule{\linewidth}{0.1mm}}
\newcommand{\llm}[2][]{$\llbracket${#2}$\rrbracket^{#1}$}
\newcommand{\ul}{$\ulcorner$}
\newcommand{\ur}{$\urcorner\ $}
\newcommand{\urn}{$\urcorner$}
\newcommand{\sub}[1]{\textsubscript{#1}}
\newcommand{\sups}[1]{\textsuperscript{#1}}
\newtheorem{proposition}{\textbfb{Proposition}}[section]
\newtheorem{definition}[proposition]{\textbf{Definition}}
\newcommand{\bfw}{\begin{fullwidth}}
\newcommand{\efw}{\end{fullwidth}}

\begin{document}

\frenchspacing

\begin{fullwidth}
\noindent\Large Section 6,  Omega Sequence Paradoxes \large \\[.3cm]
\noindent  David Boylan \hfill{11-12, 66-154}

\noindent\HRule
\end{fullwidth}


\section{Paradoxes}

\begin{itemize}


\item \emph{What is a paradox?}

A paradox is an argument that appears to be valid, and goes from seemingly true premises to a seemingly false conclusion.

A good paradox is one where the premises seem very hard to deny but the result is completely untenable.


\item \emph{What can we learn from a paradox?}

Good paradoxes can turn up surprising results; sometimes they show that seemingly plausible principles give rise to absurdities.

\end{itemize}


\section{Zeno's Paradox}


\begin{itemize}
\item Let's start with an oldie (maybe not a goodie). 



\item Alice wants to go from A to B, a distance of 1k. Suppose she travels at a speed of 100 m per minute.

\item But it looks like the tortoise has to do an infinite number of tasks. 

\begin{itemize}

\item It has to traverse half the distance, 500m, which takes her 5 minutes;

\item then half the remaining distance, 250m, which takes her 2.5 minutes;

\item then half the remaining distance, 125m, which takes her 1.25 minutes;

\item and so on.


\end{itemize}


But if it has to do an infinite number of tasks,\marginnote{A task like this involving an infinite number of steps is called a \emph{supertask}.} then how can it ever get to B? Won't she never get to B?


\item Zeno concluded that motion is impossible. 

\vspace{.2cm}

\noindent But we should not. What should we say instead? How long does it take her to travel the distance?


\item What happens if she slows down? 


\end{itemize}




\section{Thomson's Lamp}


\begin{itemize}


\item Now for a more challenging paradox.


\item It is 11pm. In front of me is a lamp, currently switched off.

 Imagine I do the following:

\begin{itemize}

\item After 30 minutes I turn the lamp off;

\item After 15 minutes I turn the lamp on;

\item After 7.5 minutes I turn the lamp off;

\item And so on. 


\end{itemize}

Question: is the lamp on or off at midnight?\marginnote{Notice that part of what makes this interesting is that there is no limit here. That is why we cannot simply say the same thing as we did before.}



\item Thomson thought such a lamp was \emph{logically impossible}. There simply couldn't be any such lamp, in any sense. 

\vspace{.2cm}
\noindent Was he right?





\end{itemize}




\section{The Liar Paradox}


\begin{itemize}


\item Here is a famous, intuitive principle governing the concept of truth, the T-schema

\ex. `p' is true iff p.\marginnote{Important convention: taking a word and putting it in quotation marks is a way of coming up with a name for a linguistic expression. For example, `It's raining' is the name of a sentence of English.}

Looks obvious, right?

As obvious as it seems, this principle and some basic logic yield a paradox. 

\item  Let $s$ be the sentence `$s$ is false'. (Also called the \emph{liar sentence}.)

Question: Is $s$ true or false? Be clear in your explanation about where the T-schema comes in.


\item Notice something about `s'. It refers to itself! So maybe we should ban sentences that refer to themselves from our language.


Question: is self-reference \emph{sufficient} for paradox? Does this bear on whether we should ban self-reference?


\item Idea: maybe it is still \emph{necessary} for paradox! 

\end{itemize}

\section{Yablo's Paradox}

\begin{itemize}

\item Yablo's paradox is meant to refute our last idea.

\item Imagine a sequence of sentences like this: 

\begin{itemize}

\item $s_1$: for every $n>1$, $s_n$ is false. 

\item $s_2$: for every $n>2$, $s_n$ is false. 

\item $s_3$: for every $n>3$, $s_n$ is false. 

\item and so on.


\end{itemize}



\item Given the T-schema we know the following: 


\begin{itemize}

\item $s_1$ is true iff for every $n>1$, $s_n$ is false. 

\item $s_2$ is true iff for every $n>2$, $s_n$ is false. 

\item $s_3$ is true iff for every $n>3$, $s_n$ is false. 

\item and so on.


\end{itemize}

\item Question: is $s_k$ true or false? 


\item Yablo's paradox potentially sheds some interesting light on what is going on with the Liar paradox. The T-schema is problematic, \emph{even} if we ban self-reference in our language!

\item Question: is there really no self-reference here? Could you argue some kind of self-reference here?




\end{itemize}




\end{document}