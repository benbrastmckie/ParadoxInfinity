\documentclass[11pt]{article}

\usepackage{amsmath,amsthm,amsfonts,amssymb,amscd}

\usepackage[margin=1.5in,headsep=.5in]{geometry}

\usepackage{fancyhdr}

\setlength{\headheight}{20pt}

\usepackage[colorlinks]{hyperref} 
\usepackage{cleveref}


\theoremstyle{definition}
\newtheorem{defn}{Definition}
\newtheorem{reg}{Rule}
\newtheorem{exer}{Exercise}
\newtheorem{note}{Note}
\newtheorem*{theorem*}{Theorem}
\newtheorem{theorem}{Theorem}[section]
\newtheorem{corollary}{Corollary}[theorem]
\newtheorem{thm}{Theorem}
\newtheorem{prop}[thm]{Proposition}
\newtheorem{lem}[thm]{Lemma}
\newtheorem{conj}[theorem]{Conjecture}

\pagestyle{fancy}

\begin{document}

\pagenumbering{gobble}

\lhead{$24.118$ Paradox and Infinity }
\rhead{Recitation $3$: Omega-Sequence Paradoxes}



\begin{center}
{\LARGE \bf The Liar Paradox}
\end{center}

\smallskip

\section{The Notion of Truth}

Philosophy is the quest for truth. The notion of truth is central to almost all philosophical studies. Consider the predicate ``is true". What is a minimal condition that any satisfactory theory of this predicate should meet? \\

\noindent
In 1933, the Polish logician Alfred Tarski argues that the behavior of the truth predicate should at least satisfy the following biconditional: for any sentence $\phi$,

$$``\phi" \, \, \text{is true} \leftrightarrow \phi$$

\noindent
This biconditional is usually called ``Tarski's \textit{T-schema}". Let $\phi$ be the sentence that snow is white. Intuitively, we want to say that ``snow is white" is true just in case snow is white.

\section{The Liar Paradox}

Unfortunately, Tarski's T-schema, as humble as it seems, leads to a notorious paradox. Consider the following sentence ``$L$":

$$ L: \, \, ``L" \, \, \text{is false}.$$

\noindent
A paradox seems to be generated: $``L"$ is either true or false. Suppose $``L"$ is true. By T-schema, we get $L$, which is the same as that $``L"$ is false. This is a contradiction since $L$ cannot be both true and false. So $``L"$ is false. But this is the same as $L$. By T-schema again, $``L"$ is true. Again, this is a contradiction since $L$ cannot be both true and false.

\section{A Potential Solution?}

In the reasoning behind the above paradox we have used the following assumption:

$$\text{Bivalence: Every sentence is either true or false.}$$

\noindent
Maybe this assumption is the culprit. Perhaps our liar sentence ``$L$" should be a sentence that is neither true nor false. And this is how the paradox can be avoided. \\

\noindent
Discussion: are there any good reasons for thinking that ``$L$" should be neither true nor false?

\section{The Revenge Liar}

Suppose we adopt the above solution and believe that there is a third option in addition to being true and being false: being \textit{meaningless}. When a sentence is meaningless, it is neither true nor false. The liar sentence $L$ is an example of a meaningless sentence. Is the paradox solved? \\

\noindent
Consider the following sentence:


$$ L^*: \, \, ``L^*" \, \, \text{is either meaningless or false}.$$

\noindent
A new paradox is generated: since there are three options in total, $``L^*"$ is either true, or either meaningless or false. Assume that $``L^*"$ is true, then it is neither meaningless nor false. But by T-schema, our assumption is equivalent to $L$, and hence is equivalent to that $``L^*"$ is either meaningless or false. But this is a contradiction, since we just concluded that $``L^*"$ is neither meaningless nor false. Hence $``L^*"$ is either meaningless or false. But this is just $L$. By T-schema again, $``L^*"$ is true, and hence is neither meaningless nor false. But this is again a contradiction, since we just concluded that it is either meaningless or false.


\section{Tarski's Solution}

In both of the above two paradoxes, we have a global truth/falsity predicate: our predicates ``is true"/``is false" can apply to any sentence in our language, including those that have truth/falsify predicates in them (for example, $L$ and $L^*$). \\

\noindent
Tarski believes that the existence of a global truth predicate that satisfies the T-schema is the real reason why we have paradoxes. Tarski distinguishes between the language \textit{for} which we wish to provide a truth predicate - \textit{the object language}, and the language \textit{in} which we discuss the truth predicate of the object language - \textit{the meta-language}. For Tarski, the truth predicate of language is always part of the meta-language but never a part of the language itself. \\

\noindent
Here's roughly what Tarski's story looks like: we begin with a language, $\mathfrak{L}_0$, that contains no truth predicate. We then ``step up" to an expanded language $\mathfrak{L}_1$, which contains a truth predicate $Tr_1$, that is only applicable to sentences of $\mathfrak{L}_0$ and satisfies the T-schema (for sentences in $\mathfrak{L}_0$). It can be shown that when this restriction on the applicability of $Tr_1$ is forced, we will have no paradox. Of course, we don't have to stop here. If we want to describe truth in $\mathfrak{L}_1$, we can simply step up again to a bigger language $\mathfrak{L}_2$, which contains a truth predicate $Tr^2$ that is only applicable to sentences of $\mathfrak{L}_1$, and so on and so forth. \\

\noindent
Discussion: are you satisfied with Tarski's hierarchy of truth?

\end{document}