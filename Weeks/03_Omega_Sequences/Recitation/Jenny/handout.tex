\documentclass[justified]{tufte-handout} 
\usepackage{amsfonts, amssymb, stmaryrd, fitch, natbib, qtree}
\usepackage{linguex, color, setspace, graphicx}
\usepackage{enumitem}
\usepackage{bussproofs}
\usepackage{turnstile}
\usepackage[super]{nth}
\thispagestyle{plain}
\definecolor{darkred}{rgb}{0.7,0,0.2}
\bibpunct{(}{)}{,}{a}{}{,}

\input xy
 \xyoption{all}

%New Symbols
\DeclareSymbolFont{symbolsC}{U}{txsyc}{m}{n}
\DeclareMathSymbol{\strictif}{\mathrel}{symbolsC}{74}
\DeclareMathSymbol{\boxright}{\mathrel}{symbolsC}{128}
\DeclareMathSymbol{\Diamondright}{\mathrel}{symbolsC}{132}
\DeclareMathSymbol{\Diamonddotright}{\mathrel}{symbolsC}{134}
\DeclareMathSymbol{\Diamonddot}{\mathord}{symbolsC}{144}
\renewcommand{\labelitemi}{$\triangleright$}
\renewcommand{\labelitemii}{$\circ$}
\renewcommand{\labelitemiii}{$\triangleright$}

%New commands
\newcommand{\bitem}{\begin{itemize}}
\newcommand{\eitem}{\end{itemize}}
\newcommand{\lang}{$\langle$}
\newcommand{\rang}{$\rangle$}
\newcommand{\back}{$\setminus$}
\newcommand{\HRule}{\rule{\linewidth}{0.1mm}}
\newcommand{\llm}[2][]{$\llbracket${#2}$\rrbracket^{#1}$}
\newcommand{\ul}{$\ulcorner$}
\newcommand{\ur}{$\urcorner\ $}
\newcommand{\urn}{$\urcorner$}
\newcommand{\sub}[1]{\textsubscript{#1}}
\newcommand{\sups}[1]{\textsuperscript{#1}}
\newtheorem{proposition}{\textbfb{Proposition}}[section]
\newtheorem{definition}[proposition]{\textbf{Definition}}
\newcommand{\bfw}{\begin{fullwidth}}
\newcommand{\efw}{\end{fullwidth}}

\begin{document}

\begin{fullwidth}
\noindent\LARGE Omega Sequences  \normalsize \\[.3cm]
\noindent  \textsc{24.118 Recitation Section $\bullet$ Matthias Jenny\\  {\texttt{\href{mailto:mjenny@mit.edu}{mjenny@mit.edu}}} $\bullet$ Office:  32-D927 $\bullet$ Hours: Thu 11:30-12:30} \hfill{October 24, 2014}
\noindent\HRule
\end{fullwidth}


\section{Pset 6, problem 1}

\noindent Is there a positive integer $n$ such that after $n$ seconds Lazy has reached point $B$?

\noindent \emph{Notes:}  \underline{\hspace{15.4cm}}\\\\\underline{\hspace{16.43cm}}\\\\\underline{\hspace{16.43cm}}\\\\\underline{\hspace{16.43cm}}\\

\section{Pset 6, problem 2}

\noindent In which direction will the red line be pointing at noon, after an infinite number of rotations?

\noindent \emph{Notes:}  \underline{\hspace{15.4cm}}\\\\\underline{\hspace{16.43cm}}\\\\\underline{\hspace{16.43cm}}\\\\\underline{\hspace{16.43cm}}\\

\section{Pset 6, problem 3}

\noindent What does the wheel look like at 2pm, after an infinite number of rotations?

\noindent \emph{Notes:}  \underline{\hspace{15.4cm}}\\\\\underline{\hspace{16.43cm}}\\\\\underline{\hspace{16.43cm}}\\\\\underline{\hspace{16.43cm}}\\

\section{Pset 6, problem 4}

\noindent Is there a difference between the way the wheel looks like at 2pm if you erase lines and the way the wheel in the previous exercise looked at 2pm?

\noindent \emph{Notes:}  \underline{\hspace{15.4cm}}\\\\\underline{\hspace{16.43cm}}\\\\\underline{\hspace{16.43cm}}\\\\\underline{\hspace{16.43cm}}\\\\\underline{\hspace{16.43cm}}\\

\section{Pset 6, problem 5}

\noindent How much money will you have at midnight if you have to burn the bill with the lowest number each time you receive bills?

\noindent \emph{Notes:}  \underline{\hspace{15.4cm}}\\\\\underline{\hspace{16.43cm}}\\\\\underline{\hspace{16.43cm}}\\\\\underline{\hspace{16.43cm}}\\\\\underline{\hspace{16.43cm}}\\

\section{Pset 6, problem 6}

\noindent How much money will you have at midnight if you give back a bill each time you receive bills?

\noindent \emph{Notes:}  \underline{\hspace{15.4cm}}\\\\\underline{\hspace{16.43cm}}\\\\\underline{\hspace{16.43cm}}\\\\\underline{\hspace{16.43cm}}\\\\\underline{\hspace{16.43cm}}\\

\end{document}
