\documentclass[12pt,letterpaper]{article}
\usepackage{../rset_2024}

%Josh: add back in the commented out questions 
%note that I have left out the Team A question involving a causal explanation...seems too fraught philosohically perhaps! 


%Questions and Answers
\qa{q} % a="answers only"; q ="questions only"; b="both"
\usepackage{../qa}


    % Given the set of integers $\Z$, suppose one were to identify the \textit{rational numbers} with the ordered pairs $\tuple{x,y}$ for $x,y \in \Z$ where $\tuple{x,y}$ aims to represent the fraction $\frac{x}{y}$. 
    % This definition has two problems:
    % (1) we cannot divide by $0$; and
    % (2) $\frac{1}{2} = \frac{2}{4}$ even though $\tuple{1,2} \neq \tuple{2,4}$.
    % In order to avoid these problems we may first 

\begin{document}

\psintro{Response Set 2: Type Theory}

\newpage

\vspace{.2in}

\fbox{\parbox{150mm}{
  Ramsey (1925, p.~29) writes, ``The principal mathematical methods which appear to require the Axiom of Reducibility are mathematical induction and Dedekindian section, the essential foundations of arithmetic and analysis respectively. Mr Russell has succeeded in dispensing with the axiom in the first case, but holds out no hope of a similar success in the second. Dedekindian section is thus left as an essentially unsound method, as has often been emphasized by Weyl, and ordinary analysis crumbles into dust. That these are its consequences is the second defect in the theory of \textit{Principia Mathematica}, and, to my mind, an absolutely conclusive proof that there is something wrong. For as I can neither accept the Axiom of Reducibility nor reject ordinary analysis, I cannot believe in a theory which presents me with no third possibility.''
}}
\vspace{.15in}

\fbox{\parbox{150mm}{
  In ``Mathematics and Logic'' (1946, p.~6), Weyl accuses (in English as opposed to his earlier German publications) Russell of having, ``cured the disease but, as shown by the Dedekind example, also imperiled the very life of the patient. Classical analysis, the mathematics of real variables as we know it and as it is applied in geometry and physics, has simply no use for a continuum of real numbers of different levels.'' 
}}

\begin{enumerate}

  \question{
  \item 
    Briefly describe the disease that Russel is trying to cure.
    What part of Russell's cure does Weyl take to imperil the patient?
    Explain what the problem is.
  }
  \answer{
  \item[\tt (A)] 
    Russell is trying to cure all of the ``reflexive'' paradoxes, including those that require type distinctions like Russell's Paradox, as well as those that require dividing the types into orders like the Liar Paradox, or Weyl's Herterological Paradox.

    Weyl takes the division of types into orders to imperil the patient by making mathematics (at least as commonly practiced) impossible.

    The problem is that mathematics often defines properties by quantifying over further properties, and yet Russell's ramified types explicitly prohibit any means of quantifying over all such properties.

    Prohibiting quantification over all properties of any order would require heavy revisions to mathematical practice, making the practices of mathematics next to impossible.
  }

  \question{
  \item 
    Use Dedekind's construction of the real numbers to explain what Weyl means by `different levels'.
    Why is this an unacceptable consequence for mathematics.
  }
  \answer{
  \item[\tt (A)]
    In considering the properties that a real number $x$ may have, we do not want to have to distinguish between higher-order \textit{being the greatest lower bound of a set $X$} and lower-order properties like \textit{being greater than a real number $y$}.
    Rather, we should like for these properties to be \textit{on a par} with each other, for this is how mathematics has proceeded.

    However, the definition of \textit{lower bound} quantifies over all real numbers: $x$ is a \textit{lower bound} of $X$ \textit{iff} $x < y$ for all $y \in X$.
    Similarly, the definition of \textit{greatest lower bound} quantifies over all lower bounds: $x$ is a \textit{greatest lower bound} of $X$ \textit{iff} $x$ is a lower bound of $X$ and for every lower bound $y$ of $X$, $y \leq x$.
    This makes \textit{greatest lower bound} a higher-order property than that of \textit{being greater than $0$}.
    Whereas the working mathematician is accustomed to quantifying over all properties that a real numbers might have at once, this is explicitly forbidden by the orders in Russell's ramified theory of types.
}

\item 
  \question{
    How does the Axiom of Reducibility avoid creating numbers of different levels?
    Why does Weyl reject the Axiom of Reducibility if it avoids dividing numbers into different levels?
    Do you find Weyl's rejection of the Axiom compelling?
  }
  \answer{thing}

\end{enumerate}

\fbox{\parbox{150mm}{
  In ``The Foundations of Mathematics'' (1925, p.~20), Ramsey divides the paradoxes with which Russell is concerned into two groups.
  % He then goes on to writes, ``The principal mathematical methods which appear to require the Axiom of Reducibility are mathematical induction and Dedekindian section, the essential foundations of arithmetic and analysis respectively. Mr Russell has succeeded in dispensing with the axiom in the first case, but holds out no hope of a similar success in the second. Dedekindian section is thus left as an essentially unsound method, as has often been emphasized by Weyl, and ordinary analysis crumbles into dust. That these are its consequences is the second defect in the theory of \textit{Principia Mathematica}, and, to my mind, an absolutely conclusive proof that there is something wrong. For as I can neither accept the Axiom of Reducibility nor reject ordinary analysis, I cannot believe in a theory which presents me with no third possibility'' (p.~29).
}}

\begin{enumerate}

\setcounter{enumi}{3}

\item 
  \question{
    Explain why Ramsey divides the paradoxes into the groups that he does.
    How does his division relate to Russell's original solution?
    Do you feel that Ramsey is right to distinguish between paradoxes, or do you agree with Russell that they should be handled together?
  }
  \answer{thing}

% \item 
%   \question{
%   }
%   \answer{thing}

\end{enumerate}

\end{document}

