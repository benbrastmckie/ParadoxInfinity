%%%%%%%%%%%%%%%%%%%%%%%%%%%%%%%%%%%%%%%%%
% Inzane Syllabus Template
% LaTeX Template
% Version 1.2 (8.22.2019)
%
% This template has been downloaded from:
% http://www.LaTeXTemplates.com
%
% Original author:
% Carmine Spagnuolo (cspagnuolo@unisa.it) with major modifications by 
% Zane Wolf (zwolf.mlxvi@gmail.com)
%
% I (Zane) have left a lot of instructions both in the .tex file and the .cls file that can guide you to customize this document to suite your tastes and requirements. Here is a brief guide: 
%  - Changing the Main Color: .cls line 39
%  - Adding more FAQs: .cls line 126 and .tex line 99
%  - Adding TA emails: uncomment .cls lines 220 & 224 and .tex lines 85 and 89
%  - Deleting the FAQ sidebar entirely: .tex line 188
%  - Removing the Lab/TA Info and placing a brief Overview/About section in their place:        uncomment .tex line 91 and .cls line 227, and comment .cls lines for the LAB/TA info        that you no longer want (c. lines 184-227)

% I am also happy to help with crafting/designing modifications to this template to help suite your personal needs in a syllabus. Feel free to reach out! 
%
% License:
% The MIT License (see included LICENSE file)
%
%%%%%%%%%%%%%%%%%%%%%%%%%%%%%%%%%%%%%%%%%

%----------------------------------------------------------------------------------------
%	PACKAGES AND OTHER DOCUMENT CONFIGURATIONS
%----------------------------------------------------------------------------------------

\documentclass[letterpaper]{infinity_syllabus} % a4paper for A4

\usepackage{booktabs, colortbl, xcolor}
\usepackage{tabularx}
\usepackage{enumitem}
\usepackage{ltablex} 
\usepackage{multirow}

\setlist{nolistsep}

\usepackage{lscape}
\newcolumntype{r}{>{\hsize=0.9\hsize}X}
\newcolumntype{w}{>{\hsize=0.6\hsize}X}
\newcolumntype{m}{>{\hsize=.9\hsize}X}

\renewcommand{\familydefault}{\sfdefault}
\renewcommand{\arraystretch}{2.0}
%----------------------------------------------------------------------------------------
%	 PERSONAL INFORMATION
%----------------------------------------------------------------------------------------

\profilepic{fish.jpg} % Profile picture, if the height of the picture is less than that of the cirle, it will have a flat bottom. 


% To remove any of the following, you need to comment/delete the lines in the .cls file (c. line 186). Commenting/deleting the lines below will produce an error. 

%To add different lines, you will need to create the new command, e.g. \profPhone, in the .cls file c. line 76, and command to create the line in the side bar in the .cls file c. line 186

\classname{Paradox and Infinity} 
\classnum{24.118 ~---~ Spring 2024} 

%%%%%%%%%%%%%%% PROF INFO
\profname{Benjamin Brast-McKie}
\officehours{Office Hrs: Mon \& Wed 11-12pm} 
\office{32-D966}
\siteA{\href{https://canvas.mit.edu/courses/22144}{Canvas Website}} 
\email{brastmck@mit.edu}

%%%%%%%%%%%%%%% COURSE INFO
\prereq{Recommended: 6.100A, 18.01}
\classdays{Lecture: Mon \& Wed}
\classhours{10am - 11am}
\classloc{32-141}

%%%%%%%%%%%%%%% TA INFO
\taAname{Bess Ann Rothman}
\recA{Recitation: Fri 10am}
\taAofficehours{Office Hrs: TBD} %Tues \& Thurs 10-11a
\taAoffice{26-142}
\TAemail{bessroth@mit.edu}

\taBname{Katie Zhou}
\recB{Recitation: Fri 11am}
\taBofficehours{Office Hrs: Th 2-3pm} %Tues \& Thurs 10-11a
\taBoffice{26-142}
\TBemail{katie\_z@mit.edu}

\taCname{Kenneth Nathaniel Black}
\recC{Recitation: Fri 12pm}
\taCofficehours{Office Hrs: Th 3-4pm} %Tues \& Thurs 10-11a
\taCoffice{26-142}
\TCemail{black199@mit.edu}

%%%%%%%%%%%%%%% PROBLEM SET INFO
\labdays{Due Fridays}
\labhours{5pm sharp}
\labloc{Online}


% \about{Fish make up the largest group of vertebrates on the planet, easily outnumbering mammals, marsupials, birds, and reptiles combined. Not only are they abundant, but they've diversified into an extraordinary array of sizes, shapes, lifestyles, and habitats. You can find them in the coldest, deepest parts of the ocean, and in the hottest freshwater ponds in the desert. This course will explore fish diversity and their biology. } 


%---------------------------------------------------------------------------------------
%	 FAQs
%----------------------------------------------------------------------------------------
%to add more questions or remove this section, go to the .cls file and start with lines comment
%lines 226-250. Also comment out this section as well as line 152(ish), the command \makeSide

\qOne{What is a Paradox?}
\aOne{
  Not any false or contradictory statement is a paradox since it also has to be interesting!
  An interesting paradox ought to teach us something we didn't know before, suggesting ways to revise our assumptions to avoid that contradiction.
}

\qTwo{But do the paradoxes have anything to show for themselves?}
\aTwo{
  To take one example, Russell's paradox played a critical role in driving the development of type theory and set theory, putting mathematics on a solid foundation (ZFC is accepted by most working mathematicians).
  }

\qThree{How much math will you need to know?}
\aThree{
  Elements of basic set theory will be introduced where some basic familiarity with logical notation will be assumed.
}

\qFour{What are the risks?}
\aFour{
  Some philosophers devote their careers to studying paradoxes.
  Some might consider this a risk!
}

% \qFive{Why care about paradoxes?}
% \aFive{Logic seeks to describe an ideal for reasoning. Of course, we are all engaged in reasoning. Learning logic is something akin to upgrading your firmware. It will literally change how you think.}

%----------------------------------------------------------------------------------------

\begin{document}

%----------------------------------------------------------------------------------------
%	 DESCRIPTION
%----------------------------------------------------------------------------------------

\makeprofile % Print the sidebar

%----------------------------------------------------------------------------------------
%	 OVERVIEW
%----------------------------------------------------------------------------------------
\section{Overview}

This course will cover a number of important paradoxes from some of the technical topics in philosophy. % including those which have inspired the development of the theories of sets, numbers, and types.
We will begin by focusing on the infinite and the theories that these paradoxes have inspired.
We will then move to consider paradoxes raised by the possibility of time travel as well as some decision theoretic paradoxes.

%----------------------------------------------------------------------------------------
%	 READING MATERIAL
%----------------------------------------------------------------------------------------
\vspace{0.5cm} %I make liberal use of the \vspace{} command to partition and place sections just how I want them. Alter as you see fit. 
\section{Material}

{\color{myCOLOR} Textbook}\\
\textit{On the Brink of Paradox} by Agust\'{i}n Rayo (2019)\\

{\color{myCOLOR} Other}\\
All of the other required readings will be provided on Canvas.
% I will also post supplemental readings written in a more philosophical vein.

%----------------------------------------------------------------------------------------
%	 GRADING SCHEME
%----------------------------------------------------------------------------------------
\vspace{0.5cm}
\section{Grading Scheme}

%below is the \twentyshort environment - a list with only two inputs. However, there is a \twenty environment, which creates a list with four inputs. You can find/alter details of that table in the .cls file c. lines 320. 
\begin{twentyshort}
	%\twentyitemshort{X\%}{Attendance/Participation}
  \twentyitemshort{20\%}{Problem Sets (4/5 assignments at \%5 each)}
  \twentyitemshort{20\%}{Reading Responses (4/5 assignments at \%5 each)}
  \twentyitemshort{20\%}{Short Essays (2 assignments at \%10 each)}
  \twentyitemshort{10\%}{Essay Final Exam (1 assignments at \%10)}
  \twentyitemshort{24\%}{Recitation Attendance (12/14 meetings at \%2 each)}
  \twentyitemshort{6\%}{Pop Quizzes (3/4 quizzes at \%2 each)}
\end{twentyshort}

% Grades for the course will be set as follows: 
Grades will not be curved.
A+ = 97-100; A = 93-96; A- = 90-92; B+ = 87-89;\\ 
B = 83-86; B- = 80-82; C+ = 77-79; C = 73-76; C- = 70-72; D = 60-69; F $<$ 60.

%----------------------------------------------------------------------------------------
%	 EXTRAS
%----------------------------------------------------------------------------------------

\vspace{0.5cm}
\section{Problem Sets}

There will be 5 problem sets where the lowest grade will be dropped.
Some weeks will split the problem set between a quiz due on Canvas before recitation and a written portion to be submitted as a PDF on Canvas after recitation.
You are welcome to work with at most two other students \textit{in preparation}, but everything you submit must be your own work.
You can find others to work with on \href{psetpartners.mit.edu}{\texttt{psetpartners.mit.edu}}.

\vspace{0.5cm}
\section{Reading Responses and Short Essays}

There will be 5 reading responses (1000 words) where the lowest grade is dropping. 
There will also be 2 short essays (1500 words).
You are encouraged to collaborate with at most two other students, talking through the topics and reading each other's work \textit{in preparation}, but all submitted work must be entirely your own.

\vspace{0.5cm}
\section{Recitations and Pop Quizzes}

There will be 14 recitations throughout the term.
You are expected to attend at least 12 of the 14 recitations, both preparing ahead of time and participating during.

There will also be 4 pop quizzes throughout the term where you can drop the lowest.
Consider these easy points if you are up on your reading and present in class.


\vspace{0.5cm}
\section{Piazza}

Please enroll in Piazza where you can ask and answer questions:\\
\href{https://piazza.com/mit/spring2024/24118}{\texttt{https://piazza.com/mit/spring2024/24118}}.


%%%%%%%%%%%%%%%%%%%%%%%%%%%%%%%%%%%%%%%%%%%%%%%%%%%%%%%%%%%%%%%%%%%%%%%%%%%%%
%                SECOND PAGE
%%%%%%%%%%%%%%%%%%%%%%%%%%%%%%%%%%%%%%%%%%%%%%%%%%%%%%%%%%%%%%%%%%%%%%%%%%%%%

\newpage % Start a new page

\makeSide % Print the FAQ sidebar; To get rid of, simply comment out and uncomment \makeFullPage

% \makeFullPage

\vspace{0.5cm}
\section{Learning Objectives}

%use \begin{outline} or \begin{outline}[enumerate] to create a list with subitems. 
\begin{itemize}
  \item Understand what drives the sense of perplexity or clash of intuitions in each of the paradoxes that we consider though out the course.
  \item Evaluate the range of responses for each paradox as well as the advantages and disadvantages had by each response.
  \item Practice closely reading of a number of philosophical papers, presenting the puzzles, arguments, and lessons which they contain in your own words.
\end{itemize}

\vspace{0.5cm}
\section{Laptop Policy and Notes}

Laptops, phones, and other distracting devices are not permitted in the lecture.
Although it can be helpful to take down notes here are there, I will provide lecture notes after each class, so don't feel that you have to take down everything.

\vspace{0.5cm}
\section{Academic Integrity}

Blindly copying another's answer is cheating.
By contrast, you are encouraged to talk through the topics covered in this course step-by-step with a classmate or two where in doing so everyone involved comes to understand each part for themselves.
However, when it comes time to write up and submit the solutions, it is important that you do this for yourself without consulting others throughout the process.

There will be a number of written exercises throughout this course.
You are not permitted to use ChatGPT (or other LLMs), though these will be of little help anyhow.
Moreover, do not expect a high grade if what you turn in amounts to little more than a string of grammatical sentences that are loosely related in topic but otherwise fail to contribute to any kind of recognizable argument or analysis.

In writing philosophy papers and responses, you should cut straight to the chase, clearly stating what you will argue for while avoiding rambling introductions filled with platitudes or other loosely related facts.
At the same time, you should make all necessary assumptions/definitions/implications/etc. completely explicit so that your work can stand on its own and be interpreted without any guess work.

\vspace{0.5cm}
\section{Make-up Policy}

Extensions for late work will not be granted without the official support of S$^3$ in which case you can assume that you will have my full support.
Making arrangements IN ADVANCE of the due date is required except in particularly difficult circumstances.

\vspace{0.5cm}
\section{Diversity and Inclusivity Statement}

In all course-related activities and communications, you will be treated with respect.
I welcome individuals of all ages, backgrounds, cultures, beliefs, ethnicities, gender identities and expressions, national origins, religious affiliations, abilities, sexual orientations, and other visible and non-visible differences.
All members of this class are expected to help create a respectful, welcoming, and inclusive environment that can be enjoyed and shared by every member of the class.

\vspace{0.5cm}
\section{Accommodations for Students with Disabilities}

If you are a student with learning needs that require accommodation, please contact Disability and Access Services at \texttt{das-student@mit.edu} (or for assistive technology, \texttt{atic-staff@mit.edu}) as soon as possible, to make an appointment to discuss your needs and to obtain an accommodations letter.
Please also e-mail me as soon as possible to set up a time to discuss your learning needs.
As someone who has used these services in the past, you can assume that you will have my full support.


%%%%%%%%%%%%%%%%%%%%%%%%%%%%%%%%%%%%%%%%%%%%%%%%%%%%%%%%%%%%%%%%%%%%%%%%%%%%%
%                COURSE SCHEDULE
%%%%%%%%%%%%%%%%%%%%%%%%%%%%%%%%%%%%%%%%%%%%%%%%%%%%%%%%%%%%%%%%%%%%%%%%%%%%%
\newpage
\makeFullPage
\section{Class Schedule}
  % \vspace{.1in}

Note: All supplementary readings will be provided on Canvas.\\

\begin{center}
\begin{tabularx}{\textwidth}{p{2.5cm}p{7cm}p{10cm}} %change the width of the comments by changing these cm measurements. Add/substract columns by adding/deleting p{} sections. 
\arrayrulecolor{myCOLOR}\hline
%%%%%%%%%%%%%%%%%%%%%%%% MODULE 1 %%%%%%%%%%%%%%%%%%%%%%%%
\multicolumn{3}{l}{\textbf{\textcolor{myCOLOR}{\large Part 1: Cantor's Paradise}}} \\
\hline
% Week & Topic & Readings \\ \hline 
%%Alternatively, instead of Week #, you can do Class date for meeting

\textbf{(Week 1)} & Infinite Cardinalities & \textit{On the Brink of Paradox}, Ch.~1 \\
Feb 05, 07 &  & Problem Set 1 Due Friday 02/09\\
\arrayrulecolor{maingray}\hline

\textbf{(Week 2)} & The Higher Infinite & \textit{On the Brink of Paradox}, Ch.~2 \\
Feb 12, 14 &  & Problem Set 2 Due Friday 02/16  \\
\arrayrulecolor{maingray}\hline

\textbf{(Week 3)} & Omega Sequences & \textit{On the Brink of Paradox}, Ch.~3 \\
Feb 20, 21 &  & Problem Set 3 Due Friday 02/23 \\
\arrayrulecolor{maingray}\hline

~\\
\arrayrulecolor{maingray}\hline
%%%%%%%%%%%%%%%%%%%%%%%% MODULE 2 %%%%%%%%%%%%%%%%%%%%%%%%
\multicolumn{2}{l}{\textbf{\textcolor{myCOLOR}{\large Part 2: Paradox in Paradise}}} \\
\hline

% % %%% THE LIAR
% \textbf{(Week 4)} & The Liar Paradox & Selected readings \\
% Feb 26, 28 &  & Response 1 Due Sunday 03/03\\ 
% \arrayrulecolor{maingray}\hline
%
% \textbf{(Week 5)} & Russell's Paradox & Selected readings \\
% Mar 4, 6 &  & Response 2 Due Sunday 03/10\\
% \arrayrulecolor{maingray}\hline
%
% % \textbf{(Week 6)} & Self Reference & Selected readings \\
% % Mar 11, 13 &  & Response 3 Due Sunday 03/17\\
% % \arrayrulecolor{maingray}\hline
%
% \textbf{(Week 6)} & The Iterative Conception of Set & Selected readings \\
% Mar 11, 13 &  & Response 3 Due Sunday 03/17\\
% \arrayrulecolor{maingray}\hline
%
% \textbf{(Week 7)} & Absolute Generality & Selected readings \\
% Mar 18, 20 &  &  Essay 1 Due Friday 03/22\\
% \arrayrulecolor{maingray}\hline
%
% % \textbf{(Week 7)} & The Iterative Conception of Set & Selected readings \\
% % Mar 18, 20 &  &  Essay 1 Due Friday 03/22\\
% % \arrayrulecolor{maingray}\hline

%%% ABSOLUTE GENERALITY
\textbf{(Week 4)} & Self Reference & "Mathematical Logic as Based on the Theory of Types" (Russell) \\
Feb 26, 28 &  & Response 1 Due Sunday 03/03\\ 
\arrayrulecolor{maingray}\hline

  \textbf{(Week 5)} & A Theory of Types & "Mathematics and Logic" (Weyl) and $\S\S1-3$ of "The Foundations of Mathematics" (Ramsey) \\
Mar 4, 6 &  & Response 2 Due Sunday 03/10\\
\arrayrulecolor{maingray}\hline

\textbf{(Week 6)} & Iterative Conception of Set & "The iterative conception of set" (Boolos) \\
Mar 11, 13 &  & Response 3 Due Sunday 03/17\\
\arrayrulecolor{maingray}\hline

\textbf{(Week 7)} & Absolute Generality & "Speaking of Everything" (Cartwright) \\
Mar 18, 20 &  &  Essay 1 Due Friday 03/22\\
\arrayrulecolor{maingray}\hline

~\\
\arrayrulecolor{maingray}\hline
%%%%%%%%%%%%%%%%%%%%%%%% MODULE 3 %%%%%%%%%%%%%%%%%%%%%%%%
\multicolumn{2}{l}{\textbf{\textcolor{myCOLOR}{\large Part 3: Time Travel}}} \\
\hline

  \textbf{(Week 8)} & \mbox{Watch the film "Timecrimes" (2007) over break} &  \\
Mar 25, 27 & --- SPRING BREAK --- &  \\
\arrayrulecolor{maingray}\hline

\textbf{(Week 9)} & Time Travel & \textit{On the Brink of Paradox}, Ch.~4 and \hfill\strut\hspace{1.7in} "The Paradoxes of Time Travel" Lewis (1976) \\
Apr 1, 3 &  &  Problem Set 4 Due Friday 04/05\\
\arrayrulecolor{maingray}\hline

\textbf{(Week 10)} & The Metaphysics of Time & "The Unreality of Time" McTaggart (1908) \\
Apr 8, 10 &  & Response 4 Due Sunday 04/14\\
\arrayrulecolor{maingray}\hline

\textbf{(Week 11)} & Time and Change & "Time Without Change" Shoemaker (1969) \\
Apr 17 &  & Essay 2 Due Sunday 04/21\\
\arrayrulecolor{maingray}\hline

\pagebreak

~\\
\arrayrulecolor{maingray}\hline
%%%%%%%%%%%%%%%%%%%%%%%% MODULE 4 %%%%%%%%%%%%%%%%%%%%%%%%
\multicolumn{2}{l}{\textbf{\textcolor{myCOLOR}{\large Part 4: Newcomb's Problem}}} \\
\hline

\textbf{(Week 12)} & Newcomb's Problem & \textit{On the Brink of Paradox}, Ch.~5 \\
Apr 22, 24 &  & Problem Set 5 Due Friday 04/26 \\
\arrayrulecolor{maingray}\hline
 
\textbf{(Week 13)} & One-Boxing & "Counterfactuals and Newcomb's Problem" Horgan (1981) \\
Apr 29, May 1 &  & Response 5 Due Sunday 05/05 \\
\arrayrulecolor{maingray}\hline

\textbf{(Week 14)} & Two-Boxing & "Why Take Both Boxes" Spencer and Wells (2019) \\
May 6, 8 &  & No Assignment \\
\arrayrulecolor{maingray}\hline

\textbf{(Week 15)} & Prisoners' Dilemma & "Prisoners' Dilemma is a Newcomb Problem" Lewis (1979) \\
May 13 &  & Essay Exam in Finals Week \\
\arrayrulecolor{maingray}\hline

\end{tabularx}
\end{center}


%----------------------------------------------------------------------------------------

\end{document} 



