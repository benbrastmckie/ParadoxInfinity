\documentclass[a4paper, 11pt]{article} % Font size (can be 10pt, 11pt or 12pt) and paper size (remove a4paper for US letter paper)

\usepackage[protrusion=true,expansion=true]{microtype} % Better typography
\usepackage{graphicx} % Required for including pictures
\usepackage{wrapfig} % Allows in-line images
\usepackage[top=1.25in, bottom=1in, left=1.65in, right=1.65in]{geometry} %%Margins
\usepackage{mathpazo} % Use the Palatino font
\usepackage[T1]{fontenc} % Required for accented characters
\linespread{1.05} % Change line spacing here, Palatino benefits from a slight increase by default

\makeatletter
\renewcommand\@biblabel[1]{\textbf{#1.}} % Change the square brackets for each bibliography item from '[1]' to '1.'
\renewcommand{\@listI}{\itemsep=0pt} % Reduce the space between items in the itemize and enumerate environments and the bibliography

\renewcommand{\maketitle}{ % Customize the title - do not edit title and author name here, see the TITLE block below
\begin{flushright}
{\large\@author} % Author name
\\\@date % Date
\end{flushright}

\begin{flushleft} % Right align
{\Large\@title} % Increase the font size of the title
\end{flushleft}
}

%----------------------------------------------------------------------------------------
%	TITLE
%----------------------------------------------------------------------------------------

\title{\textbf{Letter of Recommendation}} % Subtitle

\author{April 19, 2024} % Author

\date{\it Benjamin Brast-McKie} % Date

%----------------------------------------------------------------------------------------

\begin{document}

\maketitle % Print the title section

\pagenumbering{gobble}
\vspace{0pt} % Some vertical space between the abstract and first section

\noindent
NOTES

- good setup of the essay question

  - unpacked meanings of terms in the question

  - engaged students, helping to articulate and unpack their answers

  - students seemed engaged and interested

  - helped to distinguish between defining and characterizing time, discouraging students from attempting to do the former

  - good pacing, giving students time to think, while also presenting material and raising questions

  - did a good job engaging with students questions and answers. question about position being an intrinsic property/change was tricky, and well handled in the discussion. follow up about property of a multiple body system.

  - covered arguments that support essay claim that there cannot be time without change. 

  - called all the students by name

  - got the students to put up the shoemaker example in detail, discussing it's implications and how compelling the argument is. discussed theory selection in general, and what is at stake in the shoemaker example.

  - encouraged and drew out student reactions, contrasting plausibility considerations with simplicity.

  - worry about A-theory skepticism about change. answered that this is not something shoemaker is worried about, clarifying the dialectic between these two accounts.

  - 

















\end{document}
