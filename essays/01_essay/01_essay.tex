\documentclass[12pt,letterpaper]{article}
\usepackage{../../assets/styles/essay_2024}
\usepackage{bibentry}
\usepackage[round]{natbib} %%Or change 'round' to 'square' for square backers

%Josh: add back in the commented out questions 
%note that I have left out the Team A question involving a causal explanation...seems too fraught philosohically perhaps! 


%Questions and Answers
\qa{q} % a="answers only"; q ="questions only"; b="both"
\usepackage{../../assets/styles/qa}


    % Given the set of integers $\Z$, suppose one were to identify the \textit{rational numbers} with the ordered pairs $\tuple{x,y}$ for $x,y \in \Z$ where $\tuple{x,y}$ aims to represent the fraction $\frac{x}{y}$. 
    % This definition has two problems:
    % (1) we cannot divide by $0$; and
    % (2) $\frac{1}{2} = \frac{2}{4}$ even though $\tuple{1,2} \neq \tuple{2,4}$.
    % In order to avoid these problems we may first 

\begin{document}

\psintro{Essay \#1: Russell's Paradox}


\vspace{.2in}


Choose one of the following two approaches:

\begin{enumerate}

  \question{
  \item 
    Write a single essay that presents Russell's paradox, your favored solution to the paradox, objections to that solution, and your responses and reflections on those objections.
    Your essay may be structured in any manner that clearly accommodates these elements, giving due diligence to each.
  }
  \answer{
  \item[\tt (A)] 
    answer
  }

  \question{
  \item 
    Write three mini essays where: the first presents Russell's paradox along with a solution to the paradox; the second presents objections to that solution along with an alternative solution; and the third presents objections to the alternative solution, comparing these solutions overall.
  }
  \answer{
  \item[\tt (A)] 
    answer
  }

\end{enumerate}

\vfill
\bibliographystyle{Phil_Review} %%bib style found in bst folder, in bibtex folder, in texmf folder.
\nobibliography{Zotero} %%bib database found in bib folder, in bibtex folder
\end{document}

