
\documentclass[12pt,a4paper]{article}
\usepackage{pset-2023}


%Questions and Answers
\qa{q} % a="answers only"; q ="questions only"; b="both"
\usepackage{qa}


\begin{document}

\psintro{Problem Sets 7 and 8: Non-Measurable Sets}

%%%%%%%%%%%%%%%%%%%%%%%%%%%

\question{
\subsection*{Preliminaries} 

The line segment $[a,b]$ is the set of real numbers $x$ such that $a \leq x \leq b$. The Borel Sets are the members of the smallest set $\mathscr{B}$ such that: $(i)$ every line segment is in $\mathscr{B}$, $(ii)$ if a set $A$ is in $\mathscr{B}$, then so is $\mathbb{R} - A$, and $(iii)$ if a countable family of sets is in $\mathscr{B}$, then so is its union.

The Lebesgue Measure, $\lambda$, is the unique function on the Borel Sets that satisfies these three conditions:
\begin{description}
\item[Length on Segments] $\lambda([a,b]) = b-a$ for every $a,b \in \mathbb{R}$.\label{gloss:lls}


\item[Countable Additivity]
\[\lambda\left(\bigcup\{A_1,  A_2 , A_3,\ldots\}\right) = \lambda(A_1) + \lambda(A_2) + \lambda(A_3) + \ldots\] whenever $A_1,A_2,\dots$ is a countable family of disjoint sets for each of which $\lambda$ is defined.


\item[Non-Negativity]
$\lambda(A)$ is either a non-negative real number or the infinite value $\infty$, for any set $A$ in the domain of $\lambda$.

\end{description}
}


\subsection*{Part I} 



\begin{enumerate}

\item \question{Of each of the following sets of real numbers, determine whether it is a Borel Set. If it is a Borel Set, specify its Lebesgue Measure. (You may assume that every Borel Set has a Lebesgue Measure.) (2 points each)}

\begin{enumerate}
\item \question{$\left[\frac{1}{4},\frac{1}{3}\right]$}

\answer{Borel set.  By Length on Segments, $\lambda{\left[\frac{1}{4},\frac{1}{3}\right]} = \frac{1}{3} - \frac{1}{4} = \frac{4}{12} - \frac{3}{12} = \frac{1}{12}$.}

\item  \question{$\{\frac{1}{3}\}$}

\answer{Borel set. By Length on Segments $\lambda\{\frac{1}{3}\} = \lambda[\frac{1}{3}-\frac{1}{3}] = \frac{1}{3} - \frac{1}{3} = 0$. }


\item  \question{$\left[\frac{1}{4},\frac{1}{3}\right)$}

\answer{Borel set. By Additivity, $\lambda{\left[\frac{1}{4},\frac{1}{3}\right]} = \lambda{\left[\frac{1}{4},\frac{1}{3}\right) + \lambda\left\{\frac{1}{3}\right\}}$. So it follows from the previous exercises that $\lambda{\left[\frac{1}{4},\frac{1}{3}\right)} =  \frac{1}{12}$.}


\item  \question{$\left[\frac{1}{4},\frac{5}{6}\right] - [\frac{1}{3},\frac{1}{2}]$}

\answer{
Borel set. By Additivity, $\lambda\left[\frac{1}{4},\frac{5}{6}\right] = \lambda\left(\left[\frac{1}{4},\frac{5}{6}\right] - [\frac{1}{3},\frac{1}{2}]\right) + \lambda [\frac{1}{3},\frac{1}{2}]$. By Length on Segments, $\lambda[\frac{1}{4},\frac{5}{6}] = \frac{5}{6} - \frac{1}{4} = \frac{10}{12} - \frac{3}{12} = \frac{7}{12}$ and $\lambda[\frac{1}{3},\frac{1}{2}] = \frac{1}{2} - \frac{1}{3} = \frac{1}{6}$. So $\lambda\left(\left[\frac{1}{4},\frac{5}{6}\right] - [\frac{1}{3},\frac{1}{2}]\right) = \frac{7}{12} - \frac{1}{6} = \frac{5}{12}$.
}

\item  \question{$\left[\frac{1}{4},\frac{5}{6}\right] - \{\frac{1}{3},\frac{1}{2}\}$}

\answer{Borel set. By Additivity, $\lambda{\left[\frac{1}{4},\frac{5}{6}\right]} = \lambda\left(\left[\frac{1}{4},\frac{5}{6}\right] - \{\frac{1}{3},\frac{1}{2}\}\right) + \lambda\{\frac{1}{3},\frac{1}{2}\}$. By Length on Segments, $\lambda[\frac{1}{4},\frac{5}{6}] = \frac{5}{6} - \frac{1}{4} = \frac{10}{12} - \frac{3}{12} = \frac{7}{12}$  and $ \lambda\{\frac{1}{3},\frac{1}{2}\} = 0$. So $ \lambda\left(\left[\frac{1}{4},\frac{5}{6}\right] - \{\frac{1}{3},\frac{1}{2}\}\right) = \frac{7}{12} - 0 = \frac{7}{12}$.}


\item  \question{$\left\{0, 1, 2, \dots \right\}$}

\answer{
Borel set. Since $\left\{0,1,2,\dots \right\}$ is countable, Additivity entails $\lambda(\left\{0,1,2,\dots \right\}) = \lambda(\{0\}) + \lambda(\{1\}) + \lambda(\{2\}) + \dots$. By Length on Segments $\lambda(\{n\}) = 0$. So $\lambda(\left\{0,1,2,\dots \right\}) = 0$.
}

\item  \question{$\left[\frac{4}{5},\frac{5}{6}\right] \cup \left\{\frac{1}{2^1}, \frac{1}{2^2}, \frac{1}{2^3}, \dots \right\}$}

\answer{
Borel set. $\lambda\left(\left[\frac{4}{5},\frac{5}{6}\right] \cup \left\{\frac{1}{2^1}, \frac{1}{2^2}, \frac{1}{2^3}, \dots \right\}\right) = \lambda\left[\frac{4}{5},\frac{5}{6}\right] + \lambda\left\{\frac{1}{2^1}, \frac{1}{2^2}, \frac{1}{2^3}, \dots \right\} = \left(\frac{5}{6}-\frac{4}{5}\right) + 0 = \frac{1}{30}$.
}


\item  \question{$\left[0,\frac{1}{2}\right] - \left\{\frac{1}{2^1}, \frac{1}{2^2}, \frac{1}{2^3}, \dots \right\}$}

\answer{
Borel set. Since $\lambda\left(\left[0,\frac{1}{2}\right]\right) = \lambda\left(\left[0,\frac{1}{2}\right] - \left\{\frac{1}{2^1}, \frac{1}{2^2}, \frac{1}{2^3}, \dots \right\}\right) + \lambda\left\{\frac{1}{2^1}, \frac{1}{2^2}, \frac{1}{2^3}, \dots \right\}$, we have $\lambda\left(\left[0,\frac{1}{2}\right] - \left\{\frac{1}{2^1}, \frac{1}{2^2}, \frac{1}{2^3}, \dots \right\}\right) = \frac{1}{2} - 0 = \frac{1}{2}$.
}


\item  \question{a Vitali set}

\answer{Not a Borel set. This can be verified by noting that all Borel sets have Lebesgue measure, but Vitali sets do not.}

\item  \question{the complement of a Vitali set}

\answer{Not a Borel set. If it were a Borel set, its complement would be a Borel set too, and we know that it isn't.}

\end{enumerate}

\question{
(As usual, $[a,b] = \{x : a \leq x \leq b\}$ and $[a,b) = \{x : a \leq x < b\}$.)
}


\end{enumerate}





\subsection*{Part II} 

\begin{enumerate}
  \setcounter{enumi}{1}
  
  
  
\item \question{\emph{Choosing Socks, Choosing Shoes}

Recall that a \textbf{choice set} for set $A$ is a set containing exactly one element from each member of $A$. And recall the Axiom of Choice:

\begin{description}
\item[Axiom of Choice]
Any set of non-empty, non-overlapping sets has a choice set.
\end{description}


On a first reading, the Axiom of Choice is likely to sound trivial. This exercise is aimed at helping you understand why it is not. (It is a variant of an explanation given long ago by British philosopher Bertrand Russell.)

Here are two standard set-theoretic axioms:

\begin{description}

\item[Union]
If a set $A$ exists, then so does its union, $\bigcup A$. 

(Recall that $\bigcup A$ is the set $\{x : \text{$x$ is a member of some element of $A$}\}$, i.e.~the set of members of members of $A$.)


\item[Separation]
Let $\phi(x)$ be any formula of the form ``$x$ is such and such'' (for instance, ``$x$ is a natural number''). Then if set $A$ exists, the following set also exists: $\{x : x \in A \text{ and } \phi(x)\}$ (i.e.~the set of objects that are members of $A$ and satisfy condition $\phi(x)$).


\end{description}
Here is an example of how Separation might be used. Let $\phi(x)$ be the formula ``$x$ is female''. Separation entails that if the set $O = \{x : \text{$x$ is an octopus}\}$ exists, so does 
\[\{x : x \in O \text{ and $x$ is female}\}\]
which is the set of female octopuses.
}

\begin{enumerate}

\item \question{
Let $S$ be an infinite set, each member of which is a set of two shoes: a right shoe and a left shoe. Assume that no two elements of $S$ have any shoes in common. Now suppose you'd like to have a choice set for $S$. The Axiom of Choice guarantees that a choice set exists, but it doesn't give you much information about what it looks like. 

Let's see if we can do better than that. It follows from Union that $\bigcup S$ exists. Is there an application of Separation to $\bigcup S$ that delivers a choice set for $S$? If so, sketch a proof. If not, explain informally why not. (5 points)} \label{shoes} 

\answer{Yes. Separation entails the existence of $$\{x : x \in \bigcup S \text{ and $x$ is a right shoe}\}$$ which is a choice set for $S$.}

\item \question{
Let $S$ be an infinite set, each member of which is a set of two socks. We will assume that all socks are alike, and, in particular, that there is no such thing as a ``right'' sock or a ``left'' sock. (We will also assume, unrealistically, that socks are not located in space and therefore that socks cannot be distinguished by their spatial locations.) Assume that no two elements of $S$ have any socks in common. Now suppose you'd like to have a choice set for $S$. The Axiom of Choice guarantees that a choice set exists, but it doesn't give you much information about what it looks like. 

Let's see if we can do better than that. It follows from Union that $\bigcup S$ exists. Is there an application of Separation to $\bigcup S$ that delivers a choice set for $S$? If so, sketch a proof. If not, explain informally why not. \label{socks} (5 points)
}

\answer{No. We can't use Separation to get a choice set because we are not able to articulate a feature of socks that applies to exactly one element from each pair, so we are not able to come up with a suitable choice for $\phi(x)$.}

\end{enumerate}



\item   \question{\emph{Back to Bacon} \label{bacon}

This problem is about Bacon's Puzzle, which is discussed in Section 3.4 of the textbook. (Before reading further, you might consider having a look at the text, to refresh your memory of the puzzle.)

Towards the end of the discussion in the book, I write:

\begin{quote}
What is the probability that an individual who follows the strategy will answer correctly? I don't know the answer to this question but I suspect that when one follows the strategy one's probability of success is best thought of ill-defined. (Section 3.4.9)
\end{quote}
Throughout this problem, I will explain why I harbor such suspicions. 


Let $S$ be the set of all functions from $\mathbb{N}$ to $\{0,1\}.$  We partition $S$ into orbits, as follows: for any $f, g \in S$, $f$ is in the {same orbit} as $g$ if and only if there are at most finitely many numbers $k$ such that $f(k) \neq g(k)$. %Since the relation of ``being in the same orbit'' is an equivalence relation, the orbits {partition} $S$. In other words: $S$ is the union of the orbits, and the intersection of any two orbits is empty.
}


\begin{enumerate}





\item \label{pretrick} \question{Let $f_0(n) = 0$ for each $n \in \mathbb{N}$, and let $O_0$ be $f_0$'s orbit. Describe a bijection $\mu_0$ from $\mathbb{N}$ to $O_0$. (5 points)}

\answer{

Let $\mu_0(n)$ be the function $f \in O_0$ such that the sequence of digits $f(0),f(1),f(2),\ldots$ corresponds to $n$'s binary notation, written backwards.


}

\item \question{Given an arbitrary orbit $O$ and a function $f^* \in O$, describe a bijection $\mu$ from $\mathbb{N}$ to $O$.  (5 points)} \label{trick}

\answer{We can use a simple variant of the procedure we used in the previous exercise. Let $\mu(n)$ be the function $f \in O$ such that the sequence of digits $|f^*(0)-f(0)|, |f^*(1)-f(1)|, |f^*(2)-f(2)|, \ldots$ corresponds to $n$'s binary notation, written backwards.}





\end{enumerate}
\question{
The lesson of problem~(\ref{trick}) is that any representative from a given orbit can be used to define a well-ordering of that orbit. 

Let us now consider the problem of how one might go about choosing a representative from each orbit. Ask yourself: is the set of orbits analogous to the set of pairs of shoes of problem~(\ref{shoes}), or is it analogous to the set of pairs of socks of problem~(\ref{socks})? In other words: is there a formula $\phi(x)$ such that an application of Separation based on $\phi(x)$ could be used to specify a set that contains exactly one representative for each orbit? 

As it turns out, the answer to this question is ``no''. It is impossible to set forth an explicit rule that singles out exactly one representative from each orbit: the only way to show that a set of representatives exists is to use the Axiom of Choice.
}

\begin{enumerate}
  \setcounter{enumii}{2}


\item \label{extra}
\question{\emph{Extra Credit:} Show that one cannot prove that a set with exactly one representative from each orbit exists without using the Axiom of Choice. You may avail yourself of the following important result, due to Robert Solovay: one cannot prove that a non-measurable set exists without using Axiom of Choice. (5~points)


}

\answer{

We assume that a choice set for the set of orbits exists, and use this assumption to prove the existence of a set that is not Lebesgue measurable. Since one can only prove the existence of sets that are not Lebesgue measurable using the Axiom of Choice, this means that the Axiom of Choice must be required to show that a choice set for the set of orbits exists. 

We shall ignore the orbit containing functions ending in an infinite sequence of ones. (Note that if a choice set exists for the original set of orbits, it must also exist for the restricted set of orbits.) It is easy to verify that there is a bijection between the functions in the remaining orbits and the real numbers in $[0,1)$: simply assign each function to the real number whose binary notation is ``0.'' followed by the sequence of digits that corresponds to that function. (Because we are ignoring functions ending in an infinite sequence of ones, we don't have to worry about numbers with more than one binary expansion.) I will henceforth speak interchangeably of real numbers and the corresponding functions.

Let $O$ be an orbit and let $a,b \in O$. Assume with no loss of generality that $a \leq b$. We can now make two observations:

\begin{enumerate}
\item  $a$ and $b$ are both real numbers, so $b-a$ must also be a real number. Since $a$ and $b$'s binary expansions match after a certain point, the binary expansion of $b-a$ must consist entirely of zeroes after a certain point. (Such binary expansions are periodic, so $b-a$ must be a rational number.)

\item If $q$ is a rational number whose binary expansion consists entirely of zeroes after a certain point, then $a+q$ (or $(a+q) -1$, if $a+q \geq 1$) is a member of $O$.

\end{enumerate}
These two observations allow us to adapt the proof that Vitali sets are non-measurable to show that a choice set $C$ for the set of orbits must be non-measurable. Briefly, for each rational number $q$ whose binary expansion consists entirely of zeroes after a certain point, the set
$$C^q = \{x : \text{for some $c \in C$, $x = c+q$ (or $(c+q) - 1$, if $c+c \geq 1$)} \}$$
will count as a ``clone'' of $C$. It is then a consequence of Uniformity that $C$ and its ``clones'' must have the same Lebesgue measure, if they have Lebesgue measure at all. But there are countably many clones, so it follows from Countable Additivity that there is no single Lebesgue measure all clones could have.
}
\end{enumerate}
\question{
Back to Bacon's Puzzle.  The question we wish to consider is: what is the probability that an individual who follows the strategy will answer correctly? To fix ideas, let the individual in question be $P_0$ and assume that she has been given a blue hat. Let the function $f_\text{\at}$ represent the actual distribution of hats and let $O_\text{\at}$ be $f_\text{\at}$'s orbit. Then our question can be reformulated as follows: what is the probability that orbit $O_\text{\at}$ was assigned a representative $\rho$ such that $\rho(0) = 1$?

In fact, there is a natural way of answering this question, \emph{relative} to a well-ordering of $O_\text{\at}$. Let $\mu$ be a bijection from $\mathbb{N}$ to $O_\text{\at}$. Then $\mu$ can be used to characterize the following probability function: 
\[
\begin{array}{ccc}
p(Z) &=_{df} &\lim \limits_{n \to \infty}\dfrac{|Z\cap \{\mu(0),\mu(1),\dots, \mu(n)\}|}{|\{\mu(0),\mu(1),\dots \mu(n)\}|}\\
\end{array}
\]
(Here $Z$ is a subset of $O_\text{\at}$. If you'd like a refresher on this type of probability function, see Section~6.4.1.3 of the textbook.)

Let $X$ be the proposition that $O_\text{\at}$ was assigned a representative $\rho$ such that $\rho(0) = 1$. (Formally: $X = \{f \in O_\text{\at} : f(0) = 1\}$.) In the next couple of questions I'll ask you to calculate the value of $p(X)$ relative to different orderings.
}



\begin{enumerate}
  \setcounter{enumii}{3}



\item\label{first-ans}
 \question{
Suppose that $O_\text{\at}$ is orbit $O_0$ from problem (\ref{pretrick}) and that $\mu$ is the bijection $\mu_0$ you gave in your answer to (\ref{pretrick}).  What is the value of $p(X)$? (5~points)
}

\answer{
The answer depends on the student's choice of $\mu_0$ in their answer to (\ref{pretrick}). As $\mu_0$ is defined in my answer to (\ref{pretrick}) above, $[\mu_0(k)](0) = 1$ iff $k$ is odd. So 
$$
\begin{array}{ccl}
p(X) &=_{df} &\lim \limits_{n \to \infty}\dfrac{|\{\mu_0(1), \mu_0(3), \dots\}\cap \{\mu_0(0),\mu_0(1),\dots, \mu_0(n)\}|}{|\{\mu_0(0),\mu_0(1)_0,\dots \mu_0(n)\}|}\\
&=_{df} &\lim \limits_{n \to \infty}\dfrac{|\{\mu_0(1), \mu_0(3), \dots, \mu_0(n')\}|}{|\{\mu_0(0),\mu_0(1)_0,\dots \mu_0(n)\}|}\\
&=_{df} &0.5
\end{array}
$$
where $n' = n$ if $n$ is odd and $n-1$ otherwise.
}


\item\label{sec-ans}
 \question{
For a given integer $k \geq 2$, define a bijection $\mu$ from $\mathbb{N}$ to $O_0$ such that $p(X) = \frac{1}{k}$.
 (10~points)}

\answer{
Start by defining a function $\mu^X$, which assigns each $n \in \mathbb{N}$ to some function in $X$. For instance, $\mu^X(n)$ might be the function $f \in X$ such that the sequence of digits $f(1),f(2),f(3),\ldots$ corresponds to $n$'s binary notation, written backwards. 

Next, define a function $\mu^{\overline{X}}$, which assigns each $n \in \mathbb{N}$ to some function in $O_0 - X$. For instance, $\mu^{\overline{X}}(n)$ might be the function $f \in O_0 - X$ such that the sequence of digits $f(1),f(2),f(3),\ldots$ corresponds to $n$'s binary notation, written backwards. 

We can now define $\mu$ by starting with the values of $\mu^{\overline{X}}$,
$$\mu^{\overline{X}}(0), \mu^{\overline{X}}(1), \mu^{\overline{X}}(2), \mu^{\overline{X}}(3), \mu^{\overline{X}}(4), \mu^{\overline{X}}(5)\dots$$
and interspersing the values of $\mu^{X}$ at intervals of length $k-1$. For example, when $k = 3$, we get:
$$\mu^X(0),\mu^{\overline{X}}(0), \mu^{\overline{X}}(1), \mu^X(1), \mu^{\overline{X}}(2), \mu^{\overline{X}}(3), \mu^X(2),\mu^{\overline{X}}(4), \mu^{\overline{X}}(5),\mu^X(3),\dots$$
We then define $\mu(n)$ as the $n$th member of the resulting sequence.

The formal details follow, but all that is required for full credit is a grasp of the basic idea.
$$
\mu(n) = 
\begin{cases}
\mu^X\left(\frac{n}{k}\right), \text{ if $n = 0$ modulo $k$}\\
\mu^{\overline{X}}\left(\frac{(k-1)(n-1)}{k}\right), \text{ if $n = 1$ modulo $k$}\\
\mu^{\overline{X}}\left(\frac{(k-1)(n-2)}{k}+1\right), \text{ if $n = 2$ modulo $k$}\\
\mu^{\overline{X}}\left(\frac{(k-1)(n-3)}{k}+2\right), \text{ if $n = 3$ modulo $k$}\\
\vdots\\
\mu^{\overline{X}}\left(\frac{(k-1)(n-(k-1))}{k}+k-2\right), \text{ if $n = k-1$ modulo $k$}
\end{cases}
$$
}


\end{enumerate}
\question{As problem~(\ref{sec-ans}) suggests, you can get $p(X)$ to have \emph{any value you want}, by picking a sufficiently devious $\mu$. So the probability function $p(\dots)$ can only be assumed to assign a sensible probability to proposition $X$ if it is defined using a sensible choice of $\mu$.

When it comes to particular orbits, you may well think that there are choices of $\mu$ that stand out as particularly natural. Perhaps you think that when it comes to the specific orbit $O_0$ of problem (\ref{pretrick}), the choice of $\mu_0$ that you supplied in your answer is a particularly natural way of ordering $O_0$. (Maybe it even delivers the comforting result that  $p(X)=0.5$.)

But what about the general case? Is there a \emph{general recipe} that can be used to specify a natural ordering of each of our uncountably many orbits. Unfortunately, the answer is ``no'':

}


\begin{enumerate}
  \setcounter{enumii}{5}

\item \question{Show that it is impossible to explicitly characterize a relation $<$ such that each orbit $O$ is well-ordered by $<$. You may make use of problem (\ref{extra}). (10 points)

}

\answer{Suppose one can explicitly characterize $<$. Then one can use $<$ to specify the ``smallest'' element of each orbit: the $<$-smallest element of each orbit is the unique $f$ in each orbit such that no element of the orbit bears $<$ to $f$. But then one could use an application of Separation based on the formula ``$x$ is the $<$-smallest element of its orbit'' to specify a set that contains exactly one representative from each orbit. And it follows from problem (\ref{extra}) that this is impossible.

}




\end{enumerate}
\question{In the absence of a recipe for specifying a natural ordering for each orbit, I have no idea how one might go about characterizing sensible probability functions over the members of our orbits. \emph{That}'s why I suspect that the probability of success in Bacon's Puzzle, given that one follows the strategy, is not, in general, well-defined. 

}



% COMMENTED AWAY FOR VARIETY
\com{
\item \question{What would go wrong if we just stipulated that a Vitali set has measure $1/3$? (10~points)}

\answer{As the proof of the Vitali theorem shows, one would have to give up on the Axiom of Choice, or on at least one of the following:

\begin{description}
\item[Length on Line Segments]
$\lambda([a,b]) = b-a$.

\item[Countable Additivity]
Let $A_1,A_2,\dots$ be a countable family of disjoint sets. Whenever $\lambda(A_i)$ is defined for each $A_i$, we have:
\[
\lambda\left(\bigcup\{A_1,  A_2 , A_3,\ldots\}\right) = \lambda(A_1) + \lambda(A_2) + \lambda(A_3) + \ldots
\]

\item[Real or Infinite Values]
For any set $A$ in the domain of $\lambda$, $\lambda(A)$ is either a non-negative real number, or the infinite value $\infty$.
\end{description}
}
}

% COMMENTED AWAY FOR VARIETY -- AND ALSO TO TREAT THE BANACH-TARSKI CHAPTER AS A FREEBE.
\com{
\item\question{
Let $D$ be a disc of radius 1 with center $d$. (In other words: $D$ is the set of points at a distance no greater than 1 from $d$). Let $D^-$ be the result of removing two points from $D$. (Assume that the points are in the interior of $D$. In other words: the distance from the center of $D$ to each point is strictly smaller than 1.)  Is it possible to decompose $D^-$ into two distinct parts and reassemble the parts (without changing their sizes or shapes) so as to get $D$? If so, explain how to do it. If not, explain why not. (10 points)

\emph{Hint:} You may assume that for any points $p_1$  and $p_2$ in the interior of $D$, there is a circle $C_r$ of radius $r$ which goes through $p_1$ and $p_2$ and is entirely within $D$. (This is a consequence of Apolonius's Problem.) You may assume, moreover, that the circumference of $C_r$ is irrational with respect to the length of $\delta^{p_1}_{p_2}$, which is the shortest segment of $C_r$ with endpoints $p_1$ and $p_2$. (In other words, you may assume that there is no rational number $q$ such that the circumference of $C_r$ is $q \cdot \delta^{p_1}_{p_2}$.)
}


\answer{It follows from our assumption that there must be a circle $C^*$ that goes through $p_1$ and $p_2$, is entirely within $D$, and is such that the distance between $p_1$ and $p_2$ on $C^*$ is irrational with respect to the circumference of $C^*$.  Once a suitable $C^*$ is in place, we can solve the problem by using a procedure analogous to the procedure in Warm-Up Case~2 in the chapter on the Banach-Tarski Theorem.}

}





\item \question{
\emph{The Square of Evil}\footnote{The construction is due to the Polish mathematician Wac\l aw Sierpi\'nski. I learned about it in Frank Arntzenius, Adam Elga and John Hawthorne's ``Bayesianism, Infinite Decisions, and Binding".}

Say that a \textbf{countable ordinal} is an ordinal with countably many members, and let $\aleph_1$ be the set of all countable ordinals. $\aleph_1$ is itself an ordinal. From this it follows that $\aleph_1$ must have uncountably many members. (For suppose otherwise, then $\aleph_1$ is a countable ordinal, and therefore a member of itself. But no ordinal is a member of itself.)

Think of the \textbf{Continuum Hypothesis} as the claim that $\aleph_1$ has the same cardinality as [0,1], and therefore that there is a bijection $f$ from [0,1] to $\aleph_1$. Assume that the Continuum Hypothesis is true, and define the following ordering $<^e$ of [0,1]:
\[
\text{for any $a,b \in [0,1]$, $a <^e b$ if and only if $f(a) \in f(b)$}
\]
Since the ordinals in $\aleph_1$ are well-ordered by $\in$, it is an immediate consequence of this definition that $<^e$ is a well-ordering of [0,1].
}



\begin{enumerate}


\item \question{Show that $<^e$ has the following additional property: for each $x \in [0,1]$, there are at most countably many $y \in [0,1]$ such that $y <^e x$.\label{special} (5 points)}

\answer{
We know from the definition of $<^e$ that $y <^e x$ if and only if $f(y) \in f(x)$. But $f(x)$ is a member of $\aleph_1$ which is the set of countable ordinals. So $f(x)$ has at most countably many members. So there are at most countably many $f(y) \in \aleph_1$ such that $f(y) \in f(x)$. So there are at most countably many $y \in [0,1]$ such that $y <^e x$.
}

\end{enumerate}



\question{
We will now use $<^e$ to color the unit square [0,1]$\times$[0,1], using the following criterion:
\begin{quote}
For each point $\seq{x,y} \in [0,1]\times [0,1]$, color $\seq{x,y}$ white if $x <^e y$, and black otherwise.
\end{quote} I will refer to the colored square as the \textbf{Square of Evil}. Now let $\seq{x_0,y_0}$ be a particular point on the Square of Evil:
}
\begin{enumerate}
\setcounter{enumii}{1}

\item \question{How many white points are there in the row $\{\seq{z,y_0} : z \in [0,1]\}$? (5 points)}

\answer{
There are at most countably many. For a point $\seq{x,y_0}$ in row $\{\seq{z,y_0} : z \in [0,1]\}$ is white if and only if $x <^e y_0$, and it follows from question~\ref{special} that there are at most countable many $x \in [0,1]$ such that $x <^e y_0$.

}

\item  \question{How many white points are there in the column $\{\seq{x_0,z} : z \in [0,1]\}$? (5 points)} 

\answer{There are uncountably many. Since every point is either black or white, we can prove this by verifying that there are at most countably many black points in column $\{\seq{x_0,z} : z \in [0,1]\}$. This is easily done: a point $\seq{x_0,y}$ in column $\{\seq{x_0,z} : z \in [0,1]\}$ is black if and only if $y \leq^e x_0$, and it follows from question~\ref{special} that there are at most countably many $z \in [0,1]$ such that $z \leq^e x_0$.

}


\end{enumerate}
\question{Suppose that a point is selected at random from the Square of Evil. (It is selected by twice applying the Standard Coin Toss Procedure of Section~7.1.4.1 of the reading materials, once to pick an $x$ coordinate, and once to pick a $y$ coordinate.)}


\begin{enumerate}
\setcounter{enumii}{3}


\item \question{Someone tells you which row the selected point is in, and asks you to bet on whether the selected point is black or white. How should you bet?  (5 points)} \label{black}

\answer{Here is one good answer: ``You should bet on black, since almost every point on that row is black." Another good answer would show awareness that contradiction looms, and formulate some attempt to get out of it.}

\item \question{Someone tells you which column the selected point is in, and asks you to bet on whether the selected point is black or white. How should you bet?  (5 points)} \label{white}

\answer{Here is one good answer: ``You should bet on white, since almost every point on that column is white." Another good answer would show awareness that contradiction looms, and formulate some attempt to get out of it.}

\end{enumerate}


\question{It will rain or snow, but you don't know which. If it rains, you should wear outfit $A$ rather than outfit $B$. If it snows, you should also wear outfit $A$ rather than outfit $B$. So you should wear outfit $A$!

Here is a generalization of that seemingly attractive idea:

\begin{description}

\item[Dominance] Let $\Pi$ be a set of possible states of the world over which you have no control. You know that exactly one member of $\Pi$ obtains, but you don't know which.  Suppose you have two options, $A$ and $B$. Suppose, moreover, that for each $\pi \in \Pi$ you should choose $A$ over $B$ on the assumption that $\pi$ obtains. Then you should choose $A$ over $B$.




\end{description}
}
\begin{enumerate}
\setcounter{enumii}{5}
\item
\question{Use the Square of Evil to show that Dominance is false. (10 points)
}

\answer{

Suppose Dominance is true, as stated. Let $\Pi$ consist of states of the world [the row of the selected point is $r$] for $r \in [0,1]$. (\ref{black}) tells us that, for each $\pi \in \Pi$, you should choose ``bet black'' over ``bet white'', on the assumption that $\pi$ obtains. So, by Dominance, you should choose ``bet black'' over ``bet white''.

But an analogous argument, based on (\ref{white}) yields the conclusion that you should choose ``bet white'' over ``bet black''. But it can't be the case that you should both choose black and choose white. So Dominance must be mistaken.




}

\end{enumerate}


\end{enumerate} 



\end{document}




\vspace{7mm}
\begin{center}
\begin{tikzpicture}
\draw (0,0) circle (3cm); % B circle
\node at (3,2) {$B$}; % B label

\draw (1,0.5) circle (1cm); % main circle
\draw[thick] (1,1.5) circle (0.02cm); % p1 circle
\node at (1,1.8) {$p_1$}; % p1 label
\draw[thick] (1,-0.5) circle (0.02cm); % p2 circle
\node at (1,-0.8) {$p_2$}; % p2 label

\draw[dotted] (0,0.5) -- (2,0.5);

\end{tikzpicture}
\end{center}




