
\documentclass[12pt,letterpaper]{article}
\usepackage{pset-2023}
\usepackage{mathabx}
%\usepackage{hyperref}
\usepackage{changepage}  %for adjustwidth environment



%Questions and Answers
\qa{q} % a="answers only"; q ="questions only"; b="both"
\usepackage{qa}


\begin{document}

\psintro{PSets 7\&8: (Non)-Measurable Sets and the Axiom of Choice}

%gameplan for bacon problem: give the fN in part (a), so that part (d) involves a fixed function (will aid grading!). ask them part (b), d, e, and f. 
%part e seems hard so might want to give a hint

%ultimate plan: for part e, give them an example of the fN and ask them to explain its significance, e.g. that we can get the probability to be anything we want! so can make the `answer' the expository text that rayo has in 2019 version AFTER part (e) 

%even easier idea: increase problem 1 to four points each, then I only have 30 points left to distribute and make the bacon puzzle parts b, d, and f. including exposition on parts c and e. this will also ease grading! 

%alternative is to increase point value for borel sets problem and then bail on part (e)! 

%%%%%%%%%%%%%%%%%%%%%%%%%%%

\question{
\subsection*{Preliminaries} 

The line segment $[a,b]$ is the set of real numbers $x$ such that $a \leq x \leq b$. The \textbf{Borel Sets} are the members of the smallest set $\mathscr{B}$ such that: $(i)$ every line segment is in $\mathscr{B}$, $(ii)$ if a set $A$ is in $\mathscr{B}$, then so is its complement $\mathbb{R} - A$, and $(iii)$ if a countable family of sets is in $\mathscr{B}$, then so is its union.

The size of a set of real numbers $S$, if it's well-defined, is called its \textbf{Lebesgue measure}.  We will use $\lambda(S)$ to denote the Lebesgue measure of $S$. The Lebesgue Measure, $\lambda$, is the unique function on the Borel Sets that satisfies the following three conditions:
%The size of a set of real numbers $S$, if it's well-defined, is called its Lebesgue measure.  We will use $\lambda(S)$ to denote the Lebesgue measure of $S$. A Lebesgue measure function satisfies the following conditions:
\begin{description}
\item[Length on Segments] $\lambda([a,b]) = b-a$ for every $a,b \in \mathbb{R}$.\label{gloss:lls}


\item[Countable Additivity]
\[\lambda\left(\bigcup\{A_1,  A_2 , A_3,\ldots\}\right) = \lambda(A_1) + \lambda(A_2) + \lambda(A_3) + \ldots\] whenever $A_1,A_2,\dots$ is a countable family of disjoint sets for each of which $\lambda$ is defined.


\item[Non-Negativity]
$\lambda(A)$ is either a non-negative real number or the infinite value $\infty$, for any set $A$ in the domain of $\lambda$.

\item[Uniformity] 
$\lambda(A^{t})=\lambda(A)$, whenever $\lambda(A)$ is defined and $A^{t}$ is the result of adding $t\in\mathbb{R}$ to every member of $A$ (i.e. translating the set $A$ by $t$ units).

\end{description}
}

\vspace{2em}

The notion of measure can be viewed as a generalization of \emph{probability}. Recall that the following is a random procedure to select an individual point from the line-segment $[0,1]$:
\begin{description}
\item[Standard Coin-Toss Procedure] You toss a fair coin once for each natural number. Each time the coin lands Heads you write down a zero, and each time it lands Tails you write down a one. This gives you an infinite sequence $\langle d_{1},d_{2},d_{3},\ldots\rangle$. The selection procedure is then as follows: pick whichever number in $[0,1]$ that has $0.d_{1}d_{2}d_{3}\ldots$ as its binary expansion.
\end{description}
Assume that the Procedure is carried out anew if the output corresponds to a binary expansion ending in all 1s. Let $A$ be a subset of $[0,1]$, and let $p(A)$ be the probability that the next point selected from $[0,1]$ by our random procedure is a member of the set $A$. \\

\noindent
Let $E_{a_{1}a_{2}\ldots a_{n}}$, $a_{i}\in\{0,1\}$, be the event that the first $n$ coins come up $a_{1},a_{2},\ldots,a_{n}$. Thus, $E_{0}$ is the event that the first coin comes up Heads. $E_{001}$ is the event that the first two coins land Heads and the third lands Tails. Assume that the probability of $E_{a_{1}a_{2}\ldots a_{n}}$ is $\frac{1}{2^{n}}$. 


\clearpage

\subsection*{Part I (44 points. On Canvas!)} 



\begin{enumerate}

%JH: using slightly modified longer version from Rayo 2019 and 2021 pset versions:


\item \question{Of each of the following sets of real numbers, determine whether it is a Borel Set. If it is a Borel Set, specify its Lebesgue Measure. (You may assume that every Borel Set has a Lebesgue Measure. And as usual, $[a,b] = \{x : a \leq x \leq b\}$ and $[a,b) = \{x : a \leq x < b\}$.) (3 points each)}

\begin{enumerate}
%\item \question{$\left[\frac{1}{4},\frac{1}{3}\right]$}

%\answer{Borel set.  By Length on Segments, $\lambda{\left[\frac{1}{4},\frac{1}{3}\right]} = \frac{1}{3} - \frac{1}{4} = \frac{4}{12} - \frac{3}{12} = \frac{1}{12}$.}

\item \question{$\left[\frac{7}{8},\frac{8}{9}\right]$}

\answer{Borel set.  By Length on Segments, $\lambda{\left[\frac{7}{8},\frac{8}{9}\right]} = \frac{8}{9} - \frac{7}{8} = \frac{64}{72} - \frac{63}{72} = \frac{1}{72}$.}

\item  \question{$\{\frac{7}{8}\}$}

\answer{Borel set: note that its complement is $(-\infty, \frac{7}{8}) \bigcup (\frac{7}{8}, \infty)$ which is a union of two Borel sets. By Length on Segments $\lambda\{\frac{7}{8}\} = \lambda[\frac{7}{8}-\frac{7}{8}] = \frac{7}{8} - \frac{7}{8} = 0$. }


\item  \question{$\left[\frac{7}{8},\frac{8}{9}\right)$}

\answer{Borel set: note that its complement is $(-\infty, \frac{7}{8}) \bigcup \{\frac{8}{9} \} \bigcup (\frac{8}{9}, \infty)$, which is a union of Borel sets and hence Borel. 

By Additivity, $\lambda{\left[\frac{7}{8},\frac{8}{9}\right]} = \lambda{\left[\frac{7}{8},\frac{8}{9}\right) + \lambda\left\{\frac{8}{9}\right\}}$. So it follows from exercise (a) and the fact that $\lambda\{\frac{8}{9}\} = 0$ that $\lambda{\left[\frac{7}{8},\frac{8}{9}\right)} =  \frac{1}{72}$.}


\item  \question{$\left[\frac{1}{4},\frac{5}{6}\right] - [\frac{1}{3},\frac{1}{2}]$}

\answer{Borel set (it may help to draw a line-segment picture): this set is $[\frac{1}{4}, \frac{1}{3}) \bigcup (\frac{1}{2}, \frac{5}{6}]$, which is a union of Borel sets and hence Borel. 

By Additivity, $\lambda\left[\frac{1}{4},\frac{5}{6}\right] = \lambda\left(\left[\frac{1}{4},\frac{5}{6}\right] - [\frac{1}{3},\frac{1}{2}]\right) + \lambda [\frac{1}{3},\frac{1}{2}]$. By Length on Segments, $\lambda[\frac{1}{4},\frac{5}{6}] = \frac{5}{6} - \frac{1}{4} = \frac{10}{12} - \frac{3}{12} = \frac{7}{12}$ and $\lambda[\frac{1}{3},\frac{1}{2}] = \frac{1}{2} - \frac{1}{3} = \frac{1}{6}$. So $\lambda\left(\left[\frac{1}{4},\frac{5}{6}\right] - [\frac{1}{3},\frac{1}{2}]\right) = \frac{7}{12} - \frac{1}{6} = \frac{5}{12}$.
}

\item  \question{$\left[\frac{1}{4},\frac{5}{6}\right] - \{\frac{1}{3},\frac{1}{2}\}$}

\answer{Borel set. By Additivity, $\lambda{\left[\frac{1}{4},\frac{5}{6}\right]} = \lambda\left(\left[\frac{1}{4},\frac{5}{6}\right] - \{\frac{1}{3},\frac{1}{2}\}\right) + \lambda\{\frac{1}{3},\frac{1}{2}\}$. By Length on Segments, $\lambda[\frac{1}{4},\frac{5}{6}] = \frac{5}{6} - \frac{1}{4} = \frac{10}{12} - \frac{3}{12} = \frac{7}{12}$  and $ \lambda\{\frac{1}{3},\frac{1}{2}\} = 0$. So $ \lambda\left(\left[\frac{1}{4},\frac{5}{6}\right] - \{\frac{1}{3},\frac{1}{2}\}\right) = \frac{7}{12} - 0 = \frac{7}{12}$.}


\item  \question{the set of natural numbers, $\mathbb{N}$}

\answer{
Borel set. Since $\left\{0,1,2,\dots \right\}$ is countable, Additivity entails $\lambda(\left\{0,1,2,\dots \right\}) = \lambda(\{0\}) + \lambda(\{1\}) + \lambda(\{2\}) + \dots$. By Length on Segments $\lambda(\{n\}) = 0$. So $\lambda(\left\{0,1,2,\dots \right\}) = 0$.
}

\item  \question{$\left[\frac{4}{5},\frac{5}{6}\right] \bigcup \left\{\frac{1}{2^1}, \frac{1}{2^2}, \frac{1}{2^3}, \dots \right\}$}

\answer{
Borel set. $\lambda\left(\left[\frac{4}{5},\frac{5}{6}\right] \cup \left\{\frac{1}{2^1}, \frac{1}{2^2}, \frac{1}{2^3}, \dots \right\}\right) = \lambda\left[\frac{4}{5},\frac{5}{6}\right] + \lambda\left\{\frac{1}{2^1}, \frac{1}{2^2}, \frac{1}{2^3}, \dots \right\} = \left(\frac{5}{6}-\frac{4}{5}\right) + 0 = \frac{1}{30}$.
}


\item  \question{$\left[0,\frac{1}{2}\right] - \left\{\frac{1}{2^1}, \frac{1}{2^2}, \frac{1}{2^3}, \dots \right\}$}

\answer{
Borel set. Since $\lambda\left(\left[0,\frac{1}{2}\right]\right) = \lambda\left(\left[0,\frac{1}{2}\right] - \left\{\frac{1}{2^1}, \frac{1}{2^2}, \frac{1}{2^3}, \dots \right\}\right) + \lambda\left\{\frac{1}{2^1}, \frac{1}{2^2}, \frac{1}{2^3}, \dots \right\}$, we have $\lambda\left(\left[0,\frac{1}{2}\right] - \left\{\frac{1}{2^1}, \frac{1}{2^2}, \frac{1}{2^3}, \dots \right\}\right) = \frac{1}{2} - 0 = \frac{1}{2}$.
}


\item  \question{a Vitali set}

\answer{Not a Borel set. This can be verified by noting that all Borel sets have Lebesgue measure, but Vitali sets do not.}

\item  \question{the complement of a Vitali set}

\answer{Not a Borel set. If it were a Borel set, its complement would be a Borel set too, and we know that it isn't.}

\end{enumerate}


%***********************************************************************************
\com{ %simplified version used in 2022 %********************************************************

\item \question{Of each of the following sets of real numbers, specify its Lebesgue Measure, if it has one; if it doesn't have one, give a brief justification why it doesn't. (3 points each)}

\begin{enumerate}
\item \question{$\left[\frac{1}{4},\frac{1}{3}\right]$}

\answer{By Length on Segments, $\lambda{\left[\frac{1}{4},\frac{1}{3}\right]} = \frac{1}{3} - \frac{1}{4} = \frac{4}{12} - \frac{3}{12} = \frac{1}{12}$.}

\item  \question{$\{\frac{1}{3}\}$}

\answer{By Length on Segments $\lambda\{\frac{1}{3}\} = \lambda[\frac{1}{3}-\frac{1}{3}] = \frac{1}{3} - \frac{1}{3} = 0$. }

\com{
\item  \question{$\left[\frac{1}{4},\frac{1}{3}\right)$}

\answer{Borel set. By Additivity, $\lambda{\left[\frac{1}{4},\frac{1}{3}\right]} = \lambda{\left[\frac{1}{4},\frac{1}{3}\right) + \lambda\left\{\frac{1}{3}\right\}}$. So it follows from the previous exercises that $\lambda{\left[\frac{1}{4},\frac{1}{3}\right)} =  \frac{1}{12}$.}


\item  \question{$\left[\frac{1}{4},\frac{5}{6}\right] - [\frac{1}{3},\frac{1}{2}]$}

\answer{
Borel set. By Additivity, $\lambda\left[\frac{1}{4},\frac{5}{6}\right] = \lambda\left(\left[\frac{1}{4},\frac{5}{6}\right] - [\frac{1}{3},\frac{1}{2}]\right) + \lambda [\frac{1}{3},\frac{1}{2}]$. By Length on Segments, $\lambda[\frac{1}{4},\frac{5}{6}] = \frac{5}{6} - \frac{1}{4} = \frac{10}{12} - \frac{3}{12} = \frac{7}{12}$ and $\lambda[\frac{1}{3},\frac{1}{2}] = \frac{1}{2} - \frac{1}{3} = \frac{1}{6}$. So $\lambda\left(\left[\frac{1}{4},\frac{5}{6}\right] - [\frac{1}{3},\frac{1}{2}]\right) = \frac{7}{12} - \frac{1}{6} = \frac{5}{12}$.
}

\item  \question{$\left[\frac{1}{4},\frac{5}{6}\right] - \{\frac{1}{3},\frac{1}{2}\}$}

\answer{Borel set. By Additivity, $\lambda{\left[\frac{1}{4},\frac{5}{6}\right]} = \lambda\left(\left[\frac{1}{4},\frac{5}{6}\right] - \{\frac{1}{3},\frac{1}{2}\}\right) + \lambda\{\frac{1}{3},\frac{1}{2}\}$. By Length on Segments, $\lambda[\frac{1}{4},\frac{5}{6}] = \frac{5}{6} - \frac{1}{4} = \frac{10}{12} - \frac{3}{12} = \frac{7}{12}$  and $ \lambda\{\frac{1}{3},\frac{1}{2}\} = 0$. So $ \lambda\left(\left[\frac{1}{4},\frac{5}{6}\right] - \{\frac{1}{3},\frac{1}{2}\}\right) = \frac{7}{12} - 0 = \frac{7}{12}$.}

}
\item  \question{the set of natural numbers, $\mathbb{N}$}

\answer{
Since $\left\{0,1,2,\dots \right\}$ is countable, Additivity entails $\lambda(\left\{0,1,2,\dots \right\}) = \lambda(\{0\}) + \lambda(\{1\}) + \lambda(\{2\}) + \dots$. By Length on Segments $\lambda(\{n\}) = 0$. So $\lambda(\left\{0,1,2,\dots \right\}) = 0$.
}

\item  \question{$\left[\frac{4}{5},\frac{5}{6}\right] \cup \left\{\frac{1}{2^1}, \frac{1}{2^2}, \frac{1}{2^3}, \dots \right\}$}

\answer{
$\lambda\left(\left[\frac{4}{5},\frac{5}{6}\right] \cup \left\{\frac{1}{2^1}, \frac{1}{2^2}, \frac{1}{2^3}, \dots \right\}\right) = \lambda\left[\frac{4}{5},\frac{5}{6}\right] + \lambda\left\{\frac{1}{2^1}, \frac{1}{2^2}, \frac{1}{2^3}, \dots \right\} = \left(\frac{5}{6}-\frac{4}{5}\right) + 0 = \frac{1}{30}$.
}

\com{
\item  \question{$\left[0,\frac{1}{2}\right] - \left\{\frac{1}{2^1}, \frac{1}{2^2}, \frac{1}{2^3}, \dots \right\}$}

\answer{
Borel set. Since $\lambda\left(\left[0,\frac{1}{2}\right]\right) = \lambda\left(\left[0,\frac{1}{2}\right] - \left\{\frac{1}{2^1}, \frac{1}{2^2}, \frac{1}{2^3}, \dots \right\}\right) + \lambda\left\{\frac{1}{2^1}, \frac{1}{2^2}, \frac{1}{2^3}, \dots \right\}$, we have $\lambda\left(\left[0,\frac{1}{2}\right] - \left\{\frac{1}{2^1}, \frac{1}{2^2}, \frac{1}{2^3}, \dots \right\}\right) = \frac{1}{2} - 0 = \frac{1}{2}$.
}

}

\item  \question{a Vitali set}

\answer{Doesn't have a Lebesgue measure. (Contrary to what the prompt says, you can give full credit even if no justification is provided.)}
\com{
\item  \question{the complement of a Vitali set}

\answer{Not a Borel set. If it were a Borel set, its complement would be a Borel set too, and we know that it isn't.}
}
\end{enumerate}

\question{
(As usual, $[a,b] = \{x : a \leq x \leq b\}$.)
}
} %end com for simplified version! %********************************************************
%***********************************************************************************

%\com{ Left out in 2022 version; used in 2021
\item \question{\emph{The Cantor Set}

Intuitively speaking, the Cantor Set $\mathcal{C}$ is the result of starting with the unit line segment $[0,1]$, eliminating the middle third (so we're left with the set $[0,1/3] \cup [2/3,1]$), eliminating the middle third from each of the remaining line segments (so we're left with the set $[0,1/9] \cup [2/9,1/3] \cup [2/3,7/9] \cup [8/9,1]$), and so forth. Formally: %\footnote{\url{https://en.wikipedia.org/wiki/Cantor_set}.}
\begin{align*}
C_0 &= [0,1]\\
C_{n+1} &= \frac{C_{n}}{3} \cup \rseq{\frac{2}{3} + \frac{C_{n}}{3}} \ (n \in \mathbb{N})\\
\mathcal{C} &= \bigcap^{\infty}_{n=1} C_n
\end{align*}

Answer the following questions (2 points each):
}


\begin{enumerate}

\item \question{Is $C_k (k \in \mathbb{N})$ a Borel Set?}

\answer{Yes. Each $C_k$ is a (finite) union of line segments, each of which is a Borel set.}

\item \question{Is $\mathcal{C}$ a Borel Set?}

\answer{Yes. $\mathcal{C}$ is the countable intersection of Borel Sets, and it follows from one of the exercises in the course materials that any countable intersection of Borel Sets is also Borel Set.}

\item \question{Does $\mathcal{C}$ have a Lebesgue Measure? (If so, what is it?)}

\answer{0. 

One can verify by this by noting that each $C_{n+1}$ subtracts a third from its predecessor. So the complement of $\mathcal{C}$ has Lebesgue measure:
$$\sum_{n=0}^\infty \rseq{\frac{2^n}{3^{n+1}}} = \frac{1}{3} \sum_{n=0}^\infty \rseq{\frac{2^n}{3^{n}}}$$
which is a geometric series. So, by the general formula for geometric series:
$$\frac{1}{3} \sum_{n=0}^\infty \rseq{\frac{2^n}{3^{n}}} = \frac{1}{3} \rseq{\frac{1}{1 - \frac{2}{3}}} = 1$$
But if the complement of  $\mathcal{C}$ has Lebesgue measure 1,  $\mathcal{C}$ itself must have Lebesgue measure 0. }


\item \question{What is the cardinality of $\mathcal C$?}

\answer{$|\mathcal C| = |[0,1]|$

Every number in $[0,1]$ whose base-3 expansion contains no 1s is in $\mathcal C$. But there are uncountably many such numbers. This can be verified by noting: (1) that the set of base-3 expansions containing no 1s is isomorphic to the set of binary expansions, which are uncountable, and (2) that all but countably many numbers in $[0,1]$ have exactly one base-3 expansion.



}
\end{enumerate}
%} %end com 

\clearpage

\item\question{Answer the following questions with respect to the Standard Coin-Toss Procedure. Recall from the Preliminaries that, when $A$ is a subset of $[0,1]$, $p(A)$ is the probability that the next point selected from $[0,1]$ by our random procedure is a member of $A$.  %(\emph{don't forget to justify your answers}). 

\begin{enumerate}
\item What is the probability $p([\frac{7}{8},1])$ that the next selected point will be in $[\frac{7}{8},1]$? (2 points)

\answer{The Standard Koin-Toss Procedure yields a number in $[\frac{7}{8},1]$ iff the first three tosses land Tails. To see this, note that the number $\frac{7}{8} = 0.875$ in base 10 is represented by the binary expansion $0.111(0)$. (Recall that we are ignoring events that end in a countable infinity of 1s.) For an explanation of this fact and worked examples, see the following: \\ https://www.rapidtables.com/convert/number/decimal-to-binary.html. It may help to note that `7' in binary is 111 (while `8' in base two is 1000). Thus, each valid binary sequence not of the form $0.111d_{4}d_{5}\ldots$ represents a real number smaller than $\frac{7}{8}$, and each valid binary sequence of the form $0.111d_{4}d_{5}\ldots$ represents a real number equal to or larger than $\frac{7}{8}$. Since the probability that the first three tosses land Tails is $\frac{1}{8}$ (see Preliminaries), the probability that one gets a number in $[\frac{7}{8},1]$ as output is $\frac{1}{8}$.}

\item What is the probability $p(\{\frac{7}{8}\}$) that the next selected point will be $\frac{7}{8}$? (2 points)

\answer{The unique binary expansion of $\frac{7}{8}$ that is a valid output of the Standard Koin-Toss Procedure is $0.111(0)$. So the only way for $\frac{7}{8}$ to be the outcome of the koin toss is for the sequence to result in exactly the sequence $\langle 1,1,1,0,0,0,\ldots\rangle$. And the probability that this will happen is zero. (It's alright if students refer to this as a known result from lecture.) So the probability of selecting $\frac{7}{8}$ is 0. }

\item What is the probability $p(\{\frac{7}{8^{n}}:n\in\mathbb{N}^{+}, \text{i.e. excluding  } n=0\})$ that the next selected point will be in $\{\frac{7}{8^{n}}:n\in\mathbb{N}^{+}\}$? (You may assume that this probability satisfies Countable Additivity.) (2 points)

\answer{Note first that $\{\frac{7}{8^{n}}:n\in\mathbb{N}\}$ is a countable set. 
So by Countable Additivity, $p(\{\frac{7}{8^{n}}:n\in\mathbb{N}\})=p(\{\frac{7}{8}\})+p(\{\frac{7}{64}\})+\cdots$. But for each $n\in\mathbb{N}^{+}$, $p(\{\frac{7}{8^{n}}\})=0$. So $p(\{\frac{7}{8^{n}}:n\in\mathbb{N}^{+}\})=0$.}

\end{enumerate}
}


\com{ %version of this question used in 2021, in terms of 3/4 rather than 7/8
\begin{enumerate}
\item What is the probability $p([\frac{3}{4},1])$ that the next selected point will be in $[\frac{3}{4},1]$? (2 points)

\answer{The Standard Coin-Toss Procedure yields a number in $[\frac{3}{4},1]$ iff the first two tosses land Tails. To see this, note that the number $\frac{3}{4}$ is represented by the binary expansion $0.11(0)$. (Recall that we are ignoring events that end in 1s.) Thus, each valid binary sequence not of the form $0.11d_{3}d_{4}\ldots$ represents a real number smaller than $\frac{3}{4}$, and each valid binary sequence of the form $0.11d_{3}d_{4}\ldots$ represents a real number equal to or larger than $\frac{3}{4}$. Since the probability that the first two tosses land Tails is $\frac{1}{4}$ (see Preliminaries), the probability that one gets a number in $[\frac{3}{4},1]$ as output is $\frac{1}{4}$.}
\item What is the probability $p(\{\frac{3}{4}\}$) that the next selected point will be $\frac{3}{4}$? (2 points)

\answer{The unique binary expansion of $\frac{3}{4}$ that is a valid output of the Standard Coin-Toss Procedure is $0.11(0)$. So the only way for $\frac{3}{4}$ to be the outcome of the toin coss is for the sequence to result in exactly the sequence $\langle 1,1,0,0,0,\ldots\rangle$. And the probability that this will happen is zero. (It's alright if students refer to this as a known result from lecture.) So the probability of selecting $\frac{3}{4}$ is 0. }
\item What is the probability $p(\{\frac{1}{2^{n}}:n\in\mathbb{N}\})$ that the next selected point will be in $\{\frac{1}{2^{n}}:n\in\mathbb{N}\}$? (You may assume that the probability satisfies Countable Additivity.) (2 points)

\answer{Note first that $\{\frac{1}{2^{n}}:n\in\mathbb{N}\}$ is a countable set. 
So by Countable Additivity, $p(\{\frac{1}{2^{n}}:n\in\mathbb{N}\})=p(\{1\})+p(\{\frac{1}{2}\})+p(\{\frac{1}{4}\})+\cdots$. But $p(\{\frac{1}{2^{n}}\})=0$ for each $n\in\mathbb{N}$. So $p(\{\frac{1}{2^{n}}:n\in\mathbb{N}\})=0$.}

\end{enumerate}
} %end com of 2021 version

  
\end{enumerate}





\subsection*{Part II (56 points)} 

Recall that a \textbf{choice set} for set $A$ is a set containing exactly one element from each member of $A$. And recall the Axiom of Choice:

\begin{description}
\item[Axiom of Choice]
Any set of non-empty, non-overlapping sets has a choice set.
\end{description}


On a first reading, the Axiom of Choice is likely to sound trivial. The following two exercises are aimed at helping you understand why it is not. (The first is a variant of an explanation given long ago by Bertrand Russell.)

Here are two standard set-theoretic axioms:

\begin{description}

\item[Union]
If a set $A$ exists, then so does its union, $\bigcup A$. 

(Recall that $\bigcup A$ is the set $\{x : \text{$x$ is a member of some element of $A$}\}$, i.e.~the set of members of members of $A$.)


\item[Separation]
Let $\phi(x)$ be any formula of the form ``$x$ is such and such'' (for instance, ``$x$ is a natural number''). Then if set $A$ exists, the following set also exists: $\{x : x \in A \text{ and } \phi(x)\}$ (i.e.~the set of objects that are members of $A$ and satisfy condition $\phi(x)$).


\end{description}
Here is an example of how Separation might be used. Let $\phi(x)$ be the formula ``$x$ is female''. Separation entails that if the set $O = \{x : \text{$x$ is an octopus}\}$ exists, so does 
\[\{x : x \in O \text{ and $x$ is female}\}\]
which is the set of female octopuses.

\clearpage

\begin{enumerate}
  \setcounter{enumi}{3}
  
  \item \question{\emph{Choosing Socks, Choosing Shoes}}

\begin{enumerate}

\item \label{shoes}  \question{
Let $S$ be an infinite set, each member of which is a set of two shoes: a right shoe and a left shoe. Assume that no two elements of $S$ have any shoes in common. Now suppose you'd like to have a choice set for $S$. The Axiom of Choice guarantees that a choice set exists, but it doesn't give you much information about what it looks like. 

Let's see if we can do better than that. It follows from Union that $\bigcup S$ exists. Is there an application of Separation to $\bigcup S$ that delivers a choice set for $S$? If so, write out the relevant instance of Separation. If not, give a brief explanation why not. (7~points)}  

\answer{Yes. Separation entails the existence of $$\{x : x \in \bigcup S \text{ and $x$ is a right shoe}\}$$ which is a choice set for $S$.}

\item \label{socks} \question{
Let $S$ be an infinite set, each member of which is a set of two socks. We will assume that all socks are perfect duplicates of one another---and, in particular, that there is no such thing as a ``right'' sock or a ``left'' sock. (We will also assume, unrealistically, that socks are not located in space and therefore that different socks cannot be distinguished by their spatial locations.) Assume that no two elements of $S$ have any socks in common. Now suppose you'd like to have a choice set for $S$. The Axiom of Choice guarantees that a choice set exists, but it doesn't give you much information about what it looks like. 

Let's see if we can do better than that. It follows from Union that $\bigcup S$ exists. Is there an application of Separation to $\bigcup S$ that delivers a choice set for $S$? If so, write out the relevant instance of Separation. If not, give a brief explanation why not. (7~points)}

\answer{No. We can't use Separation to get a choice set because we are not able to articulate a feature of socks that applies to exactly one element from each pair, so we are not able to come up with a suitable choice for $\phi(x)$. }

\end{enumerate}



\item   \question{\emph{Back to Bacon} \label{bacon}

This problem provides a lengthy, scenic, guided tour of one of the many puzzling features of Bacon's Puzzle: the probability that an \textit{individual-who-follows-the-strategy} (hereafter, a `\textbf{Baconite}') will answer correctly does not seem to be well-defined. \\ We'll see that this situation parallels an instance of Bertrand's paradox (recall the issue we saw in Lecture with \textsc{Mystery Square}). There are four parts, (a)--(d)!

Let $S$ be the set of all functions from $\mathbb{N}$ to $\{0,1\}.$ Pictorially, each function corresponds to a putative assignment of hats to our countable infinity of Baconites: \\ $0$ for a red hat, $1$ for a blue hat. 

We partition the set $S$ into orbits as follows: for any $f, g \in S$, $f$ is in the {same orbit} as $g$ if and only if there are at most finitely many numbers $k$ such that $f(k) \neq g(k)$. (Recall that the relation of ``being in the same orbit'' is an equivalence relation, so the orbits {partition} $S$. In other words: $S$ is the union of the orbits, and the intersection of any two orbits is empty.)

Let $O_0$ be the orbit that contains the distinguished function $f_0$ that spits out zero for each natural number, i.e. $f_0(n) = 0$ for each $n \in \mathbb{N}$. Pictorially, $f_0$ corresponds to the belief that every Baconite has a red hat. % (where $0$ denotes red hat; $1$ denotes blue hat). 

Ultimately, to define the probability that a given Baconite answers correctly, we'll well-order the functions belonging to an arbitrary orbit $O$, i.e. associate each function in the orbit with a natural number in $\mathbb{N}$. To keep things simple (and uniform for the purposes of grading!), we'll hook you up with one such well-ordering $\mu_0$ of our special orbit $O_0$: 

\begin{adjustwidth}{2cm}{1cm}
\label{pretrick} \textsc{Tool:} Define a well-ordering $\mu_0$ of $O_0$ by letting $\mu_0(n)$ be the function $ f \in O_0$ such that when written in binary notation, the natural number $n$ corresponds to $\ldots*f(3)*f(2)*f(1)*f(0)$ where `$*$' concatenates these digits. In other words, \textit{when written backwards}, the sequence of digits $f(0),f(1),f(2),\ldots$ corresponds to $n$ in binary. For instance,  $\mu_0(8)$ equals the function $f$ that returns $f(3) = 1$ and 0 for all other values of $n$: i.e. $0000\ldots01000 := (0)1000$. 
\end{adjustwidth}

%Note that the Orbit $O_0$ consists of all the functions from the naturals to $\{0,1\}$ that output finitely-many ones, since these are the functions that differ at finitely-many places from $f_0$
%so f never spits out a countable infinity of 1s, so we get to exclude the binary representations of naturals that end in countable infinity of 1s. 

%\text{Let} \mu_0(n) \text{be the function} f \in O_0 \text{such that when written in binary notation, the natural number n corresponds to} \dots*f(3)*f(2)*f(1)*f(0), \text{where we concatenate these digits}.


}


\begin{enumerate}




\com{ %elevating this subpart to given info, to make grading uniform (since later parts depend on the chosen well-ordering
\item \label{pretrick} \question{Let $f_0(n) = 0$ for each $n \in \mathbb{N}$, and let $O_0$ be $f_0$'s orbit. Describe a bijection $\mu_0$ from $\mathbb{N}$ to $O_0$. (\emph{Hint}: Every $n\in\mathbb{N}$ can be written in binary notation.) (3 points)}

\answer{
Let $\mu_0(n)$ be the function $f \in O_0$ such that the sequence of digits $f(0),f(1),f(2),\ldots$ corresponds to $n$'s binary notation, written backwards.
}
}%end com

\item \question{Given an arbitrary orbit $O$ and a function $h \in O$, describe a bijection $\mu$ from $\mathbb{N}$ to $O$ (e.g. show well-defined, injection, surjection!) (14~points)} \label{trick}
%%should i note that they need not provide hardcore justification that it is a bijection? i.e. well-defined, injective, and surjective? 

\answer{We can use a simple variant of the well-ordering given by $\mu_0$ in \textsc{Tool}. Let $\mu(n)$ be the function $f \in O$ such that the sequence of digits $|h(0)-f(0)|, |h(1)-f(1)|, |h(2)-f(2)|, \ldots$ corresponds to $n$'s binary notation, written backwards. Note that for a given $n \in \mathbb{N}$, its binary representation is an infinite number of zeros followed by a finite string of ones and zeros that begins with 1. E.g. `8' in binary is $0000\ldots01000 = (0)1000$. 

\textbf{Well-defined} (i.e. defined for each natural number while outputing a function in our orbit $O$): Using the binary representation of $n$ and the values of $h(n)$, we can calculate the values of $f(n)$ for each $n \in \mathbb{N}$. e.g. if the 4th digit from the rightmost digit of $n$ in binary is 1, and $h(3)=1$, then $f(3) = 0$. Next we must show that $f$ is actually in the same orbit $O$ as $h$, i.e. differs at most finitely-many values from $h$. Since $n$'s binary representation begins with a countable infinity of 0s, we know that there exists some $m \in \mathbb{N}$ such that for all $n > m$, $|h(n)-f(n)| =0$, entailing that $h(n) = f(n)$ for all $n > m$. Hence, $f$ differs at most finitely-many values from $h$, and hence belongs to the same orbit $O$. 

\textbf{Injection}: note that if $n=m$, then $\mu(n)$ and $\mu(m)$ both yield functions $f$ and $g$, respectively, that result in the same sequence of digits $|h(n)-f(n)| = |h(n)-g(n)| $, $\forall n \in \mathbb{N}$. Since $f$ and $g$ always output 0 or 1, this is possible just in case $f(n) = g(n)$ $\forall n \in \mathbb{N}$, which is to say that $f=g$, so $\mu(n)=\mu(m)$.  

\textbf{Surjection}: consider an arbitrary function $g \in O$. We must show that there exists $m \in \mathbb{N}$ such that $\mu(m) = g$. Let $m$ be the natural number whose representation in binary notation is given by the sequence of digits $|h(0)-g(0)|, |h(1)-g(1)|, |h(2)-g(2)|, \ldots$ written backwards. By assumption, $g$ is in the same orbit as $h$, so they disagree on only finitely many values of $n$. This guarantees that the sequence $\ldots |h(2)-g(2)| |h(1)-g(1)| |h(0)-g(0)| $ begins with infinitely many 0s and ends with a finite string of ones and zeros that begins with a 1. Hence, this sequence does define an $m \in \mathbb{N}$. 

}

%To see that this function $\mu$ is an injection, note that if $\mu(n) = f = g$, then both $f$ and $g$ yield the same sequence of digits, i.e. $|h(n)-f(n)| = |h(n)-g(n)| $ $\forall n \in \mathbb$. Since $f$ and $g$ always output 0 or 1, this is possible just in case $f(n) = g(n)$ $\forall n \in \mathbb$, which is to say that $f=g$. 



\end{enumerate}

\question{
Problem~(\ref{trick}) shows that any representative from a given orbit can be used to define a well-ordering of that orbit. This will be important later. Hold on to your bacon!


Recall our main question here: what is the probability that an arbitrary Baconite will answer correctly? To fix ideas, let $P_0$  be the individual in question, and assume that she has been given a blue hat. Let the function $f_\text{\at}$ represent the \textbf{actual} distribution of hats and let $O_\text{\at}$ be $f_\text{\at}$'s orbit. Then our question can be reformulated as follows: what is the probability that orbit $O_\text{\at}$ was assigned a representative $\rho$ such that $\rho(0) = 1$? It is precisely in these cases---where the representative function $\rho$ views Baconite number $0$ as wearing a blue hat---that this Baconite will answer correctly. 

There is a natural way of answering this question, \emph{relative} to a well-ordering of $O_\text{\at}$. Let $\mu$ be one such well-ordering, i.e. a bijection from $\mathbb{N}$ to $O_\text{\at}$. Then we can use $\mu$ to characterize the following probability function, where  $Z$ is a subset of $O_\text{\at}$: 
\[
\begin{array}{ccc}
p(Z) &=_{df} &\lim \limits_{n \to \infty}\dfrac{|Z\cap \{\mu(0),\mu(1),\dots, \mu(n)\}|}{|\{\mu(0),\mu(1),\dots \mu(n)\}|}\\
\end{array}
\]
%(Here $Z$ is a subset of $O_\text{\at}$. If you'd like a refresher on this type of probability function, see Section~6.4.1.2 of the textbook.)

%Let Let's now consider the value of $p(X)$ relative to different orderings. 
% In the next couple of questions I'll ask you to calculate the value of $p(X)$ relative to different orderings.

}


\begin{enumerate}
  \setcounter{enumii}{1}
\item\label{first-ans}
 \question{
Suppose that $O_\text{\at}$ is orbit $O_0$ introduced earlier and that $\mu$ is the bijection $\mu_0$ generously provided in \textsc{Tool}. What is the value of $p(X)$, where $X$ is the proposition that $O_\text{\at}$ was assigned a representative $\rho$ such that $\rho(0) = 1$? \\ (Formally: $X = \{f \in O_\text{\at} : f(0) = 1\}$) (10~points)
}


\answer{
%The answer depends on the student's choice of $\mu_0$ in their answer to (\ref{pretrick}). 
Using $\mu_0$ defined in \textsc{Tool} above, $[\mu_0(k)](0) = 1$ iff $k$ is odd. To see this, recall that $\mu_0(k)$ is the function $f \in O_0$ such that $k$ in binary equals $\ldots f(3) f(2) f(1) f(0)$. So $[\mu_0(k)](0)  = f(0)$ which is the rightmost digit of $k$. And in binary notation, a natural number ends in a 1 just in case it is odd (Even natural numbers end in 0). 


Hence, $X := \{f \in O_\text{\at} : f(0) = 1\} = \{\mu_0(1), \mu_0(3), \mu_0(5), \dots\}$, so
$$
\begin{array}{ccl}
p(X) &=_{df} &\lim \limits_{n \to \infty}\dfrac{|\{\mu_0(1), \mu_0(3), \dots\}\cap \{\mu_0(0),\mu_0(1),\dots, \mu_0(n)\}|}{|\{\mu_0(0),\mu_0(1)_0,\dots \mu_0(n)\}|}\\
&=_{df} &\lim \limits_{n \to \infty}\dfrac{|\{\mu_0(1), \mu_0(3), \dots, \mu_0(n')\}|}{|\{\mu_0(0),\mu_0(1)_0,\dots \mu_0(n)\}|}\\
&=_{df} &0.5
\end{array}
$$
where $n' = n$ if $n$ is odd and $n-1$ otherwise.
}

%\item\label{sec-ans}
 
\end{enumerate}

\question{Now if all went well, you should have gotten a pleasing answer for $p(X)$ above. One might wonder: are matters this nice in general? To answer this question, define a function $\mu^X$, which assigns each $n \in \mathbb{N}$ to some function in $X$. 
 %For instance, $\mu^X(n)$ might be the function $f \in X$ such that the sequence of digits $f(1),f(2),f(3),\ldots$ corresponds to $n$'s binary notation, written backwards. 
 Next, define a function $\mu^{\overline{X}}$, which assigns each $n \in \mathbb{N}$ to some function in $O_0 - X$ (i.e. the functions in the orbit $O_0$ that don't assign the 0th Baconite a blue hat). For instance, $\mu^{\overline{X}}(n)$ might be the function $g \in O_0 - X$ such that the sequence of digits $g(0), g(1),g(2),g(3),\ldots$ corresponds to $n$'s binary notation, written backwards. 
 
We can now define various modified well-orderings $\mu_k$ by starting with values of $\mu^{\overline{X}}$,
$$\mu^{\overline{X}}(0), \mu^{\overline{X}}(1), \mu^{\overline{X}}(2), \mu^{\overline{X}}(3), \mu^{\overline{X}}(4), \mu^{\overline{X}}(5)\dots$$
and interspersing values of $\mu^{X}$ at intervals of length $k-1$, for $k\in \mathbb{N}$. For example, when $k = 3$, we put two values of $\mu^{\overline{X}}$ in between each value of $\mu^{X}$, yielding:
$$\mu^X(0),\mu^{\overline{X}}(0), \mu^{\overline{X}}(1), \mu^X(1), \mu^{\overline{X}}(2), \mu^{\overline{X}}(3), \mu^X(2),\mu^{\overline{X}}(4), \mu^{\overline{X}}(5),\mu^X(3),\dots$$
We then define $\mu_k(n)$ as the $n$th member of the resulting sequence.
}

\begin{enumerate}
  \setcounter{enumii}{2}
\item\label{sec-ans}
 \question{Explain the significance of our modified well-orderings $\mu_k$, e.g. for $k=3$. \\ What is $p(X)$ according to $\mu_k$, e.g. for $k=3$? (8~points)}
 
 \answer{Significance: using our modified well-orderings $\mu_k$, you can get $p(X)$ to equal any of a \emph{large class of values}, by picking an appropriate $\mu_k$. For a given $\mu_k$, $p(X) = 1/k$. We might then wonder whether any choice of well-ordering is canonical or privileged, and if we could specify a canonical well-ordering in general for an arbitrary orbit. 
 %So the probability function $p(\dots)$ can only be assumed to assign a sensible probability to proposition $X$ if it is defined using a sensible choice of $\mu$.
 
 For $k=3$, $p(X) = 1/3$ according to $\mu_k$. Ideally the student will show this using the definition of $p(\ldots)$: for each $n$, the set in the numerator has 1/3 as many elements as the set in the denominator, since we have interspersed two elements of $\mu^{\overline{X}}$ in between each element of $\mu^X$. And it is only these $\mu^X$ elements that say the 0th Baconite is assigned a blue hat, since by construction the $\mu^{\overline{X}}$ elements belong to $O_0 - X$ (i.e. the functions in the orbit $O_0$ that DON'T assign the 0th Baconite a blue hat). 
 
 In case it helps you explain or discuss this problem, the formal, general definition of $\mu_k$ for arbitrary $k\in \mathbb{N}$ is as follows:
$$
\mu(n) = 
\begin{cases}
\mu^X\left(\frac{n}{k}\right), \text{ if $n = 0$ modulo $k$}\\
\mu^{\overline{X}}\left(\frac{(k-1)(n-1)}{k}\right), \text{ if $n = 1$ modulo $k$}\\
\mu^{\overline{X}}\left(\frac{(k-1)(n-2)}{k}+1\right), \text{ if $n = 2$ modulo $k$}\\
\mu^{\overline{X}}\left(\frac{(k-1)(n-3)}{k}+2\right), \text{ if $n = 3$ modulo $k$}\\
\vdots\\
\mu^{\overline{X}}\left(\frac{(k-1)(n-(k-1))}{k}+k-2\right), \text{ if $n = k-1$ modulo $k$}
\end{cases}
$$

Some additional fodder for discussion/reflection (not needed in a good student answer): When it comes to particular orbits, you may well think that there are choices of $\mu$ that stand out as particularly natural. Perhaps you think that when it comes to the specific orbit $O_0$, our choice of $\mu_0$ in \textsc{Tool} is a particularly natural way of ordering $O_0$ (it delivers the comforting result that  $p(X)=0.5$). Yet as we go on to show next, there's seemingly no general recipe for specifying natural orderings of arbitrary orbits. 
 }
 
 
 \end{enumerate}

%\question{Now if all went well, you should have gotten a pleasing answer for $p(X)$ above. However, matters are not so nice in general! By modifying the well-ordering $\mu$, you can get $p(X)$ to have \emph{any value you want}.  So the probability function $p(\dots)$ can only be assumed to assign a sensible probability to proposition $X$ if it is defined using a sensible choice of $\mu$.}



\question{

Let us now consider the problem of how one might go about choosing a representative from each orbit, in general. Is there a \emph{general recipe} that can be used to specify a natural ordering of each of our uncountably many orbits? To get a handle on this question, ask yourself: is the set of orbits analogous to the set of pairs of shoes of problem~(\ref{shoes}), or is it analogous to the set of pairs of socks of problem~(\ref{socks})? In other words: is there a formula $\phi(x)$ such that an application of \textsc{Separation} based on $\phi(x)$ specifies a set that contains exactly one representative for each orbit? 

As it turns out, the answer to this question is ``no.'' An important result due to Robert Solovay entails that it is impossible to set forth an explicit rule that singles out exactly one representative from each orbit: the only way to show that a choice set for the set of orbits exists is to assume the Axiom of Choice (specifically, Solovay showed that one cannot prove the existence of a non-measurable set without using the Axiom of Choice). 
}




\begin{enumerate}
  \setcounter{enumii}{3}

\item \question{Show that it is impossible to explicitly characterize a relation $<$ such that each orbit $O$ is well-ordered by $<$. (You may rely on the above \textit{key fact} that the only way to show that a choice set exists for the set of orbits is to use the Axiom of Choice). (10~points)

%Fact: one cannot show that a choice set for the set of orbits exists without using the Axiom of Choice. 

}

\answer{Suppose for reductio that one can explicitly characterize $<$. Then one can use $<$ to specify the ``smallest'' element of each orbit: the $<$-smallest element of each orbit is the unique $f$ in each orbit such that no element of the orbit bears $<$ to $f$. But then one could use an application of Separation based on the formula ``$x$ is the $<$-smallest element of its orbit'' to specify a set that contains exactly one representative from each orbit. We would thereby have specified a choice set without using the Axiom of Choice. But it follows from our key fact that this is impossible.

}
\end{enumerate}

\question{\textit{Dramatic (and long-awaited) Upshot}: In the absence of a general recipe for specifying a natural ordering for each orbit, it seems impossible to characterize a sensible probability function over the members of our orbits. For this reason, it seems plausible that the probability of success in Bacon's Puzzle---given that one follows the strategy---is not well-defined in general. Bummer? 
}


% COMMENTED AWAY FOR VARIETY
\com{
\item \question{What would go wrong if we just stipulated that a Vitali set has measure $1/3$? (10~points)}

\answer{As the proof of the Vitali theorem shows, one would have to give up on the Axiom of Choice, or on at least one of the following:

\begin{description}
\item[Length on Line Segments]
$\lambda([a,b]) = b-a$.

\item[Countable Additivity]
Let $A_1,A_2,\dots$ be a countable family of disjoint sets. Whenever $\lambda(A_i)$ is defined for each $A_i$, we have:
\[
\lambda\left(\bigcup\{A_1,  A_2 , A_3,\ldots\}\right) = \lambda(A_1) + \lambda(A_2) + \lambda(A_3) + \ldots
\]

\item[Real or Infinite Values]
For any set $A$ in the domain of $\lambda$, $\lambda(A)$ is either a non-negative real number, or the infinite value $\infty$.
\end{description}
}
}

% COMMENTED AWAY FOR VARIETY -- AND ALSO TO TREAT THE BANACH-TARSKI CHAPTER AS A FREEBE.
\com{
\item\question{
Let $D$ be a disc of radius 1 with center $d$. (In other words: $D$ is the set of points at a distance no greater than 1 from $d$). Let $D^-$ be the result of removing two points from $D$. (Assume that the points are in the interior of $D$. In other words: the distance from the center of $D$ to each point is strictly smaller than 1.)  Is it possible to decompose $D^-$ into two distinct parts and reassemble the parts (without changing their sizes or shapes) so as to get $D$? If so, explain how to do it. If not, explain why not. (10 points)

\emph{Hint:} You may assume that for any points $p_1$  and $p_2$ in the interior of $D$, there is a circle $C_r$ of radius $r$ which goes through $p_1$ and $p_2$ and is entirely within $D$. (This is a consequence of Apolonius's Problem.) You may assume, moreover, that the circumference of $C_r$ is irrational with respect to the length of $\delta^{p_1}_{p_2}$, which is the shortest segment of $C_r$ with endpoints $p_1$ and $p_2$. (In other words, you may assume that there is no rational number $q$ such that the circumference of $C_r$ is $q \cdot \delta^{p_1}_{p_2}$.)
}


\answer{It follows from our assumption that there must be a circle $C^*$ that goes through $p_1$ and $p_2$, is entirely within $D$, and is such that the distance between $p_1$ and $p_2$ on $C^*$ is irrational with respect to the circumference of $C^*$.  Once a suitable $C^*$ is in place, we can solve the problem by using a procedure analogous to the procedure in Warm-Up Case~2 in the chapter on the Banach-Tarski Theorem.}

}



\com{

\item \question{
\emph{The Square of Evil}\footnote{The construction is due to the Polish mathematician Wac\l aw Sierpi\'nski. Rayo learned about it in Frank Arntzenius, Adam Elga and John Hawthorne's ``Bayesianism, Infinite Decisions, and Binding".}

Say that a \textbf{countable ordinal} is an ordinal with countably many members, and let $\aleph_1$ be the set of all countable ordinals. $\aleph_1$ is itself an ordinal. From this it follows that $\aleph_1$ must have uncountably many members. (For suppose otherwise, then $\aleph_1$ is a countable ordinal, and therefore a member of itself. But no ordinal is a member of itself.)

Think of the \textbf{Continuum Hypothesis} as the claim that $\aleph_1$ has the same cardinality as [0,1], and therefore that there is a bijection $f$ from [0,1] to $\aleph_1$. Assume that the Continuum Hypothesis is true, and define the following ordering $<^e$ of [0,1]:
\[
\text{for any $a,b \in [0,1]$, $a <^e b$ if and only if $f(a) \in f(b)$}
\]
Since the ordinals in $\aleph_1$ are well-ordered by $\in$, it is an immediate consequence of this definition that $<^e$ is a well-ordering of [0,1].
}



\begin{enumerate}


\item \question{Show that $<^e$ has the following additional property: for each $x \in [0,1]$, there are at most countably many $y \in [0,1]$ such that $y <^e x$.\label{special} (5 points)}

\answer{
We know from the definition of $<^e$ that $y <^e x$ if and only if $f(y) \in f(x)$. But $f(x)$ is a member of $\aleph_1$ which is the set of countable ordinals. So $f(x)$ has at most countably many members. So there are at most countably many $f(y) \in \aleph_1$ such that $f(y) \in f(x)$. So there are at most countably many $y \in [0,1]$ such that $y <^e x$.
}

\end{enumerate}



\question{
We will now use $<^e$ to color the unit square [0,1]$\times$[0,1], using the following criterion:
\begin{quote}
For each point $\seq{x,y} \in [0,1]\times [0,1]$, color $\seq{x,y}$ white if $x <^e y$, and black otherwise.
\end{quote} I will refer to the colored square as the \textbf{Square of Evil}. Now let $\seq{x_0,y_0}$ be a particular point on the Square of Evil:
}
\begin{enumerate}
\setcounter{enumii}{1}

\item \question{How many white points are there in the row $\{\seq{z,y_0} : z \in [0,1]\}$? (5 points)}

\answer{
There are at most countably many. For a point $\seq{x,y_0}$ in row $\{\seq{z,y_0} : z \in [0,1]\}$ is white if and only if $x <^e y_0$, and it follows from question~\ref{special} that there are at most countable many $x \in [0,1]$ such that $x <^e y_0$.

}

\item  \question{How many white points are there in the column $\{\seq{x_0,z} : z \in [0,1]\}$? (5 points)} 

\answer{There are uncountably many. Since every point is either black or white, we can prove this by verifying that there are at most countably many black points in column $\{\seq{x_0,z} : z \in [0,1]\}$. This is easily done: a point $\seq{x_0,y}$ in column $\{\seq{x_0,z} : z \in [0,1]\}$ is black if and only if $y \leq^e x_0$, and it follows from question~\ref{special} that there are at most countably many $z \in [0,1]$ such that $z \leq^e x_0$.

}


\end{enumerate}
\question{Suppose that a point is selected at random from the Square of Evil. (It is selected by twice applying the Standard Coin Toss Procedure, once to pick an $x$ coordinate and once to pick a $y$ coordinate.)}


\begin{enumerate}
\setcounter{enumii}{3}


\item \question{Someone tells you which row the selected point is in, and asks you to bet on whether the selected point is black or white. How should you bet?  (5 points)} \label{black}

\answer{Here is one good answer: ``You should bet on black, since almost every point on that row is black." Another good answer would show awareness that contradiction looms, and formulate some attempt to get out of it.}

\item \question{Someone tells you which column the selected point is in, and asks you to bet on whether the selected point is black or white. How should you bet?  (5 points)} \label{white}

\answer{Here is one good answer: ``You should bet on white, since almost every point on that column is white." Another good answer would show awareness that contradiction looms, and formulate some attempt to get out of it.}

\end{enumerate}


\question{It will rain or snow, but you don't know which. If it rains, you should wear outfit $A$ rather than outfit $B$. If it snows, you should also wear outfit $A$ rather than outfit $B$. So you should wear outfit $A$!

Here is a generalization of that seemingly attractive idea:

\begin{description}

\item[Dominance] Let $\Pi$ be a set of possible states of the world over which you have no control. You know that exactly one member of $\Pi$ obtains, but you don't know which.  Suppose you have two options, $A$ and $B$. Suppose, moreover, that for each $\pi \in \Pi$ you should choose $A$ over $B$ on the assumption that $\pi$ obtains. Then you should choose $A$ over $B$.




\end{description}
}
\begin{enumerate}
\setcounter{enumii}{5}
\item
\question{Use the Square of Evil to show that Dominance is false. (10 points)
}

\answer{

Suppose Dominance is true, as stated. Let $\Pi$ consist of states of the world [the row of the selected point is $r$] for $r \in [0,1]$. (\ref{black}) tells us that, for each $\pi \in \Pi$, you should choose ``bet black'' over ``bet white'', on the assumption that $\pi$ obtains. So, by Dominance, you should choose ``bet black'' over ``bet white''.

But an analogous argument, based on (\ref{white}) yields the conclusion that you should choose ``bet white'' over ``bet black''. But it can't be the case that you should both choose black and choose white. So Dominance must be mistaken.




}

\end{enumerate}

}
\end{enumerate} 



\end{document}




\vspace{7mm}
\begin{center}
\begin{tikzpicture}
\draw (0,0) circle (3cm); % B circle
\node at (3,2) {$B$}; % B label

\draw (1,0.5) circle (1cm); % main circle
\draw[thick] (1,1.5) circle (0.02cm); % p1 circle
\node at (1,1.8) {$p_1$}; % p1 label
\draw[thick] (1,-0.5) circle (0.02cm); % p2 circle
\node at (1,-0.8) {$p_2$}; % p2 label

\draw[dotted] (0,0.5) -- (2,0.5);

\end{tikzpicture}
\end{center}




