
\documentclass[12pt,letterpaper]{article}
\usepackage{pset-2023-no-quiz}
\usepackage{enumitem}
\usepackage{hyperref}
\newcommand{\test}[1]{#1}


%Questions and Answers
\qa{b} % a="answers only"; q ="questions only"; b="both"
\usepackage{qa}


\begin{document}

\psintro{Problem Set 9: Computability}

%%%%%%%%%%%%%%%%%%%%%%%%%%%

%\fbox{\parbox{150mm}{\emph{Important:} Answers that call for a Turing Machine program will only be given credit if they are submitted as part of a PDF document with code that can be copied and pasted onto the following simulator: \url{morphett.info/turing/turing.html}.
%Please {test} your PDF {using Adobe's Acrobat Reader} before submitting it, by making sure your code works as intended after being copied and pasted into the simulator. (\LaTeX \ users might consider using the ``verbatim'' environment, or an environment intended for computer code.)}}


% Idea for next time: something that gets students to see the basic point behinds Rice's Theorem. (For example, something to do with the informal proof (in the Wikipedia page for Rice's Theorem) that the function ascertaining whether a given machine computes the squaring function is not computable.) One idea would be to guide the student through the proof I sent Michele on 5/20/2021.

\subsection*{Part I (no need to justify answers; show work for partial credit)} 


\begin{enumerate}

%2*3*25*7*(11^2)*(13^2)*17*19*23*29


\item The aim of this problem is to get you to think about how to code Turing Machines as natural numbers. Please use the coding system described in the course materials.

\begin{enumerate}

\item 

\begin{enumerate}

\item \question{Which natural number codes the following Turing Machine?  (4~points)
\vspace{1mm}
\begin{verbatim}
0 _ 1 r 1
1 _ _ r 0  
\end{verbatim}
\vspace{1mm}
}

\answer{
%Josh: changing `l' to `L' because the `l' prints in a font that looks like numeral 1

Note that you'll probably get a bunch of student answers who give a number for each command line, rather than a SINGLE natural number for the Turing Machine. This might be a great thing to clarify in section. 

In the relevant coding system, ``\ub'' is coded as 0, and``r" is coded as 1. So our Turing Machine corresponds to the sequence \(\langle 0, 0, 1, 0, 1, 1, 0, 0, 0, 0 \rangle\), which gets coded as 
\[2^{0+1} \cdot 3^{0+1} \cdot 5^{1+1} \cdot 7^{0+1} \cdot 11^{1+1} \cdot 13^{1+1} \cdot 17^{0+1} \cdot 19^{0+1} \cdot 23^{0+1} \cdot 29^{0+1}= 4625830659450 \approx 4.62583066 \cdot 10^{12}\]
Give full points if they express most of the digits in scientific notation, e.g. it's fine to truncate the last few digits. 
}

\item \question{Give an informal description of the behavior of that Turing Machine, when run on an empty input. (4~points)}

\answer{When this Turing Machine runs on an empty input, it replaces every other blank with a 1, going on forever. It never halts. So it produces a tape that looks like 1\ub 1\ub1\ub1\ub \dots . 
}


\end{enumerate}

\item 

\begin{enumerate}
\item \question{Which Turing Machine is coded by the number 1,205,247,120? (4~points)
}

\answer{The Turing Machine is 3 \ub \, \ub \, $\ell$ 3

In the relevant coding system, ``\ub'' is coded as 0, state ``3'' is coded as 3, and ``$\ell$" is coded as 2. The Machine above gets assigned the sequence \(\langle 3,0,0,2,3\rangle\), which is coded as 
\[
2^{3+1} \cdot 3^{0+1} \cdot 5^{0+1} \cdot 7^{2+1} \cdot 11^{3+1} = 1,205,247,120
\]

Note that to calculate the answer, just systemically divide 1,205,247,120 by primes until you hit one. e.g. start with the first prime $2$ and you'll see you can divide by 2 four times without remainder, telling you that the initial state is called ``3''. Then you'll see you can divide by 3 once, by 5 once, by 7 three times, and 11 four times. 

}


\item \question{Give an informal description of the behavior of that Turing Machine, when run on an empty input. (4~points)}

\answer{Award full points to either of the following answers: 
i) It immediately halts, since no directions are given for the initial state 0. This is the convention that Morphett follows (but it seems rather uninteresting to \textit{require} the initial state to be labeled by `0').

ii) It goes left forever, keeping blanks blank. So it does the same thing as the TM coded by 113,190, namely 0 \ub \, \ub \, $\ell$ 0.

 }
\end{enumerate}


\end{enumerate}

\com{ %Version used in 2019, 2021, 2022. Hunt writing a new version for 2023, used above


\item The aim of this problem is to get you to think about how to code Turing Machines as natural numbers. Please use the coding system described in the course materials.

\begin{enumerate}

\item 

\begin{enumerate}

\item \question{Which natural number codes the following Turing Machine?  (3~points)
\vspace{1mm}
\begin{verbatim}
0 _ _ L 1
\end{verbatim}
\vspace{1mm}
}

\answer{
%Josh: changing `l' to `L' because the `l' prints in a font that looks like numeral 1
In the relevant coding system, ``\ub'' is coded as 0, and``L" is coded as 2. So our Turing Machine corresponds to the sequence \(\langle 0,0,0,2,1\rangle\), which gets coded as 
\[2^{0+1} \cdot 3^{0+1} \cdot 5^{0+1} \cdot 7^{2+1} \cdot 11^{1+1} = 1,245,090\]
}

\item \question{Give an informal description of the behavior of that Turing Machine, when run on an empty input. (3~points)}

\answer{When this Turing Machine runs on an empty input, it goes leftward for one step and halts, leaving the tape unchanged.
}


\end{enumerate}

\item 

\begin{enumerate}
\item \question{Which Turing Machine is coded by the number 11,550? (3~points)
}

\answer{The Turning Machine: 0 \_ 1 r 0

In the relevant coding system, ``\ub'' is coded as 0, ``1'' is coded as 1, and ``$r$" is coded as 0. the Machine above gets assigned the sequence \(\langle 0,0,1,0,0\rangle\), which is coded as 
\[
2^{0+1} \cdot 3^{0+1} \cdot 5^{1+1} \cdot 7^{0+1} \cdot 11^{0+1} = 11,550
\]}


\item \question{Give an informal description of the behavior of that Turing Machine, when run on an empty input. (3~points)}

\answer{It goes right forever, replacing blanks with ones.}
\end{enumerate}


\end{enumerate}
} %end com of old version 










% LEFT OUT, FOR VARIETY
\com{

\item \question{Give an informal description of the behavior of the following Turing Machine, when run on an empty input. (Try to be as succinct as possible.) (5 points)
\[
\begin{array}{ccccc}
0 &\ub &1 &r &1\\
1 &\ub &1 &l &2\\
2 &1 &\ub &r &3\\
3 &1 &\ub &l &0
\end{array}
\]

}

\answer{A succinct---and correct---answer might say something like ``It goes back and forth forever, changing blanks to ``1"s and ``1"s to blanks''.

A longer one is also acceptable. For instance: ``This Turing Machine starts in state 0, enters a ``1" in the blank cell, goes right, and enters state 1. It now again enters a ``1" in the blank cell, but then goes left. So now it is back at the cell at which it began. It changes the ``1" back to a blank, and moves right, where it enters state 3. It then changes the second ``1" it wrote back to a blank, goes left, and starts the whole process over again. So it goes right changing blanks into ``1"s, left changing ``1"s into blanks, over and over."}
}


% LEFT OUT, FOR VARIETY
% could auto-grade these in canvas quiz? 
\com{
\item \question{For each Turing Machine on the list below, say whether it halts when run on an empty input. (5~points each)}

\begin{enumerate}

\item 
\question{
\(
\begin{array}{ccccc}
0 &\ub &\ub &r &0
\end{array}
\)
}

\answer{Doesn't halt. (This machine just goes right indefinitely.)}
\vspace{3mm}

\item
\question{
\(
\begin{array}{ccccc}
0 &\ub &1 &r &1
\end{array}
\)
}


\answer{Halts. (This machine writes a ``1" and then halts, since it goes to state 1 but there are no instructions for state 1.)}


\vspace{3mm}


\item
\question{
\(
\begin{array}{ccccc}
0 &\ub &1 &r &1\\
1 &\ub &1 &r &2\\
2 &\ub &1 &r &3\\
3 &\ub &1 &r &4
\end{array}
\)
}



\answer{Halts. (This machine writes four ``1"s and then halts, since it goes to state 4 but there are no instructions for state 4.)}

\vspace{3mm}


\item 
\question{
\(
\begin{array}{ccccc}
0 &\ub &1 &r &1\\
1 &\ub &1 &r &2\\
2 &\ub &1 &r &3\\
3 &\ub &1 &r &0
\end{array}
\)
}


\answer{Doesn't halt. (This machine writes ``1"s indefinitely, since after writing four ``1"s it goes back to state 0 and starts again.)}

\end{enumerate}
}


%Please {test} your PDF {using Adobe's Acrobat Reader} before submitting it, by making sure your code works as intended after being copied and pasted into the simulator. (\LaTeX \ users might consider using the ``verbatim'' environment, or an environment intended for computer code.)}}

\item The following problems are meant to give you some practice coding Turing machines. (6~points each). \emph{Important:} Your answers will only be given credit if they are submitted as part of a PDF document with \textbf{code that can be copied and pasted into the following simulator}: \href{http://www.morphett.info/turing/turing.html}{morphett.info/turing/turing.html}. Please {test} your PDF before submitting it, by making sure your code works as intended after being copied and pasted into the simulator. (\LaTeX \ users might consider using the ``verbatim'' environment, or an environment intended for computer code.)

\begin{enumerate}
\com{ %left out 2022 and 2023
\item \question{

Ternary notation is to 3 what decimal notation is to 10 and what binary notation is to 2. In other words: one works with the digits ``0'', ``1'', and ``2'', and lets the string ``$d_k\dots d_1 d_0$'' refer to the number $d_0 \cdot 3^0 + d_1 \cdot 3^1 \cdot \dots \cdot d_k \cdot 3^k$.

Design a Turing Machine that does the following: when given as input a natural number $n \geq 1$ in ternary notation, followed by a blank, followed by natural number $m \geq 1$ in ternary notation, it halts with the number $n + m$ in ternary notation on an otherwise blank tape.

Here is an example, to illustrate how your Turing Machine ought to work. Suppose $n = 47$ and $m = 64$. In ternary notation, ``1202'' refers to the number 47 and ``2101'' refers to the number 64. So your machine should start out with the following string of symbols on an otherwise blank tape:
$$1202\ 2101$$
Since $47 + 64 = 111$ (and since, in ternary notation, ``11010'' refers to the number 111), your Turing Machine should halt with the following string of symbols on an otherwise blank tape:
$$11010$$
And, of course, you want this to work for arbitrary $n$ and $m$.  You may use auxiliary symbols, if you need them. 

}}


\item \question{Design a Turing-machine that does the following: when given as input a string of $n$ ones ($n>0$), it outputs a string of $n+3$ ones and halts with the reader at the leftmost one.}

\answer{This one will do:  \begin{verbatim}
0 1 1 r 0
0 _ 1 r 1
1 _ 1 r 2
2 _ 1 r 3
3 _ _ l 4
4 1 1 l 4
4 _ _ r 5
\end{verbatim}

}

\item \question{Design a Turing-machine that does the following: when given as input a string of $n$ ones ($n>0$), it outputs a string of $n-1$ ones and halts with the reader at the leftmost one (if there is one).}

\answer{This one will do: \begin{verbatim}
0 1 1 r 0
0 _ _ l 1
1 1 _ l 2
2 1 1 l 2
2 _ _ r 3
\end{verbatim}}

\item \question{Design a Turing Machine that behaves as follows: when given as input a natural number $n \geq 1$ in \textbf{binary notation}, it halts with the number $4n +2$ in \textbf{binary notation} on an otherwise blank tape. (No need to worry about where the reader ends up.)

}

\answer{It suffices to adjoin `10' to the right end of the tape, so when you run through morphett just check that this is what the student's program does. Below are two ways, with explanation in comments:
\begin{verbatim}
0 0 0 r 0 ; get to end (move to right if see a 0 on the tape)
0 1 1 r 0 ; get to end (move to right if see a 1 on the tape)
0 _ 1 r 1 ; adjoin a 1 to the end, multiplies by 2 and adds 1 (i.e. 2n+1)
1 _ 0 * halt; adjoin a 0 as last digit, which multiples by 2 in binary 

Or the following, not using the wildcard and halt notations:

0 0 0 r 0 ; get to end
0 1 1 r 0 ; get to end
0 _ 1 r 1 ; adjoin a 1 to the end
1 _ 0 r 2; adjoin a 0 as last digit
\end{verbatim}}

\item[] Again, remember to write your Turing-machine code for parts (a)--(c) such that it can be copy/pasted into morphett by your TA!!!



%********************************************************
\com{case of 4n+3

\item \question{Design a Turing Machine that has no more than two states and behaves as follows: when given as input a natural number $n \geq 1$ in binary notation, it halts with the number $4n +3$ in binary notation on an otherwise blank tape. (No need to worry about where the reader ends up.)

}

\answer{The following will do:
\begin{verbatim}
0 0 0 r 0 ; get to end
0 1 1 r 0 ; get to end
0 _ 1 r 1 ; add first digit
1 _ 1 * halt; add second digit
\end{verbatim}}
}%end com

\com{
\item \question{
Design a Turing-machine that does the following: when given as input a string of $n$ ones, followed by a blank, followed by a string of $m$ ones, it outputs a string of $n\times m$ ones. You may assume that $n$ and $m$ are greater than zero and use auxiliary symbols, if you need them. 
}

\answer{There are many different ways to do it, but this one will do:

\input{mutliplication-machine}

}} %end com 

\end{enumerate}
%********************************************************

%\clearpage 

\vspace{-1em}

  \item \question{When people characterize Turing Machines, they usually assume that the tape is ``infinitely long,'' or that it can be ``extended indefinitely.'' The purpose of the following two questions is get you to think about that. (4~points each)}
\begin{enumerate}

\item \question{Find a (total) function $f(n)$ from natural numbers to natural numbers which is Turing-computable, and which is such that, for any number $m$, there exists a number $k$ such that a Turing Machine would need a tape with more than $m$ cells to compute $f(k)$. }

\answer{Let $f$ be the identify function. In order to compute $f(n)$ a Turing Machine must print $n$ ones as output, and therefore use at least $n$ many cells on the tape. So for arbitrary $m$, one can let $k = m+1$.}


\item \question{Let $f(n)$ be whatever function you used to answer the previous question, and suppose that Turing Machine $M$ computes $f(n)$. Is there some number $k$ such that $M$ must use an infinite amount of tape to compute $f(k)$? }

\answer{ There can be no such number because $M$ can only compute $f(k)$ if it halts after a finite number of steps.}

\end{enumerate}
 

\com{ %included in 2019, 2021. Left out 2022, 2023, except part c. 
\item \question{Suppose you're interested in minimizing the number of states required by your Turing Machine. One strategy is to come up with a clever algorithm. Another strategy is to start with an algorithm that requires many states and bring down the number of states by increasing the number of auxiliary symbols that your Turing Machine is allowed to print on the tape. Use either of these strategies (or a combination of the two) to solve the following problems. Your Turing Machines are allowed to use as many auxiliary symbols as they need. (8~points each)}

\begin{enumerate}


\item \question{Design a Turing Machine that has no more than two states and behaves as follows: whenever it is given a sequence of $n$ ones as input ($n > 0$), it halts with a sequence of $n$ ones, followed by a blank, followed by a one, with the reader positioned at the left-most one of the initial sequence. 
}

\answer{The following will do:
\begin{verbatim}
0 1 A r 1
0 _ 1 l 0
0 B _ l 0
0 > 1 l 0
0 A 1 * halt

1 1 > r 1
1 _ B r 0
\end{verbatim}
}

\item \question{Design a Turing Machine that has no more than two states and behaves as follows: for some $n \geq 20$, when run on an empty input, it halts with $n$ cells of the tape containing non-blanks. (No need to worry about where the reader ends up.)
}

\answer{The following will do:
\begin{verbatim}
0 _ z l 1; turn left at right end
0 a a r 0; keep going right
0 z y r 0; keep going right
0 y x r 0; keep going right
0 x w r 0; keep going right
0 w v r 0; keep going right
0 v u r 0; keep going right
0 u t r 0; keep going right
0 t s r 0; keep going right
0 s r r 0; keep going right
0 r q r 0; keep going right

1 _ a r 0; turn right at left end
1 a a l 1; keep going left
1 y y l 1; keep going left
1 x x l 1; keep going left
1 w w l 1; keep going left
1 v v l 1; keep going left
1 u u l 1; keep going left
1 t t l 1; keep going left
1 s s l 1; keep going left
1 r r l 1; keep going left
1 q q l 1; keep going left
\end{verbatim}
}

\item \question{Design a Turing Machine that has no more than two states and behaves as follows: when given as input a natural number $n \geq 1$ in binary notation, it halts with the number $4n +3$ in binary notation on an otherwise blank tape. (No need to worry about where the reader ends up.)

}

\answer{The following will do:
\begin{verbatim}
0 0 0 r 0 ; get to end
0 1 1 r 0 ; get to end
0 _ 1 r 1 ; add first digit
1 _ 1 * halt; add second digit
\end{verbatim}
}



\end{enumerate}
} % end com 

\com{ %left out 2022 and 2023
  \item \question{The course materials include a proof that the Busy Beaver function is not Turing-computable. The following problems are aimed at getting you to think about that proof.
  }

\begin{enumerate}

\item  \label{first}
\question{For $k$ an arbitrary positive integer, design a (one-symbol) Turing Machine with exactly $k$ states that does the following, given an empty input: it produces a sequence of $k$ ones, brings the reader to the beginning of the sequence, and halts. (4~points)}

\answer{An example of such a machine is supplied in the answers to the exercises in Section~9.2.2:

\[
\begin{array}{ccccc}
0 &\ub &1 &l &1 \\
1 &\ub &1 &l &2 \\
\ldots \\
(k-1) &\ub &1 &* &\text{halt} \\
\end{array}
\]
[This machine has $k$ states.]
}

\item \label{doubler} \question{Design a (one-symbol) Turing machine that does the following, given a sequence of $n$ ones as input: it produces a sequence of $2n$ ones, brings the reader to the beginning of the sequence, and halts. \label{second} (4~points)}


\answer{An example of such a machine is supplied in the answers to the exercises in Section~9.1 of the book:

\[
\begin{array}{ccccc}
0 &1 &\ub &r &1\\
1 &1 &1 &r &1\\
1 &\ub &\ub &r &2\\
2 &1 &1 &r &2\\
2 &\ub &1 &r &3\\
3 &\ub &1 &l &4\\
4 &1 &1 &l &4\\
4 &\ub &\ub &l &5\\
5 &1 &1 &l &5\\
5 &\ub &\ub &r &6\\
6 &1 &\ub &r &1\\
6 &\ub &\ub &r &\text{halt}
\end{array}
\]


[This machine has seven states.]
}


\item \label{last} \question{Design a (one-symbol) Turing machine that does the following, given a sequence of $n$ ones as input: it produces a sequence of $n+1$ ones, brings the reader to the beginning of the sequence, and halts. (4~points)}

\answer{

\[
\begin{array}{ccccc}
0 &\ub &1 &* &\text{halt}\\
0 &1 &1 &l &1\\
1 &\ub &1 &* &\text{halt}
\end{array}
\]


[This machine has two states.]
}







\item \label{construction}
 \question{The proof in the course materials works with a hypothetical Turing Machine $M^I$. The characterization of $M^I$ presupposes a Turing Machine $M^{BB}$, which computes the Busy Beaver Function. Since the Busy Beaver Function is not Turing-Computable,  $M^{BB}$ doesn't actually exist (and so neither does $M^I$). For the purposes of this exercise, however, I'd like you to pretend that $M^{BB}$ does exist, and has the following program:


\begin{verbatim}
; fake version of BB machine
0 _ _ l 1 
0 1 1 r 0
1 _ _ r halt
1 1 1 l 1
\end{verbatim}


Let $k=b + c + d$, where $b$ is the number of states in your answer to \ref{doubler}, $c$ is the number of states in your answer to \ref{last}, and $d$ is the number of states in the fake version of $M^{BB}$ described above. 

Explicitly write out a program for $M^I$ by using  your answers to problems $(\ref{first})$--$(\ref{last})$ as subroutines and by pretending that the fake version of $M^{BB}$ above really does compute the Busy Beaver Function. Make sure you annotate your program so as to make it easy for your TA to understand which part of your  code corresponds to each of your answers to problems $(\ref{first})$--$(\ref{last})$ and which corresponds to the fake version of $M^{BB}$ above. (4~points)
}

\answer{
On the answers supplied above, $b = 7, c = 2$. And everyone should agree that $d = 2$. So we'll work with $k = 11$.


\begin{verbatim}
; (5a) machine
0 _ 1 l 1 
1 _ 1 l 2 
2 _ 1 l 3
3 _ 1 l 4 
4 _ 1 l 5 
5 _ 1 l 6 
6 _ 1 l 7 
7 _ 1 l 8 
8 _ 1 l 9 
9 _ 1 l 10 
10 _ 1 * 20 

; (5b) machine
20 1 _ r 21 
21 1 1 r 21
21 _ _ r 22
22 1 1 r 22
22 _ 1 r 23
23 _ 1 l 24
24 1 1 l 24
24 _ _ l 25
25 1 1 l 25
25 _ _ r 26
26 1 _ r 21
26 _ _ r 30

; fake BB machine
30 _ _ l 31 
30 1 1 r 30
31 _ _ r 40
31 1 1 l 31

; (5c) machine
40 _ 1 * halt 
40 1 1 l 41
41 _ 1 * halt
\end{verbatim}


}






\item \question{Let $k$ be as in your answer to problem~\ref{construction}. If the two-state machine I supplied in problem~(\ref{construction}) had really computed the Busy Beaver Function, then your answer to (\ref{construction}) would have computed the function $BB(2k) + 1$. So your code would have been more productive---by one---than the most productive Turing Machine with $2k$ states or fewer.
How many states does your answer to (\ref{construction}) have? Please give your answer as a function of $k$. (1~point)
}

\answer{$2k$.
}


\end{enumerate}
} %end com 




\end{enumerate}





\vspace{-1.8em}

\subsection*{Part II (justify all answers!)} 

\begin{enumerate}
  \setcounter{enumi}{3}
  

\item   \question{Could there be a function \(f\) such that, for distinct numbers \(n\) and \(m\), \(n\) and \(m\) both code Turing Machines that compute \(f\)? Note that your justification here can be succinct; you need not provide a concrete example (8~points). 
}
       
\answer{  Yes. One way to see this is to note that if a Turing Machine \(M\) with two or more command lines computes \(f\), then any Turing Machine that results from changing the order of \(M\)'s command lines will also compute \(f\), and is assigned a code different from the code  assigned to \(M\).

Another way: have the second Turing machine move the reader-head to a different position before halting. 
}

  
  
  \item \question{The following questions concern the limits of computability. Note that you do \textbf{NOT} have to write explicit command lines to justify your answers! (8~points each)}
  \begin{enumerate}
  
 
\item  \question{ Suppose $f$ is a Turing-computable function from natural numbers to natural numbers. Can there be a Turing Machine that, when given input $n$, outputs a one, if $f(n)=0$, and prints a string of two ones, if $f(n)\neq 0$? If yes, give a general description of how to design such a machine; if no, explain why not.}

\answer{Yes. Let $M$ be a Turing Machine that computes $f$. We construct a Turing Machine, $N$, that works as desired, as follows. Here is how $N$ works when given input $n$:\begin{itemize}
\item $N$ uses $M$ as a subroutine to check whether $f(n)=0$. 
\item If the answer is yes, then $N$ outputs a one. 
\item If the answer is no, then $N$ outputs a string of two ones.
\end{itemize}
}

\com{
\item \question{Can there be an algorithm that determines, for any Turing Machine $M$ and input $m$, whether $M$ halts when given input $m$? (Don't forget to justify your answer; a large part of your score will be determined by the justification you give.)}
}
\item\question{Let $M_{n}$ be the $n$th Turing Machine and let $f:\mathbb{N}\rightarrow\mathbb{N}$ be the following function:

$$
f(n) = 
\begin{cases}
1, \, \, \, \, \text{if the output of $M_{n}$ on input $n$ is 0.} \\
0, \, \, \, \, \text{if otherwise.}
\end{cases}
$$

Is $f$ Turing-computable? (Don't forget to justify your answer.)}

\answer{No, $f$ is not Turing-computable. Suppose, for \emph{reductio}, that it is, and let $M_{k}$ (i.e. the $k$th Turing Machine) be a Turing Machine that computes $f$.  Now suppose $M_{k}$ is given input $k$. Then if the output of $M_{k}$ is 0, then since $M_{k}$ computes $f$, the output of $M_{k}$ is 1. And if the output of $M_{k}$ is not zero, then again since $M_{k}$ computes $f$, the output of $M_{k}$ is 0. So the output of $M_{k}$ on input $k$ is 0 iff it is not 0. So the assumption that there is a Turing Machine that computes $f$ is false.}

   \end{enumerate}
 
%\com{   %left out in 2022: 
 \item  \question{For each of the descriptions below, determine whether there could be a Turing Machine satisfying that description.  (8~points each; don't forget to justify your answers)
}

\begin{enumerate}

\item \question{A Turing Machine $M$ that behaves as follows when given the code number of a Turing Machine $G$ as input (i.e. initialize $M$ on a string of 1s that code for the machine $G$):

\begin{itemize}
\item If $G$ halts when run on an empty input, $M$ halts.
\item If $G$ doesn't halt when run on an empty input, $M$ doesn't halt.  
\end{itemize}
}

\answer{Yes. All one needs is a Universal Turing Machine. Recall that we defined this universal machine $M^U$ as taking input a pair $\langle m, n \rangle$ where $m$ codes for the $m$th Turing Machine and $n$ is the input that the $m$th machine is run on (we can represent a pair $\langle m, n \rangle$ using a blank space in between $m$-many 1s and $n$-many 1s). So in this case, since we run $G$ on an empty input, we run the Universal Machine $M^U$ on $\langle g, 0 \rangle$, i.e. we initialize the Universal Turing machine on the code number for $G$, as the problem requires. 

By definition, $M^U$ halts on $\langle g, 0 \rangle = g$ provided the $g$th machine $G$ halts on an empty input, and $M^U$ does not halt if the $g$th machine $G$ does not halt on an empty input. So $M^U$ meets the required criteria.}

\item \question{For a given Turing Machine $G$, a Turing Machine $M$ that behaves as follows on an empty input:

\begin{itemize}
\item If $G$ halts when run on an empty input, $M$ outputs a $1$
\item If $G$ doesn't halt when run on an empty input, $M$ outputs a 0 
\end{itemize}
}

\answer{Yes. Note that we don't need a single machine $M$ that works no matter whether $G$ halts or not! If $G$ halts on an empty input, $M_1$ can be $\seq{0,\_,1,*,\text{halt}}$. \\ - If $G$ doesn't halt on an empty input, $M_2$ can be $\seq{0,\_,0,*,\text{halt}}$.

It may help to note a key difference in the logical form of part b vs. part a. Part b has the form $\forall G \, \exists M$, i.e. for every $G$, there is some $M$ that meets the criteria. This means that we might need different Ms to handle different $G$s with different behavior. And clearly part b involves different $G$s, since the two if-then statements involve mutually incompatible behavior, i.e. no single Turing machine $G$ could both halt when run on empty input AND not halt when run on empty input. 

Whereas in part a, the question has the form $\exists M \, \forall G$, i.e. it asks for a single $M$ that works for all $G$s that meet the criteria. }

\com{ %2023: deciding to bail on the following problem, to make grading easier 

\item \question{A Turing Machine $M$ that behaves as follows when given the code of a Turing Machine $G$ as input:

\begin{itemize}
\item If $G$ halts when run on an empty input, $M$ outputs a $1$
\item If $G$ doesn't halt when run on an empty input, $M$ outputs a 0 
\end{itemize}
}

\answer{No, since such a machine would allow us to compute the halting function, which we know not to be Turing computable.

To see this, we will suppose for reductio that $M$ exists and use it to prove the existence of a Turing machine $M^H$, which computes the (one-place) halting function $H(n)$.

Here's one way to define $M^H$, given input $n$:

\begin{itemize}

\item The first thing $M^H$ does is use its input to print out a representation of the Turing machine with code number $n$ on its tape. (Whether or not this can be done depends on the coding system one is presupposing, but it definitely can be done for the coding system provided in the course materials: one factors $n$ into primes and uses the exponents to write out command lines, using an easy translation method.)

\item Next, $M^H$ modifies the code represented on its tape by adding an initial subroutine that prints out n ones, returns the reader to the first one, and then runs the program originally on the tape. (That's easy to do: it's just a matter of adding n additional command lines, using up the first $n$ states, and then relabelling the original command lines so that each state $k$ becomes state $k+n$.)

\item $M^H$ now represents on its tape the code for some Turing machine $M^*$. The next thing $M^H$ has to do is use the information on its tape to find the number $m^*$ that codes $M^*$ on our coding scheme. (Again, whether or not that's possible depends on one's coding scheme. But it can definitely be done with the coding scheme in the course materials. $M^H$ starts by turning each of the command-line symbols on its tape into numbers, and use the resulting numbers as the exponents of a big multiplication of consecutive prime numbers.)

\item Finally, $M^H$ uses $M$ as a subroutine to establish whether the machine encoded by $m^*$ halts on an empty input. If the answer is yes, $M^H$ yields an output indicating that the machine coded by $n$ halts on input $n$; otherwise $M^H$ yields an output indicating that the machine coded by $n$ doesn't halt on input $n$.
\end{itemize}
}
} %end com 






\end{enumerate}

%\clearpage

\item A method, $M$, for solving some problem is called ``effective'' just in case:
\begin{enumerate}[label=(\roman*)]
\item $M$ is set out in terms of a finite number of exact instructions (each instruction being expressed by means of a finite number of symbols);
\item $M$ will, if carried out without error, produce the solution in a finite number of steps;
\item $M$ can (in practice or in principle) be carried out by a human being unaided by any machinery except paper and pencil;
\item $M$ demands no insight, intuition, or ingenuity, on the part of the human being carrying out the method.
\end{enumerate}
Consider the following thesis:
\begin{description}
\item[Church--Turing Thesis] Every function that can be calculated by an effective method is Turing-computable.
\end{description}
(Note that the Church--Turing Thesis stated here is slightly different from the one stated in the course materials. For the purpose of this problem, please refer to the thesis as it is stated here.)

The following questions are meant to get you to think about the Church-Turing Thesis.

\begin{enumerate}
\item \question{Suppose there is a physical process that allows for the calculation of functions not computable by any Turing machine. (By ``physical process'' we mean any process that is in accordance with the actual laws of physics.)  (10~points)



\begin{enumerate}
\item Would the existence of such a process refute the Church-Turing Thesis? \\ \textbf{Consider both} Case $\alpha$) the process provides an effective method \\ \textbf{and} Case $\beta$) the process does not provide an effective method.

\item In Case $\beta$), \textbf{comment on which} of the four criteria for an \textit{effective method}---criteria (i)-(iv) above---\textbf{is most likely to fail} for a physical process.

\end{enumerate}
}
% \\ -- Would the existence of such a process refute the Church-Turing Thesis? \\ \textbf{Consider both} Case $\alpha$) the process provides an effective method \\ \textbf{and} Case $\beta$) the process does not provide an effective method. \\ -- In Case $\beta$), \textbf{comment on which} of the four criteria for an \textit{effective method}---criteria (i)-(iv) above---\textbf{is most likely to fail} for a physical process (10~points)}

\answer{Case $\alpha$): If the physical process provides an effective method for calculating a function that is not Turing computable, then it WOULD provide a counter-example to the Church-Turing Thesis. We would have found a function that CAN be calculated by an effective method but that is not Turing computable. 

Case $\beta$) If the physical process does NOT provide an effective method for calculating the function, then it poses no threat to the Church-Turing Thesis. 

Provided a student writes something cogent, accept all student comments on which of the four criteria (i)--(iv) is most likely to fail. 

The hope is that students will comment briefly on the nature of a physical process that computes a function in a NON-effective way. Which of the four conditions for an effective method is most likely to fail? Will it vary case-by-case? 

If supertasks can be physically realized (e.g. perhaps by accelerating an object arbitrarily close to the speed of light or by relying on time dilation effects with black holes), then condition (ii) could be violated: we could perform infinitely-many tasks in finite time. 

Perhaps condition (iii) can be violated: carrying out a physical process typically requires more than pencil and paper (e.g. we might need some lab equipment). But what if we could simply \textit{simulate} the physical process according to some theory: this is presumably something we could do in principle with pencil and paper. So condition (iii) might seem more restrictive than it really is. 

Regarding condition (iv), although it might require ingenuity to discover such a physical process, normally we then learn how to write directions step-by-step for reproducing a physical phenomena (e.g. a standard lab procedure). 

%A good answer shows some awareness of the scope of the Church-Turing Thesis and gives a succinct explanation of why the imagined scenario  would or wouldn't refute it, depending on what we take the scenario to be.  For example, one good answer might explain that it depends on what exactly the physical process is. If the physical process is not an effective method (or couldn't be simulated by an effective method), then its existence doesn't contradict the thesis. If it is an effective method (or could be simulated by an effective method), then its existence does contradict the thesis.
}

\item \question{Imagine that scientists discover that the behavior of the brain can be completely described by some mathematical function. Given the Church--Turing Thesis, would this discovery entail that the behavior of the brain can be simulated by a Turing Machine? (8~points)}



\answer{A good answer that would deserve full credit: the discovery would not necessarily entail that the behavior of the brain can be simulated by a Turing Machine, because the function in question might not be calculable by an effective method. 

An affirmative answer might also deserve close to full credit provided it is well-justified, but I have a hard time imagining how someone could do this\dots}

%\item \question{Food for thought (not graded):  Which of the four conditions for an \textit{effective method} is a physical process most likely to violate?}


\end{enumerate}




  
  
  
  
  
  \end{enumerate} 



\end{document}






