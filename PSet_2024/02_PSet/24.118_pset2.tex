

\documentclass[12pt,letterpaper]{article}
\usepackage{../pset_2024}


%Questions and Answers
\qa{q} % a="answers only"; q ="questions only"; b="both"
\usepackage{../qa}


\begin{document}

\psintro{Problem Set 2: The Higher Infinite}

%%%%%%%%%%%%%%%%%%%%%%%%%%%

\newpage



\subsection*{Part I (Quiz on Canvas: 46 points)} 
%currently 46 = 10 +12 +12 +4 +8 points
%part II: 54 = 10+10+5+21+8 



\begin{enumerate}




%left out for variety
%\com{

\item[] \question{\emph{Notation:} \(\emptyset\) is the empty set. If \(A\) and \(B\) are sets, $\powerset(A)$  is the set of $A$'s subsets and \(A-B\) is the set whose members are the elements of \(A\) that are not also elements of \(B\) (so, for instance, \(\{1,2\}-\{1\}=\{2\}\)).}

\item \question{Answer the following questions. (2 points each)}

\begin{enumerate}

\item \question{Is $\powerset(\mathbb{Z})-\powerset(\mathbb{N})$  well-ordered by $\subseteq$?}

\answer{\noindent No, it is not. For instance, neither $\{-1 \} \in \{-2 \}$ nor $\{-2 \} \in \{-1 \}$. So the set is not a total order under the subset relation and hence not a well-order}

\item \question{Is $\powerset(\powerset(\emptyset))$ well-ordered by $\in$?}

\answer{Yes, it is: Note that $\powerset(\powerset(\emptyset)) = \{ \emptyset,  \{ \emptyset \} \}$. Since $ \emptyset \in \{ \emptyset \}$, we can totally order these elements as $\emptyset <  \{ \emptyset \}$. Clearly, every non-empty subset has a $<$-smallest member. }

\com{ %Used in 2022; seems better discussed in lecture 
\item \question{Is $\powerset(\emptyset)$ well-ordered by $\in$?}

\answer{Yes, it is.}
}%end com

\item  \question{Is \(\powerset^4(\emptyset)=\powerset(\powerset(\powerset( \powerset (\emptyset))))\) well-ordered by \(\in\)?} 
  
 \answer{ No, it is not.   Note that $\powerset^4(\emptyset)$ contains 16 elements, including $\emptyset$ and $\{\{\emptyset\}\}$. Yet neither $\emptyset \in\{\{ \emptyset\}\}$ nor  $\{\{ \emptyset\}\} \in \emptyset$. So it is not even a total order.   }

\com{ %Used in 2022
\item  \question{Is \(\powerset(\powerset( \powerset (\emptyset)))\) well-ordered by \(\in\)?} 
  
 \answer{ No, it is not.        \(\mathcal P (\mathcal P (\mathcal P (\emptyset))) = \{ \emptyset, \{\emptyset\}, \{\{ \emptyset \}\}, \{ \emptyset, \{\emptyset\} \} \}\) 
      and neither \(\emptyset \in \{\{\emptyset\}\}\) nor \(\{\{\emptyset\}\}\ \in\{\emptyset, \{\emptyset\} \} \).}
}%end com


\item  \question{Is \(\powerset^3(\emptyset)- \{ \{ \{ \emptyset\}\}\} =\powerset (\powerset (\powerset (\emptyset))) - \{ \{ \{ \emptyset\}\}\}\) well-ordered by \(\in\)?} 

\answer{    Yes, it is. Note that \(\mathcal P (\mathcal P (\mathcal P (\emptyset))) - \{ \{ \{ \emptyset\}\}\}\) is NOT the result of removing \(\{ \{ \{ \emptyset\}\}\}\) from amongst the elements of \(\mathcal P (\mathcal P (\mathcal P (\emptyset)))\), but the result of removing \( \{ \{ \emptyset\}\}\) from amongst the elements of \(\mathcal P (\mathcal P (\mathcal P (\emptyset)))\). (This is because \(A - B\) is not the result of eliminating set \(B\) from amongst the elements of \(A\) but the result of removing any \emph{elements} of \(B\) from amongst the elements of \(A\).) Since
\(\mathcal P (\mathcal P (\mathcal P (\emptyset))) = \{ \emptyset, \{\emptyset\}, \{\{ \emptyset \}\}, \{ \emptyset, \{\emptyset\} \} \}\)
this means that \(\mathcal P (\mathcal P (\mathcal P (\emptyset))) - \{ \{ \{ \emptyset\}\}\} = \{ \emptyset, \{\emptyset\},  \{ \emptyset, \{\emptyset\} \} \} \), which is well-ordered by \(\in\). (In fact, it's an ordinal.)

}

\item \question{Is \(\powerset^3(\emptyset)- \{ \{ \{ \emptyset\}\}\} \) an ordinal number?}

\answer{ Yes! Since \(\mathcal P (\mathcal P (\mathcal P (\emptyset))) - \{ \{ \{ \emptyset\}\}\} = \{ \emptyset, \{\emptyset\},  \{ \emptyset, \{\emptyset\} \} \} \), which we denote as $0''$.}





\com{ %Used in 2022 
\item \question{Is $\powerset(\mathbb{N})$  well-ordered by $\subseteq$?}

\answer{No, it is not.}
}%end com 



 
 \end{enumerate}

%}



%
\item \question{Recall the following definitions:

\begin{itemize}



\item $\mathbb{N}$ is the set of natural numbers. 

\item $\powerset^n(A) = \underbrace{\powerset(\powerset(\dots\powerset(}_{\text{\tiny $n$ times}}A\underbrace{)\dots))}_{\text{\tiny $n$ times}}$ ($n \in \mathbb{N}$).

\item $\bigcup A = \set{x : x \in B \text{ for } B \in A}$. 



\end{itemize}
Determine whether each of the following statements is true or false. You may assume that neither the natural numbers nor the integers are sets. e.g. The number `$118$' is not a set, nor is $-5$. (2~points each)}


\begin{enumerate} %start of 2023 version; see below for an earlier version with the naturals 

\item  \question{$\powerset^n(\mathbb{Z}) \subseteq \powerset^m(\mathbb{Z})$, for $n < m$ and $n,m \in \mathbb{N}$.}

\answer{False. For instance, $\set{0} \in \powerset^1(\mathbb{Z})$, but $\set{0} \notin \powerset^2(\mathbb{Z})$, since we're assuming that $0$ is not a set, and therefore that $0 \notin \powerset^1(\mathbb{Z})$, which consists entirely of sets.}


\item \question{$|\powerset^n(\mathbb{Z})| < |\powerset^m(\mathbb{Z})|$, for $n < m$ and $n,m \in \mathbb{N}$.}

\answer{True, by Cantor's Theorem.}

\item \question{For arbitrary $n \in \mathbb{N}$, $\powerset^n(\mathbb{Z}) \subseteq \set{\powerset^m(\mathbb{Z}): m \in \mathbb{N}}$.}

\answer{False. For instance, $\powerset^1(\mathbb{Z})$  contains $\mathbb{Z} - \set{0}$, which is not a member of $\set{\powerset^m(\mathbb{Z}): m \in \mathbb{N}}$.}

\item \question{For arbitrary $n \in \mathbb{N}$, $\powerset^n(\mathbb{Z}) \subseteq \bigcup \set{\powerset^m(\mathbb{Z}): m \in \mathbb{N}}$.}

\answer{True. $\powerset^n(\mathbb{Z}) = \set{x : x \in \powerset^n(\mathbb{Z})} \subseteq \set{x : x \in \powerset^m(\mathbb{Z}) \text{ for arbitrary } m \in \mathbb{N}} = $ \\ $\bigcup \set{\powerset^m(\mathbb{Z}): m \in \mathbb{N}}$.
}

\item \question{$|\powerset^n(\mathbb{Q})| < | \set{\powerset^m(\mathbb{Q}): m \in \mathbb{N}}|$, $n \in \mathbb{N}$}

\answer{False. $| \set{\mathbb{Q}, \powerset^1(\mathbb{Q}), \powerset^2(\mathbb{Q}), \dots}| = |\mathbb{N}|$. But, by Cantor's Theorem, $|\mathbb{N}| < |\powerset^n(\mathbb{N})|$ whenever $0 < n$, and since $|\mathbb{Q} | = |\mathbb{N}|$, $|\powerset^n(\mathbb{Q})| = |\powerset^n(\mathbb{N})|$ }

\item \question{$|\powerset^n(\mathbb{R})| < |\bigcup \set{\powerset^m(\mathbb{R}): m \in \mathbb{N}}|$, $n \in \mathbb{N}$}

\answer{True, since $\bigcup \set{\mathbb{R}, \powerset^1(\mathbb{R}), \powerset^2(\mathbb{R}), \dots}$ includes everything in $\powerset^{n+1}(\mathbb{R})$ and, by Cantor's Theorem, $|\powerset^n(\mathbb{R})| < |\powerset^{n+1}(\mathbb{R})|$.}

%\item \question{$|\powerset^n(\mathbb{Q})| < |\bigcup \set{\powerset^m(\mathbb{Q}): m \in \mathbb{N}}|$, $n \in \mathbb{N}$}

%\answer{True, since $\bigcup \set{\mathbb{Q}, \powerset^1(\mathbb{Q}), \powerset^2(\mathbb{Q}), \dots}$ includes everything in $\powerset^{n+1}(\mathbb{Q})$ and, by Cantor's Theorem, $|\powerset^n(\mathbb{Q})| < |\powerset^{n+1}(\mathbb{Q})|$.}

\end{enumerate}


\com{ %2022 version; modifying for 2023 to be based on integers! 

Determine whether each of the following statements is true or false. You may assume that the natural numbers are not sets. (2~points each)

\begin{enumerate}

\item  \question{$\powerset^n(\mathbb{N}) \subseteq \powerset^m(\mathbb{N})$, for $n < m$ and $n,m \in \mathbb{N}$.}

\answer{False. For instance, $\set{0} \in \powerset^1(\mathbb{N})$, but $\set{0} \notin \powerset^2(\mathbb{N})$, since we're assuming that $0$ is not a set, and therefore that $0 \notin \powerset^1(\mathbb{N})$, which consists entirely of sets.}


\item \question{$|\powerset^n(\mathbb{N})| < |\powerset^m(\mathbb{N})|$, for $n < m$ and $n,m \in \mathbb{N}$.}

\answer{True, by Cantor's Theorem.}

\item \question{$\powerset^n(\mathbb{N}) \subseteq \set{\powerset^m(\mathbb{N}): m \in \mathbb{N}}$, $n \in \mathbb{N}$.}

\answer{False. For instance, $\powerset^1(\mathbb{N})$  contains $\mathbb{N} - \set{0}$, which is not a member of $\set{\powerset^m(\mathbb{N}): m \in \mathbb{N}}$.}

\item \question{$\powerset^n(\mathbb{N}) \subseteq \bigcup \set{\powerset^m(\mathbb{N}): m \in \mathbb{N}}$, $n \in \mathbb{N}$.}

\answer{True. $\powerset^n(\mathbb{N}) = \set{x : x \in \powerset^n(\mathbb{N})} \subseteq \set{x : x \in \powerset^m(\mathbb{N}) \text{ for } m \in \mathbb{N}} = $ \\ $\bigcup \set{\powerset^m(\mathbb{N}): m \in \mathbb{N}}$.
}

\item \question{$|\powerset^n(\mathbb{N})| < | \set{\powerset^m(\mathbb{N}): m \in \mathbb{N}}|$, $n \in \mathbb{N}$}

\answer{False. $| \set{\mathbb{N}, \powerset^1(\mathbb{N}), \powerset^2(\mathbb{N}), \dots}| = |\mathbb{N}|$. But, by Cantor's Theorem, $|\mathbb{N}| < |\powerset^n(\mathbb{N})|$ whenever $0 < n$.}

\item \question{$|\powerset^n(\mathbb{N})| < |\bigcup \set{\powerset^m(\mathbb{N}): m \in \mathbb{N}}|$, $n \in \mathbb{N}$}

\answer{True, since $\bigcup \set{\mathbb{N}, \powerset^1(\mathbb{N}), \powerset^2(\mathbb{N}), \dots}$ includes everything in $\powerset^{n+1}(\mathbb{N})$ and, by Cantor's Theorem, $|\powerset^n(\mathbb{N})| < |\powerset^{n+1}(\mathbb{N})|$.}

\end{enumerate}

}%end of long com for this problem 






% Left out for variety %%Hunt using as practice problems in 2023 lecture!; also included in the hidden HW questions for Topic 2, MITx, Problem 2. 
\com{

\item \question{Determine whether each of the following statements is true or false. (2 points each)}

\begin{enumerate}

\item \question{\(0' + 0''' = 0''' + 0'\)}

\answer{True. \(0' + 0''' = 0'''' = 0''' + 0'\).}    


\item  \question{\(0' \times 0''' = 0''' \times 0'\)}

\answer{True. \(0' \times 0''' = 0''' = 0''' \times 0'\).}


    
\item  \question{\(0' + \omega = \omega' + 0\)}
   
\answer{ False. \(0' + \omega = \omega\), but \(\omega' + 0\) is \(\omega'\).}
    
\item  \question{\(0' + \omega = 0 + \omega'\)}
   
\answer{ False. \(0' + \omega = \omega\), but \(0 + \omega'\) is \(\omega'\).}


    
      
%left out for variety
\com{
\item  \question{\(0' + \omega = \omega + 0'\)}
   
\answer{ False. \(0' + \omega = \omega\), but \(\omega + 0'\) is greater than \(\omega\).}

}

%left out for variety
\com{
\item \question{$0'' + 0' <_o 0 + 0''' $}

\answer{False. \(0 + 0''' =  0''' = 0'' + 0'\).}
}


%left out for variety
\com{
\item  \question{\((\omega + 0'') + \omega <_o (\omega + \omega) + 0''\)} 

\answer{True. \((\omega + 0'') + \omega = \omega + \omega\), which is smaller than \((\omega + \omega) + 0''\). }
}


%left out for variety
\com{
\item \question{$\omega \times 0'''$ = $0''' \times \omega$}

\answer{False. $\omega \times 0''' = \omega + \omega + \omega$ but $0''' \times \omega = \omega$}
}


\item \question{\(0''' \times \omega = (\omega + \omega) + \omega\)}
       
\answer{False.  \(0''' \times \omega = \omega\), which is smaller than \((\omega + \omega) + \omega\).}


\item  \question{\(\omega \times 0''' = \omega + (\omega + \omega)\)}

\answer{True. \(\omega \times 0''' = \omega + (\omega + \omega)\).}  


\item  \question{\((\omega \times 0'') + \omega <_o (\omega \times \omega) + 0''\)} 

\answer{True. \((\omega \times 0'') + \omega = ((\omega + \omega) + \omega)\), which is much smaller than \(\omega \times \omega\) (and therefore much smaller than \((\omega \times \omega) + 0''\).)}


 \item  \question{\(\omega \times \omega <_o \omega \times (0'' \times \omega)\)} 

\answer{False. \(\omega \times (0'' \times \omega) = \omega \times \omega\). }


 \item   \question{\(\omega \times (\omega + \omega) = (\omega \times \omega) + (\omega \times \omega)\)}

\answer{True. \(\omega \times (\omega + \omega)\) is the result of using \((\omega + \omega)\) as a template, and filling each position with an \(\omega\)-sequence. So we get an \(\omega\)-sequence of \(\omega\)-sequences followed by an \(\omega\)-sequence of \(\omega\)-sequences. \((\omega \times \omega) + (\omega \times \omega)\) is the result of starting with a sequence of type \((\omega \times \omega)\) (i.e. an \(\omega\)-sequence of \(\omega\)-sequences), and appending another sequence of type \((\omega \times \omega)\) (i.e. another \(\omega\)-sequence of \(\omega\)-sequences) to the right. So, again, we get an \(\omega\)-sequence of \(\omega\)-sequences followed by an \(\omega\)-sequence of \(\omega\)-sequences.}

\item  \question{\(\alpha + 0' = \alpha \cup \{\alpha\}\) ($\alpha$ an ordinal)}

\answer{True. It follows from the Construction Principle that \(\alpha \cup \{\alpha\} = \alpha'\), and it follows from the definition of addition that  \(\alpha + 0' = (\alpha + 0)' = \alpha'\). Putting the two together, \(\alpha \cup \{\alpha\} = \alpha'\) .}



\end{enumerate}

}







\item \question{Recall the following definitions:

\begin{itemize}

\item $\alpha <_o \beta \leftrightarrow_{\text{\emph{df}}} \alpha \in \beta$ ($\alpha, \beta$ ordinals)

\item $|A| < |B| \leftrightarrow_{\text{\emph{df}}}$ there is an injection from $A$ to $B$ but no bijection


 \item \(
\mathfrak{B}_\alpha=
\begin{cases}
\mathbb{N}, \text{ if $\alpha = 0$}\\
\powerset(\mathfrak{B}_\beta), \text{ if $\alpha = \beta'$}\\
\bigcup \{\mathfrak{B}_\gamma : \gamma <_o \alpha\} \text{ if $\alpha$ is a limit ordinal greater than $0$}
\end{cases}
\)

\item $\beth_\alpha \text{ is the $<_o$-smallest ordinal of cardinality } |\mathfrak{B}_\alpha|$


%\item $\beth_\alpha < \beth_\beta \leftrightarrow_{\text{\emph{df}}} |\beth_\alpha | < |\beth_\beta|$

\end{itemize}
Which of the following are true? (2~points each)
}

\begin{enumerate}

\item \question{$\omega <_o \omega +\omega$}

\answer{True. Since $\omega + \omega = \set{0, 0',\dots, \omega, \omega+1, \omega+2,\dots}$, $\omega \in \omega+\omega$. So $\omega <_o \omega+\omega$.}

\item \question{$|\omega| < |\omega \times \omega|$}

\answer{False. $\omega$ and $\omega \times \omega$ both have the size of the natural numbers. (You can show that $\omega \times \omega$ is countable by using a technique similar to the one we used to show that the rational numbers are countable.)}


\item \question{$\omega \times \omega <_o \beth_{0}$}

\answer{False. $\beth_0$ is just $\omega$, since by definition $\beth_0$ is the smallest-ordinal of cardinality $|\mathfrak{B}_0|$, and by definition $\mathfrak{B}_0 = \mathbb{N}$ }

\item \question{$|\omega \times \omega| < |\beth_{0}|$}

\answer{False. Since $\omega \times \omega$ and $\beth_0 = \omega$ are both countably infinite, we have $|\omega \times \omega| = |\beth_{0}|$.}

\item \question{Let `$118^o$' denote the ordinal named by $0$ followed by 118 prime symbols. Claim: $\beth_{118^o} <_o \beth_{\omega}$}

\answer{True. Since $118^o <_o \omega$, $|\mathfrak{B}_{118^o}| < |\mathfrak{B}_\omega|$. So $\beth_{118^o} <_o \beth_{\omega}$.}

\item \question{$|\beth_{118^o}| < |\beth_{\omega}|$}

\answer{True. Since $118^o <_o \omega$, $|\mathfrak{B}_{118^o}| < |\mathfrak{B}_\omega|$. And we have $|\beth_\alpha| = |\mathfrak{B}_\alpha|$.}

\com{ %modifying slightly for 2023 version

\item \question{$\beth_0 <_o \beth_{\omega}$}

\answer{True. Since $0 <_o \omega$, $|\mathfrak{B}_0| < |\mathfrak{B}_\omega|$. So $\beth_0 <_o \beth_{\omega}$.}

\item \question{$|\beth_0| < |\beth_{\omega}|$}

\answer{True. Since $0 <_o \omega$, $|\mathfrak{B}_0| < |\mathfrak{B}_\omega|$. And we have $|\beth_\alpha| = |\mathfrak{B}_\alpha|$.}
} %end com 

\end{enumerate}

\item \question{Let $\mathscr{U} = \bigcup \set{\powerset^m(\mathbb{N}): m \in \mathbb{N}}$. Answer the following questions (2 points each).}

\begin{enumerate}

\item \question{Does $\mathscr{U}$ also contain a set $\underbrace{\{\{\ldots\{\{}_{\mbox{\scriptsize $\infty$ times}}118\underbrace{\}\}\ldots\}\}}_{\mbox{\scriptsize $\infty$ times}}$? }

\answer{No. For the only members of $\mathscr{U}$ are members of $\mathcal{P}^n(\mathbb{N})$ for some $n$, and no $\mathcal{P}^n(\mathbb{N})$ contains $\underbrace{\{\{\ldots\{\{}_{\mbox{\scriptsize $\infty$ times}}118\underbrace{\}\}\ldots\}\}}_{\mbox{\scriptsize $\infty$ times}}$.}


\item \question{Is the cardinality of the following set greater than the cardinality of $\powerset^n(\mathscr{U})$ for each $n \in \mathbb{N}$? \[
\bigcup \{\mathbb{N}, \powerset(\mathbb{N}), \powerset(\powerset(\mathbb{N})), \dots, \mathscr{U}, \powerset(\mathscr{U}), \powerset(\powerset(\mathscr{U})), \dots\}
\] }

\answer{Yes. One can verify that it's bigger than each set on our list because it contains that set's successor (which is bigger by Cantor's Theorem.)}


\end{enumerate}


\com{ 
\item \question{Where $\mathscr{U} = \bigcup \set{\powerset^m(\mathbb{N}): m \in \mathbb{N}}$, answer the following questions. (5~points each; don't forget to justify your answers)
}

\begin{enumerate}

\item \question{Show that $\mathscr{U}$ contains the set $\underbrace{\{\{\ldots\{\{}_{\mbox{\scriptsize $n$ times}}17\underbrace{\}\}\ldots\}\}}_{\mbox{\scriptsize $n$ times}}$ for each $n>0$.}

\answer{Let $n$ be an arbitrary natural number greater than 0. Then $\mathcal{P}^n(\mathbb{N})$ contains the set $\underbrace{\{\{\ldots\{\{}_{\mbox{\scriptsize $n$ times}}17\underbrace{\}\}\ldots\}\}}_{\mbox{\scriptsize $n$ times}}$. Since $\mathscr{U}$ contains every member of $\mathcal{P}^n(\mathbb{N})$, $\mathscr{U}$ must also contain  $\underbrace{\{\{\ldots\{\{}_{\mbox{\scriptsize $n$ times}}17\underbrace{\}\}\ldots\}\}}_{\mbox{\scriptsize $n$ times}}$.}

\item \question{Does $\mathscr{U}$ also contain a set $\underbrace{\{\{\ldots\{\{}_{\mbox{\scriptsize $\infty$ times}}17\underbrace{\}\}\ldots\}\}}_{\mbox{\scriptsize $\infty$ times}}$? }

\answer{No. For the only members of $\mathscr{U}$ are members of $\mathcal{P}^n(\mathbb{N})$ for some $n$, and no $\mathcal{P}^n(\mathbb{N})$ contains $\underbrace{\{\{\ldots\{\{}_{\mbox{\scriptsize $\infty$ times}}17\underbrace{\}\}\ldots\}\}}_{\mbox{\scriptsize $\infty$ times}}$.}


\item \question{
Give an example of a set whose cardinality is greater than the cardinality of  $\powerset^n(\mathscr{U})$ for each $n \in \mathbb{N}$.}

\answer{
This will do it:
\[
\bigcup \{\mathbb{N}, \powerset(\mathbb{N}), \powerset(\powerset(\mathbb{N})), \dots, \mathscr{U}, \powerset(\mathscr{U}), \powerset(\powerset(\mathscr{U})), \dots\}
\] 
One can verify that it's bigger than each set on our list because it contains that set's successor (which is bigger by Cantor's Theorem.)}


\end{enumerate}
} %end long com 

%Used to be on Part II, but easily made true/false question for Part 1
\item \question{Determine whether the following are true or false (2 points each)}

%To justify your answer, you may use a diagram (or use prose) to give an informal characterization of the respective well-order types. (3 points each.)

%\answer{ In the answers below, ``omega-sequence'' is defined as in the problem statement of Problem 3. Note: students may also use the recursive definitions of ordinal addition and multiplication from the lecture notes.}

\begin{enumerate}
\item \question{$(\omega+0'')+(\omega+0')=(\omega+\omega)+(\omega+0')$ } 

\answer{ False.  Left-hand side: two consecutive omega sequences followed by one item ($\omega+\omega+0'$). Right-hand side: three consecutive omega sequences followed by one item ($\omega+\omega+\omega+0'$).} 

\item \question{$(0'+\omega)\times 0'''=(0'\times 0''')+(\omega\times 0''')$}

\answer{True. Both sides: three consecutive omega sequences ($\omega+\omega+\omega$).}

\item \question{$(\omega+0'')\times 0''''=(\omega\times 0'')+(0''\times 0'''')$}

\answer{False. Left-hand side: four consecutive sequences of the following type: an omega sequence followed by two items; or, equivalently, four consecutive omega sequences followed by two items ($\omega+\omega+\omega+\omega+0''$). Right-hand side: two  omega sequences followed by eight items ($\omega+\omega+0'''''''')$. }

\item \question{$(\omega + 0')+0''=(0'+\omega)+0''$}

\answer{ False. Left-hand side: one omega sequence followed by three items ($\omega+0'''$). Right-hand side: one omega sequence followed by two items ($\omega+0''$).}
\end{enumerate}




\end{enumerate}



%%%%%%%%%%%
%PART II
%%%%%%%%%%%%
\subsection*{Part II (Submit PDF on Canvas: 54 points)} 

\begin{enumerate}
  \setcounter{enumi}{5} % use this to set the counter for what number appears first

%used to be on Part 1 but can't code into Canvas Quiz. 
\item \question{Draw a diagram (or use prose) to give an informal characterization of the well-ordering types represented by each of the following ordinals. (2 points each; no need to justify answer but feel free to show work) \\ [1ex]
What does it mean to use prose to give an informal characterization of a well-order type? Suppose, for example, that the well-order type in question corresponded to $\omega$. Then you might say something like ``A countably infinite sequence of items which is ordered like the natural numbers, with an $n$th member for each $n \in \mathbb{N}$)---but no last member." Drawing diagrams is probably easier than using your words!}

\answer{In the answers below, ``omega-sequence'' is defined as in the problem statement.}

\begin{enumerate}

\item \question{$(\omega + 0'') \times 0'''$ }

\answer{Three consecutive sequences of the following type: an omega sequence followed by two items}



\item \question{$(\omega \times 0''') \times 0''$ }

\answer{Two consecutive sequences of the following type: three consecutive omega-sequences}



\item \question{$(0'''''' \times \omega) \times 0''$ }

\answer{Two omega-sequences}


\item \question{$(\omega \times \omega) + \omega$}

\answer{An omega-sequence of omega-sequences, followed by an omega-sequence}


\item \question{$(\omega \times \omega) \times \omega$}

\answer{An omega-sequence of sequences of the following type: an omega-sequence of omega-sequences}

\end{enumerate}
%\question{What does it mean to use prose to give an informal characterization of a well-order type? Suppose, for example, that the well-order type in question corresponded to $\omega$. Then you might say something like ``A countably infinite sequence of items which is ordered like the natural numbers, with an $n$th member for each $n \in \mathbb{N}$)---but no last member."}



%\com{ 
  \item \question{Recall that a relation constitutes a strict total order on a set just in case it is \\ (i) asymmetric (so irreflexive as well and hence `strict'), (ii) transitive, and (iii) total. \\ Using an example not found in the course material/lecture:}
  
  \begin{enumerate}
  
  \item \question{Specify (i) a set whose members are not numbers and (ii) an ordering on that set that is not a strict total ordering. (5 points)}
  
  \answer{Check whether student example works. They should supply a CONCRETE, specific counterexample to the set being a strict total ordering.}
  
  
  %\answer{Check whether student example works. A proof that it works is not strictly required; merely specifying a set and an appropriate ordering is sufficient.}

  \item \question{Specify (i) a set whose members are not numbers and (ii) a strict total ordering on that set that is not a well-ordering. You do not need to rigorously prove that the relation constitutes a strict total ordering. (5 points)}
  
   \answer{Check whether student example works. They should supply a CONCRETE, specific counterexample to the set being a well-ordering. They don't need to hardcore prove it is a strict total ordering.}
  
   
 % \answer{Check whether student example works. A proof that it works is not strictly required; merely specifying a set and an appropriate ordering is sufficient.}
  
  \end{enumerate}
  
%  }



%Left out for variety
%\com{
\item \question{Recall that  $\alpha <_o \beta$ is defined as $\alpha \in \beta$, for $\alpha$ and $\beta$  ordinals. Does $\alpha <_o \beta$ entail $|\alpha| < |\beta|$? If so explain, why. If not, give a counterexample. (5 points)}
%}

\answer{$\omega <_{o} \omega'$ but $|\omega| = |\omega'| = |\mathbb{N}|$.}

\item \question{Recall that we think of the ordinals as introduced in stages,  in accordance with the following principles:

\begin{description}
\item[Open-Endedness Principle] However many stages have occurred, there is always a ``next'' stage: a first stage after every stage considered so far.


\item[Construction Principle] 
At each stage, we introduce a new\footnote{A new ordinal is an ordinal that has not been introduced at previous stages.} ordinal, namely: the set of all ordinals that have been introduced at previous stages. 

\end{description}
Use these principles to give an informal justification of each of the following propositions. (7~points each; don't forget to justify your answers)}

\begin{enumerate}


\item \question{No ordinal is a member of itself.}

\answer{Since the Construction Principle is our only way of introducing new ordinals, the construction process guarantees that each time we introduce an ordinal we introduce a ``new'' ordinal, that is, an ordinal that hadn't been introduced at previous stages. And, again by the Construction Principle, that new ordinal contains all and only ordinals that had been introduced earlier. So it cannot be a member of itself.}

\item \question{For any ordinal $\alpha$, either $\alpha = \set{}$ or $\set{} <_o \alpha$. }

\answer{$\emptyset$ is an ordinal since it is introduced at the first stage of the process (i.e.~the ``next'' stage after nothing has happened). Since the Construction Principle is our only principle of set introduction, it guarantees that every ordinal is the set of ordinals introduced at previous stages. So the ordinal introduced at the first stage (which is $\emptyset$) must be a member of every other ordinal. And,  by definition, $\omega <_o \alpha \leftrightarrow \omega \in \alpha$. }


\item \question{If $\alpha$ is an ordinal with infinitely many members, then either $\alpha = \omega$ or  $\omega <_o \alpha$.}

\answer{Since each ordinal is the set of its predecessors, no finite stage of the process can yield an infinite ordinal. So the first stage at which our process  yields an infinite ordinal must be an infinite stage. Indeed, it is the first infinite stage---the stage at which $\omega$ is introduced---since $\omega$ is indeed infinite. So any infinite ordinal distinct from $\omega$ must be introduced at some later stage of the process. But, by the construction principle, every ordinal that is introduced after $\omega$ must contain $\omega$. And, by definition, $\omega <_o \alpha \leftrightarrow \omega \in \alpha$.
}



\end{enumerate}


\item \question{Give an example of a set whose cardinality is ``much greater'' than $|\mathfrak{B}_{\omega \times \omega^\omega}|$, in the sense that there are infinitely many sizes of infinity between the set you identify and $|\mathfrak{B}_{\omega \times \omega^\omega}|$.   (8~points; don't forget to justify your answer.)
}

\answer{One such set is $\mathfrak{B}_{(\omega \times \omega^\omega)+\omega}$, since 
$$|\mathfrak{B}_{\omega \times \omega^\omega}| < |\mathfrak{B}_{(\omega \times \omega^\omega)+0'}| < |\mathfrak{B}_{(\omega \times \omega^\omega)+0''}| < \dots  < |\mathfrak{B}_{(\omega \times \omega^\omega)+\omega}|$$
}

\com{ %2019 version; also on MITx homework:
\item \question{Give an example of a set whose cardinality is ``much greater'' than $|\mathfrak{B}_{\omega \times \omega}|$, in the sense that there are infinitely many sizes of infinity between the set you identify and $|\mathfrak{B}_{\omega \times \omega}|$.   (2~points; don't forget to justify your answer.)
}

\answer{One such set is $\mathfrak{B}_{(\omega \times \omega)+\omega}$, since 
$$|\mathfrak{B}_{\omega \times \omega}| < |\mathfrak{B}_{(\omega \times \omega)+0'}| < |\mathfrak{B}_{(\omega \times \omega)+0''}| < \dots  < |\mathfrak{B}_{(\omega \times \omega)+\omega}|$$
}
}%end com 

\com{ %Used in 2022; will discuss in 2023 lecture 
\item \question{Give an example of a set whose cardinality is ``much greater'' than $|\mathfrak{B}_{\omega^\omega}|$, in the sense that there are infinitely many sizes of infinity between the set you identify and $|\mathfrak{B}_{\omega^\omega}|$.   (3~points; don't forget to justify your answer.)


}


\answer{One such set is $\mathfrak{B}_{(\omega^\omega)+\omega}$, since 
$$|\mathfrak{B}_{\omega^\omega}| < |\mathfrak{B}_{(\omega^\omega)+0'}| < |\mathfrak{B}_{(\omega^\omega)+0''}| < \dots  < |\mathfrak{B}_{(\omega^\omega)+\omega}|$$

}
}%end com 



\com{ %start of long com for Russell's paradox question 
\item \question{Russell's Paradox is the observation that the following principle leads to contradiction when $F$ is the predicate ``is not a member of itself'' (and therefore cannot be true in general):

\begin{description}
\item[Naive Comprehension Principle]
There is a set $\set{x : F(x)}$, which consists of all and only objects that are $F$.
\end{description}

Consider an alternative principle:

\begin{description}
\item[Iterative Comprehension Principle]
For each ordinal $\alpha$, there is a set $\set{x \in V_\alpha: F(x)}$, which consists of all and only objects in $V_\alpha$ that are $F$.
\end{description}
where:
$$
V_\alpha=
\begin{cases}
\set{}, \text{ if $\alpha = 0$}\\
\powerset(V_\beta), \text{ if $\alpha = \beta'$}\\
\bigcup \{V_\gamma : \gamma <_o \alpha\} \text{ if $\alpha$ is a limit ordinal greater than $0$}
\end{cases}
$$


}

\begin{enumerate}

\item \question{Does the Iterative Comprehension Principle lead to contradiction when $F$ is the predicate ``is not a member of itself''? If it does, provide a proof; if it doesn't---or if you're not sure whether it does or doesn't---explain how it blocks the reasoning that leads to Russell's Paradox. (5~points)}

\answer{We have no reason to think that the Iterative Comprehension Prin\-ci\-ple leads to contradiction. Russell's Paradox arises because $\set{x : x \not\in x}$  is meant to be the set of \emph{all} entities that are not members of themselves, which leads to the conclusion that it is a member of itself if and only if it isn't. The Iterative Com\-pre\-hension Principle escapes this problem because $\set{x \in V_\alpha: x \not \in x}$ includes only entities that are members of $V_\alpha$ and therefore allows for the conclusion that $\set{x \in V_\alpha: x \not \in x}$ is not a member of itself (and is instead a member of some $V_\beta$ for $\alpha <_o \beta$).}


\item \question{Does the Iterative Comprehension Principle entail that there is a set of all sets? (5~points; don't forget to justify your answer.)} 

\answer{It does not. To see this note that: (1) the Iterative Com\-pre\-hension Principle only entails the existence of sets all of whose members are in some $V_\alpha$, and (2) no $V_\alpha$ contains all sets (since it follows from Cantor's Theorem that $V_\alpha \subsetneq V_{\alpha'}$).
}

\end{enumerate}
} %end com 

\com{ %start of long com for a difficult series of questions 
\item \question{Recall the following definitions:

\[\begin{array}{rclcl}
& & 0 &= &\set{}\\ 
& & \alpha' &= &\alpha \cup \set{\alpha}\\ 
& & \omega &= &\{0^{\overbrace{'\dots'}^{\text{$n$-times}}} : n \in \mathbb{N}\} = \set{0, 0', 0'', \dots}\\ \\
 \alpha &+ &0 &= &\alpha \\ \alpha &+ &\beta' &= &(\alpha + \beta)'\\ \alpha &+ &\lambda &= & \bigcup \{\alpha + \beta : \beta < \lambda\} \text{ ($\lambda$ a limit ordinal)}\\ \\ 
%
\alpha &\times &0 &= &0 \\ 
\alpha &\times &\beta' &= &(\alpha \times \beta) + \alpha\\ 
\alpha &\times &\lambda &= & \bigcup \{\alpha \times \beta : \beta < \lambda\} \text{ ($\lambda$ a limit ordinal)} \\ \\ 
%
& & \alpha^0 & = & 0' \\ 
& & \alpha^{\beta'} &= &(\alpha^\beta) \times \alpha \\ 
& & \alpha^{\lambda}  &= & \bigcup \{\alpha^\beta : \beta < \lambda\} \text{ ($\lambda$ a limit ordinal)} \end{array}\]

Use the above definitions to give a \underline{fully rigorous} proof of each of the following identities. In doing so, you may \underline{not} make use of results mentioned in the lecture notes, e.g.~Associativity. You'll have to prove everything from scratch!\footnote{You may use mathematical induction.}  (8 points each)
}

\begin{enumerate}

\item \label{mult-left} \question{$0^{\overbrace{'\dots'}^{\text{$n$-times}}} \times 0' = 0^{\overbrace{'\dots'}^{\text{$n$-times}}}$, for arbitrary $n \in \mathbb{N}$ }

\answer{
\begin{align}
0^{\overbrace{'\dots'}^{\text{$n$-times}}} \times 0' &= (0^{\overbrace{'\dots'}^{\text{$n$-times}}} \times 0) + 0^{\overbrace{'\dots'}^{\text{$n$-times}}} &(\times \ \prime)\nonumber\\
&= 0 + 0^{\overbrace{'\dots'}^{\text{$n$-times}}} &(\times \ 0)\nonumber\\
&= (0 + 0^{\overbrace{'\dots'}^{\text{$n-1$-times}}})' &(+ \ ')\nonumber\\
&= (0 + 0^{\overbrace{'\dots'}^{\text{$n-2$-times}}})'' &(+ \ ')\nonumber\\
&\vdots \nonumber\\
&= (0 + 0)^{\overbrace{'\dots'}^{\text{$n$-times}}} &(+ \ ')\nonumber\\
&= 0^{\overbrace{'\dots'}^{\text{$n$-times}}} &(+ \ 0)\nonumber
\end{align}


}

\item \label{mult-left} \question{$0' \times 0^{\overbrace{'\dots'}^{\text{$n$-times}}} = 0^{\overbrace{'\dots'}^{\text{$n$-times}}}$, for arbitrary $n \in \mathbb{N}$ }

\answer{The proof is by induction on $n$. When $n = 0$, the result is immediate:

\begin{align}
0' \times 0 &= 0 &(\times \ 0) \nonumber
\end{align}
We now assume $0' \times 0^{\overbrace{'\dots'}^{\text{$n$-times}}} = 0^{\overbrace{'\dots'}^{\text{$n$-times}}}$ and show $0' \times 0^{\overbrace{'\dots'}^{\text{$n+1$-times}}} = 0^{\overbrace{'\dots'}^{\text{$n+1$-times}}}$.
\begin{align}
0' \times 0^{\overbrace{'\dots'}^{\text{$n+1$-times}}} &= (0' \times 0^{\overbrace{'\dots'}^{\text{$n$-times}}}) + 0' &(\times \ \prime) \nonumber\\
&= (0^{\overbrace{'\dots'}^{\text{$n$-times}}}) + 0' &(\text{Ind.~Hyp.}) \nonumber\\
&= (0^{\overbrace{'\dots'}^{\text{$n$-times}}} + 0)' &(+ \ \prime) \nonumber\\
&= (0^{\overbrace{'\dots'}^{\text{$n$-times}}})' &(+ \ 0) \nonumber\\
&= 0^{\overbrace{'\dots'}^{\text{$n+1$-times}}} &(\text{notation})
\end{align}



}

\item \question{$w^{0''} = \omega \times \omega$}

\answer{


\begin{align}
\omega^{0''} &= \omega^{0'} \times \omega &(\wedge \ \prime) \nonumber\\
&= (\omega^{0} \times \omega) \times \omega &(\wedge \ \prime) \nonumber\\
&= (0' \times \omega) \times \omega \nonumber &(\wedge \ 0)\\
&= \rseq{\bigcup \{0' \times \beta : \beta < \omega\}} \times \omega \nonumber &(\times \ \lambda)\\
&= \bigcup \{0^{\overbrace{'\dots'}^{\text{$n$-times}}} : n\in \mathbb{N}\} \times \omega \nonumber &(\text{exercise~\ref{mult-left}}, \omega)\\
&= \{0^{\overbrace{'\dots'}^{\text{$n$-times}}} : n \in \mathbb{N}\} \times \omega &(') \nonumber\\
&= \omega \times \omega &(\text{$\omega$})\nonumber
\end{align}



}

\end{enumerate}
} %end of com 


\end{enumerate}


\end{document}





