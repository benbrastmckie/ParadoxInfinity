\documentclass[12pt]{article}

\usepackage{amsmath,amsfonts,amssymb,amscd}

\usepackage{amsthm}

\usepackage[margin=1.5in,headsep=.5in]{geometry}

\usepackage{fancyhdr}

\setlength{\headheight}{20pt}

\usepackage[colorlinks]{hyperref} 
\usepackage{cleveref}

\usepackage{setspace}
\usepackage{enumitem,linegoal}

\usepackage{amssymb}
\newcommand{\counterfactual}{\ensuremath{%
  \Box\kern-1.5pt
  \raise1pt\hbox{$\mathord{\rightarrow}$}}}

\newtheorem{theo}{Theorem}[section] 

\theoremstyle{definition}
\newtheorem{defin}[theo]{Definition} 
\newtheorem{lema}[theo]{Lemma} 
\newtheorem{cor}[theo]{Corollar}
\newtheorem{prop}[theo]{Proposition}
\newtheorem{exer}{Exercise}

\pagestyle{fancy}

\begin{document}

\pagenumbering{gobble}

\lhead{Xinhe Wu (xinhewu@mit.edu)}
\rhead{$24.118$ Paradox and Infinity $|$ Recitation $11$}

\begin{center}
{\Large \bf The Church-Turing Thesis}
\end{center}

\smallskip

\section{The Halting Problem}

\begin{defin}
The halting function $H: \mathbb{N} \leftarrow \mathbb{N}$ is the following function:
\begin{equation}
H(n) = \begin{cases} 1 &\mbox{if the Turing Machine coded by $n$ halts on input $n$} \\ 
0 & \mbox{if otherwise} \end{cases}.
\end{equation}
\end{defin}

\begin{theo}
The halting function is not Turing-computable.
\end{theo}

\section{The Church-Turing Thesis}

\begin{description}[style=unboxed,leftmargin=0cm]
\item[Church-Turing Thesis] A function is Turing-computable if and only if it can be computed algorithmically.
\end{description}

For a problem to be solvable algorithmically is for it to be possible to specify a finite list of instructions for solving the problem such that:
\begin{enumerate}
\item Following the instructions is guaranteed to yield a solution to the problem in a finite amount of time.
\item The instructions are specified in such a way that carrying them out requires no ingenuity of any kind: they can be followed mechanistically.
\item No Special resources are required to carry out the instructions: they could, in principle, be carried out by an ordinary human (albeit a human equipped with unlimited supplies of pencils and paper and patience).
\item No special physical conditions are required for the computation to succeed (no need for faster-than-light travel, special solutions to Einstein's equations, etc.)
\end{enumerate}

An example: the truth-table test for tautologies in propositional calculus.

\section{Different Views on the Thesis}

\begin{itemize}
\item Trivially true: Turing-computability \textit{defines} our ordinary notion of ``following an algorithm". (Church) 
\item Does not admit of a mathematical proof, but we'd better believe it as it is supported by tons of evidence: (Post)
\begin{itemize}
\item Any attempt to give an exact analysis of the intuitive notion of ``following an algorithm" has turned out to be equivalent.
\item Lambda calculus, recursive functions, process calculus, etc.
\item C++, Pascal, Java, Python, R, etc.
\item Minecraft, (Infinite) Minesweeper, Fortress, etc.
\end{itemize}
\item Does not admit of a mathematical proof, but justified by appeals to intuition. (Turing)
\item Does admit of a mathematical proof: ``human computers" (Kripke)
\end{itemize}

\section{Is Human Mind a Turing Machine?}

\begin{description}
\item[Assumption One] The human mind is such that it can in principle calculate the halting function. 
\item[Assumption Two] The human mind is such that it can in principle be simulated by a computer.
\end{description}

\section{Super Turing Machines}

\begin{itemize}
\item Accelerating Turing Machines: exactly like standard Turing machines except that their speed of operation accelerates as the computation proceeds.
\item Extended Turing Machines: exactly like standard Turing machines except that, whereas standard Turing machines stores only a single discrete symbol on each (non-blank) square of its tape (e.g., `0’ or `1’), a single square of an ETM’s tape can store any desired real number, for example $\pi$, or even an uncomputable real number.
\end{itemize}



\end{document}