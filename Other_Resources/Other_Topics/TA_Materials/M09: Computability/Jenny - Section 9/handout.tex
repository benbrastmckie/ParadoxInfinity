\documentclass[justified]{tufte-handout} 
\usepackage{amsfonts, amssymb, stmaryrd, fitch, natbib, qtree}
\usepackage{linguex, color, setspace, graphicx}
\usepackage{enumitem}
\usepackage{bussproofs}
\usepackage{turnstile}
\usepackage[super]{nth}
\thispagestyle{plain}
\definecolor{darkred}{rgb}{0.7,0,0.2}
\bibpunct{(}{)}{,}{a}{}{,}

\input xy
 \xyoption{all}

%New Symbols
\DeclareSymbolFont{symbolsC}{U}{txsyc}{m}{n}
\DeclareMathSymbol{\strictif}{\mathrel}{symbolsC}{74}
\DeclareMathSymbol{\boxright}{\mathrel}{symbolsC}{128}
\DeclareMathSymbol{\Diamondright}{\mathrel}{symbolsC}{132}
\DeclareMathSymbol{\Diamonddotright}{\mathrel}{symbolsC}{134}
\DeclareMathSymbol{\Diamonddot}{\mathord}{symbolsC}{144}
\renewcommand{\labelitemi}{$\triangleright$}
\renewcommand{\labelitemii}{$\circ$}
\renewcommand{\labelitemiii}{$\triangleright$}

%New commands
\newcommand{\bitem}{\begin{itemize}}
\newcommand{\eitem}{\end{itemize}}
\newcommand{\lang}{$\langle$}
\newcommand{\rang}{$\rangle$}
\newcommand{\back}{$\setminus$}
\newcommand{\HRule}{\rule{\linewidth}{0.1mm}}
\newcommand{\llm}[2][]{$\llbracket${#2}$\rrbracket^{#1}$}
\newcommand{\ul}{$\ulcorner$}
\newcommand{\ur}{$\urcorner\ $}
\newcommand{\urn}{$\urcorner$}
\newcommand{\sub}[1]{\textsubscript{#1}}
\newcommand{\sups}[1]{\textsuperscript{#1}}
\newtheorem{proposition}{\textbfb{Proposition}}[section]
\newtheorem{definition}[proposition]{\textbf{Definition}}
\newcommand{\bfw}{\begin{fullwidth}}
\newcommand{\efw}{\end{fullwidth}}

\begin{document}

\begin{fullwidth}
\noindent\LARGE More on Computability  \normalsize \\[.3cm]
\noindent  \textsc{24.118 Recitation Section $\bullet$ Matthias Jenny\\  {\texttt{\href{mailto:mjenny@mit.edu}{mjenny@mit.edu}}} $\bullet$ Office:  32-D927 $\bullet$ Hours: Thu 11:30-12:30} \hfill{November 21, 2014}
\noindent\HRule
\end{fullwidth}

\section{More on algorithmic computability}

\noindent The notion of algorithmic computability is often explained with another informal notion, that of an \emph{effective procedure}. \marginnote{Piccinini (2011), ``The Physical Church-Turing Thesis: Modest or Bold?'', in: \emph{The British Journal for the Philosophy of Science} 62:733--769.}Philosopher Gual\-tiero Piccinini  argues that an effective procedure is a finite set of instructions that meets the following conditions:

\begin{description}
\item[Executability:] The procedure's instructions are deterministic, i.e. they determine ``a unique next step in the procedure'', ``which have finite and unambiguous specifications commanding the execution of a finite number of primitive operations.''
\item[Automaticity:] ``The procedure requires no intuitions about the domain (e.g. intuitions about numbers), no ingenuity, no invention and no guesses.''
\item[Uniformity:] ``The procedure is the same for each possible argument of the function.''
\item[Reliability:] When the procedure terminates, it ``generates the correct value of the function for each argument after a finite number of primitive operations are performed.''
\end{description}

\section{Why believe in the Church-Turing thesis?}

\noindent Recall that the Church-Turing thesis, as put by Turing, says that the algorithmically computable functions are exactly the Turing computable functions. Why do logicians and computer scientists commonly take this thesis to be true? There are roughly four sources of evidence:

\begin{enumerate}
\item The intuitive plausibility of the thesis.\marginnote{Some think that Turing's original article (1936) contains an actual informal \emph{proof} of his the thesis.}
\item The fact that every function we've expected to be algorithmically computable has turned out to be Turing computable.
\item The fact \marginnote{In addition to the analysis in terms of Turing machines and the analysis in terms of recursive functions, there's also analysis in terms of the so-called $\lambda$-calculus, due to Alonzo Church (1936).} that various proposed analyses of algorithmic computability have turned out to be equivalent.
\item A function from natural numbers to natural numbers is Turing computable iff it can be computed by an ordinary computer (given unlimited memory and running time).\marginnote{This was first proved by Turing's student Robin Gandy (1980).}
\end{enumerate}

\section{Recursive functions}

\noindent Recall how primitive recursion and minimization works.\\

\noindent For $f$ a one-place function and for $g$ a three-place function, the function obtained by primitive recursion is the unique two-place function $h$ satisfying the following conditions:

\begin{enumerate}[label=(\roman*)]
\item $h(x,0)=f(x)$, for all $x$.
\item $h(x,y+1)=g(x,y,h(x,y))$ for all $x,y$.\\
\end{enumerate}

\noindent For $f$ an $n+1$-place function, the function obtained by minimization from $f$ is the $n$-place function $h$ given by:

\[
    h(x_1,x_2,\dots,x_n)= 
\begin{cases}
   y & \text{if }f(x_1,\dots,x_n,y)=0,\text{ and for all }t<y, f(x_1,\dots,x_n,t)\text{ is defined and }\neq0,\\
    \text{undef.}              &  \text{if there is no such $y$}.
\end{cases}
\]

\section{Listing the recursive functions}

\noindent Recall the definition of the recursive functions:

\begin{description}
\item[Basic functions] The basic recursive functions are:
	\begin{itemize}
	\item The zero function $z$.
	\item The successor function $s$.
	\item The countably many identity functions $id^n_i$.
	\end{itemize}
\item[Complex functions] We can build new recursive functions out of old recursive functions as follows:
 	\begin{itemize}
	\item For $f$ an $m$-place function and $g_1,g_2,\dots,g_m$ $n$-place functions, the composition $Cn[f,g_1,g_2,\dots,g_m]$ is an $n$-place recursive function. 
	\item For $f$ a one-place function and for $g$ a three-place function, the function $Pr[f,g]$ obtained by primitive recursion is a two-place recursive function.
	\item For $f$ an $n+1$-place function, the function $Mn[f]$ obtained by minimization from $f$ is an $n$-place recursive function.\\
	\end{itemize}
\end{description}

\noindent Can we list all recursive functions to prove that there are only countably many?\marginnote{Hint: Think of the proof that the rationals are countable, but in more dimensions.}

\end{document}
