\documentclass[12pt]{article}

\usepackage{amsmath,amsfonts,amssymb,amscd}

\usepackage{amsthm}

\usepackage[margin=1.5in,headsep=.5in]{geometry}

\usepackage{fancyhdr}

\setlength{\headheight}{20pt}

\usepackage[colorlinks]{hyperref} 
\usepackage{cleveref}

\usepackage{setspace}
\usepackage{enumitem,linegoal}

\usepackage{amssymb}
\newcommand{\counterfactual}{\ensuremath{%
  \Box\kern-1.5pt
  \raise1pt\hbox{$\mathord{\rightarrow}$}}}

\newtheorem{theo}{Theorem}[section] 

\theoremstyle{definition}
\newtheorem{defin}[theo]{Definition} 
\newtheorem{lema}[theo]{Lemma} 
\newtheorem{cor}[theo]{Corollar}
\newtheorem{prop}[theo]{Proposition}
\newtheorem{exer}{Exercise}

\pagestyle{fancy}

\begin{document}

\pagenumbering{gobble}

\lhead{Xinhe Wu (xinhewu@mit.edu)}
\rhead{$24.118$ Paradox and Infinity $|$ Recitation $9$}

\begin{center}
{\Large \bf Review on Chapter Six and Seven}
\end{center}

\smallskip

\section{Problem Set 6}

\begin{defin}
A \textit{probability function} is a function from the set of propositions to $[0, 1]$ such that 
\begin{enumerate}
\item $p(A)=1$ if A is a necessary truth. (Necessity)
\item $P(A \lor B) = p(A) + p(B)$ if $A$ and $B$ are incompatible. (Finite Additivity)
\end{enumerate}
\end{defin}

\begin{defin}
Let $p$ be a probability function. Let $p(B)>0$. The conditional probability of A given that B has occurred is given by $p(A|B)=p(A \land B)/p(B)$ (The Bayes' Law).
\end{defin}

\begin{exer}
Prove that $p(B|A)=1-p(\overline{B}|A)$. (Assume $p(A) \neq 0$.)
\end{exer}

\noindent
\underline{Common Mistakes}:
\begin{enumerate}
\item The `given' in conditional probability is not a logical connective. Hence `$A|B$' is not a proposition.
\item  You cannot always `substitute by identity' when doing probability calculus.
\item You cannot always assume $p(A|B)$ is defined. 
\end{enumerate}

\begin{exer}
A deck of card has been thoroughly shuffled. And you have no more information. Which normative principle tells you what your credence should be in the proposition `the top card is the three of hearts'?
\end{exer}

\noindent
\underline{Common Mistakes}:
\begin{enumerate}
\item Not all truths are necessary.
\item We don't have a principle that tells you equal credence follows from `physical symmetry'.
\end{enumerate}

\section{Review on Chapter 7}

\subsection{Borel Sets}
\begin{defin}
For $a, b \in \mathbb{R}$, the line segment $[a,b] = \{x \in \mathbb{R} \; | \; a \leqslant x \leqslant b\}$. The \textit{Borel Sets} are the members of the smallest set $\mathfrak{B}$ such that:
\begin{enumerate}
\item[(i)] every line segment is in $\mathfrak{B}$;
\item[(ii)] if a set $A$ is in $\mathfrak{B}$, then $\mathbb{R} - A$ is in $\mathfrak{B}$;
\item[(iii)] if a countable set $S$ is such that all its members are in $\mathfrak{B}$, then $\bigcup S$ is in $\mathfrak{B}$.
\end{enumerate}
\end{defin}

\begin{exer}
Let $S = \{A_1, A_2, A_3, ...\}$ be a countable set of Borel sets. Show that $\bigcap S$ is a Borel set.
\end{exer}

\begin{exer}
Let $A, B$ be Borel sets. Show that $A - B$ ($\{x \; | \; x \in A \land x \notin B \}$) is a Borel set.
\end{exer}


\subsection{The Lebesque Measure}
\begin{defin}
The Lebesque measure, $\lambda$, is the unique function from $\mathfrak{B}$ to $\bigcup \{\mathbb{R}^+, +\infty\}$ such that:
\begin{enumerate}
\item[(i)] $\lambda([a, b])=b-a$
\item[(ii)] $\lambda(\bigcup\{A_1, A_2, A_3, ...\})=\lambda(A_1)+\lambda(A_2)+\lambda(A_3)+...$, where $A_1$, $A_2$, ... are countably many disjoint Borel sets.
\end{enumerate}
\end{defin}

\begin{exer}
Every countable Borel set has Lebesgue measure $0$.
\end{exer}

\begin{exer}
Let $A, B$ be Borel sets and $B\subseteq A$. Show that $\lambda(B) \leqslant \lambda(A)$.
\end{exer}

\subsection{The Axiom of Choice}

\begin{prop} 
[The Axiom of Choice] Every set of non-empty, non-overlapping sets has a choice set.
\end{prop}

\begin{prop}
[The Well-Ordering Principle] Every set can be well-ordered.
\end{prop}

\begin{theo}
The axiom of choice is equivalent to the well-ordering principle.
\end{theo}



\end{document}