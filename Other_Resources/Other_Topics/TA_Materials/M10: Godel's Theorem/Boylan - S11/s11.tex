\documentclass[justified]{tufte-handout} 
\usepackage{amsfonts, amssymb, stmaryrd, natbib, qtree, amsxtra}
\usepackage{linguex, color, setspace, graphicx}
\usepackage{enumitem}
\usepackage{bussproofs}
\usepackage{turnstile}
\usepackage{phaistos}
\usepackage{protosem}
\usepackage{txfonts}
\usepackage{pxfonts}
\usepackage[super]{nth}
\thispagestyle{plain}
\definecolor{darkred}{rgb}{0.7,0,0.2}
\bibpunct{(}{)}{,}{a}{}{,}


\input xy
 \xyoption{all}

%New Symbols
\DeclareSymbolFont{symbolsC}{U}{txsyc}{m}{n}
\DeclareMathSymbol{\strictif}{\mathrel}{symbolsC}{74}
\DeclareMathSymbol{\boxright}{\mathrel}{symbolsC}{128}
\DeclareMathSymbol{\Diamondright}{\mathrel}{symbolsC}{132}
\DeclareMathSymbol{\Diamonddotright}{\mathrel}{symbolsC}{134}
\DeclareMathSymbol{\Diamonddot}{\mathord}{symbolsC}{144}
\renewcommand{\labelitemi}{$\triangleright$}
\renewcommand{\labelitemii}{$\circ$}
\renewcommand{\labelitemiii}{$\triangleright$}

%New commands
\newcommand{\bitem}{\begin{itemize}}
\newcommand{\eitem}{\end{itemize}}
\newcommand{\lang}{$\langle$}
\newcommand{\rang}{$\rangle$}
\newcommand{\back}{$\setminus$}
\newcommand{\HRule}{\rule{\linewidth}{0.1mm}}
\newcommand{\llm}[2][]{$\llbracket${#2}$\rrbracket^{#1}$}
\newcommand{\ul}{$\ulcorner$}
\newcommand{\ur}{$\urcorner\ $}
\newcommand{\urn}{$\urcorner$}
\newcommand{\sub}[1]{\textsubscript{#1}}
\newcommand{\sups}[1]{\textsuperscript{#1}}
\newtheorem{proposition}{\textbfb{Proposition}}[section]
\newtheorem{definition}[proposition]{\textbf{Definition}}
\newcommand{\bfw}{\begin{fullwidth}}
\newcommand{\efw}{\end{fullwidth}}

\begin{document}

\frenchspacing

\begin{fullwidth}
\noindent\LARGE Section 11, G\"{o}del's Incompleteness Theorem \normalsize \\[.3cm]
\noindent  David Boylan \hfill{5.12.2017}

\noindent\HRule
\end{fullwidth}

\section{An Arithmetical Languge}

\begin{itemize}

\item Here are the parts of our language:

\begin{itemize}
\item Arithmetical vocabulary: $0,1, ,\times$

\item Logical vocabulary: $=, \neg, \&, \forall, x_n$

\item Parentheses: $(,)$

\end{itemize}

\vspace{.4cm} 

Some sentences in our language:

\ex.  \a.[] $\forall x_1: (x_1=x_1)$
\b.[] Everything is equal to itself

\ex. \a.[]$\forall x_1\forall x_2:  (x_1 \times x_2 = x_2 \times x_1)$
\b.[] Addition is commutative.

\vspace{.4cm} 


We can also introduce some abbreviations. Here are a few: 

\ex.[] $\exists x_n Fx_n \equiv_{df} \neg\forall x_n\neg Fx_n $ 

\ex.[] $ p \vee r \equiv_{df} \neg(\neg p \& \neg r)$

\ex.[]$x_i < x_j \equiv_{df} \exists x_k:(x_i + x_k= x_j$ $\&$ $\neg x_k = 0 )$  


\vspace{.4cm} 


 Exercises:

\begin{itemize}


\item For a given formula $\phi(x_i)$ introduce an abbreviation $\exists ! x_i (\phi(x_i))$ by finding a sentence of $\mathcal{L }$ which is true if and only if there is exactly one number $n$ such
that $\phi(n)$ is the case.


\item Introduce an abbreviation $x_i|x_j$, by finding a formula of $\mathcal{L}$ that is true of $x_i$ and $x_j$ if and only if $x_i$ divides $x_j$ , with no remainder.


\item Introduce an abbreviation ``$Odd(x_i)$", by finding a formula of $\mathcal{L}$ that is true of $x_i$ if and only if $x_i$ is an even natural number.


\end{itemize}



\end{itemize}



\section{G\"{o}del's Theorem}


\begin{itemize}

\item Question: suppose we taught a computer this language.\marginnote{Remember as well that any computer can be simulated by a Turing machine. So if there were such a computer, there would be a Turing machine that could compute do this for us also.} Could we write a program which would make the computer output \emph{all} arithmetical truths eventually and no falsehoolds?

\vspace{.2cm}

\noindent Answer: no! G\"{o}del's Incompleteness Theorem shows that this is not possible.

\vspace{.7cm}


 \noindent Strategy: we show that if we had such a machine existed, we could use it to design a machine that computed the halting function.

\vspace{.2cm}

\noindent Since we already know the halting function is \emph{not} computable, this shows that our machine cannot exist.


\vspace{.7cm}


 \noindent For this we need an important lemma:

\vspace{.2cm}


\begin{quote}

We can define a formula \emph{Halt x} which is true iff the $k$th Turing machine halts on input $k$.

\end{quote}

\vspace{.2cm}

\noindent This requires a good deal of work to show but the basic insight is that we can use sequences of numbers to represent Turing machines, their states, their input tapes and their outputs.

 \vspace{.2cm}

\noindent But $\mathcal{L}$ has no problem talking about these sequences.

\vspace{.7cm}


\item Step 1: we construct a machine $M^H$.

\vspace{.2cm}

\noindent Suppose $M^H$ has as input a sequence of $k$ ones.

\vspace{.4cm}

\noindent Start running $M$ and every time $M$ outputs something check whether it is of the form \emph{Halt k} or $\neg$\emph{Halt k}. 

\vspace{.4cm}

\noindent If you find \emph{Halt k}, delete everything on the tape, print a one and halt. If you find $\neg$\emph{Halt k} delete everything on the tape, print a 0 and halt. 

\vspace{.7cm}


\item Step 2: we confirm that $M^H$ would compute the halting function.\\ We show that the machine outputs a one if the $k$th Turing halts on $k$; and outputs 0 otherwise.

\vspace{.4cm}

\noindent Suppose the $k$th machine halts on $k$. We know that $M$ will then output \emph{Halt k}. So $M^H$ outputs 1.


\vspace{.4cm}

\noindent Suppose the $k$th machine doesn't halt on $k$. We know that $M$ will then output $\neg$\emph{Halt k}. So $M^H$ outputs 0.

\vspace{.4cm}

 \noindent And we're done! We have seen that if $M$ exists, then $M^H$ exists. But $M^H$ exists.

\end{itemize}

\section{GT Again}

\begin{itemize}


\item Here's a different (and to some more familiar) statement: 

\begin{quote}

No axiomatization of elementary arithmetic (in a
rich enough language) can be both consistent and complete.

\end{quote}

Lots of new ideas here.

\item  So far we have a language $\mathcal{L}$. But we don't have a way of saying what sequence of $\mathcal{L}$-sentences count as a proof. An axiomatisation helps us determine this. 

\vspace{.2cm}

\noindent An axiomatisation has two components:

\vspace{.2cm}

\noindent We have a set of axioms. These are sentences of $\mathcal{L}$. They are supposed to be intuitive, obvious truths about arithmetic and logic. 

Examples: $(p\vee \neg p)$; $1=1$; $\forall x_n \exists x_{n+1} ( x_{n+1} = x_{n} +1)$.

\vspace{.3cm}

\noindent We also have a set of rules of inference. These tell us what sentences let us infer what.

Examples: Modus ponens: from $p$ and $p\supset q$, infer $q$.

\vspace{.2cm}


\item When we have an axiomatisation, we can say when a sentence is \emph{provable}.

A sentence $s$ is provable just in there is a sequence of sentences ending with $s$ where every sentence is either an axiom or can be derived from the axioms and previous sentences in the sequence using the rules of inference.

\vspace{.2cm}

 \noindent We also say that an axiomatisation for a language is \emph{complete} just in case all the true sentences of $\mathcal{L}$ are provable.

\vspace{.2cm}

\item We can show that there is no complete axiomatisation of $\mathcal{L}$.
\vspace{.2cm}

 \noindent Suppose there were. We can code sentences and sequences of sentences as single numbers, with each sequence coded as a unique number. 

\vspace{.2cm}

 \noindent We then program a Turing machine to go through the natural numbers and find ones that code sequences. If it does code a sequence, then it checks if that sequence is a proof. If it is, it outputs the last sentence of the sequence. Since we're supposing every sentence is provable, every sentence will eventually be outputted.
 
 \vspace{.2cm}

 \noindent But notice that this is $M^H$!



\end{itemize}
 
 
 

\end{document}