\documentclass[12pt]{article}

\usepackage{amsmath,amsfonts,amssymb,amscd}

\usepackage{amsthm}

\usepackage[margin=1.5in,headsep=.5in]{geometry}

\usepackage{fancyhdr}

\setlength{\headheight}{20pt}

\usepackage[colorlinks]{hyperref} 
\usepackage{cleveref}

\usepackage{setspace}
\usepackage{enumitem,linegoal}

\usepackage{amssymb}
\newcommand{\counterfactual}{\ensuremath{%
  \Box\kern-1.5pt
  \raise1pt\hbox{$\mathord{\rightarrow}$}}}

\newtheorem{theo}{Theorem}[section] 

\theoremstyle{definition}
\newtheorem{defin}[theo]{Definition} 
\newtheorem{lema}[theo]{Lemma} 
\newtheorem{cor}[theo]{Corollar}
\newtheorem{prop}[theo]{Proposition}
\newtheorem{exer}{Exercise}

\newcommand{\proofsketch}{\vspace*{-1ex} \noindent {\bf Proof Sketch: }}


\pagestyle{fancy}

\begin{document}

\pagenumbering{gobble}

\lhead{Xinhe Wu (xinhewu@mit.edu)}
\rhead{$24.118$ Paradox and Infinity $|$ Recitation $12$}

\begin{center}
{\Large \bf G\"{o}del's Incompleteness Theorems}
\end{center}

\smallskip

\section{The First Incompleteness}

\begin{description}
\item[G\"{o}del's First Incompleteness (V3)] No axiomatization $A$ of $L$ is both consistent and complete.
\begin{itemize}
\item Complete: every true sentence of $L$ is provable in $A$.
\item Consistent: it is not the case that for some sentence $\phi$, both $\phi$ and its negation $\neg \phi$ are provable in $A$.
\end{itemize}
\end{description}

\noindent
\begin{description}
\item[G\"{o}del's First Incompleteness (standard version)] No axiomatization $A$ of $L$ is both consistent and complete*.
\begin{itemize}
\item Complete*: every sentence of $L$ is decidable - either provable or refutable, in $A$.
\item $A$ refutes $\phi$ iff $A$ proves $\neg \phi$.
\end{itemize}
\end{description}

\section{The G\"{o}del Sentence}

\begin{theo} [ G\"odel's Fixed Point Lemma]
For any formula $\phi(x)$ definable in $L$, there is a sentence $\psi \in L$ such that $\phi \leftrightarrow \psi (\ulcorner \phi \urcorner).$
\end{theo}

\begin{theo}
We can define in $L$ a formula $Prov(x)$, which is true of a number $n$ if and only if $n$ codes a sentence that is provable in $A$.
\end{theo}

\begin{lema} 
We can construct a TM such that it halts on an input $n$ if and only if $Prov(n)$ is true.
\end{lema}

\begin{theo}
There exists a sentence $G$ (the G\"{o}del Sentence) in $L$ such that
$$G \leftrightarrow \neg Prov(\ulcorner G \urcorner).$$
\end{theo}

\begin{theo}
If $A$ is ($\omega$-)consistent, $G$ is neither provable nor refutable in $A$.
\end{theo}

\begin{proof} 
Suppose $G$ is provable. Then the TM we construct in 2.3 will halt on $\ulcorner G \urcorner$. As it turns out, when $A$ is good enough (at least as strong as Robinson Arithmetic), we can prove in $A$ that this TM halts on $\ulcorner G \urcorner$. Hence $Prov(\ulcorner G \urcorner)$ is provable. Hence $\neg G$ is provable. Contradiction.

Suppose $G$ is refutable. Hence $\neg G$ is provable. Hence $Prov (\ulcorner G \urcorner)$ is provable. But that means we can prove in $A$ that the TM in 2.3 halts on $\ulcorner G \urcorner$, which means $G$ is provable. Contradiction.
\end{proof}

\section{The Second Incompleteness}

\begin{description}
\item[G\"{o}del's Second Incompleteness] If an axiomatization $A$ of $L$ (that contains elementary arithmetic) is consistent, then it does not prove its own consistency.
\end{description}

\begin{defin}
$Con(A)$ (that $A$ is consistent) is the sentence $\neg Prov(0=1)$.
\end{defin}

\begin{theo}
If $A$ is consistent, then $A$ does not prove $Con(A)$.
\end{theo}

\proofsketch The proof of 2.5 can be formulated in $A$. Hence we have:
$$ Prov(\ulcorner G \urcorner) \rightarrow \neg Con(A)$$
This is equivalent to
$$ Con(A) \rightarrow \neg Prov(\ulcorner G \urcorner)$$
But that means
$$ Con(A) \rightarrow G$$
Hence if $A$ can prove its own consistency, it can prove the G\"{o}del sentence. But we just showed it's not provable in $A$.

\section{Philosophical Lessons}

\subsection{Hilbert's Program}

\begin{itemize}
\item A formalization of all of mathematics in axiomatic form, together with a proof that this axiomatization of mathematics is consistent.
\end{itemize}

\subsection{We Are Not Computers 2.0}
\begin{itemize}
\item Given any machine which is consistent and capable of doing simple arithmetic, there is a formula it is incapable of producing as being true. But we can see that it is true.
\end{itemize}


\end{document}