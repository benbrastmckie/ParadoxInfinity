\documentclass[justified]{tufte-handout} 
\usepackage{amsfonts, amssymb, stmaryrd, natbib, qtree, amsxtra}
\usepackage{linguex, color, setspace, graphicx}
\usepackage{enumitem}
\usepackage{bussproofs}
\usepackage{turnstile}
\usepackage{phaistos}
\usepackage{protosem}
\usepackage{txfonts}
\usepackage{pxfonts}
\usepackage[super]{nth}
\thispagestyle{plain}
\definecolor{darkred}{rgb}{0.7,0,0.2}
\bibpunct{(}{)}{,}{a}{}{,}

\input xy
 \xyoption{all}

%New Symbols
\DeclareSymbolFont{symbolsC}{U}{txsyc}{m}{n}
\DeclareMathSymbol{\strictif}{\mathrel}{symbolsC}{74}
\DeclareMathSymbol{\boxright}{\mathrel}{symbolsC}{128}
\DeclareMathSymbol{\Diamondright}{\mathrel}{symbolsC}{132}
\DeclareMathSymbol{\Diamonddotright}{\mathrel}{symbolsC}{134}
\DeclareMathSymbol{\Diamonddot}{\mathord}{symbolsC}{144}


%New commands
\newcommand{\bitem}{\begin{itemize}}
\newcommand{\eitem}{\end{itemize}}
\newcommand{\lang}{$\langle$}
\newcommand{\rang}{$\rangle$}
\newcommand{\back}{$\setminus$}
\newcommand{\HRule}{\rule{\linewidth}{0.1mm}}
\newcommand{\llm}[2][]{$\llbracket${#2}$\rrbracket^{#1}$}
\newcommand{\ul}{$\ulcorner$}
\newcommand{\ur}{$\urcorner\ $}
\newcommand{\urn}{$\urcorner$}
\newcommand{\sub}[1]{\textsubscript{#1}}
\newcommand{\sups}[1]{\textsuperscript{#1}}
\newtheorem{proposition}{\textbfb{Proposition}}[section]
\newtheorem{definition}[proposition]{\textbf{Definition}}
\newcommand{\bfw}{\begin{fullwidth}}
\newcommand{\efw}{\end{fullwidth}}

\begin{document}

\frenchspacing

\begin{fullwidth}
\noindent\Large Section 7,  Probability \large \\[.3cm]
\noindent  David Boylan \hfill{11-12, 66-154}

\noindent\HRule
\end{fullwidth}


\section{Probability Functions}

\begin{itemize}


\item Examples of subjective probability:

\ex. I'm very confident it's going to rain.

\ex. I'm 50/50 the coin will come up heads.


\item Examples of objective probability: 


\ex. It's 70\% likely this atom will decay. 

\ex. It's 100\% likely that this coin will come up heads or 100\% likely it will come up tails. I just don't know which.


\item Why do we call both of these things \emph{probabilities}? 

Answer: we use the same kind of mathematical tools to model both. 



\item Say that $P$ is a probability function iff for all propositions $p$: 

\begin{itemize}

\item For every $p$, $0 \leq P(p) \leq 1$. 

\item If $p$ is a necessary truth, then $P(p)=1$


\item If $p$ and $q$ are incompatible then $P(p\vee q)= P(p)+P(q)$.

\end{itemize}

We use probability functions to model both subjective and objective probability.

\item We also define $P(q|p)$, the probability of $q$ \emph{conditional on} $p$:

$P(q|p)= \frac{P(p \& q)}{P(p)}$



\item Note that, since the formalism is the same in both cases, any mathematical results we prove about one case will carry over to the other. 

For instance, Bayes Law, which says


\begin{quote}

$P(A \& B) = P(A) \times P(B|A)$ 

\end{quote}

holds for either kind of probability.

\item \textbf{Exercises}:

\begin{itemize}


\item Prove Bayes' Law.

\item We say $A$ and $B$ are probabilistically independent iff $P(A|B) = P(A|\neg B)$ and dependent otherwise.

Give an example of an independent pair of events and of a dependent pair.


\item Here are three definitions of independence: 

\begin{itemize}

\item $A$ and $B$ are independent iff $P(A|B) = P(A|\neg B)$

\item  $A$ and $B$ are independent iff $P(A) = P(A|B)$

\item  $A$ and $B$ are independent iff $P(A \&B) = P(A)\times P(B)$

\end{itemize}

Show that these are equivalent.


\end{itemize}




\end{itemize}




\section{Subjective Probability}

\begin{itemize}

\item We can be more or less confident about things. We can imagine a function assigning a number to each proposition, where the number represents how confident we are in a given proposition. (We would take 1 to be certainty.)

\item We are rational only if that function is a probability function. The laws of probability then tell us how we should organise our beliefs in order to be rational at a given time.

\item The laws of probability tell us how our beliefs should hang together at any one time. But Bayesians also have a rule for how to change your beliefs over time, the rule of conditionalisation:

\ex. Where $P_p$ is your current probability function updated with $p$: $P_p(q)= P(q|p)$.


\item Subjective probabilities are involved in rational decision making. We defined expected utility:


\ex.[]$EU(A) = \Sigma_{o\in O} P(o)\times U(o \wedge A)$

and said that you should chose the option with the highest expected utility.

If this is supposed to be a good guide decisions, then the relevant probability is subjective.



\item Questions:

\begin{itemize}


\item Suppose I become fully confident in $p$. Can I then learn anything that will make me less confident? Is this a good result?




\item Suppose I am about to throw a point-sized dart at a dartboard with real many points on it. How likely am I to get a bullseye? What does this mean?

\item Suppose I offer you the following bet: I'm going to flip a going. If it comes up heads, you win \$1m; if you lose you pay me \$.9m. No money changes hands if you decide not to play. 

If you were an expected value maximiser, what would you do here? Is that what you should actually do?

\item Suppose my prior confidence in $p$ is $0.5$ but your prior confidence in $p$ is $0.7$. According to our theory so far, which one of us is rational? Are either of us rational? Are both of us rational?


\end{itemize}


\end{itemize}

\section{Objective Probability}

\begin{itemize}

\item It's somewhat harder to get a grip on what exactly objective probability is.


\item One idea is that is has something to do with frequency. 

But this is difficult to make work. After all, it's possible when flipping three fair coins to get three heads 100 times in a row. But the probability of getting three heads on the next one should still be $\frac{1}{8}$.

\item We know at least this much about objective probability: 

\begin{quote}Objective/subjective connection: The objective probability of an event at time t is the subjective probability
that a perfectly rational agent would assign to that event, if she had perfect
information about events before t and no information about events after t.\end{quote}

Some will suggest that this is all there really is to objective probability.


\item Questions: 

\begin{itemize}

\item Suppose determinism is true. Now I am about to flip a coin. What is the objective probability that it comes up heads?


\item Does the objective/subjective allow them to come apart? I.e. can you find some proposition $p$ such that its objective probability is different from its subjective probability?

\end{itemize}

\end{itemize}




\end{document}