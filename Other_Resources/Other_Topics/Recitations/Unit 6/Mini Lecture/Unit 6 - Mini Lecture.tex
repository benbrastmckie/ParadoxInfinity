\documentclass[11pt]{article}

\usepackage{amsmath,amsthm,amsfonts,amssymb,amscd}

\usepackage[margin=1.5in,headsep=.5in]{geometry}

\usepackage{fancyhdr}

\setlength{\headheight}{20pt}

\usepackage[colorlinks]{hyperref} 
\usepackage{cleveref}

\usepackage{enumitem}
\setlist[enumerate]{itemsep=0mm}

\theoremstyle{definition}
\newtheorem{defn}{Definition}
\newtheorem{reg}{Rule}
\newtheorem{exer}{Exercise}
\newtheorem{note}{Note}
\newtheorem*{theorem*}{Theorem}
\newtheorem{theorem}{Theorem}[section]
\newtheorem{corollary}{Corollary}[theorem]
\newtheorem{thm}{Theorem}
\newtheorem{prop}[thm]{Proposition}
\newtheorem{lem}[thm]{Lemma}
\newtheorem{conj}[theorem]{Conjecture}

\pagestyle{fancy}

\begin{document}

\pagenumbering{gobble}

\lhead{$24.118$ Paradox and Infinity }
\rhead{Recitation $6$: Probability}



\begin{center}
{\LARGE \bf What is Objective Chance?}
\end{center}

\smallskip

\section{Finite Frequentism}

\begin{description}
\item[Finite Frequentism] The objective probability of an outcome $A$ in a finite reference class $B$ is the frequency of actual occurrences of (outcomes of the same type as) $A$ within $B$.
\end{description}

\begin{enumerate}[leftmargin=4.4 \parindent]
\item[Problem 1:] Unrepeatable events: for example, the 2020 presidential election. 
\item[Problem 2:] What should be the reference class?

\indent Consider, say, my chance of being less than $6$ feet tall. One reference class: the set of all humans. Another reference class: the set of me and Eiffel Tower.
\end{enumerate}


\section{Hypothetical Frequentism}
\begin{description}
\item[Infinite Hypothetical Frequentism] The objective probability of an outcome $A$ in an infinite reference class $B$ (of hypothetical events) is the frequency of occurrences of (outcomes of the same type as) $A$ within $B$.
\end{description}

\begin{enumerate}[leftmargin=4.5\parindent]
\item[Problem 1:] Is there a matter of fact of what such counterfactual relative frequencies are?
\item[Problem 2:] Reordering 

For example, suppose we have an hypothetical infinite sequence of coin-tossing. And suppose the relative frequency sequence for heads converges to $1/2$. But by suitably reordering this sequence, we can make it converge to any value in $[0, 1]$ that we like.

\item[Problem 3:] Again, what should be the reference class?

\end{enumerate}



\section{Rationalism}
\begin{description}
\item[Rationalism] The objective probability of a proposition $A$ \textit{just is} the subjective probability that a perfectly rational agent would assign to $A$, if she had perfect information about events before at $t$ and no information about events after $t$.
\end{description}

\begin{enumerate}[leftmargin=4.5\parindent]
\item[Problem 1:] Sometimes it might be permissible for there to be more than one probability that a perfectly rational agent could assign to a certain proposition. Which one, then, should be the objective chance?
\end{enumerate}

\section{The Best-System Account}

\begin{description}
\item[The Best System Account] The objective probability of a proposition $A$ is the objective probability of $A$ under our best theory for the relevant phenomenon.
\end{description}

\begin{itemize}
\item A theory is best when (1) it is `fit' with the actual phenomenon; (2) it delivers an optimal combination of simplicity and strength.
\end{itemize}

\begin{enumerate}[leftmargin=4.5\parindent]
\item[Problem 1:] How do we balance simplicity and strength?
\item[Problem 2:] Is there an objective standard for simplicity? Or is it dependent upon human psychology?
\item[Problem 3:] Is simplicity language relative? After all, any theory can have the simplest specification possible: simply abbreviate it as $T$!
\end{enumerate}


\section{Primitivism}
\begin{description}
\item[Primitivism] The notion of objective probability is primitive and cannot be further analyzed.
\end{description}

\begin{enumerate}[leftmargin=4.5\parindent]
\item[Problem 1:] Potential explanatory loss: why rational belief ought to track objective probability?
\end{enumerate}



\end{document}