\documentclass[11pt]{article}

\usepackage{amsmath,amsthm,amsfonts,amssymb,amscd}

\usepackage[margin=1.5in,headsep=.5in]{geometry}

\usepackage{fancyhdr}

\setlength{\headheight}{20pt}

\usepackage[colorlinks]{hyperref} 
\usepackage{cleveref}

\usepackage{enumitem}
\setlist[enumerate]{itemsep=0mm}

\theoremstyle{definition}
\newtheorem{defn}{Definition}
\newtheorem{reg}{Rule}
\newtheorem{exer}{Exercise}
\newtheorem{note}{Note}
\newtheorem*{theorem*}{Theorem}
\newtheorem{theorem}{Theorem}[section]
\newtheorem{corollary}{Corollary}[theorem]
\newtheorem{thm}{Theorem}
\newtheorem{prop}[thm]{Proposition}
\newtheorem{lem}[thm]{Lemma}
\newtheorem{conj}[theorem]{Conjecture}
\newtheorem*{axm}{Axiom of Choice}
\newtheorem{exm}{Example}

\pagestyle{fancy}

\begin{document}

\pagenumbering{gobble}

\lhead{$24.118$ Paradox and Infinity }
\rhead{Recitation $8$: Computability}



\begin{center}
{\LARGE \bf The Church-Turing Thesis}
\end{center}

\smallskip

\section{The Informal Concept}

Pretheoretically, we have the informal concept of a problem being solvable ``algorithically"/``effectively"/``mechanically"/``systematically". Roughly, it means that it is possible to specify a finite list of instructions such that:

\begin{enumerate}
\item Following the instructions is guaranteed to yield a solution to the problem in a finite amount of time.
\item The instructions are specified in such a way that carrying them out requires no ingenuity of any kind: they can be followed mechanistically.
\item No Special resources are required to carry out the instructions: they could, in principle, be carried out by an ordinary human (albeit a human equipped with unlimited supplies of pencils and paper and patience).
\item No special physical conditions are required for the computation to succeed (no need for faster-than-light travel, special solutions to Einstein's equations, etc.)
\end{enumerate}

Example: The function $f^{+2}: \mathbb{N} \rightarrow \mathbb{N}$ that takes any natural number $n$ to $n+2$.

\section{The Thesis}

In his famous paper of 1936, Alan Turing presented a formally exact predicate with which the informal predicate ``can be done by means of an effective method” may be replaced (Turing 1936). Alonzo Church, working independently, did the same (Church 1936). \\

The formal concept proposed by Turing was ``\textit{computability by Turing machine}". The formal concept proposed by Church was ``\textit{lambda-definability}'. Despite being very different superficially, these two concepts turned out to be equivalent.

\begin{description}[style=unboxed,leftmargin=0cm]
\item[The Church-Turing Thesis] A function is Turing-computable if and only if it can be computed algorithmically, in the informal sense. 
\end{description}

\section{Why Accepting the Thesis?}

\subsection{By Intuition/Definition}

According to Turing, his thesis is not susceptible to mathematical proof. Rather, “[a]ll arguments which can be given” for the thesis, are ``fundamentally, appeals to intuition". (Turing 1936). He says,

\begin{quote}
The statement is … one which one does not attempt to prove. Propaganda is more appropriate to it than proof, for its status is something between a theorem and a definition. (Turing 1954)
\end{quote}

He also says that the phrase ``systematic method"

\begin{quote}
is a phrase which, like many others e.g., ‘vegetable’ one understands well enough in the ordinary way. But one can have difficulties when speaking to greengrocers or microbiologists or when playing ‘twenty questions’. Are rhubarb and tomatoes vegetables or fruits? Is coal vegetable or mineral? What about coal gas, marrow, fossilised trees, streptococci, viruses? Has the lettuce I ate at lunch yet become animal? … The same sort of difficulty arises about question c) above [Is there a systematic method by which I can answer such-and-such questions?]. An ordinary sort of acquaintance with the meaning of the phrase ‘systematic method’ won’t do, because one has got to be able to say quite clearly about any kind of method that might be proposed whether it is allowable or not. (Turing in Copeland 2004)

\end{quote}

Here Turing emphasizes that the term ``systematic method" is not exact, and so in that respect is like the term ``vegetable" but unlike mathematically precise terms, such as ``Turing-machine computable". The appropriate way to support a statement that pairs ``systematic methods" with items falling under a mathematically precise description is to offer ``propaganda", rather than to attempt to prove it. \\

Somewhat similar to Turing's view, Church states, more explicitly, that we should simply regard the Thesis as a definition, rather than a conjecture or a hypothesis. He proposes that we

\begin{quote}
define the notion … of an effectively calculable function of positive integers by identifying it with the notion of a recursive function of positive integers (or of a $\lambda$-definable function of positive integers). (Church 1936)
\end{quote}

On this view, then, the Church-Turing Thesis is trivially and necessarily true.

\subsection{Loads of Evidence}

A different view is that the Church-Turing Thesis should be regarded as a ``working hypothesis" that admits of empirical evidence. Emil Post, for example, criticized Church for masking this hypothesis as a definition:

\begin{quote}
[T]o mask this identification under a definition … blinds us to the need of its continual verification. (Post 1936)
\end{quote}

Since 1936, much evidence has amassed for this ``working hypothesis". In summary:

\begin{enumerate}
\item Every effectively calculable function that has been investigated in this respect has turned out to be computable by Turing machine.
\item All known methods or operations for obtaining new effectively calculable functions from given effectively calculable functions are paralleled by methods for constructing new Turing machines from given Turing machines.
\item All attempts to give an exact analysis of the intuitive notion of an effectively calculable function have turned out to be equivalent, in the sense that each analysis offered has been proved to pick out the same class of functions, namely those that are computable by Turing machine.
\end{enumerate}

(3) is often considered to be very strong evidence for the thesis, because of the diversity of the various formal analyses involved.  For example, there are analyses in terms of register machines (Shepherdson and Sturgis 1963), Post’s canonical and normal systems (Post 1943, 1946), combinatory definability (Schönfinkel 1924; Curry 1929, 1930, 1932), Markov algorithms (Markov 1960), and Gödel’s notion of reckonability (Gödel 1936; Kleene 1952). \\

On this view, the Church-Turing Thesis has a status similar to statements like ``the Earth is not flat". We should accept it because it is supported by loads of evidence. But its truth is neither trivial nor necessary.

\subsection{Provable Mathematically}

Contrary to Turing's own view, some people believe that the Church-Turing Thesis is not only empirically verifiable but also mathematically provable. For example, Saul Kripke has sketched the following four-step argument. \\

The first premise of Kripke's argument - the key step - is that effective computation is itself a form of mathematical deduction:

\begin{quote}
[A] computation is a special form of mathematical argument. One is given a set of instructions, and the steps in the computation are supposed to follow—follow deductively—from the instructions as given. So a computation is just another mathematical deduction, albeit one of a very specialized form. (Kripke 2013)
\end{quote}

The second premise is what Kripke calls Hilbert's Thesis: this is the thesis that any mathematical deduction can be expressed in a first-order language. Applying the second premise to the first, we have:

\begin{quote}
Every (effective) computation can be formalized as a valid deduction in a first order-language.
\end{quote}

The third premise of this argument is G\"odel's Completeness Theorem. Roughly, it states that every valid deduction (in a first-order language) is provable in first-order calculus. Applying this to the above we get:

\begin{quote}
Every (effective) computation is provable in first-order calculus.
\end{quote}

The final premise is Turing’s provability theorem: every formula provable in first-order calculus can be proved by the universal Turing machine (Turing 1936). Applying to the above we get (the hard direction of) the Church-Turing Thesis:

\begin{quote}
Every (effective) computation can be done by Turing machine. 
\end{quote}

(The other direction of the Church-Turing Thesis is quite uncontroversial, since a Turing machine is itself a specification of an effective computation.)

\end{document}

