\documentclass[11pt]{article}

\usepackage{amsmath,amsthm,amsfonts,amssymb,amscd}

\usepackage[margin=1.5in,headsep=.5in]{geometry}

\usepackage{fancyhdr}

\setlength{\headheight}{20pt}

\usepackage[colorlinks]{hyperref} 
\usepackage{cleveref}

\usepackage{enumerate}

\usepackage{enumitem}
\setlist[enumerate]{itemsep=0mm}

\theoremstyle{definition}
\newtheorem{defn}{Definition}
\newtheorem{reg}{Rule}
\newtheorem{exer}{Exercise}
\newtheorem{note}{Note}
\newtheorem*{theorem*}{Theorem}
\newtheorem{theorem}{Theorem}[section]
\newtheorem{corollary}{Corollary}[theorem]
\newtheorem{thm}{Theorem}
\newtheorem{prop}[thm]{Proposition}
\newtheorem{lem}[thm]{Lemma}
\newtheorem{conj}[theorem]{Conjecture}
\newtheorem*{wo}{The Well-Ordering Principle}

\pagestyle{fancy}

\newcommand{\counterfactual}{\ensuremath{%
  \Box\kern-1.5pt
  \raise1pt\hbox{$\mathord{\rightarrow}$}}}

\begin{document}

\pagenumbering{gobble}

\lhead{$24.118$ Paradox and Infinity}
\rhead{Recitation $7$: Axiom of Choice}



\begin{center}
{\LARGE \bf Answers to In-class Exercises}
\end{center}

\smallskip


\begin{defn}[Well-Order]
A well-order $<$ on a set $S$ is an order that satisfies the following properties:
\begin{enumerate}
\item Anti-Symmetry and Transitivity;
\item Total: for any $a \neq b \in S$, either $a < b$ or $b <a$;
\item Every non-empty subset $X \subseteq S$ has a $<-$smallest element, that is, an element $x \in X$ such that for every other $y \in X$, $x <y$.
\end{enumerate}
\end{defn}

\begin{wo}
Every set is well-orderable, i.e. for every set $S$, there is some ordering $<$ on $S$ that is a well-order.
\end{wo}

\begin{exer}
Show that if the Well-Ordering Principle holds, then the Axiom of Choice holds. In particular, let $S = \{S_i \, \, | \, \, i \in I \}$ be a set of non-empty sets (indexed by $I$). Show that there is a choice function $c: S \rightarrow \bigcup S$ on $S$, that is, a function such that for any $i \in I$, $c(S_i) \in S_i$.
\end{exer}

\begin{proof}[Answer]
The Well-Ordering Principle says that all sets are well-orderable. Therefore the set $\bigcup S$ is well-orderable. Let $<$ be a well-order on $\bigcup S$. Now consider an arbitrary $S_i$. Obviously $S_i \subseteq \bigcup S$. By the definition of a  well-order, $S_i$ has a $<-$smallest element. Let it be $c(S_i)$.

\end{proof}

\begin{exer}
Show that if the Axiom of Choice holds, then the Well-Ordering Principle holds. In particular, if $S$ is a set, then there is a well-order $<$ on $S$.
\end{exer}

\noindent
This proof is harder than the other direction, and uses some technique that we haven't introduced yet. So I will guide you through parts of the proof, and let you fill in the rest of the details. \\

\noindent
To start with, we consider the powerset $\mathcal{P}(S)$ of $S$. Let $P = \mathcal{P}(S) \setminus \emptyset$, the set of all subsets of $S$ except the empty set. By the Axiom of Choice, there is a choice function $c: P \rightarrow S$ on $P$. \\

\noindent
Now we define a function $f$ from ordinals to $S$ using a process called transfinite recursion\footnote{For those of you who are familiar with the notion of recursion, this is indeed what you are imagining: transfinite induction is the extension of mathematical induction to ordinals.}. The main idea is the follows: the first step is to stipulate the value of $f(0)$, then for any ordinal $\alpha$, we define the value of $f(\alpha)$ based on the values that has been defined in all of the previous steps, that is, all the $f(\beta)$'s such that $\beta \in \alpha$. In particular, we let $f(0) = c(S)$. And for any $\alpha$, we let $f(\alpha) = c(S \setminus \{f(\beta) \, \, | \, \, \beta \in \alpha \})$, if $S \setminus \{f(\beta) \, \, | \, \, \beta \in \alpha \}$ is not empty, where $S \setminus \{f(\beta) \, \, | \, \, \beta \in \alpha \}$ is the subset of $S$ that contains everything except the $f(\beta)$'s, for $\beta \in \alpha$. When $S \setminus \{f(\beta) \, \, | \, \, \beta \in \alpha \}$ is empty, we end the process (in which case $f(\alpha)$ is undefined).

\begin{exer}
Show that this function $f$ is injective.
\end{exer}

\begin{proof}[Answer]
Suppose $\alpha$ and $\gamma$ are two distinct ordinals such that both $f(\alpha)$ and $f(\gamma)$ are defined. WLOG we assume that $\gamma \in \alpha$. Then $f(\alpha) = c(S \setminus \{f(\beta) \, \, | \, \, \beta \in \alpha \})$. Since $c$ is a choice function, $f(\alpha) \in S \setminus \{f(\beta) \, \, | \, \, \beta \in \alpha \}$. Since $\gamma \in \alpha$, $f(\gamma) \in \{f(\beta) \, \, | \, \, \beta \in \alpha \}$. Hence $f(\alpha) \neq f(\gamma)$.

\end{proof}

\begin{exer}
Show that the process has to end at some point, in the following sense: $f(\alpha)$ cannot be defined for every ordinal $\alpha$. (Hint: the Burali-Forti paradox)
\end{exer}

\begin{proof}[Answer]
Suppose otherwise, the by the previous exercise $f$ is an injection from the collection of all ordinals to $S$. This means there is a bijection between the collection of all ordinals and some subset $S'$ of $S$. Let $\alpha$ be the cardinality of $S'$. Restricting $f$ to $\alpha +1$, then, we get an injection from $\alpha +1$ to $\alpha$, meaning that $\alpha +1 \leqslant \alpha$, which is a contradiction since $\alpha < \alpha +1$.

\end{proof}

\begin{exer}
Let $\gamma$ be the least ordinal on which $f(\gamma)$ is undefined. Show that $f: \gamma \rightarrow S$ is a bijection.
\end{exer}

\begin{proof}[Answer]
We have already shown that $f$ is injective. Suppose $f$ is not surjective, then $\{f(\alpha) \, \, | \, \, \alpha \in \gamma\}$, the image of $f$, is a proper subset of $S$. Hence $S \setminus \{f(\alpha) \, \, | \, \, \alpha \in \gamma\}$ is not empty, and hence $f(\gamma)$ is defined, contradicting our assumption.

\end{proof}

\begin{exer}
Using the bijection $f: \gamma \rightarrow S$, show that there is a well order $<$ on $S$.
\end{exer}

\begin{proof}[Answer]
We know that every ordinal is well-ordered by $\in$. Define $<$ on $S$ as follows: for any $a, b \in S$, $a < b$ just in case $\alpha \in \beta$, where $f(\alpha) = a$ and $f(\beta) = b$. It is easy to show that $<$ is a well-order, since $\in$ is a well-order.

\end{proof}


\end{document}