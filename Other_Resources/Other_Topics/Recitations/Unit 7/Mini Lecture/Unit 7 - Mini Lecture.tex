\documentclass[11pt]{article}

\usepackage{amsmath,amsthm,amsfonts,amssymb,amscd}

\usepackage[margin=1.5in,headsep=.5in]{geometry}

\usepackage{fancyhdr}

\setlength{\headheight}{20pt}

\usepackage[colorlinks]{hyperref} 
\usepackage{cleveref}

\usepackage{enumitem}
\setlist[enumerate]{itemsep=0mm}

\theoremstyle{definition}
\newtheorem{defn}{Definition}
\newtheorem{reg}{Rule}
\newtheorem{exer}{Exercise}
\newtheorem{note}{Note}
\newtheorem*{theorem*}{Theorem}
\newtheorem{theorem}{Theorem}[section]
\newtheorem{corollary}{Corollary}[theorem]
\newtheorem{thm}{Theorem}
\newtheorem{prop}[thm]{Proposition}
\newtheorem{lem}[thm]{Lemma}
\newtheorem{conj}[theorem]{Conjecture}
\newtheorem*{axm}{Axiom of Choice}
\newtheorem{exm}{Example}

\pagestyle{fancy}

\begin{document}

\pagenumbering{gobble}

\lhead{$24.118$ Paradox and Infinity }
\rhead{Recitation $7$: Axiom of Choice}



\begin{center}
{\LARGE \bf The Axiom of Choice}
\end{center}

\smallskip

\section{A Brief Recap}

\begin{axm}
Let $S = \{S_i \, \, | \, \, i \in I \}$ be a set of non-empty sets (indexed by $I$). Then there is a choice set $X = \{x_i \, \, | \, \, i \in I\}$ that selects an $x_i$ from each $S_i$.
\end{axm}

\begin{exm}
Let $S$ be an infinite set of pairs of reals. One choice set $X$ is the one that contains the smaller one of each pair of reals.
\end{exm}

\section{Disasters with Choice}

We've seen two in this chapter: the existence of non-measurable sets and the Banach-Tarski Paradox. But there are more:

\begin{enumerate}
\item There is a well-ordering of the reals. (In fact, AC entails that there is a well-ordering of any set!)
\item In general, the existece of certain ``intangible'' mathematical objects (objects that are proved to exist, but which cannot be explicitly constructed). This is against some philosophical view about the nature of mathematics, say, constructivism - that mathematics is a product of mental construction.

\end{enumerate}

\section{Disasters without Choice}

But why do we need Choice? What life is like without Choice?

\begin{enumerate}
\item A set can be infinite, but have no countably infinite subset.
\item Thus, it can be incorrect to say that $\aleph_o$ is the smallest infinite cardinality, since there can be infinite sets of incomparable size with  $\aleph_o$.
\item There can be an equivalence relation on $\mathbb{R}$, such that the number of equivalence classes is strictly greater than the size of $\mathbb{R}$.
\item There can be a vector space with no basis.
\item There can be a vector space with bases of different cardinalities.
\item The reals can be a countable union of countable sets.
\item Consequently, the theory of Lebesgue measure can fail totally.
\end{enumerate}

\section{Consistency and Independence Results}

\begin{enumerate}
\item[1904] Zermelo introduces axioms of set theory, explicitly formulates $AC$ and uses it to prove the well-ordering theorem, thereby raising a storm of controversy.
\item[1914] Hausdorff derives from $AC$ the existence of nonmeasurable sets.
\item[1924] Building on the work of Hausdorff, Banach and Tarski derive from $AC$ their paradoxical decompositions of the sphere.
\item[1935] Gödel establishes relative consistency of $AC$ with the axioms of set theory.
\item[1963] Cohen proves independence of $AC$ from the standard axioms of set theory.
\end{enumerate}

Gödel and Cohen collectively show that $AC$ is independent from the other axioms of set theory ($ZF$), meaning:
\begin{enumerate}
\item $ZFC$ = $ZF$ + $AC$ is consistent, and;
\item $ZF$ + the negation of $AC$ is also consistent.
\end{enumerate}

So what? What is the philosophical implication of these independence results? Is there a fact of the matter whether $AC$ is true? \\


Gödel: Yes! Sets are objects in the world. Just as there is a fact of the matter whether there are more than $8$ billion people on Earth, there is a fact of the matter whether $AC$ truly describes how sets behave.  \\


Cohen: No! The independence results effectively settle the question by showing that there is no answer. Which system to adopt is a matter of taste: we only need to consider practical reasons like simplicity, mathematical fruitfulness, etc.

\section{Some Cool Quotes}

\begin{quote}
The axiom of choice is obviously true, the well-ordering principle obviously false, and who can tell about Zorn's lemma.
\begin{flushright}
    Jerry Bona
\end{flushright}
\end{quote}

\begin{quote}
The Axiom of Choice is necessary to select a set from an infinite number of pairs of socks, but not an infinite number of pairs of shoes.
\begin{flushright}
   Bertrand Russell
\end{flushright}
\end{quote}

\begin{quote}
The axiom gets its name not because mathematicians prefer it to other axioms.
\begin{flushright}
   A. K. Dewdney
\end{flushright}
\end{quote}

\end{document}