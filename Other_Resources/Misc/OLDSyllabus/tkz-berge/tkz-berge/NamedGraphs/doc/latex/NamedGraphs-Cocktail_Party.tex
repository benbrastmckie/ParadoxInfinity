\newpage\section{Cocktail Party graph}\label{cocktail}
%<––––––––––––––––––––––––––––––––––––––––––––––––––––––––––––––––––––––––––>
%<–––––––––––––––––––––––––––––   Cocktail Party  –––––––––––––––––––––––––––>
%<––––––––––––––––––––––––––––––––––––––––––––––––––––––––––––––––––––––––––>
\begin{NewMacroBox}{grCocktailParty}{\oarg{options}\var{integer}}

\medskip
From MathWord : \url{http://mathworld.wolfram.com/CocktailPartyGraph.html}  

\emph{The cocktail party graph of order , also called the hyperoctahedral graph (Biggs 1993, p. 17) is the graph consisting of two rows of paired nodes in which all nodes but the paired ones are connected with a graph edge. It is the graph complement of the ladder graph , and the dual graph of the hypercube graph.\hfill\break
This graph arises in the handshake problem. It is a complete n-partite graph that is denoted  by Brouwer et al. (1989, pp. 222-223), and is distance-transitive, and hence also distance-regular.\hfill\break
The cocktail party graph of order  is isomorphic to the circulant graph.}
\href{http://mathworld.wolfram.com/topics/GraphTheory.html}%
           {\textcolor{blue}{MathWorld}} by \href{http://en.wikipedia.org/wiki/Eric_W._Weisstein}%
           {\textcolor{blue}{E.Weisstein}}

\medskip
The Chvátal graph is implemented in \tkzname{tkz-berge} as \tkzcname{grCocktailParty} with two forms.
\end{NewMacroBox}

\subsection{\tkzname{Cocktail Party graph form 1 }}
\tikzstyle{VertexStyle}   = [shape          = circle,
                             shading         = ball,%
                             ball color      = green,%
                             minimum size    = 24pt,%
                             draw]
\SetVertexMath
\tikzstyle{EdgeStyle}     = [thick,%
                             double          = orange,%
                             double distance = 1pt] 
\begin{center}
   \begin{tkzexample}[vbox]
    \begin{tikzpicture}
       \grCocktailParty[RA=3,RS=5]{4}
 \end{tikzpicture}
\end{tkzexample} 
\end{center}

\vfill\newpage
\subsection{\tkzname{Cocktail Party graph form 2 }}

\vspace*{2cm}
\begin{center}
   \begin{tkzexample}[vbox]
   \begin{tikzpicture}
      \grCocktailParty[form=2,RA=4,RS=6]{4}
     \end{tikzpicture}
\end{tkzexample} 
\end{center}
\endinput
