\documentclass[12pt,a4paper]{article}


%Spacing Packages
\usepackage{fullpage}
\usepackage{a4wide}

%Other Packages
\usepackage{amssymb}
\usepackage{amsmath}
\usepackage{euscript}

\usepackage[lf]{venturis} %% lf option gives lining figures as default; 
			  %% remove option to get oldstyle figures as default
\usepackage[T1]{fontenc}

\begin{document}

\begin{center} {\large 24.118: Paradox and Infinity -- Spring, 2016\\ \vspace{1mm}}
 {\large Problem Set 1: Answer Key\\ \vspace{1mm}}
 \end{center}
 
 \begin{enumerate}
 
 \item The main thing you needed to do, in answering this question, was show an appreciation for the fact that the case for premise 3 in the Grandfather Paradox is, arguably, stronger than the case for premise 3 in the Tabitha-Hitler argument. This is because it is plausible that a) for someone to be able to do something, there's got to be a possible scenario in which they do it, and b) there is no such possible scenario, when it comes to Tim killing grandfather, but there is such a possible scenario when it comes to Tabitha killing Hitler.
 
Now, that is just how things seem \textsl{prima facie}. You might think that how things seem \textsl{prima facie} is right. But you might not; you might think, like Lewis, that there \emph{is} a possible scenario in which Tim kills Grandfather, in which case you might regard both premise 3 of the Grandfather paradox and premise 3 of the Tabitha-Hitler argument as true, at least in some contexts. Or you might think that, actually, there is no possible scenario in which Tabitha kills Hitler, as, actually, whether Tabitha gets born at all depends very sensitively on the history of the world, and any change as significant as killing Hitler makes it impossible for Tabitha to be born.

You might \emph{even} think that the principle that, in order for someone to be able to do something, it must be possible for that person to do it, is wrong. MIT's own Jack Spencer argues for exactly that (though not in this context; last time I checked, he wasn't sure whether Tim is able to kill Grandfather or not).

\item Lewis says: ``able'' is context sensitive. When you say some person $P$ is able to do something $\phi$, you are saying that it is compatible with some contextually determined set of facts that P $\phi$.

So, assuming Lewis is right, is the argument valid? If we fix the sense of `able', then yes; but not if we don't. If we fix the sense of able, is the argument sound? It depends on the context. In particular, whether premise 2 is true or not depends on the context. If the context is one in which the relevant set of facts includes the initial state of the universe, in all detail, and the laws of physics, then yes. But if the relevant set of facts does \emph{not} include all that, then no. So premise 2 is not true in all contexts, and neither is the conclusion of the argument.

\item This question is relatively open; as long as you said something thoughtful and clear, you should have done fine.

Here is something you \emph{could} have said. Lewis is certainly right in claiming that there are many different senses of ``able''. But let us stipulate that the ``able'' we are talking about is the one that is relevant to decision making, whatever that is. That's the one we care about. What would be nice is an argument that in \emph{that} sense, premise 2 is false; a Lewis-style response does not give you such an argument, by itself.

A further argument you might make, which I think isn't bad, is that the the relevant set of facts, when we \emph{normally} make claims about what someone is able to do, does not include all details about the initial state of the universe. And the one we normally use is, in fact, the relevant one to decision making -- after all, it serves us pretty well. So premise 2 is false, when we are using ``able'' in the sense relevant to decision making.

\item The main thing you needed to do, in answering this question, was be clear on what exactly the Vihvelin-Skow objection \emph{is}. It is to object to premise 2, on the grounds that it is only narrow abilities that Tim must share with Tim2, in the same position, not abilities in general. Narrow abilities are abilities, the having of which depends only on what is going on close to the agent, both in space and time. 

Does the objection work? You are free to take your own view on this matter. I'll just report my own view: I'm inclined to agree with Skow in thinking that it only works if ``the position'' is not specified too precisely, and that it does not work if we mean something extremely precise by ``the position''. If that's what you mean, then I think you should accept the conclusion of this argument, but reject premise 1 of the Grandfather Paradox.

\item In answering this question, you just needed to say something clear about the difference between epistemic and metaphysical possibility, and its possible relevance to what the merchant was able to do, in the Death in Damascus story.

Here is something you could have said. Yes, the agent was able to avoid her fate, in the sense that it was a metaphysical possibility for her to do so. It was within her power to stay in Damascus, rather than go to Aleppo, and had she stayed she would have avoided her date with Death. But as soon as Death warned her, the merchant was in a position to know that she was, in fact, going to end up meeting Death tomorrow, and avoiding her death became epistemically impossible for her. In that sense, it was not a possibility for the merchant to avoid her date with Death, and she was not able to.

\end{enumerate}



\end{document}

