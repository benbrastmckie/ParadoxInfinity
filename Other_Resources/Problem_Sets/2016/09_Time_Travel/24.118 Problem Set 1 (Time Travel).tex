


\documentclass[12pt,a4paper]{article}


%Spacing Packages
\usepackage{fullpage}
\usepackage{a4wide}

%Other Packages
\usepackage{amssymb}
\usepackage{amsmath}
\usepackage{euscript}

\usepackage[lf]{venturis} %% lf option gives lining figures as default; 
			  %% remove option to get oldstyle figures as default
\usepackage[T1]{fontenc}

\begin{document}

\begin{quote}
\begin{center} {\large 24.118: Paradox and Infinity -- Spring, 2016\\ \vspace{1mm}}
 {\large Problem Set 1: Time Travel\\ \vspace{1mm}}
 
\end{center}
\vspace{3mm}

\noindent How these problems will be graded:

\begin{itemize} 

\item Assessment will be based on both whether you give the correct answer and on the \emph{reasons} you give in support of your answers. (Note that not every question has a single correct answer.) Even if it is unclear whether your answer is correct, it can be clear whether or not the reasons you have given in support of your answer are good ones. 

\item  \emph{No answer may consist of more than 150 words}. For each answer, all words after the first 150 will be ignored.


\end{itemize} 

These two constraints are often in competition: it may sometimes seem to you that you can't argue for your answer properly in 150 words or less. Learning to deal with this problem is a skill you will acquire with practice. The ability to distill what is essential about a point in a few words requires clear thinking, and it is clear thinking that we are after.

\end{quote}

Throughout this problem set, ``time-travel'' means time-travel within a single universe, which has only one dimension of time.

\subsection*{Problems:}

\begin{enumerate}

\item Let us suppose that Tabitha's grandfather is \emph{not} Hitler, and that Tabitha was born in 1985. Let Tabitha be ``in the position'' if she is aiming her rifle at Hitler in 1938, training in mind, hate in heart, etc. Consider the following argument:
\begin{center}
\begin{tabular}{l}
1. If time-travel is possible, then it's possible for Tabitha to be in the position.\\
2. If it's possible for Tabitha to be in the position, then Tabitha is able to kill Hitler.\\
3. Tabitha is not able to kill Hitler.\\ \hline
4. Time-travel is not possible.
\end{tabular}
\end{center}

Is this argument importantly different from the Grandfather paradox? Explain your answer. 

\item Let something be \textsl{physically possible} if and only if it is consistent with the initial conditions of the universe, and the laws of physics. Consider the following argument:
\begin{center}
\begin{tabular}{l}
1. If determinism is true, then it is not physically possible for you to do anything other\\ \hspace{10mm} than what you in fact do.\\
2. If it is not physically possible for you to do something, you are not able to do it.\\ \hline
3. If determinism is true, you are not able to do anything other than what you in fact do.
\end{tabular}
\end{center}  

Construct an objection to this argument along the lines of Lewis's objection to the Grandfather paradox.

\item Do you think a Lewis-style response to the argument in Problem 2 works? Is there a better way to resist the conclusion? Explain your answer.

\item Let Tim2 be a molecule-for-molecule duplicate of Tim that appears spontaneously in 1921, and let ``the position'' be holding a rifle, in the apartment overlooking Grandfather's route, hate in heart, etc. Recall the following argument. 
\begin{center}
\begin{tabular}{l}
1. If it's possible from Tim2 to be in the position, then Tim2 is able to kill grandfather.\\
2. If Tim2 is able to do something, then Tim, in the same position, is able to do the same thing.\\ \hline
3. If it's possible for Tim to be in the position, Tim is able to kill Grandfather.
\end{tabular}
\end{center}

Does the Vihvelin-Skow objection to this argument work? Explain.

\item Here is the story of Death in Damascus:

\begin{quote}
One day a travelling merchant met Death in the souks of Damascus. Death appeared surprised, but she quickly recovered her characteristic cool and intoned with cadaverous solemnity, ``Prepare yourself; I am coming for you tomorrow.'' \vspace{.05in}

The merchant was terrified, and fled that very night to Aleppo. \vspace{.05in}

The next day, the merchant woke up and --- horror of horrors! --- found Death at her bedside. Her voice quaking, she managed to squeak, ``I thought you were looking for me in Damascus!'' \vspace{.05in}

``No, I was merely shopping in Damascus,'' said Death. ``That's why I was surprised to see you: it is written that our final meeting is in Aleppo.'' 
\end{quote}

Question: as the merchant pondered her fate in Damascus, was she able to avoid her meeting with Death in Aleppo? Reference the distinction between epistemic and metaphysical possibility in your answer.


\end{enumerate}



\end{document}

