

\documentclass[12pt,a4paper]{article}


%Spacing Packages
\usepackage{fullpage}
\usepackage{a4wide}

%Other Packages
\usepackage{amssymb}
\usepackage{amsmath}
\usepackage{euscript}

\usepackage[lf]{venturis} %% lf option gives lining figures as default; 
			  %% remove option to get oldstyle figures as default
\usepackage[T1]{fontenc}

\begin{document}

\begin{quote}

\begin{center} {\large 24.118x -- Paradox and Infinity \\ \vspace{1mm}}
 {\large Problem Set 2: Answer sheet \\ \vspace{1mm}}
 
\end{center}
\vspace{3mm}



\end{quote} 

\subsection*{Problems:}

\begin{enumerate}
  
  \item Yes, Hilbert's Hotel can accommodate all the infinite busloads of infinitely many guests. You can prove it using a similar technique to the one we used to prove that the rationals have the same cardinality as the natural numbers. Form a table with a row 0, row 1, row 2,\ldots and a column 0, column 1, column 2,\ldots. The first row can represent the first busload of infinite guests, the second row the second busload of infinite guests, and so on. Now, assign a natural number to each cell of the table, the way we did with the rational numbers.
  
To fit all the new guests into Hilbert's Hotel, have old guest $n$ move to room $2n$, for all $n$. Then put the new guests into all the odd rooms in the order defined by the assignment of natural numbers to cells in the infinite table.
  
\item No, the fact that an ordered set is dense does \emph{not} entail that it is bigger than the set of natural numbers. The rationals, under the standard ordering, are dense, but they have the same cardinality as the naturals, so they provide a counterexample.
 
 \item You can express a subset of the natural numbers as an infinite string of ``0''s and ``1''s. A ``0'' in the first place means that the number zero is not in the subset, a ``1'' means that the number zero \emph{is} in the subset; a ``0'' in the second place means that the number one is not in the subset, a ``1'' in the second place means that the number one \emph{is} in the subset; and in general, a ``0'' in the $n^\text{th}$ place means that $n-1$ is not in the subset, and a ``1'' in the $n^\text{th}$ place means that $n-1$ \emph{is} in the subset.
 
Now, there is an obvious bijection between these strings of ``0''s and ``1''s and the names of real numbers between zero and one, in binary notation. And the set of names of real numbers between zero and one in binary notation has the same cardinality as the set of real numbers between zero and one (for similar reasons to the reason that the set of names of numbers between zero and one in decimal notation has the same cardinality as the set of numbers between zero and one). And the set of real numbers between zero and one has the same cardinality as the set of real numbers.

So: the cardinality of the power set of the naturals is the same as the cardinality of the reals.

\item There are many ways to do this. Here's one: 
\begin{itemize}
\item map $[0, \frac{1}{2})$ to $[0,1)$ by $f(x)=2x-0$, 
\item map $[\frac{1}{2},\frac{3}{4})$ to $[1,2)$ by $f(x)=4x-1$, 
\item map $[\frac{3}{4},\frac{7}{8})$ to $[2,3)$ by $f(x)=8x-4$,
\item and so on.
\end{itemize}
(In general, map $[1-\frac{1}{2^n}, 1-\frac{1}{2^n+1})$ to $[n,n+1)$ by $f(x)=2^{n+1}x-(2^{n+1}-n-2)$.) 

This is a bijection; to see this, note that it is clearly a surjection, and it is an injection, as no two points in $[0,1)$ will be mapped to the same point in $[0,\infty)$.

\item Yes, there is a bijection between the natural numbers and the nodes of the tree. An easy way to pair every natural number with a distinct node, and vice-versa, is to just start with the root node, then go to the next row, left-to-right, then go to the next row, left-to-right, and so on. So the root node (in row zero) gets 0, the first node in row one on the left gets 1, the second gets 2, and in general nodes in row $n$ get natural numbers $2^{n-1} +1$ through to $2^n$.

Here is a slightly fancier way to show that there is a bijection between the nodes and the natural numbers which helps with the next problem. You can represent each node on the tree except the root (which we'll worry about that later) with a finite string of ``0''s and ``1''s, where ``0'' means ``go left'' and ``1'' means ``go right''; the string as a whole is to be interpreted as directions for finding the relevant node, starting from the root node. There is a bijection from such strings to the positive natural numbers given by the following mapping: map string $$d_nd_{n-1}d_{n-2}\ldots d_0$$ to $$(d_n+1)\times 2^n + (d_{n-1}+1)\times 2^{n-1} + \ldots + (d_0 +1)\times 2^0$$ (Write out the first few strings and their equivalent natural number and you will be convinced that this is a bijection between the non-root nodes and the positive naturals).

So map the non-root nodes to the positive naturals as above, and map the root node to 0; that gives you a bijection from the nodes to the naturals.

\item There is \emph{not} a bijection between the paths and the natural numbers. 

Each path can be represented with an infinite string of ``0''s and ``1''s --- one digit for each natural number. You can think of such a string as instructions for how to walk the path, starting from the root node: ``0''s means go left (or right, if you were facing the direction of the path) and ``1'' means go right (or left, if you were facing the direction of the path).

Call the set of such strings ``$S$''. As we saw in problem 3, the cardinality of $S$ equals the cardinality of the real numbers. So the cardinality of $S$ is strictly greater than the cardinality of the natural numbers, so the cardinality of the set of paths is strictly greater than the cardinality of the set of natural numbers, so there is no bijection from the paths to the natural numbers.

\end{enumerate}



\end{document}

