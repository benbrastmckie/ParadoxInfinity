\documentclass[a4paper, 11pt]{article} % Font size (can be 10pt, 11pt or 12pt) and paper size (remove a4paper for US letter paper)

\usepackage[protrusion=true,expansion=true]{microtype} % Better typography
\usepackage{graphicx} % Required for including pictures
\usepackage{wrapfig} % Allows in-line images
\usepackage{enumitem} %%Enables control over enumerate and itemize environments
\usepackage{setspace}
\usepackage{amssymb, amsmath, mathrsfs,mathabx} %%Math packages
\usepackage{stmaryrd}
\usepackage{mathtools}
\usepackage{multicol} 
\usepackage{mathpazo} % Use the Palatino font
\usepackage[T1]{fontenc} % Required for accented characters
\usepackage{array}
\usepackage{bibentry}
\usepackage{prooftrees} 
\usepackage[round]{natbib} %%Or change 'round' to 'square' for square backers
\setcitestyle{aysep=}
% \usepackage{fitchproof} 

% \linespread{1.05} % Change line spacing here, Palatino benefits from a slight increase by default

\newcommand{\tuple}[1]{\langle#1\rangle} %%Angle brackets
\newcommand{\corner}[1]{\ulcorner#1\urcorner} %%Angle brackets
\newcommand{\set}[1]{\lbrace#1\rbrace} %%Set brackets
\newcommand{\abs}[1]{|#1|} %%Set brackets
\newcommand{\interpret}[1]{\llbracket#1\rrbracket} %%Double brackets
\newcommand{\N}{\mathbb{N}}
\renewcommand{\L}{\mathcal{L}}
\newcommand{\D}{\mathbb{D}}
\newcommand{\Z}{\mathbb{Z}}
\renewcommand{\Pr}{\mathbb{P}}
\newcommand{\Q}{\mathbb{Q}}
\newcommand{\R}{\mathbb{R}}
\newcommand{\B}{\mathfrak{B}}
\renewcommand{\max}[1]{\texttt{max}\set{#1}}

\makeatletter
\renewcommand\@biblabel[1]{\textbf{#1.}} % Change the square brackets for each bibliography item from '[1]' to '1.'
\renewcommand{\@listI}{\itemsep=0pt} % Reduce the space between items in the itemize and enumerate environments and the bibliography

\renewcommand{\maketitle}{ % Customize the title - do not edit title and author name here, see the TITLE block below
\begin{flushright} % Right align
{\LARGE\@title} % Increase the font size of the title

\vspace{10pt} % Some vertical space between the title and author name

{\@author} % Author name
\\\@date % Date

\vspace{-20pt} % Some vertical space between the author block and abstract
\end{flushright}
}

%----------------------------------------------------------------------------------------
%	TITLE
%----------------------------------------------------------------------------------------

\title{\textbf{Set Theory}} % Subtitle

\author{\textsc{Paradox and Infinity}\\ \em Benjamin Brast-McKie} % Institution

\date{\today} % Date

%----------------------------------------------------------------------------------------

\begin{document}

\maketitle % Print the title section

\thispagestyle{empty}

%----------------------------------------------------------------------------------------

% NEXT:
  % extensionality gives uniqueness
  % no universal set
  % iterative conception of set


\section*{Motivations}

  \begin{itemize}
    \item[\it Dialectic:] Recall Ramsey's simple theory of types.
      \item Sets are replaced with properties.
      \item `$x \in x$' is treated as `$x(x)$' which cannot be typed.
      \item If `$\in$' is intelligible in its own right, the paradox remains.
      \item Set theory is more intuitive than simple type theory.
      \item Worth developing in place of or alongside type theory.
  \end{itemize}





\section*{Naive Set Theory}

  \begin{itemize}
    \item[\it Language:] $\forall, \exists, \neg, \vee, \wedge, \rightarrow, \leftrightarrow, =, \in, S, x_1, x_2,\ldots, (,)$ are primitive symbols. 
      \item For purposes of illustration we may add other predicates and names.
      \item But strictly speaking, the language of set theory is very austere.
    \item[\it Comprehension:] Every open sentence with one free variable corresponds to a set.
      \item $\exists y\forall x(x \in y \leftrightarrow \varphi)$ where `$y$' does not occur in `$\varphi$'.
      \item From open sentences to predicates \textit{vs.} definite descriptions.
      \item Sets are objects not properties.
      \item[\bf Question:] Why assume uniqueness.
    \item[\it Extensinoality:] Sets are defined by their members.
      \item $\forall z(z \in x \leftrightarrow z \in y) \rightarrow x = y$.
      \item The set of fish that walk is identical to the set of pigs that fly.
      \item Properties need not be identified with sets.
      \item Set theory restricts attention to the extensions of predicates.
  \end{itemize}




\section*{Russell's Paradox}

  \begin{itemize}
    \item[\it Russell Set:] The open sentence `$x \notin x$' can be used to define a set.
      \item $R\coloneq\set{x : x \notin x}$, i.e., $\exists y\forall x(x \in y \leftrightarrow x \notin x)$.
      \item $R \in R$ \textit{iff} $R \notin R$. 
      \item Restricting comprehension looks \textit{ad hoc}.
  \end{itemize}






\section*{The Iterative Conception of Set}

  \begin{itemize}
    \item[\it Intuitions:] No reason to expect our intuitions to be univocal.
      \item Lots of reasons to expect otherwise, e.g., \textit{same number as}, etc.
      \item First impressions often have to be revised.
    \item[\it Extensinoality:] Compare the following metaphysical theses.
      \item \textit{What it is to be} (identical to) a set is to have the members it has.
      \item \textit{What it is for} a set to exist is for its members to exist.
      \item Sets \textit{ontologically depend} on their members: part of what it is for a set to exist is for all of its members to exist.
      \item Put otherwise: for a set to exist it is \textit{necessary for} its members to exist.
      \item But then sets can't be members of themselves.
    \item[\it Sufficiency:] Is the existence of some entities \textit{sufficient for} a set to exist?
      \item Do you have to ``put a lasso around'' some entities to make a set?
      \item Intuitionists of a certain stripe might claim so.
      \item What about the empty set? Is it a product of our conceptual exertion.
      \item Platonists reject this, taking sets to exist objectively and necessarily.
      \item Existence of the members to be sufficient for the existence of a set.
  \end{itemize}




\section*{Separation Axiom}

  \begin{itemize}
    \item[\it Construction:] Sets are constructed not in time, but in nature.
      \item The ingredients precede the product in constitution.
      \item Given any things, we have a set from which we can build new sets.
    \item[\it Comprehension:] $\forall z\exists y\forall x(x \in y \leftrightarrow (y \in z \wedge \varphi))$, i.e., some $k \coloneq \set{x \in z : \varphi}$ for any set $z$.
      \item What of Russell's paradox?
      \item For any set $z$, there is a set $R_z \coloneq \set{x \in z : x \notin x}$.
    \item[\it Indefinite Extensibility:] Whence the contradiction?
      \item Assume $R_z \in z$ for contradiction.
      \item Then $R_z \in R_z$ \textit{iff} $R_z \in z$ and $R_z \notin R_z$ \textit{iff} $R_z \notin R_z$. 
      \item So $R_z \notin z$.
      \item Letting $z'\coloneq z\cup\set{R_z}$, then $R_{z'} \notin z'$, so $z'' \coloneq z'\cup{R_{z'}}$, etc.
    \item[\it Solution:] Consider the following conclusions.
      \item There is no universal set.
      \item No set belongs to itself and so $R_z = z$ for any set $z$.
      \item Every set is indefinitely extensible.
  \end{itemize}


\end{document}



