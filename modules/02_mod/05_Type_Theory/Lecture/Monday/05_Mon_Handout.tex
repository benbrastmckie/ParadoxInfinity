\documentclass[a4paper, 11pt]{article} % Font size (can be 10pt, 11pt or 12pt) and paper size (remove a4paper for US letter paper)

\usepackage[protrusion=true,expansion=true]{microtype} % Better typography
\usepackage{graphicx} % Required for including pictures
\usepackage{wrapfig} % Allows in-line images
\usepackage{enumitem} %%Enables control over enumerate and itemize environments
\usepackage{setspace}
\usepackage{amssymb, amsmath, mathrsfs,mathabx} %%Math packages
\usepackage{stmaryrd}
\usepackage{mathtools}
\usepackage{multicol} 
\usepackage{mathpazo} % Use the Palatino font
\usepackage[T1]{fontenc} % Required for accented characters
\usepackage{array}
\usepackage{bibentry}
\usepackage{prooftrees} 
\usepackage[round]{natbib} %%Or change 'round' to 'square' for square backers
\setcitestyle{aysep=}
% \usepackage{fitchproof} 

% \linespread{1.05} % Change line spacing here, Palatino benefits from a slight increase by default

\newcommand{\tuple}[1]{\langle#1\rangle} %%Angle brackets
\newcommand{\corner}[1]{\ulcorner#1\urcorner} %%Angle brackets
\newcommand{\set}[1]{\lbrace#1\rbrace} %%Set brackets
\newcommand{\abs}[1]{|#1|} %%Set brackets
\newcommand{\interpret}[1]{\llbracket#1\rrbracket} %%Double brackets
\newcommand{\N}{\mathbb{N}}
\renewcommand{\L}{\mathcal{L}}
\newcommand{\D}{\mathbb{D}}
\newcommand{\Z}{\mathbb{Z}}
\renewcommand{\Pr}{\mathbb{P}}
\newcommand{\Q}{\mathbb{Q}}
\newcommand{\R}{\mathbb{R}}
\newcommand{\B}{\mathfrak{B}}
\renewcommand{\max}[1]{\texttt{max}\set{#1}}

\makeatletter
\renewcommand\@biblabel[1]{\textbf{#1.}} % Change the square brackets for each bibliography item from '[1]' to '1.'
\renewcommand{\@listI}{\itemsep=0pt} % Reduce the space between items in the itemize and enumerate environments and the bibliography

\renewcommand{\maketitle}{ % Customize the title - do not edit title and author name here, see the TITLE block below
\begin{flushright} % Right align
{\LARGE\@title} % Increase the font size of the title

\vspace{10pt} % Some vertical space between the title and author name

{\@author} % Author name
\\\@date % Date

\vspace{0pt} % Some vertical space between the author block and abstract
\end{flushright}
}

%----------------------------------------------------------------------------------------
%	TITLE
%----------------------------------------------------------------------------------------

\title{\textbf{Type Theory}} % Subtitle

\author{\textsc{Paradox and Infinity}\\ \em Benjamin Brast-McKie} % Institution

\date{\today} % Date

%----------------------------------------------------------------------------------------

\begin{document}

\maketitle % Print the title section

\thispagestyle{empty}

%----------------------------------------------------------------------------------------


% \section*{Rationals from the Integers}
%
% \begin{enumerate}
%   \item[\it Construction:] The aim is to construct the numbers.
% \end{enumerate}


\section*{Reals from the Rationals}

\begin{enumerate}
  \item[\it Construction:] Dedekind aimed to construct the real numbers to ground analysis.
  \item[\it Definition:] A \textit{real number} is any nonempty proper subset $X \subset \Q$ where: 
    \item $X$ is downward closed, i.e., $x \in X$ whenever $y\in X$ and $x < y$. 
    \item $X$ has no greatest element, i.e., for every $x \in X$, there is some $y \in X$ where $x < y$. 
  \item[\it Roots:] Let $\sqrt{2} = \set{x \in \Q : x^2 < 2}$.
  % \item[\it Least Upper Bound:] $\bigcup X$ is the \textit{least upper bound} of a set of real numbers $X$.
  \item[\it Higher-Order:] The real numbers are of higher-order than the rationals.
  \item[\it Subsumption:] The rationals are \textit{subsumed} by the reals $\frac{2}{5} = \set{x \in \Q : x < \frac{2}{5}}$.
\end{enumerate}





\section*{Greatest Lower Bounds}

\begin{enumerate}
  \item[\it Ordering:] For $x, y \in \R$, $x \leq y$ \textit{iff} $x \subseteq y$. 
  \item[\it Lower Bound:] $x$ is a \textit{lower bound} of $X \subseteq R$ \textit{iff} $x \leq y$ for all $y \in X$.  
  \item[\it Greatest Lower Bound:] $x$ is a \textit{greatest lower bound} of $X \subseteq R$ \textit{iff} $x$ is a lower bound of $X$ and $x \geq y$ for any lower bound of $y$ of $X$.
    \item[$\bullet$] The definition of the glb of $X$ quantifies over real numbers in $X$.
    \item[$\bullet$] Mathematics is full of definitions of properties of every higher-order.
    \item[$\bullet$] We want to quantify over all properties of real numbers at once.
  \item[\it Properties:] `$P(6)$' vs. `$6 \in \pi$' where $\pi=\set{x \in \N : P(x)}$.
  \item[\it Type Restrictions:] ``Whatever contains an apparent variable must be of a different type from the possible values of that variable\ldots'' 
\end{enumerate}




\section*{Predicative Properties}

  \begin{itemize}
    \item[\it Typed Expresions:] Recall the recursive clauses from last time:
      \item If $(t_1^{o_1},\ldots,t_n^{o_n})^a$ is predicative and $\varphi_1 : t_1^{o_1}$, \ldots, $\varphi_n : t_n^{o_n}$ for expressions $\varphi_1,\ldots,\varphi_n$, then $z(\varphi_1,\ldots,\varphi_n) : ((t_1^{o_1},\ldots,t_n^{o_n})^a,t_1^{o_1},\ldots,t_n^{o_n})^{a+1}$.
      \item If $\varphi : (t_1^{o_1},\ldots,t_n^{o_n})^a$, then $\forall x : t_i^{o_i} \varphi : (t_1^{o_1},t_{i-1}^{o_{i-1}},\ldots,t_{i+1}^{o_{i+1}},t_n^{o_n})^a$.
    \item[\it Stratification:] Quantification is stratified by ramified types.
    \item[\it Predicative Types:] $t^o$ is \textit{predicative} if $o$ is as low as it can be, c.f., $(0^0)^1$ and $(0^0)^2$.  
  \end{itemize}





\section*{Axiom of Reducibility}

\begin{itemize}
  \item[\it Axiom of Reducibility:] ``[E]very propositional function is equivalent, for all its values, to some predicative function.'' (pp.~242-3)
    \item Want all properties of real numbers to be on a par.
    \item Russell assumes there always are equivalent predicative properties.
    \item What is the property \textit{being the glb of $X$} equivalent to?
    \item Neither Russell nor anyone else can say.
  \item[\it Construction:] This undermines the spirit of Dedekind's project.
    \item The aim is to construct the numbers (ultimately from the empty set).
    \item But Russell was a logicist, not an constructivist/intuitionist.
    \item Logicists have a much more realist conception of mathematics where logic describes objective universal principles (the laws of thought) and mathematics reduces to logic.
    \item Nevertheless, one might worry that the AR is not logical, i.e., it is thinkable (consistent) for it to be false.
\end{itemize}



\section*{Mathematics or Metaphysics}

\begin{itemize}
  \item[\it Planets Example:] Let Hesperus and Phosphorus have the same first-order properties.
    \item Hesperus is shining \textit{iff} Phosphorus is shining, etc.
    \item Could they differ in higher-order properties?
    \item Maybe someone loves Hesperus but no one loves Phosphorus?
    \item There is a relation someone bears to Hesperus but not to Phosphorus?
    \item Given that Hesperus is Phosphorus, this might seem unlikely.
  \item[\it Iron Spheres:] Consider two iron spheres that have all the same properties. 
    \item Could there be a higher-order property upon which they differ?
  \item[\it Logic:] Logic is concerned with what must be the case.
    \item By `must' we mean: it would be contradictory for it to not be the case.
    \item Compare the axioms of set theory which are contingent.
    \item Must the axiom of reducibility hold?
    \item Many have thought it contingent (e.g., Ramsey and Wittgenstein).
  \item[\it Logicists:] Ensuring AR is a truth of logic is required for logicism.
    \item Reducing math to logic is an ambitious program.
    \item Neither logicists and intuitionists can easily accommodate AR.
  % \item[\it Wittgenstein's Example:] Imagine there are a countable infinity of things. 
  %   \item There is \textit{one} relation $R$ which holds of infinitely many things.
  %   \item $R$ does not hold between all things.
  %   \item $R$ does not hold between finitely many things. 
\end{itemize}











\end{document}



