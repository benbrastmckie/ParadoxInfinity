\documentclass[a4paper, 11pt]{article} % Font size (can be 10pt, 11pt or 12pt) and paper size (remove a4paper for US letter paper)

\usepackage[protrusion=true,expansion=true]{microtype} % Better typography
\usepackage{graphicx} % Required for including pictures
\usepackage{wrapfig} % Allows in-line images
\usepackage{enumitem} %%Enables control over enumerate and itemize environments
\usepackage{setspace}
\usepackage{amssymb, amsmath, mathrsfs,mathabx} %%Math packages
\usepackage{stmaryrd}
\usepackage{mathtools}
\usepackage{multicol} 
\usepackage{mathpazo} % Use the Palatino font
\usepackage[T1]{fontenc} % Required for accented characters
\usepackage{array}
\usepackage{bibentry}
\usepackage{prooftrees} 
\usepackage[round]{natbib} %%Or change 'round' to 'square' for square backers
\setcitestyle{aysep=}
% \usepackage{fitchproof} 

% \linespread{1.05} % Change line spacing here, Palatino benefits from a slight increase by default

\DeclareSymbolFont{symbolsC}{U}{txsyc}{m}{n}
\SetSymbolFont{symbolsC}{bold}{U}{txsyc}{bx}{n}
\DeclareFontSubstitution{U}{txsyc}{m}{n}
\DeclareMathSymbol{\boxright}{\mathrel}{symbolsC}{"80}
\DeclareMathSymbol{\circleright}{\mathrel}{symbolsC}{"91}
\DeclareMathSymbol{\diamondright}{\mathrel}{symbolsC}{"84}
\DeclareMathSymbol{\medcirc}{\mathrel}{symbolsC}{"07}

\newcommand{\tuple}[1]{\langle#1\rangle} %%Angle brackets
\newcommand{\corner}[1]{\ulcorner#1\urcorner} %%Angle brackets
\newcommand{\set}[1]{\lbrace#1\rbrace} %%Set brackets
\newcommand{\abs}[1]{|#1|} %%Set brackets
\newcommand{\interpret}[1]{\llbracket#1\rrbracket} %%Double brackets
\newcommand{\N}{\mathbb{N}}
\renewcommand{\L}{\mathcal{L}}
\renewcommand{\O}{\mathcal{O}}
\newcommand{\A}{\mathcal{A}}
\newcommand{\D}{\mathbb{D}}
\newcommand{\Z}{\mathbb{Z}}
\renewcommand{\Pr}{\mathbb{P}}
\newcommand{\Q}{\mathbb{Q}}
\newcommand{\R}{\mathbb{R}}
\newcommand{\B}{\mathfrak{B}}
\renewcommand{\max}[1]{\texttt{max}\set{#1}}

\makeatletter
\newcommand{\superimpose}[2]{%
  {\ooalign{$#1\@firstoftwo#2$\cr\hfil$#1\@secondoftwo#2$\hfil\cr}}}
\makeatother

\newcommand{\past}{\mathpalette\superimpose{{\Diamond}{\raisebox{1.5pt}{\tiny \hspace{.4pt}\textsc{p}}}}}

\newcommand{\Past}{\mathpalette\superimpose{{\Box}{\raisebox{1.2pt}{\tiny \textsc{p}}}}}

\newcommand{\future}{\mathpalette\superimpose{{\Diamond}{\raisebox{1.5pt}{\tiny \textsc{f}}}}}

\newcommand{\Future}{\mathpalette\superimpose{{\Box}{\raisebox{1.2pt}{\tiny \textsc{f}}}}}

\newcommand{\always}{\ensuremath \raisebox{1.3pt}{\rotatebox[origin=c]{180}{$\triangle$}}}

\newcommand{\sometimes}{\ensuremath \raisebox{-1.3pt}{$\triangle$}}

\makeatletter
\renewcommand\@biblabel[1]{\textbf{#1.}} % Change the square brackets for each bibliography item from '[1]' to '1.'
\renewcommand{\@listI}{\itemsep=0pt} % Reduce the space between items in the itemize and enumerate environments and the bibliography

\renewcommand{\maketitle}{ % Customize the title - do not edit title and author name here, see the TITLE block below
\begin{flushright} % Right align
{\LARGE\@title} % Increase the font size of the title

\vspace{10pt} % Some vertical space between the title and author name

{\@author} % Author name
\\\@date % Date

\vspace{-10pt} % Some vertical space between the author block and abstract
\end{flushright}
}

%----------------------------------------------------------------------------------------
%	TITLE
%----------------------------------------------------------------------------------------

\title{\textbf{Newcomb's Problem}} % Subtitle

\author{\textsc{Paradox and Infinity}\\ \em Benjamin Brast-McKie} % Institution

\date{\today} % Date

%----------------------------------------------------------------------------------------

\begin{document}

\maketitle % Print the title section

\thispagestyle{empty}

%----------------------------------------------------------------------------------------

\section*{Green Grass}

\begin{itemize}
  \item[\it Drought:] The town of Bayes has seen a terrible drought and the grass is dying.
    \item $P(\texttt{WetGrass}\ |\ \texttt{Umbrellas})=.99$ and $P(\texttt{DryGrass}\ |\ \texttt{Umbrellas}) = .01$.
    \item $P(\texttt{WetGrass}\ |\ \neg\texttt{Umbrellas})=.1$ and $P(\texttt{DryGrass}\ |\ \neg\texttt{Umbrellas}) = .9$.
    \item Assume that $v(\texttt{WetGrass}) = 10$ and $v(\texttt{DryGrass}) = -10$.
    \item $EV(\texttt{Umbrellas}) = .99 \times v(\texttt{WetGrass}) + .01 \times v(\texttt{DryGrass}) = 9.8$.
    \item $EV(\neg\texttt{Umbrellas}) = .1 \times v(\texttt{WetGrass}) + .9 \times v(\texttt{DryGrass}) = -8$.
  \item[\it Solution:] Upon learning of these numbers, the mayor calls for the Bayesians to go outside with their umbrellas so that the grass can get some water. 
    \item Something has gone wrong, but what is it?
\end{itemize}




\section*{Epilepsy}

\begin{itemize}
  \item[\it Lassie:] Angela has a dog Lassie which can reliably predict her seizures.
    \item $P(\texttt{Seizure}\ |\ \texttt{Bark})=.99$ and $P(\neg\texttt{Seizure}\ |\ \texttt{Bark}) = .01$.
    \item Assume that $v(\texttt{Seizure} + \texttt{Meds}) = 0$ and $v(\texttt{Seizure} + \neg \texttt{Meds}) = -10$.
    \item Lassie is barking and so Angela is sure to take her medication.
    \item But given what was said above, has she made a mistake?
\end{itemize}




\section*{Two Conditionals}

\begin{itemize}
  \item[\it Indication:] Umbrellas on the streets and Lassie's barks are reliable indicators.
    \item If you only knew \texttt{Umbrellas}/\texttt{Barks} you could make a safe bet.
    \item \texttt{Umbrellas} and \texttt{Barks} indicate a cause, but are not causes themselves.
    \item Their is a common cause of \texttt{Umbrellas}/\texttt{WetGrass} and \texttt{Barks}/\texttt{Seizure}.
  \item[\it Indicatives:] Indicative conditionals can be used to assert conditional knowledge.
    \item If \texttt{Barks}, then \texttt{Seizure}.
    \item If \texttt{Umbrellas}, then \texttt{WetGrass}.
  \item[\it Subjunctives:] Subjunctive conditionals can be used to track causal connections.
    \item If the Bayesians \textit{were} to go out with umbrellas, the grass \textit{would} be wet.
    \item If Lassie \textit{were} to bark, then Angela would have a seizure.
\end{itemize}

% NOTES
  % back to boxes
    % assuming you OneBox, getting rich is likely
    % but that doesn't mean OneBoxing causes the money to be there
    % it is just highly correlated with it
    % this explains what is wrong with the EV calculation
    % but what is the right way to reason and why
  % causal dependence
    % approximated by counterfactuals
    % it is likely that: given that you are out with an umbrella, the garden will 
    % if you were to go outside with an umbrella, the garden would get some water
  % 
  % 
  % 
  % 

\section*{Two Box}

\begin{itemize}
  \item[\it Action:] Faced with two boxes, the question is what are you in a position to do.
    \item LIKELY: If you choose \texttt{OneBox}, the big box will be \texttt{Full}.
    \item UNLIKELY: If you were to choose \texttt{OneBox}, the big box would be \texttt{Full}.
    % \item Choosing \texttt{OneBox} or \texttt{TwoBox} will not change the money in the boxes.
    % \item Can hope that you have been predicted to choose \texttt{OneBox} in the past.
    \item Choosing \texttt{OneBox} isn't going to fill it with money (compare \texttt{Umbrellas}).
    \item So you might as well take what is there, and hence \texttt{TwoBox}.
  \item[\it Independence:] When can we use an expected utility calculation as before?
    \item Is the outcome causally or probabilistically dependent on the action?
    \item An outcome $S$ is \textit{counterfactually independent} of an action $A$ \textit{iff} either: 
    \vspace{-.1in}
    \begin{enumerate}
      \begin{multicols}{2}
        \item $A \boxright S$ and $\neg A \boxright S$.
        \item $A \boxright \neg S$ and $\neg A \boxright \neg S$.
      \end{multicols}
    \end{enumerate}
    \vspace{-.1in}
    \item $P(\texttt{OneBox} \boxright \texttt{Full}) = P(\texttt{Full})$ and $P(\texttt{TwoBox} \boxright \texttt{Full}) = P(\texttt{Full})$.
    \item $P(\texttt{OneBox} \boxright \texttt{Empty}) = P(\texttt{Empty})$ and $P(\texttt{TwoBox} \boxright \texttt{Empty}) = P(\texttt{Empty})$.
    \item By exclusivity and exhaustivity, $P(\texttt{Empty}) = 1 - P(\texttt{Full})$.
    % \item Conditional probabilities are helpful under ignorance.
    % \item But Newcomb is not a case of betting about what we don't know.
  \item[\it Causal Decision Theory:] To be rational, maximize expected causal utility if possible.
    \item Weighting utilities in proportion to their likelihood is not the problem.
    \item The problem is mistaking probabilistic for counterfactual dependence.
    \item $ECU(A) = \sum\limits_{i \in I_A} v(S_i^A)P(A \boxright S_i^A)$ instead of $EV(A) = \sum\limits_{i \in I_A} v(S_i^A)P(S_i^A | A)$.
    \item \mbox{$ECU(\texttt{OneBox}) = \$1,000,000\times P(\texttt{OneBox} \boxright \texttt{Full}) + \$0\times P(\texttt{OneBox} \boxright \texttt{Empty})$}
    \item[] \hspace{.94in} $= \$1,000,000\times P(\texttt{Full})$.
    \item \mbox{$ECU(\texttt{TwoBox}) = \$1,001,000\times P(\texttt{TwoBox} \boxright \texttt{Full}) + \$1,000\times P(\texttt{TwoBox} \boxright \texttt{Empty})$.}
    \item[] \hspace{.94in} $= \$1,001,000\times P(\texttt{Full}) + \$1,000\times P(\texttt{Empty})$
    \item[] \hspace{.94in} $= \$1,001,000\times P(\texttt{Full}) + \$1,000\times (1 - P(\texttt{Full}))$
    \item[] \hspace{.94in} $= \$1,000,000\times P(\texttt{Full}) + \$1,000$
    \item[] \hspace{.94in} $= ECU(\texttt{OneBox}) + \$1,000$
    \item Two boxing is better independent of the value of $P(\texttt{Full})$.
\end{itemize}




\section*{Short}

\begin{itemize}
  \item[\it Bonus:] Choose, but before opening, bet about the total value for a bonus.
    \item Your choice may indicate what you were inclined to choose in the past.
    \item And your past inclinations indicate the prediction made about you.
    \item Bet \texttt{Empty} if and only if you chose \texttt{TwoBox}.
    \item You might feel that you have been punished for your rationality.
    \item So it goes in mischievous though experiments!
\end{itemize}





\end{document}



