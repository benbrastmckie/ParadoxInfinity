\documentclass[a4paper, 11pt]{article} % Font size (can be 10pt, 11pt or 12pt) and paper size (remove a4paper for US letter paper)

\usepackage[protrusion=true,expansion=true]{microtype} % Better typography
\usepackage{graphicx} % Required for including pictures
\usepackage{wrapfig} % Allows in-line images
\usepackage{enumitem} %%Enables control over enumerate and itemize environments
\usepackage{setspace}
\usepackage{amssymb, amsmath, mathrsfs,mathabx} %%Math packages
\usepackage{stmaryrd}
\usepackage{mathtools}
\usepackage{multicol} 
\usepackage{mathpazo} % Use the Palatino font
\usepackage[T1]{fontenc} % Required for accented characters
\usepackage{array}
\usepackage{bibentry}
\usepackage{prooftrees} 
\usepackage[round]{natbib} %%Or change 'round' to 'square' for square backers
\setcitestyle{aysep=}
% \usepackage{fitchproof} 

% \linespread{1.05} % Change line spacing here, Palatino benefits from a slight increase by default

\newcommand{\tuple}[1]{\langle#1\rangle} %%Angle brackets
\newcommand{\corner}[1]{\ulcorner#1\urcorner} %%Angle brackets
\newcommand{\set}[1]{\lbrace#1\rbrace} %%Set brackets
\newcommand{\abs}[1]{|#1|} %%Set brackets
\newcommand{\interpret}[1]{\llbracket#1\rrbracket} %%Double brackets
\newcommand{\N}{\mathbb{N}}
\renewcommand{\L}{\mathcal{L}}
\renewcommand{\O}{\mathcal{O}}
\newcommand{\A}{\mathcal{A}}
\newcommand{\D}{\mathbb{D}}
\newcommand{\Z}{\mathbb{Z}}
\renewcommand{\Pr}{\mathbb{P}}
\newcommand{\Q}{\mathbb{Q}}
\newcommand{\R}{\mathbb{R}}
\newcommand{\B}{\mathfrak{B}}
\renewcommand{\max}[1]{\texttt{max}\set{#1}}

\makeatletter
\newcommand{\superimpose}[2]{%
  {\ooalign{$#1\@firstoftwo#2$\cr\hfil$#1\@secondoftwo#2$\hfil\cr}}}
\makeatother

\newcommand{\past}{\mathpalette\superimpose{{\Diamond}{\raisebox{1.5pt}{\tiny \hspace{.4pt}\textsc{p}}}}}

\newcommand{\Past}{\mathpalette\superimpose{{\Box}{\raisebox{1.2pt}{\tiny \textsc{p}}}}}

\newcommand{\future}{\mathpalette\superimpose{{\Diamond}{\raisebox{1.5pt}{\tiny \textsc{f}}}}}

\newcommand{\Future}{\mathpalette\superimpose{{\Box}{\raisebox{1.2pt}{\tiny \textsc{f}}}}}

\newcommand{\always}{\ensuremath \raisebox{1.3pt}{\rotatebox[origin=c]{180}{$\triangle$}}}

\newcommand{\sometimes}{\ensuremath \raisebox{-1.3pt}{$\triangle$}}

\makeatletter
\renewcommand\@biblabel[1]{\textbf{#1.}} % Change the square brackets for each bibliography item from '[1]' to '1.'
\renewcommand{\@listI}{\itemsep=0pt} % Reduce the space between items in the itemize and enumerate environments and the bibliography

\renewcommand{\maketitle}{ % Customize the title - do not edit title and author name here, see the TITLE block below
\begin{flushright} % Right align
{\LARGE\@title} % Increase the font size of the title

\vspace{10pt} % Some vertical space between the title and author name

{\@author} % Author name
\\\@date % Date

\vspace{10pt} % Some vertical space between the author block and abstract
\end{flushright}
}

%----------------------------------------------------------------------------------------
%	TITLE
%----------------------------------------------------------------------------------------

\title{\textbf{Newcomb's Problem}} % Subtitle

\author{\textsc{Paradox and Infinity}\\ \em Benjamin Brast-McKie} % Institution

\date{\today} % Date

%----------------------------------------------------------------------------------------

\begin{document}

\maketitle % Print the title section

\thispagestyle{empty}

%----------------------------------------------------------------------------------------

\section*{Aporia}

\begin{itemize}
  \item[\it Opinionated:] Associated with being informed or knowledgeable (a strong look).
    \item Can be inappropriate for sensitive, complicated, or subtle matters.
    \item Can also make one less sensitive to alternatives (confirmation bias).
  \item[\it Uninformed:] Having no opinion can be due to lack of knowledge about a case.
    \item Knowledge is valued over ignorance, and opinion signals knowledge.
    \item But opinion doesn't entail knowledge, nor is it valuable itself.
  \item[\it Aporia:] Suspension of opinion for the sake of greater sensitivity to the truth.
    \item Aporia is often a state that is achieved since we often begin with biases.
    \item Aporia can be difficult to maintain, leading back to opinion.
  \item[\it Obvious:] Taking one's assumptions/biases to be obvious is the opposite.
    \item Paradoxes are one way to achieve aporia, even if only temporarily.
    \item Aim is to see one's views in the context of many alternatives.
\end{itemize}

\section*{Two Boxes or One?}

\begin{itemize}
  \item[\it Boxes:] A small box has \$1000 and a big box could have a \$1,000,000.
    \item On Wednesday, you will choose between two boxes or just the big box.
    \item On Monday, a predictor with 99\% accuracy put \$1,000,000 in the big box if you are predicted to take the big box, and nothing otherwise.
    \item What would an ideally rational agent do?
\end{itemize}



\section*{Two Box}

\begin{itemize}
  \item[\it Dominance:] Take both boxes, of course!
    \item The big box is either \texttt{Full} or \texttt{Empty} and not both.
    \item If \texttt{Empty}, then \texttt{OneBox} gives \$0 and \texttt{TwoBox} gives \$1,000.
    \item If \texttt{Full}, then \texttt{OneBox} gives \$1,000,000 and \texttt{TwoBox} gives \$1,001,000.
    \item \texttt{TwoBox} \textit{dominates} \texttt{OneBox} since it never leaves you worse off.
  \item[\it Rational:] To be rational, choose a dominant strategy if there is one.
  % \item[\it Rational:] Given that there is a dominate strategy, an ideally rational agent will choose that strategy, and so \texttt{TwoBox}.
\end{itemize}




\section*{One Box}

\begin{itemize}
  \item[\it Probability:] But this ignores the probabilities for the outcomes \texttt{Full} and \texttt{Empty}.
    \item The actions under consideration are $\A = \set{\texttt{OneBox}, \texttt{TwoBox}}$ .
    \item \mbox{The outcomes $\O_A = \set{\texttt{Full},\texttt{Empty}}$ are exclusive and exhaustive for $A \in \A$.} 
    \item $EV(A) = \sum\limits_{i \in I_A} v(S_i^A)P(S_i^A|A)$ where $\O_A = \set{S_i^A: i \in I_A}$ and $A \in \A$.
    \item $EV(\texttt{OneBox})=\$1,000,000\times.99 + \$0\times.01 = \$990,000$.
    \item $EV(\texttt{TwoBox})=\$1,001,000\times.01 + \$1,000\times.99 = \$11,000$.
    % \item $\sum\limits_{i \in I}P(S_i|A) = 1$ requires the outcomes to be exhaustive. 
    % \item $P(S_i \cap S_j)=0$.
    \item So \texttt{OneBox} is 90 times higher expected utility than \texttt{TwoBox}.
  \item[\it Maximize:] \mbox{To be rational, choose an $A \in \A$ where $EV(A) \geq EV(B)$ for any $B \in \A$ (if any).}  
    \item Expected utility is maximized by \texttt{OneBox}.
    \item Thus a rational agent will \texttt{OneBox}.
\end{itemize}




\section*{Double or Nothing}

\begin{itemize}
  \item[\it Bets:] You are given \$1,000 with the option of drawing a stone from an urn.
    \item The urn has 51 black stones and 49 white, all well mixed.
    \item Drawing black doubles your winnings, but drawing white loses all.
    \item Should you take draw a stone or take the \$1000?
  \item[\it Utility:] Suppose a rational agent would maximize expected utility.
    \item $EV(\texttt{Take}_1) = \$1,000 \times 1.00 = \$1,000$ since $\O_{\texttt{Take}_1}=\set{T_1}$ 
    \item $EV(\texttt{Draw}_1) = \$2,000 \times .51 + \$0 \times .49 = \$1,020$ since $\O_{\texttt{Draw}_1}=\set{B_1,W_1}$. 
    \item Since $EV(\texttt{Draw}_1) > EV(\texttt{Take}_1)$, the rational agent will draw.
    \item $EV(\texttt{Take}_2) = \$2,000 \times 1.00 = \$2,000$ since $\O_{\texttt{Take}_2}=\set{T_2}$ 
    \item $EV(\texttt{Draw}_2) = \$4,000 \times .51 + \$0 \times .49 = \$2,040$ since $\O_{\texttt{Draw}_2}=\set{B_2,W_2}$. 
    \item The same reasoning may be repeated indefinitely.
  \item[\it Risk:] What are the chances that you walk away with nothing after $n$ draws? 
    \item \$1,000 after 0 draws with probability 1: $EV(0) = \$1,000$.
    \item \$2,000 after 1 draws with probability .51: $EV(1) = \$1,020$.
    \item \$4,000 after 2 draws with probability .26: $EV(2) = \$1,040$.
    % \item \$8,000 after 3 draws with probability .13: $EV(3) = \$1,061$.
    \item $EV(n) = \$1,000 \times 2^n \times .51^n \geq \$1,000$ for all $n \geq 0$.
  \item[\it Certanty:] Fails to account for the value of certainty.
    % \item $EV(\texttt{Draw}_1') = \$2,000 \times .51 + \$0 \times .49 + v(\texttt{Certain}) \times .00 = \$1,020$. 
    \item $EV(\texttt{Take}_1') = (\$1,000 + v(\texttt{Certain})) \times 1.00 = \$1,000 + v(\texttt{Certain})$.
    \item $EV(\texttt{Draw}_1') = (\$2,000 + \frac{1}{.51}v(\texttt{Certain})) \times .51 = \$1,020 + v(\texttt{Certain})$. 
    \item Certainty has a different kind of value than money.
\end{itemize}



\end{document}



