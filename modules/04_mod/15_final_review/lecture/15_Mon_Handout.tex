\documentclass[a4paper, 11pt]{article} % Font size (can be 10pt, 11pt or 12pt) and paper size (remove a4paper for US letter paper)

\usepackage[protrusion=true,expansion=true]{microtype} % Better typography
\usepackage{graphicx} % for adding extra padding to rows
\usepackage{wrapfig} % Allows in-line images
\usepackage{enumitem} %%Enables control over enumerate and itemize environments
\usepackage{setspace}
\usepackage{amssymb, amsmath, mathrsfs,mathabx} %%Math packages
\usepackage{stmaryrd}
\usepackage{mathtools}
\usepackage{multicol} 
\usepackage{mathpazo} % Use the Palatino font
\usepackage[T1]{fontenc} % Required for accented characters
\usepackage{array}
\usepackage{bibentry}
\usepackage{prooftrees} 
\usepackage[round]{natbib} %%Or change 'round' to 'square' for square backers
\setcitestyle{aysep=}
% \usepackage{fitchproof} 

% \linespread{1.05} % Change line spacing here, Palatino benefits from a slight increase by default

\DeclareSymbolFont{symbolsC}{U}{txsyc}{m}{n}
\SetSymbolFont{symbolsC}{bold}{U}{txsyc}{bx}{n}
\DeclareFontSubstitution{U}{txsyc}{m}{n}
\DeclareMathSymbol{\boxright}{\mathrel}{symbolsC}{"80}
\DeclareMathSymbol{\circleright}{\mathrel}{symbolsC}{"91}
\DeclareMathSymbol{\diamondright}{\mathrel}{symbolsC}{"84}
\DeclareMathSymbol{\medcirc}{\mathrel}{symbolsC}{"07}

\newcommand{\tuple}[1]{\langle#1\rangle} %%Angle brackets
\newcommand{\corner}[1]{\ulcorner#1\urcorner} %%Angle brackets
\newcommand{\set}[1]{\lbrace#1\rbrace} %%Set brackets
\newcommand{\abs}[1]{|#1|} %%Set brackets
\newcommand{\interpret}[1]{\llbracket#1\rrbracket} %%Double brackets
\newcommand{\N}{\mathbb{N}}
\renewcommand{\L}{\mathcal{L}}
\renewcommand{\O}{\mathcal{O}}
\newcommand{\A}{\mathcal{A}}
\newcommand{\D}{\mathbb{D}}
\newcommand{\Z}{\mathbb{Z}}
\renewcommand{\Pr}{\mathbb{P}}
\newcommand{\Q}{\mathbb{Q}}
\newcommand{\R}{\mathbb{R}}
\newcommand{\B}{\mathfrak{B}}
\renewcommand{\max}[1]{\texttt{max}\set{#1}}

\makeatletter
\newcommand{\superimpose}[2]{%
  {\ooalign{$#1\@firstoftwo#2$\cr\hfil$#1\@secondoftwo#2$\hfil\cr}}}
\makeatother

\newcommand{\past}{\mathpalette\superimpose{{\Diamond}{\raisebox{1.5pt}{\tiny \hspace{.4pt}\textsc{p}}}}}

\newcommand{\Past}{\mathpalette\superimpose{{\Box}{\raisebox{1.2pt}{\tiny \textsc{p}}}}}

\newcommand{\future}{\mathpalette\superimpose{{\Diamond}{\raisebox{1.5pt}{\tiny \textsc{f}}}}}

\newcommand{\Future}{\mathpalette\superimpose{{\Box}{\raisebox{1.2pt}{\tiny \textsc{f}}}}}

\newcommand{\always}{\ensuremath \raisebox{1.3pt}{\rotatebox[origin=c]{180}{$\triangle$}}}

\newcommand{\sometimes}{\ensuremath \raisebox{-1.3pt}{$\triangle$}}

\makeatletter
\renewcommand\@biblabel[1]{\textbf{#1.}} % Change the square brackets for each bibliography item from '[1]' to '1.'
\renewcommand{\@listI}{\itemsep=0pt} % Reduce the space between items in the itemize and enumerate environments and the bibliography

\renewcommand{\maketitle}{ % Customize the title - do not edit title and author name here, see the TITLE block below
\begin{flushright} % Right align
{\LARGE\@title} % Increase the font size of the title

\vspace{10pt} % Some vertical space between the title and author name

{\@author} % Author name
\\\@date % Date

\vspace{-30pt} % Some vertical space between the author block and abstract
\end{flushright}
}

%----------------------------------------------------------------------------------------
%	TITLE
%----------------------------------------------------------------------------------------

\title{\textbf{Surprise Exam Paradox}} % Subtitle

\author{\textsc{Paradox and Infinity}\\ \em Benjamin Brast-McKie} % Institution

\date{\today} % Date

%----------------------------------------------------------------------------------------

\begin{document}

\maketitle % Print the title section

\thispagestyle{empty}

%----------------------------------------------------------------------------------------

\section*{Disbelief}

\begin{itemize}
  \item[\it Knowledge:] Sometimes the argument is developed in terms of knowledge.
    \item Were going to stick with belief.
  \item[\it Belief:] Can the students believe the instructor?
    \item Yes, easily so long as they don't do too much reasoning (bad answer).
    \item Let $S = (E_m \wedge \neg \B_m(E_m)) \vee (E_w \wedge \neg \B_w(E_w)) \vee (E_f \wedge \neg \B_f(E_f))$.
    \item The interesting question is whether $\B_m(S)$ given \textit{Closure}, etc. 
  \item[\it Logic:] Can logically omniscient students believe $S$ on Monday?
    \item It might seem that the arguments show that the answer is `No'. 
    \item But it seems like there can be surprises, and so $S$ could be true. 
    \item So are the logically omniscient students missing out on a true belief?
    % \item If $S$ can't be believed, what could the logically omniscient believe? 
  \item[\it Repost:] Perhaps the good reasons for belief are overturned by the argument.
    \item Even the most expert testimonies can be overturned, why not this?
    \item Remains to accommodate the possibility of a surprise exam.
    \item But we also want to maintain reasonably strong epistemic principles.
\end{itemize}





\section*{Doubts}

\begin{itemize}
  \item[\it Setup:] Why can't the students believe $S$?
    \item One explanation claims that $S$ happens to be false. 
    \item But surely $S$ is possible, and if so, assume such a case. 
    \item Another strategy looks to spot the mistake in our reasoning before.
  \item[\it One Day:] Can there be an announced surprise exam on just one day?
    \item Announcement: ``There is a surprise exam on Monday.''
    \item Seems that the announcement ensures that it is false.
  \item[\it Two Days:] Can there be an announced surprise exam on one of two days?
    \item Suppose the exam is held on Monday (as opposed to Wednesday).
    \item Would it come as a surprise to the students?
    \item On Monday, how could they be sure that it wasn't on Wednesday?
    \item Because if it was on Wednesday, it wouldn't be a surprise \textit{then}.
    \item But we might be surprised \textit{today} to find out that it is on Wednesday.
  % \item[\it Argument:] Which step in the argument have we gone wrong?
\end{itemize}




\section*{Surprise}

\begin{itemize}
  \item[\it Timing:] It would come as a surprise on Monday that it is/isn't on Monday.
    \item It wouldn't be a surprise on Wednesday if it didn't happen Monday.
    \item Do we need to maintain that it is a surprise on the day of?
    \item Why not take something to be a surprise by referencing the day before?
  \item[\it Analysis:] Let $E_i$ be a \textit{surprise iff} $E_i \wedge \neg \B_{i-1}(E_i)$.
    \item Assume $m-1 = f'$ (on the week before),~ $w-1=m$,~ and $f-1=w$. 
    \item If $E_f$, then since $\neg \B_{f-1}(E_f) = \neg \B_{w}(E_f)$, so it is a surprise.
    \item Really the surprise takes place on Wednesday.
    \item On Wednesday, the surprise is about whether $E_w$ or $E_f$. 
    % \item On Friday, the surprise has come and gone, but our new definition captures this.
  % \item[\it Belief:] New analysis seems to get something right about the conn
  \item[\it Surprise:] Does this new analysis capture a natural notion of surprise?
    \item No less reasonable than the first analysis, and blocks the argument.
    \item So there can be surprise exams, just not of the first kind of surprise.
    \item Is this adequate?
  \item[\it Learning:] Compare learning something new: you go from $\neg \B(X)$ to $\B(X)$.
    \item Suppose that you learn something now about something in the future.
    \item Suppose Ali will go on a walk tomorrow.
    \item Learning this today, must we be surprised?
    \item You might say, ``I'm not surprised,'' since Ali often goes on walks.
    \item But this has the same form as before: $\texttt{Walk}_i \wedge \neg \B_{i-1}(\texttt{Walk}_i)$.
  \item[\it Belief:] Could weaken our analysis to a mere necessary condition.
    \item Partial analysis risks being fairly weak, though still true.
    \item Consider the exclamations: ``I don't believe it!'', ``I am in disbelief!''.
    \item We say these things when we believe something that surprises us.
    \item It's not just that we learn something new, it has to be anticipated.
    % \item High stakes is one reason to anticipate something.
    % \item But it also seems that we can be surprised about things we do believe.
  \item[\it Credences:] But couldn't we anticipate Ali's walk without being surprised?
    \item Merely contemplating a future event is not enough to anticipate it.
    \item Instead of changing our beliefs, consider updating our credences.
    \item The bigger the jump in credences, the more surprising.
    \item When no test is given Wednesday, we go from $\frac{1}{2}$ to $1$ that it is on Friday. 
    \item Couldn't our expectation that Ali goes on a walk be similar?
    \item What makes the exam a surprise and Ali's walk anything but?
  \item[\it Stakes:] One thought is that the \textit{stakes} play a role.
    \item The higher the stakes, the more surprising something can be.
    % \item Or so it might seem\ldots
\end{itemize}



% \bibliographystyle{Phil_Review} %%bib style found in bst folder, in bibtex folder, in texmf folder.
% \nobibliography{Zotero} %%bib database found in bib folder, in bibtex folder
\end{document}
