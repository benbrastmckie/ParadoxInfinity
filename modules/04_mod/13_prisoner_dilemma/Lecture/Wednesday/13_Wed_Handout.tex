\documentclass[a4paper, 11pt]{article} % Font size (can be 10pt, 11pt or 12pt) and paper size (remove a4paper for US letter paper)

\usepackage[protrusion=true,expansion=true]{microtype} % Better typography
\usepackage{graphicx} % for adding extra padding to rows
\usepackage{wrapfig} % Allows in-line images
\usepackage{enumitem} %%Enables control over enumerate and itemize environments
\usepackage{setspace}
\usepackage{amssymb, amsmath, mathrsfs,mathabx} %%Math packages
\usepackage{stmaryrd}
\usepackage{mathtools}
\usepackage{multicol} 
\usepackage{mathpazo} % Use the Palatino font
\usepackage[T1]{fontenc} % Required for accented characters
\usepackage{array}
\usepackage{bibentry}
\usepackage{prooftrees} 
\usepackage[round]{natbib} %%Or change 'round' to 'square' for square backers
\setcitestyle{aysep=}
% \usepackage{fitchproof} 

% \linespread{1.05} % Change line spacing here, Palatino benefits from a slight increase by default

\DeclareSymbolFont{symbolsC}{U}{txsyc}{m}{n}
\SetSymbolFont{symbolsC}{bold}{U}{txsyc}{bx}{n}
\DeclareFontSubstitution{U}{txsyc}{m}{n}
\DeclareMathSymbol{\boxright}{\mathrel}{symbolsC}{"80}
\DeclareMathSymbol{\circleright}{\mathrel}{symbolsC}{"91}
\DeclareMathSymbol{\diamondright}{\mathrel}{symbolsC}{"84}
\DeclareMathSymbol{\medcirc}{\mathrel}{symbolsC}{"07}

\newcommand{\tuple}[1]{\langle#1\rangle} %%Angle brackets
\newcommand{\corner}[1]{\ulcorner#1\urcorner} %%Angle brackets
\newcommand{\set}[1]{\lbrace#1\rbrace} %%Set brackets
\newcommand{\abs}[1]{|#1|} %%Set brackets
\newcommand{\interpret}[1]{\llbracket#1\rrbracket} %%Double brackets
\newcommand{\N}{\mathbb{N}}
\renewcommand{\L}{\mathcal{L}}
\renewcommand{\O}{\mathcal{O}}
\newcommand{\A}{\mathcal{A}}
\newcommand{\D}{\mathbb{D}}
\newcommand{\Z}{\mathbb{Z}}
\renewcommand{\Pr}{\mathbb{P}}
\newcommand{\Q}{\mathbb{Q}}
\newcommand{\R}{\mathbb{R}}
\newcommand{\B}{\mathfrak{B}}
\renewcommand{\max}[1]{\texttt{max}\set{#1}}

\makeatletter
\newcommand{\superimpose}[2]{%
  {\ooalign{$#1\@firstoftwo#2$\cr\hfil$#1\@secondoftwo#2$\hfil\cr}}}
\makeatother

\newcommand{\past}{\mathpalette\superimpose{{\Diamond}{\raisebox{1.5pt}{\tiny \hspace{.4pt}\textsc{p}}}}}

\newcommand{\Past}{\mathpalette\superimpose{{\Box}{\raisebox{1.2pt}{\tiny \textsc{p}}}}}

\newcommand{\future}{\mathpalette\superimpose{{\Diamond}{\raisebox{1.5pt}{\tiny \textsc{f}}}}}

\newcommand{\Future}{\mathpalette\superimpose{{\Box}{\raisebox{1.2pt}{\tiny \textsc{f}}}}}

\newcommand{\always}{\ensuremath \raisebox{1.3pt}{\rotatebox[origin=c]{180}{$\triangle$}}}

\newcommand{\sometimes}{\ensuremath \raisebox{-1.3pt}{$\triangle$}}

\makeatletter
\renewcommand\@biblabel[1]{\textbf{#1.}} % Change the square brackets for each bibliography item from '[1]' to '1.'
\renewcommand{\@listI}{\itemsep=0pt} % Reduce the space between items in the itemize and enumerate environments and the bibliography

\renewcommand{\maketitle}{ % Customize the title - do not edit title and author name here, see the TITLE block below
\begin{flushright} % Right align
{\LARGE\@title} % Increase the font size of the title

\vspace{10pt} % Some vertical space between the title and author name

{\@author} % Author name
\\\@date % Date

\vspace{-10pt} % Some vertical space between the author block and abstract
\end{flushright}
}

%----------------------------------------------------------------------------------------
%	TITLE
%----------------------------------------------------------------------------------------

\title{\textbf{Prisoners' Dilemma}} % Subtitle

\author{\textsc{Paradox and Infinity}\\ \em Benjamin Brast-McKie} % Institution

\date{\today} % Date

%----------------------------------------------------------------------------------------

\begin{document}

\maketitle % Print the title section

\thispagestyle{empty}

%----------------------------------------------------------------------------------------

% NOTES
  % review assignment
  % Bermudez's argument
  % critique
  % twist: assume players are optimally rational and know it
    % is there one way to be optimally rational?
    % if so, players must act the same, eliminating two options
    % does this mean the players will cooperate?
    % what matters is if the players both think there is one way to be optimally rational
    % optimal rationality shouldn't include a theory of rationality any more than it should include a theory of physics
    % so this way of eliminating two of the four options fails
    % but maybe there is another way to eliminate options?

\section*{Instance Thesis}

\begin{itemize}
  \item[\it Argument:] Lewis argues that $(P)$ is an instance of $(N)$:
    \item[$(P)$] $\texttt{Rich}_A$ \textit{iff} $\neg \texttt{Take}_B$.
    \item[$(N)$] $\texttt{Rich}_A$ \textit{iff} it is predicted that $\neg \texttt{Take}_A$.
  \item[\it Inessentials:] Lewis claims $(N)$ is equivalent to the following:
    \item[$(N)'$] $\texttt{Rich}_A$ \textit{iff} a certain potentially predictive process (which may go on before, during, or after my choice) yields an outcome which could warrant a prediction that I do not take my \$1,000.
    \item Lewis claims that $(N)'$ eliminates inessentials from $(N)$.
    \item Focusing on $(N)$, we may take $(N)'$ to elaborate what $(N)$ intends.
  \item[\it Instance:] Is $(P)$ an instance of $(N)$?
    \item Does $\neg \texttt{Take}_B$ predict that $\neg \texttt{Take}_A$? 
    \item Lewis says `yes' when prisoners $A$ and $B$ are sufficiently similar. 
    \item What is sufficient for $\neg \texttt{Take}_B$ to predict that $\neg \texttt{Take}_A$?
\end{itemize}




\section*{Prediction and Probability}

\begin{itemize}
  \item[\it Motivation:] Lewis appeals to the prisoner's dilemma to motivate CDT. 
    \item All that matters is that $\neg \texttt{Take}_B$ raises the likelihood of $\neg \texttt{Take}_A$ enough. 
    \item Letting $r = \frac{\$1,000}{\$1,000,000}$, the probability must greater than $\frac{1 + r}{2} = .5005$. 
    \item $\neg \texttt{Take}_B$ predicts $\neg \texttt{Take}_A$ \textit{if} $P(\neg \texttt{Take}_A\ |\ \neg \texttt{Take}_B) > .5005$. 
  \item[\it Coin:] Does getting heads $7/10$ times \textit{predict} heads is more likely? 
    \item If the coin is fair, heads is just as likely as tails.
    \item The fairness of the coin justifies the prediction that heads is $.5$ likely. 
  \item[\it Similarity:] What could justify that $P(\neg \texttt{Take}_A\ |\ \neg \texttt{Take}_B) > .5005$?
    \item Lewis claims that simulation is a predictive process \textit{par excellence}.
    \item ``To predict whether I will take my thousand, make a replica of me, put my replica in a replica of my predicament, and see whether my replica takes his thousand.'' ---\citet[p.~237]{Lewis1979a}
    \item Is prisoner $B$ a good enough replica of prisoner $A$?
  \item[\it Conclusion:] If so, then $(P)$ is an instance of $(N)$ as Lewis claims.
\end{itemize}






\section*{Optimal Rationality}

\begin{itemize}
  \item[\it Collaboration:] Suppose that both prisoners are (optimally) rationally.
    \item Suppose they know that they are each optimally rational.
    \item Suppose they have all the same information and values.
    \item Does this mean that they act in the same way?
    \item One sort of answers claims `yes': optimal rationality is unique.
    \item Thus there are only two possible outcomes: $\texttt{Take}_{AB}$ or $\neg \texttt{Take}_{AB}$.
    \item Moreover, $v(\neg \texttt{Take}_{AB}) \gg v(\texttt{Take}_{AB})$.
    \item Can we conclude that optimally rational prisoners will collaborate?
  \item[\it Theory:] Is optimal rationality unique?
    \item Is there just one rational action for each agent in each case?
    \item \mbox{Does optimal rationality require knowing whether optimal rationality is unique?}
    % \item Even if unique, rationality doesn't require knowing it is unique.
    % \item An agent need not know all of physics to be optimally rational.
    \item One needn't know a final linguistic theory to be fluent in English.
    \item Nor does one need to know physics in order to hit a baseball.
    \item Being rational doesn't require knowing what rationality is.
    \item In particular, one needn't know if optimal rationality is unique.
  \item[\it Uniqueness:] Can the prisoners assume that they will act in the same way?
    \item Even if optimal rationality is unique, can't assume they know this.
    \item Thus they can't conclude they will act in the same way.
    \item So the prisoner's can't run the reasoning above to $\neg \texttt{Take}_{AB}$.
    \item This reasoning also fails if rationality is not unique.
\end{itemize}




\section*{Modulo Theory}

\begin{itemize}
  \item[\it Rationality:] What is it to be rational?
    \item \mbox{Takes an epistemic state and values as input and action choice as output.}
    \item There are the various ways people act given their values and info.
    % \item Should we expect rationally to form a total order?
    \item Holding the inputs fixed, can the outputs be totally ordered?
    \item If totally ordered, must there be a maximally rational output?
  \item[\it Theory:] Is it the task of a theory of rationality to provide a total ordering?
    \item For instance, EDT and CDT recommend opposing choices.
    \item Should we assume the same theory will be universally applicable?
    \item If not, how are we to decide which theory to choose when?
    \item For instance, we saw before that a twoboxer might use CDT to choose, but then use EDT to bet against themselves.
  % \item[\it Hard Choices:] Compare buying an apartment in Boston to a farm in Vermont.
  %   \item Suppose that you value them in different ways.
  %   \item Must any two types of value be commensurate, even subjectively?
    % \item If not, are such choices irrational?
\end{itemize}



\bibliographystyle{Phil_Review} %%bib style found in bst folder, in bibtex folder, in texmf folder.
\nobibliography{Zotero} %%bib database found in bib folder, in bibtex folder
\end{document}
