\documentclass[a4paper, 11pt]{article} % Font size (can be 10pt, 11pt or 12pt) and paper size (remove a4paper for US letter paper)

\usepackage[protrusion=true,expansion=true]{microtype} % Better typography
\usepackage{graphicx} % for adding extra padding to rows
\usepackage{wrapfig} % Allows in-line images
\usepackage{enumitem} %%Enables control over enumerate and itemize environments
\usepackage{setspace}
\usepackage{amssymb, amsmath, mathrsfs,mathabx} %%Math packages
\usepackage{stmaryrd}
\usepackage{mathtools}
\usepackage{multicol} 
\usepackage{mathpazo} % Use the Palatino font
\usepackage[T1]{fontenc} % Required for accented characters
\usepackage{array}
\usepackage{bibentry}
\usepackage{prooftrees} 
\usepackage[round]{natbib} %%Or change 'round' to 'square' for square backers
\setcitestyle{aysep=}
% \usepackage{fitchproof} 

% \linespread{1.05} % Change line spacing here, Palatino benefits from a slight increase by default

\DeclareSymbolFont{symbolsC}{U}{txsyc}{m}{n}
\SetSymbolFont{symbolsC}{bold}{U}{txsyc}{bx}{n}
\DeclareFontSubstitution{U}{txsyc}{m}{n}
\DeclareMathSymbol{\boxright}{\mathrel}{symbolsC}{"80}
\DeclareMathSymbol{\circleright}{\mathrel}{symbolsC}{"91}
\DeclareMathSymbol{\diamondright}{\mathrel}{symbolsC}{"84}
\DeclareMathSymbol{\medcirc}{\mathrel}{symbolsC}{"07}

\newcommand{\tuple}[1]{\langle#1\rangle} %%Angle brackets
\newcommand{\corner}[1]{\ulcorner#1\urcorner} %%Angle brackets
\newcommand{\set}[1]{\lbrace#1\rbrace} %%Set brackets
\newcommand{\abs}[1]{|#1|} %%Set brackets
\newcommand{\interpret}[1]{\llbracket#1\rrbracket} %%Double brackets
\newcommand{\N}{\mathbb{N}}
\renewcommand{\L}{\mathcal{L}}
\renewcommand{\O}{\mathcal{O}}
\newcommand{\A}{\mathcal{A}}
\newcommand{\D}{\mathbb{D}}
\newcommand{\Z}{\mathbb{Z}}
\renewcommand{\Pr}{\mathbb{P}}
\newcommand{\Q}{\mathbb{Q}}
\newcommand{\R}{\mathbb{R}}
\newcommand{\B}{\mathfrak{B}}
\renewcommand{\max}[1]{\texttt{max}\set{#1}}

\makeatletter
\newcommand{\superimpose}[2]{%
  {\ooalign{$#1\@firstoftwo#2$\cr\hfil$#1\@secondoftwo#2$\hfil\cr}}}
\makeatother

\newcommand{\past}{\mathpalette\superimpose{{\Diamond}{\raisebox{1.5pt}{\tiny \hspace{.4pt}\textsc{p}}}}}

\newcommand{\Past}{\mathpalette\superimpose{{\Box}{\raisebox{1.2pt}{\tiny \textsc{p}}}}}

\newcommand{\future}{\mathpalette\superimpose{{\Diamond}{\raisebox{1.5pt}{\tiny \textsc{f}}}}}

\newcommand{\Future}{\mathpalette\superimpose{{\Box}{\raisebox{1.2pt}{\tiny \textsc{f}}}}}

\newcommand{\always}{\ensuremath \raisebox{1.3pt}{\rotatebox[origin=c]{180}{$\triangle$}}}

\newcommand{\sometimes}{\ensuremath \raisebox{-1.3pt}{$\triangle$}}

\makeatletter
\renewcommand\@biblabel[1]{\textbf{#1.}} % Change the square brackets for each bibliography item from '[1]' to '1.'
\renewcommand{\@listI}{\itemsep=0pt} % Reduce the space between items in the itemize and enumerate environments and the bibliography

\renewcommand{\maketitle}{ % Customize the title - do not edit title and author name here, see the TITLE block below
\begin{flushright} % Right align
{\LARGE\@title} % Increase the font size of the title

\vspace{10pt} % Some vertical space between the title and author name

{\@author} % Author name
\\\@date % Date

\vspace{-10pt} % Some vertical space between the author block and abstract
\end{flushright}
}

%----------------------------------------------------------------------------------------
%	TITLE
%----------------------------------------------------------------------------------------

\title{\textbf{Prisoners' Dilemma}} % Subtitle

\author{\textsc{Paradox and Infinity}\\ \em Benjamin Brast-McKie} % Institution

\date{\today} % Date

%----------------------------------------------------------------------------------------

\begin{document}

\maketitle % Print the title section

\thispagestyle{empty}

%----------------------------------------------------------------------------------------

\section*{Two Prisoners}

\begin{itemize}
  \item[\it Setup:] Two separated prisoners are each offered \$1,000. They will be given an additional \$1,000,000 \textit{iff} the other prisoner does not take the \$1,000.
    % \item Prisoner $A$ gets \$1,000,000 \textit{iff} prisoner $B$ does not take the \$1,000. 
    \item The prisoners' choices are causally independent.
    \item $P(\texttt{Take}_A \boxright \texttt{Take}_B) = P(\neg \texttt{Take}_A \boxright \texttt{Take}_B) = P(\texttt{Take}_B)$.
    \item $P(\texttt{Take}_A \boxright \neg \texttt{Take}_B) = P(\neg \texttt{Take}_A \boxright \neg \texttt{Take}_B) = P(\neg \texttt{Take}_B)$.
    \item We know that $P(\neg \texttt{Take}_B) = 1 - P(\texttt{Take}_B)$, but don't know $P(\texttt{Take}_B)$.
    \item Something similar may be said swapping `$A$' and `$B$' above.
    \item The prisoner's know everything except for the other's choice.
    \item What is it rational for prisoner $A$ (similarly $B$) to do?
  \item[\it Dominant:] Taking the \$1,000 is a \textit{dominant strategy} for prisoner $A$ (similarly $B$).
    \item Whether \texttt{Take}$_B$ or not, $v(\texttt{Take}_A) > v(\neg\texttt{Take}_A)$ for prisoner $A$.
    % \item Would it matter if prisoner $A$ knew the probability $P(\texttt{Take}_B)$?
    \item We get the following alternatives:
  \vspace{-.25in}
  \begin{center} \begin{tabular}{>{\centering\arraybackslash}m{2cm}|>{\centering\arraybackslash}m{5cm}|>{\centering\arraybackslash}m{5cm}|}
         & $\texttt{Take}_B$ & $\neg \texttt{Take}_B$ \\
      \hline
      $\texttt{Take}_A$ & $(A, B: \$1,000)$ & $(A: \$1,001,000)$, $(B: \$0)$ \\
      \hline
      $\neg \texttt{Take}_A$ & $(A: \$0)$, $(B: \$1,001,000)$ & $(A, B: \$1,000,000)$ \\
      \hline
    \end{tabular}
  \end{center}
    \item The setup assumes that neither prisoner cares about the other.
    \item If the prisoners cared about each other, that would be a different case.
  \item[\it Predictor:] Given the circumstances, each prisoner is a good predictor of the other.
    \item $\texttt{Take}_A$ predicts that $\texttt{Take}_B$, i.e., $P(\texttt{Take}_B\ |\ \texttt{Take}_A)$ is high.
    \item Thus $P(\neg \texttt{Rich}_A\ |\ \texttt{Take}_A)$ is high since $\texttt{Take}_B$ \textit{iff} $\neg \texttt{Rich}_A$.
    \item So if $\texttt{Take}_A$, then prisoner $A$ has good reason to bet $\neg \texttt{Rich}_A$.
    \item We don't know what the probabilities $P(\texttt{Take}_A)$ or $P(\texttt{Take}_B)$.
    \item Does $\neg \texttt{Take}_A$ change the probability $P(\neg \texttt{Take}_B) = P(\texttt{Rich}_A)$?
  \item[\it Newcomb:] $\texttt{Rich}_A$ \textit{iff} it is predicted that $\neg \texttt{Take}_A$ (by $\neg \texttt{Take}_B$). 
    \item $\neg \texttt{Take}_B$ is a \textit{prediction instance} (a way of predicting $\neg \texttt{Take}_A$).
    \item The predication amounts to probabilistic dependence (not causal).
    \item When the prediction happens does not matter to the case.
    \item Is the prisoners' dilemma a Newcomp problem?
\end{itemize}






\section*{Dominance Calculations}

\begin{itemize}
  \item[\it Expected Causal Utility:] Recall that: (a) $\texttt{Rich}_A$ \textit{iff} $\neg \texttt{Take}_B$; and (b) $\texttt{Rich}_B$ \textit{iff} $\neg \texttt{Take}_A$.
    % \item Thus $P(\texttt{Take}_A \boxright \texttt{Rich}_A) = P(\texttt{Take}_A \boxright \neg \texttt{Take}_B) = P(\neg \texttt{Take}_B)$.
    % \item Similarly, $P(\texttt{Take}_A \boxright \neg \texttt{Rich}_A) = P(\texttt{Take}_A \boxright \texttt{Take}_B) = P(\texttt{Take}_B)$. 
    % \item Lastly, $P(\neg \texttt{Take}_A \boxright \texttt{Rich}_A) = P(\neg \texttt{Take}_A \boxright \neg \texttt{Take}_B) = P(\neg \texttt{Take}_B)$.
    \item What are the \textit{expected causal utilities} of $\texttt{Take}_A$ and $\neg \texttt{Take}_A$? 
    \item \mbox{$ECU(\neg \texttt{Take}_A) = \$1,000,000\times P(\neg \texttt{Take}_A \boxright \texttt{Rich}_A) + \$0\times P(\neg \texttt{Take}_A \boxright \neg \texttt{Rich}_A)$}
    \item[] \hspace{.99in} $= \$1,000,000\times P(\neg \texttt{Take}_B)$ by (a).
    \item \mbox{$ECU(\texttt{Take}_A) = \$1,001,000\times P(\texttt{Take}_A \boxright \texttt{Rich}_A) + \$1,000\times P(\texttt{Take}_A \boxright \neg \texttt{Rich}_A)$.}
    \item[] \hspace{.87in} \mbox{$= \$1,001,000\times P(\neg \texttt{Take}_B) + \$1,000\times P(\texttt{Take}_B)$ by (a).}
    \item[] \hspace{.87in} $= \$1,001,000\times P(\neg \texttt{Take}_B) + \$1,000\times (1 - P(\neg \texttt{Take}_B))$.
    % \item[] \hspace{.87in} $= \$1,001,000\times P(\neg \texttt{Take}_B) - \$1,000\times P(\neg \texttt{Take}_B) + (\$1,000$.
    \item[] \hspace{.87in} $= \$1,000,000\times P(\neg \texttt{Take}_B) + \$1,000$.
    \item[] \hspace{.87in} $= ECU(\neg \texttt{Take}_A) + \$1,000$.
    \item Taking the money is better for prisoner $A$ (and similarly for $B$).
\end{itemize}




\section*{Accuracy}

\begin{itemize}
  \item[\it Clash:] The predication does not have to be very accurate for the expected utility calculation to clash with causal expected utility (i.e. $> .5005$).
    \item Suppose $P(\texttt{Take}_B\ |\ \texttt{Take}_A) = P(\neg \texttt{Take}_B\ |\ \neg \texttt{Take}_A) = .5006$.
    % \item Similarly $P(\texttt{Rich}_A\ |\ \texttt{Take}_A) = .5006$, so $P(\neg \texttt{Rich}_A\ |\ \neg\texttt{Take}_A) = .4994$.
    \item So $P(\texttt{Rich}_A\ |\ \neg \texttt{Take}_A) = P(\neg \texttt{Take}_B\ |\ \neg \texttt{Take}_A) = .5006$.
    \item \mbox{And $P(\texttt{Rich}_A\ |\ \texttt{Take}_A) = P(\neg \texttt{Take}_B\ |\ \texttt{Take}_A) = 1 - P(\texttt{Take}_B\ |\ \texttt{Take}_A) = .4994$.}
    % \item $P(\texttt{Rich}_A\ |\ \neg \texttt{Take}_A) = P(\neg \texttt{Rich}_A\ |\ \texttt{Take}_A)$ and $P(\neg \texttt{Rich}_A\ |\ \texttt{Take}_A) = P(\texttt{Rich}_A\ |\ \neg \texttt{Take}_A)$.
    \item $EV(\neg \texttt{Take}_A) = \$1,000,000\times P(\texttt{Rich}_A\ |\ \neg\texttt{Take}_A) + \$0\times P(\neg \texttt{Rich}_A\ |\ \neg \texttt{Take}_A)$
    \item[] \hspace{.88in} $= \$1,000,000\times P(\texttt{Rich}_A\ |\ \neg\texttt{Take}_A)$
    \item[] \hspace{.88in} $= \$500,600$.
    \item $EV(\texttt{Take}_A) = \$1,001,000\times P(\texttt{Rich}_A\ |\ \texttt{Take}_A) + \$1,000\times P(\neg \texttt{Rich}_A\ |\ \texttt{Take}_A)$
    \item[] \hspace{.76in} $= \$1,000,000\times P(\texttt{Rich}_A\ |\ \texttt{Take}_A) + \$1,000$
    \item[] \hspace{.76in} $= \$500,400$.
    \item Even if prisoner $A$ is an inaccurate predictor of prisoner $B$, the expected utility and expected causal utility calculations are bound to come apart.
\end{itemize}



\section*{Upshot}

\begin{itemize}
  \item[\it Common:] Newcomb's problem is fanciful, but prisoners' dilemmas are common.
    \item Prisoners' dilemmas support \textit{causal decision theory} on their own.
    \item No need to appeal to Newcomb cases to motivate CDT.
  \item[\it Comparison:] Should a oneboxer also avoid taking the money?
    \item Does comparing the cases put any pressure on the oneboxer to twobox?
\end{itemize}



\end{document}



