\documentclass[a4paper, 11pt]{article} % Font size (can be 10pt, 11pt or 12pt) and paper size (remove a4paper for US letter paper)

\usepackage[protrusion=true,expansion=true]{microtype} % Better typography
\usepackage{graphicx} % Required for including pictures
\usepackage{wrapfig} % Allows in-line images
\usepackage{enumitem} %%Enables control over enumerate and itemize environments
\usepackage{setspace}
\usepackage{amssymb, amsmath, mathrsfs,mathabx} %%Math packages
\usepackage{stmaryrd}
\usepackage{mathtools}
\usepackage{multicol} 
\usepackage{mathpazo} % Use the Palatino font
\usepackage[T1]{fontenc} % Required for accented characters
\usepackage{array}
\usepackage{bibentry}
\usepackage{prooftrees} 
\usepackage[round]{natbib} %%Or change 'round' to 'square' for square backers
\setcitestyle{aysep=}
% \usepackage{fitchproof} 

% \linespread{1.05} % Change line spacing here, Palatino benefits from a slight increase by default

\newcommand{\tuple}[1]{\langle#1\rangle} %%Angle brackets
\newcommand{\corner}[1]{\ulcorner#1\urcorner} %%Angle brackets
\newcommand{\set}[1]{\lbrace#1\rbrace} %%Set brackets
\newcommand{\abs}[1]{|#1|} %%Set brackets
\newcommand{\interpret}[1]{\llbracket#1\rrbracket} %%Double brackets
\newcommand{\N}{\mathbb{N}}
\renewcommand{\L}{\mathcal{L}}
\newcommand{\D}{\mathbb{D}}
\newcommand{\Z}{\mathbb{Z}}
\renewcommand{\Pr}{\mathbb{P}}
\newcommand{\Q}{\mathbb{Q}}
\newcommand{\R}{\mathbb{R}}
\newcommand{\B}{\mathfrak{B}}
\renewcommand{\max}[1]{\texttt{max}\set{#1}}

\makeatletter
\newcommand{\superimpose}[2]{%
  {\ooalign{$#1\@firstoftwo#2$\cr\hfil$#1\@secondoftwo#2$\hfil\cr}}}
\makeatother

\newcommand{\past}{\mathpalette\superimpose{{\Diamond}{\raisebox{1.5pt}{\tiny \hspace{.4pt}\textsc{p}}}}}

\newcommand{\Past}{\mathpalette\superimpose{{\Box}{\raisebox{1.2pt}{\tiny \textsc{p}}}}}

\newcommand{\future}{\mathpalette\superimpose{{\Diamond}{\raisebox{1.5pt}{\tiny \textsc{f}}}}}

\newcommand{\Future}{\mathpalette\superimpose{{\Box}{\raisebox{1.2pt}{\tiny \textsc{f}}}}}

\newcommand{\always}{\ensuremath \raisebox{1.3pt}{\rotatebox[origin=c]{180}{$\triangle$}}}

\newcommand{\sometimes}{\ensuremath \raisebox{-1.3pt}{$\triangle$}}

\makeatletter
\renewcommand\@biblabel[1]{\textbf{#1.}} % Change the square brackets for each bibliography item from '[1]' to '1.'
\renewcommand{\@listI}{\itemsep=0pt} % Reduce the space between items in the itemize and enumerate environments and the bibliography

\renewcommand{\maketitle}{ % Customize the title - do not edit title and author name here, see the TITLE block below
\begin{flushright} % Right align
{\LARGE\@title} % Increase the font size of the title

\vspace{10pt} % Some vertical space between the title and author name

{\@author} % Author name
\\\@date % Date

\vspace{50pt} % Some vertical space between the author block and abstract
\end{flushright}
}

%----------------------------------------------------------------------------------------
%	TITLE
%----------------------------------------------------------------------------------------

\title{\textbf{Time and Change}} % Subtitle

\author{\textsc{Paradox and Infinity}\\ \em Benjamin Brast-McKie} % Institution

\date{\today} % Date

%----------------------------------------------------------------------------------------

\begin{document}

\maketitle % Print the title section

\thispagestyle{empty}

%----------------------------------------------------------------------------------------

\section*{Real Change}

\begin{itemize}
  \item[\it Grid:] Consider a universe consisting of three pieces on a $3 \times 3$ grid. 
    \item Consider three successive configurations of the grid in time.
    \item Compare this to three configurations of the grid separated in space. 
    \item How does change across time differ from change across space?
  \item[\it Identity:] The spatially separated grids are not identical.
    \item By contrast, the temporally separated grids are one.
    \item The properties of one and the same grid differ at different times.
    \item Call a complete configuration a \textit{world state}.
  \item[\it Properties:] What properties are to be included in a world state?
    \item A piece $x$ is \textit{shrew} at $t$ \textit{iff} either: (1) $x$ is shaded at $t$ and $t$ is before 11am; or (2) $x$ not shaded at $t$ and $t$ is after 11am.
    \item Suppose a shaded piece goes on being shaded at 11am.
    \item Does that piece change from being shrew to not being shrew at 11am?
  \item[\it Things:] Consider the object which consists of the grid at different times.
    \item This object does not change in time, but goes on just as it is.
    \item Suppose we exclude temporally defined properties and things.
    \item We can ask what real properties every real thing has at a time.
  \item[\it Existence:] World states determines which properties everything has.
    \item But suppose one grid ends and another begins at each change.
    \item Can still say that each grid is thus and so at each time.
    \item Do the things and properties that exist also change?
  \item[\it Change:] A difference between the real properties real things have.
    \item Do two times differ only if there is a change between them?
    \item Something is some way at time $t$, and not that way at time $t'$.
    \item Could there be two times where the same things are all the same ways?
\end{itemize}




\section*{Real Possibility}

\begin{itemize}
  \item[\it Logical Possibility:] ``What is in question here is not whether it is physically possible for there to be time without change but whether this is logically or conceptually possible.'' --- Shoemaker (p.~368, 1969)
    \item Important to distinguish logical possibility in the sense of consistency.
    \item It is consistent for the atom to be gold and to have only 6 protons.
    \item There is no \textit{way for things to be} where the gold atom has only 6 protons.
  \item[\it Metaphysical Possibility:] Broadest range of objective possibilities (ways for things to be).
    \item Fixing the interpretation, how must things be for the claim to be true?
    \item Is there any way whatsoever for things to be where the claim is true?
    \item Interpretational possibility concerns truth on any interpretation.
\end{itemize}




\section*{Total Freezes}

\begin{itemize}
  \item[\it Universe:] Suppose there is a possible world with A, B, and C regions.
    \item Local freezes occur every 3rd year in region A.
    \item Local freezes occur every 4th year in region B.
    \item Local freezes occur every 5th year in region C.
  \item[\it Total:] On certain years, the freezes in different regions align.
    \item A and B freeze together every 12th year.
    \item A and C freeze together every 15th year.
    \item B and C freeze together every 20th year.
    \item A, B, and C all freeze every 60th year.
  \item[\it No Change:] On that 60th year, does time pass without change?
    \item More dramatically, could there be permanently frozen worlds?
    \item What about worlds that occupy the same world states more than once?
  \item[\it Possible Worlds:] Let $W$ be the set of \textit{world states} and $T$ a strict total order of \textit{times}. 
    \item A \textit{world evolution} is any function from $\tau : T \to W$.
    \item Which world evolutions are possible worlds? 
    \item Are constant functions permitted? What about loops?
  \item[\it Convention:] Is it a matter of convention whether there are total freezes or not?
    \item Which of two bodies is rotating around each other?
    \item The year is exactly 365 days long with a one day freeze every 4th year.
    \item Compare the continuum hypothesis or axiom of choice for sets.
\end{itemize}


\end{document}



