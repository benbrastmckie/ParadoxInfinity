\documentclass[a4paper, 11pt]{article} % Font size (can be 10pt, 11pt or 12pt) and paper size (remove a4paper for US letter paper)

\usepackage[protrusion=true,expansion=true]{microtype} % Better typography
\usepackage{graphicx} % Required for including pictures
\usepackage{wrapfig} % Allows in-line images
\usepackage{enumitem} %%Enables control over enumerate and itemize environments
\usepackage{setspace}
\usepackage{amssymb, amsmath, mathrsfs,mathabx} %%Math packages
\usepackage{stmaryrd}
\usepackage{mathtools}
\usepackage{multicol} 
\usepackage{mathpazo} % Use the Palatino font
\usepackage[T1]{fontenc} % Required for accented characters
\usepackage{array}
\usepackage{bibentry}
\usepackage{prooftrees} 
\usepackage[round]{natbib} %%Or change 'round' to 'square' for square backers
\setcitestyle{aysep=}
% \usepackage{fitchproof} 

% \linespread{1.05} % Change line spacing here, Palatino benefits from a slight increase by default

\newcommand{\tuple}[1]{\langle#1\rangle} %%Angle brackets
\newcommand{\corner}[1]{\ulcorner#1\urcorner} %%Angle brackets
\newcommand{\set}[1]{\lbrace#1\rbrace} %%Set brackets
\newcommand{\abs}[1]{|#1|} %%Set brackets
\newcommand{\interpret}[1]{\llbracket#1\rrbracket} %%Double brackets
\newcommand{\N}{\mathbb{N}}
\renewcommand{\L}{\mathcal{L}}
\newcommand{\D}{\mathbb{D}}
\newcommand{\Z}{\mathbb{Z}}
\renewcommand{\Pr}{\mathbb{P}}
\newcommand{\Q}{\mathbb{Q}}
\newcommand{\R}{\mathbb{R}}
\newcommand{\B}{\mathfrak{B}}
\renewcommand{\max}[1]{\texttt{max}\set{#1}}

\makeatletter
\renewcommand\@biblabel[1]{\textbf{#1.}} % Change the square brackets for each bibliography item from '[1]' to '1.'
\renewcommand{\@listI}{\itemsep=0pt} % Reduce the space between items in the itemize and enumerate environments and the bibliography

\renewcommand{\maketitle}{ % Customize the title - do not edit title and author name here, see the TITLE block below
\begin{flushright} % Right align
{\LARGE\@title} % Increase the font size of the title

\vspace{10pt} % Some vertical space between the title and author name

{\@author} % Author name
\\\@date % Date

\vspace{-20pt} % Some vertical space between the author block and abstract
\end{flushright}
}

%----------------------------------------------------------------------------------------
%	TITLE
%----------------------------------------------------------------------------------------

\title{\textbf{Time Travel}} % Subtitle

\author{\textsc{Paradox and Infinity}\\ \em Benjamin Brast-McKie} % Institution

\date{\today} % Date

%----------------------------------------------------------------------------------------

\begin{document}

\maketitle % Print the title section

\thispagestyle{empty}

%----------------------------------------------------------------------------------------


\section*{Time Travel}

\begin{itemize}
  \item[\bf Question:] What is it to travel in time?
  \item[\it Personal Duration:] The journey's duration in personal time: $\Delta_p$.
  \item[\it External Time:] The time for Earth's frame of reference: $t_i$.
  \item[\it External Duration:] The difference in the external start and end times: $\Delta_w \coloneq t_e - t_b$.
  \item[\it Trival Time Travel:] Occurs when $\Delta_w \neq 0$.
  \item[\it Non-Trival Time Travel:] Occurs when $\Delta_p \neq \Delta_w$ (e.g., $0 < \Delta_p < \Delta_w$).
    \item If Bob travels close to the speed of light.
  \item[\it Extraordinary Time Travel:] Occurs when $\Delta_w$ is negative or much greater than $\Delta_p$.
    % \item Suppose $\Delta_p$ is 1 hour (and in general never negative).
  \item[\it Example:] $t_b$ is 10am EST, April 1st, 2024 and $t_e$ is 10am EST, April 1st, 1924.
    \item Then $\Delta_w = -100$ years (i.e., 100 years in the past). 
\end{itemize}



\section*{Metaphysical Possibility}

\begin{itemize}
  \item[\bf Question:] Is it possible to travel in time (in an extraordinary way)?
  \item[\it Practical Possibility:] Is it possible for me to do a double backflip (on Earth's surface)?
  \item[\it Nomological Possibility:] Is it nomologically possible for me to do a double backflip?
  \item[\it Metaphysically Possibility:] Is it metaphysically possible to travel faster than the speed of light?
  \item[\it Objective Modality:] Each modality concerns a range of objective \textit{ways for things to be}.
    \item Metaphysical modality is the maximal objective modality.
  \item[\it Epistemic Modality:] Is it possible that $a^n + b^n = c^n$ for some $a,b,c \in \N^+$ and $n > 2$?
    \item Fermat's last theorem was proven to be true (1995).
    \item Moreover, it is not possible for Fermat's last theorem to be false.
    \item Nevertheless, it may be epistemically possible for the uniformed.
  % \item[\it Necessity:] Necessarily $A$ \textit{iff} it is not possible for it not to be the case that $A$. 
    % \item Fermat' last theorem is (and always was) metaphysically necessary.
    \item Or consider the epistemic possibility that $2,641 \times 31 \neq 81,971$.
\end{itemize}



\section*{The Possibility of Time Travel}

\begin{itemize}
  \item[\it Assume:] It is practically impossible to travel back in time.
  \item[\it Agnostic:] It is nomologically possible to travel back in time. 
  \item[\bf Question:] Is it metaphysically possible to travel back in time?
\end{itemize}




\section*{World Travel}

\begin{itemize}
  \item[\it Actuality:] Suppose that nobody has ever arrived from a future time.
    \item Deep in an MIT laboratory, Michele finishes her time machine.
    \item She gets in, eager to zip off into the distant past.
    \item Is it possible to travel into the past?
  \item[\it Branching Worlds:] Is it possible to change the past?
    \item Instead of traveling to the actual past, one has traveled to another past.
    \item In what sense is traveling to a branching world count as time travel?
  \item[\it Take Two:] Assume we restrict attention to time travel within one time-line.
    \item Michele can't travel back in time if she hasn't already arrived.
    \item If she has already arrived, she must travel back to those times.
  \item[\it Open Future:] Is the future open if time travelers have already arrived?
    \item At least it is not as open as it might otherwise be assumed to be.
    \item But traveling back in time may be assumed to be entirely fixed.
  \item[\it Possibility:] We don't just want to ask if our open future includes any time travel.
    \item We are asking if there are any worlds at all that include time travel.
\end{itemize}




\section*{Grandfather Paradox}

\begin{itemize}
  \item[\it Paradox:] Tim travels to a time before his grandfather and grandmother met.
  \item[\bf Question] Can Tim kill his grandfather?
    \item If so, then neither Tim's father, nor Tim would have been born.
    \item So Tim wouldn't have traveled back in time, nor killed his grandfather.
    \item But how could Tim fail if appropriately poised to kill his grandfather?
    \item It would seem that Tim both can and cannot kill his grandfather.
    \item Perhaps this shows that time travel is not possible after all. 
  \item[\it Equivocation:] Or perhaps Tim can time travel, but only do exactly what he did.
    \item Considering everything, Tim cannot kill his grandfather.
    \item But `can' is context sensitive, only taking some things into account.
    \item There are contexts which do not take everything into account where Tim \textit{can} kill his grandfather.
    \item Tim has the necessary skills, position, timing, etc., he just doesn't do it.
  \item[\it Determined:] Perhaps Tim can kill his grandfather even though it is impossible.
    \item Time travel has been defined in such a way that the future is closed.
    \item But as we will see next time, nothing forces this choice.
\end{itemize}



\end{document}



