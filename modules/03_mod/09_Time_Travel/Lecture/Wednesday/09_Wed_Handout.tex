\documentclass[a4paper, 11pt]{article} % Font size (can be 10pt, 11pt or 12pt) and paper size (remove a4paper for US letter paper)

\usepackage[protrusion=true,expansion=true]{microtype} % Better typography
\usepackage{graphicx} % Required for including pictures
\usepackage{wrapfig} % Allows in-line images
\usepackage{enumitem} %%Enables control over enumerate and itemize environments
\usepackage{setspace}
\usepackage{amssymb, amsmath, mathrsfs,mathabx} %%Math packages
\usepackage{stmaryrd}
\usepackage{mathtools}
\usepackage{multicol} 
\usepackage{mathpazo} % Use the Palatino font
\usepackage[T1]{fontenc} % Required for accented characters
\usepackage{array}
\usepackage{bibentry}
\usepackage{prooftrees} 
\usepackage[round]{natbib} %%Or change 'round' to 'square' for square backers
\setcitestyle{aysep=}
% \usepackage{fitchproof} 

% \linespread{1.05} % Change line spacing here, Palatino benefits from a slight increase by default

\newcommand{\tuple}[1]{\langle#1\rangle} %%Angle brackets
\newcommand{\corner}[1]{\ulcorner#1\urcorner} %%Angle brackets
\newcommand{\set}[1]{\lbrace#1\rbrace} %%Set brackets
\newcommand{\abs}[1]{|#1|} %%Set brackets
\newcommand{\interpret}[1]{\llbracket#1\rrbracket} %%Double brackets
\newcommand{\N}{\mathbb{N}}
\renewcommand{\L}{\mathcal{L}}
\newcommand{\D}{\mathbb{D}}
\newcommand{\Z}{\mathbb{Z}}
\renewcommand{\Pr}{\mathbb{P}}
\newcommand{\Q}{\mathbb{Q}}
\newcommand{\R}{\mathbb{R}}
\newcommand{\B}{\mathfrak{B}}
\renewcommand{\max}[1]{\texttt{max}\set{#1}}

\makeatletter
\renewcommand\@biblabel[1]{\textbf{#1.}} % Change the square brackets for each bibliography item from '[1]' to '1.'
\renewcommand{\@listI}{\itemsep=0pt} % Reduce the space between items in the itemize and enumerate environments and the bibliography

\renewcommand{\maketitle}{ % Customize the title - do not edit title and author name here, see the TITLE block below
\begin{flushright} % Right align
{\LARGE\@title} % Increase the font size of the title

\vspace{10pt} % Some vertical space between the title and author name

{\@author} % Author name
\\\@date % Date

\vspace{-20pt} % Some vertical space between the author block and abstract
\end{flushright}
}

%----------------------------------------------------------------------------------------
%	TITLE
%----------------------------------------------------------------------------------------

\title{\textbf{Time Travel}} % Subtitle

\author{\textsc{Paradox and Infinity}\\ \em Benjamin Brast-McKie} % Institution

\date{\today} % Date

%----------------------------------------------------------------------------------------

\begin{document}

\maketitle % Print the title section

\thispagestyle{empty}

%----------------------------------------------------------------------------------------

\section*{Future Possibility}

\begin{itemize}
  \item[\it Open Future:] There is a reading of `possible' that is about the future.
    \item Metaphysical modality is much broader than that reading.
    \item Future possibility reading is a subset of nomological possibility. 
    \item The laws in our world may be incompatible with time travel.
    \item We intend to ask if there is any possible world in which someone travels into the actual past of that world.
\end{itemize}




\section*{Grandfather Paradox}

\begin{itemize}
  \item[\it Paradox:] Tim travels to a time before his grandfather and grandmother met.
    \item Tim intends to kill his grandfather and is poised to do so.
  \item[\bf Question:] Can Tim kill his grandfather?
    \item If so, then neither Tim's parent, nor Tim would have been born.
    \item So Tim wouldn't have traveled back in time, nor killed his grandfather.
    \item But how could Tim fail if appropriately poised to kill his grandfather?
    \item It would seem that Tim both can and cannot kill his grandfather.
    \item Perhaps this shows that time travel is not possible after all. 
  \item[\it Equivocation:] Lewis takes this argument to equivocate on `can'.
    \item Considering everything, Tim cannot kill his grandfather.
    \item But `can' is context sensitive, only taking some things into account.
    \item There are contexts which do not take everything into account where Tim \textit{can} kill his grandfather.
    \item Tim has the necessary skills, position, timing, etc., he just doesn't do it.
  \item[\it Example:] Holding some facts fixed, I can speak Finnish.
    \item Holding others fixed, I cannot speak Finnish.
    \item Thus there are contexts in which Tim can kill his grandfather. 
    \item Nevertheless, it is impossible for Tim to kill his grandfather.
  \item[\it Change:] Do we mean to ask about Tim's abilities in this context sensitive sense?
    \item Instead we may ask: is it possible for Tim to kill his grandfather?
    \item The answer is already clear, but perhaps this is still frustrating to Tim.
\end{itemize}





\section*{Open Future}

\begin{itemize}
  \item[\it Determined:] Tim's actions are entirely determined during his journey.
    \item This includes not killing his grandfather.
    \item But it may also include a whole lot else to which he is unaware.
    \item He is only able to become aware of some of the things that he can't do.
    \item And not just for Tim: no time traveler can kill their ancestor.
    \item But if there are any time travelers, then there are a lot of them.
    \item What explains the fact that none of them succeed is consistency.
    \item But it may still seem strange to Tim that however he tries he fails.
  \item[\it Actuality:] For Lewis, each world is a space-time continuum.
    \item So the actual world is also a space-time continuum.
    \item So we all do only what we will do, though we don't know what.
    \item From this perspective, what is strange about time travel is that it puts us in a position to figure out what we cannot do.
    \item But determinism follows from the conception of possible worlds.
  \item[\it Open Future:] Is time travel compatible with the open future?
    \item Consider a set of \textit{world states} $S$ and a \textit{task relation} $\to$ over $S$.
    \item Let $\tau : \Z \to S$ be a \textit{world history iff} $\tau(x) \to \tau(x+1)$ for all $x \in \Z$. 
    \item Nothing binds us to a single world and so the future is open.
    \item Just because you travel to the actual past, nothing holds you there.
    \item But that does that mean such cases don't count as time travel?
  \item[\it Determinism:] One cannot change \textit{the} past or \textit{the} future.
    \item But there need not be a unique past nor a unique future.
    % \item If there is a \textit{directed path} from $s \Rightarrow r$ and $t \Rightarrow r$, then one path is part of the other.  
    % \item If there is a \textit{directed path} from $s \Rightarrow t$ and $s \Rightarrow r$, then one path is part of the other.  
    % % \item Could take $\tuple{S,\to}$ to be a directed forest in either direction. 
    % \item Or both directions in which it is a set of linear orderings.
\end{itemize}

\end{document}



