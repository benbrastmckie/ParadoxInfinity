\documentclass[a4paper, 11pt]{article} % Font size (can be 10pt, 11pt or 12pt) and paper size (remove a4paper for US letter paper)

\usepackage[protrusion=true,expansion=true]{microtype} % Better typography
\usepackage{graphicx} % Required for including pictures
\usepackage{wrapfig} % Allows in-line images
\usepackage{enumitem} %%Enables control over enumerate and itemize environments
\usepackage{setspace}
\usepackage{amssymb, amsmath, mathrsfs,mathabx} %%Math packages
\usepackage{stmaryrd}
\usepackage{mathtools}
\usepackage{multicol} 
\usepackage{mathpazo} % Use the Palatino font
\usepackage[T1]{fontenc} % Required for accented characters
\usepackage{array}
\usepackage{bibentry}
\usepackage{prooftrees} 
\usepackage[round]{natbib} %%Or change 'round' to 'square' for square backers
\setcitestyle{aysep=}
% \usepackage{fitchproof} 

% \linespread{1.05} % Change line spacing here, Palatino benefits from a slight increase by default

\newcommand{\tuple}[1]{\langle#1\rangle} %%Angle brackets
\newcommand{\corner}[1]{\ulcorner#1\urcorner} %%Angle brackets
\newcommand{\set}[1]{\lbrace#1\rbrace} %%Set brackets
\newcommand{\abs}[1]{|#1|} %%Set brackets
\newcommand{\interpret}[1]{\llbracket#1\rrbracket} %%Double brackets
\newcommand{\N}{\mathbb{N}}
\renewcommand{\L}{\mathcal{L}}
\newcommand{\D}{\mathbb{D}}
\newcommand{\Z}{\mathbb{Z}}
\renewcommand{\Pr}{\mathbb{P}}
\newcommand{\Q}{\mathbb{Q}}
\newcommand{\R}{\mathbb{R}}
\newcommand{\B}{\mathfrak{B}}
\renewcommand{\max}[1]{\texttt{max}\set{#1}}

\makeatletter
\newcommand{\superimpose}[2]{%
  {\ooalign{$#1\@firstoftwo#2$\cr\hfil$#1\@secondoftwo#2$\hfil\cr}}}
\makeatother

\newcommand{\past}{\mathpalette\superimpose{{\Diamond}{\raisebox{1.5pt}{\tiny \hspace{.4pt}\textsc{p}}}}}

\newcommand{\Past}{\mathpalette\superimpose{{\Box}{\raisebox{1.2pt}{\tiny \textsc{p}}}}}

\newcommand{\future}{\mathpalette\superimpose{{\Diamond}{\raisebox{1.5pt}{\tiny \textsc{f}}}}}

\newcommand{\Future}{\mathpalette\superimpose{{\Box}{\raisebox{1.2pt}{\tiny \textsc{f}}}}}

\newcommand{\always}{\ensuremath \raisebox{1.3pt}{\rotatebox[origin=c]{180}{$\triangle$}}}

\newcommand{\sometimes}{\ensuremath \raisebox{-1.3pt}{$\triangle$}}

\makeatletter
\renewcommand\@biblabel[1]{\textbf{#1.}} % Change the square brackets for each bibliography item from '[1]' to '1.'
\renewcommand{\@listI}{\itemsep=0pt} % Reduce the space between items in the itemize and enumerate environments and the bibliography

\renewcommand{\maketitle}{ % Customize the title - do not edit title and author name here, see the TITLE block below
\begin{flushright} % Right align
{\LARGE\@title} % Increase the font size of the title

\vspace{10pt} % Some vertical space between the title and author name

{\@author} % Author name
\\\@date % Date

\vspace{-20pt} % Some vertical space between the author block and abstract
\end{flushright}
}

%----------------------------------------------------------------------------------------
%	TITLE
%----------------------------------------------------------------------------------------

\title{\textbf{The Metaphysics of Time}} % Subtitle

\author{\textsc{Paradox and Infinity}\\ \em Benjamin Brast-McKie} % Institution

\date{\today} % Date

%----------------------------------------------------------------------------------------

\begin{document}

\maketitle % Print the title section

\thispagestyle{empty}

%----------------------------------------------------------------------------------------


\section*{Ameliorating Intuitions}

\begin{itemize}
  \item[\it Time:] Calling something a theory of time does not make it a theory of time.
    \item Must fill the appropriate theoretical role, conforming to a significant extent with our intuitions.
    \item Compare a theory of sets that rejects extensionality.
    \item Or a theory of identity that rejects reflexivity.
  \item[\it Pre-Theory:] Intuitions correspond to common ways of speaking about time.
    \item These ways of speaking serve our practical aims.
    \item Nothing as systematic as talk of sets used naively in mathematics.
    \item Our aim is to improve on this situation.
\end{itemize}




\section*{The B-Series}

\begin{itemize}
  \item[\it Earlier-Than:] An asymmetric and transitive relation over events. 
    \item The ordering of events by the earlier-than relation is called the B-series.
  \item[\it Events:] Queen Anne's death; the poker is hot.
    \item Events may be understood roughly as instantaneous configurations.
    \item Does not capture a natural way of speaking about extended events.
    \item Could replace `events' with `states' or `propositions'.
  \item[\it Russell:] An event is past, present, or future only in relation to an event in time. 
    \item Typically it is the event of assertion that we intend to relate.
    \item Queen Anne's death is past in relation to the present assertion event.
    \item But events are never past, present, or future \textit{simpliciter}.
  \item[\it No Change:] If $e_1$ is earlier than $e_2$, then $e_1$ is \textit{always} earlier than $e_2$. 
    \item The B-series does not change.
    \item Can a change be $\tuple{e_1,e_2}$ where $e_1$ is earlier than $e_1$?
    \item The poker being hot ($e_1$) is earlier than the poker being cool $(e_2)$.
  \item[\it Space:] Compare ``change'' over space.
    \item The tip of the poker is hot; the handle of the poker is not hot.
    \item But the poker need not change for this to be true.
    \item How does change over time differ from ``change'' over space?
\end{itemize}





\section*{The A-Series}

\begin{itemize}
  \item[\it Change:] Events change from being future, then present, then past.
    \item A-series: \textit{past}, \textit{present}, \textit{future}.
    \item Without the A-series their is no change at all.
    \item Everything in time must have each of the A-series properties.
  \item[\it Relational Properties:] Some properties include other objects.
    \item \textit{Being North of London} is a relational property (includes London).
    \item Non-relational properties may be called \textit{absolute}.
  \item[\it Atemporal:] At most, A-series properties relate events to something outside of time.
    \item If \textit{past} is a relational property, it does not relate two events in time.
    \item Let `$P(e,x)$' read `$e$ \textit{is past relative to} $x$'.
    \item If $x$ is an event, $P(e,x)$ is always or never the case, so cannot change. 
  \item[\it Spotlight:] What do the A-series properties include if not other events?
    \item \textit{Was in}, \textit{is in}, \textit{will be in} ``the spotlight.''
    \item B-series as moving through the A-series.
    \item A-series as moving over the B-series.
    \item Film projector metaphor: was projected, is projected, will be projected.
  \item[\it Absolute:] A-series properties may just as well be taken to be absolute.
    \item Either way, the A-series properties are incompatible.
    \item $Pe \vdash \neg Ne \wedge \neg Fe$; $Ne \vdash \neg Pe \wedge \neg Fe$; $Fe \vdash \neg Pe \wedge \neg Ne$.
    
\end{itemize}






\section*{Paradox}

\begin{itemize}
  \item[\it Argument 1:] The A-series is essential to the reality of time.
    \item[\bf P1] If time is real, then events change.
    \item[\bf P2] If an event changes, then its A-series properties are what change.
    \item[\bf P3] \mbox{If an event's A-series properties change, events have A-series properties.}
    \item[\bf C1] Therefore, if time is real, then events have A-series properties.
  \item[\it Argument 2:] Events do not have A-series properties.
    % \item[\bf P4] If an event has an A-series property, it has every A-series property.
    \item[\bf P4] If an event has an A-series property, it has every A-series property.
    \item[\bf P5] The A-series properties are incompatible.
    \item[\bf C2] There are no events that have A-series properties. 
  \item[\it Argument 3:] Putting these first two arguments together, McTaggard concludes:
    \item[\bf C3] Time is not real.
\end{itemize}






\end{document}



