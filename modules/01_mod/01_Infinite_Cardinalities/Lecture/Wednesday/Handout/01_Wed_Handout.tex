\documentclass[a4paper, 11pt]{article} % Font size (can be 10pt, 11pt or 12pt) and paper size (remove a4paper for US letter paper)

\usepackage[protrusion=true,expansion=true]{microtype} % Better typography
\usepackage{graphicx} % Required for including pictures
\usepackage{wrapfig} % Allows in-line images
\usepackage{enumitem} %%Enables control over enumerate and itemize environments
\usepackage{setspace}
\usepackage{amssymb, amsmath, mathrsfs,mathabx} %%Math packages
\usepackage{stmaryrd}
\usepackage{mathtools}
\usepackage{multicol} 
\usepackage{mathpazo} % Use the Palatino font
\usepackage[T1]{fontenc} % Required for accented characters
\usepackage{array}
\usepackage{bibentry}
\usepackage{prooftrees} 
\usepackage[round]{natbib} %%Or change 'round' to 'square' for square backers
\setcitestyle{aysep=}
% \usepackage{fitchproof} 

% \linespread{1.05} % Change line spacing here, Palatino benefits from a slight increase by default

\newcommand{\tuple}[1]{\langle#1\rangle} %%Angle brackets
\newcommand{\set}[1]{\lbrace#1\rbrace} %%Set brackets
\newcommand{\abs}[1]{|#1|} %%Set brackets
\newcommand{\interpret}[1]{\llbracket#1\rrbracket} %%Double brackets
\newcommand{\N}{\mathbb{N}}
\newcommand{\D}{\mathbb{D}}
\newcommand{\Z}{\mathbb{Z}}
\renewcommand{\Pr}{\mathbb{P}}
\newcommand{\Q}{\mathbb{Q}}
\newcommand{\R}{\mathbb{R}}

\makeatletter
\renewcommand\@biblabel[1]{\textbf{#1.}} % Change the square brackets for each bibliography item from '[1]' to '1.'
\renewcommand{\@listI}{\itemsep=0pt} % Reduce the space between items in the itemize and enumerate environments and the bibliography

\renewcommand{\maketitle}{ % Customize the title - do not edit title and author name here, see the TITLE block below
\begin{flushright} % Right align
{\LARGE\@title} % Increase the font size of the title

\vspace{10pt} % Some vertical space between the title and author name

{\@author} % Author name
\\\@date % Date

\vspace{-30pt} % Some vertical space between the author block and abstract
\end{flushright}
}

%----------------------------------------------------------------------------------------
%	TITLE
%----------------------------------------------------------------------------------------

\title{\textbf{Infinite Cardinalities}} % Subtitle

\author{\textsc{Paradox and Infinity}\\ \em Benjamin Brast-McKie} % Institution

\date{\today} % Date

%----------------------------------------------------------------------------------------

\begin{document}

\maketitle % Print the title section

\thispagestyle{empty}

%----------------------------------------------------------------------------------------


\section*{Cardinality Principles}

\begin{enumerate}
  \item[\it Bijection Principle:] $\abs{A}=\abs{B}$ \textit{iff} $A\simeq B$. 
    \begin{itemize}
      \item[\tt Reflexive:] $A \simeq A$.
      \item[\tt Symmetric:] if $A \simeq B$, then $B \simeq A$.
      \begin{itemize}[leftmargin=-.5in]
        \item What's an inverse of a relation?
        \item Do functions always have inverses?
        \item Observe: the inverse of a bijection is a bijection.
      \end{itemize}
      \item[\tt Transitive:] if $A \simeq B$ and $B \simeq C$, then $A \simeq C$.
    \end{itemize}
  \item[\bf Observe:] We get equivalence classes but no ordering.
  \item[\it Injection Principle:] $\abs{A} \leq \abs{B}$ \textit{iff} $A \simeq C$ for some $C\subseteq B$. 
    \begin{itemize}
      \item[\tt Reflexive:] $|A| \leq |A|$.
      \item[\tt Transitive:] if $|A| \leq |B|$ and $|B| \leq |C|$, then $|A| \leq |C|$.
      \item[\tt Anti-Symmetric:] if $|A| \leq |B|$ and $|B| \leq |A|$, then $|A| = |B|$.
        \begin{itemize}[leftmargin=.5in]
          \item[\it Cantor-Schroeder-Bernstein Theorem:] If there are injective functions $f: A \to B$ and $g: B \to A$, then there is a bijection $h: A \to B$. 
        \end{itemize}
      \item[\tt Total:] $|A| \leq |B|$ or $|B| \leq |A|$. (Requires the Axiom of Choice)
    \end{itemize}
  \item[\it Could Define:] $\abs{A} = \abs{B}$ \textit{iff} $\abs{A} \leq \abs{B}$ and $\abs{B} \leq \abs{A}$.
  \item[] $\abs{A} < \abs{B}$ \textit{iff} $\abs{A} \leq \abs{B}$ and $\abs{B} \nleq \abs{A}$.
\end{itemize}






\section*{Countably Infinite}

\begin{enumerate}
  \item[\it Countable:] A set $A$ is \textit{countable iff} $\abs{A} \leq \abs{\N}$.
  \item[\it Infinite:] A set $A$ is \textit{infinite iff} $\abs{\N} \leq \abs{A}$.
    \begin{itemize}[leftmargin=-.2in]
      \item $\N_m$ is countably infinite since $f(n) = n + m$ is a bijection.  
      \item $\N_{(m)}$ is countably infinite since $f(n) = n \times m$ is a bijection.
      \item $\Z$ is countably infinite since there is a bijection $f(n) = 
        \begin{cases}
          \frac{n}{2}       & \text{if } n \text{ is even}\\
          \frac{-(n+1)}{2}  & \text{otherwise.}
        \end{cases}$.
      % \item The set of primes $\Pr$ is countable since $\Pr \subset \N$ and yet infinite.
      \item The positive rational numbers $\Q^+$ are countably infinite since: 
        \begin{itemize}
          \item There is an injection from $\Q^+$ to $\N^2$. 
          \item And $f(\tuple{n,m})=2^n \cdot 3^m$ is an injection from $\N^2$ to $\N$.
          \item Hence $\Q^+$ is countable, and so $\Q$ is also countable. 
          \item Infinite since identity is an injection from $\N$ to $\Q$. 
        \end{itemize}
      % \item FTA: every natural number has a unique prime decomposition.
    \end{itemize}
\end{enumerate}




\section*{Real Numbers}

\begin{enumerate}
  \item[\it Real Interval:] The real interval $(0,1)$ is uncountably infinite.
    \item $\abs{\N_2} \leq \abs{(0,1)}$ since $f(x) = 1/x$ is an injection $f: \N_2 \to (0,1)$.
    % \item Let $B=\set{.b_0b_1b_2\ldots : b_i \in \set{0,1} \text{ for all } i \in \N}/\set{0,1}$.
    % \item By Cantor's diagonal argument, $\abs{\N_1} \neq \abs{B}$.
    \item $\abs{\N_2} \neq \abs{(0,1)}$ by Cantor's diagonal argument.
    \item Thus $\abs{\N_1} < \abs{(0,1)}$.
    \item Observe that $g(x) = \pi(x - 1/2)$ is a bijection $g: (0,1) \to (-\pi/2,\pi/2)$. 
    \item Additionally~ $\texttt{tan}: (-\pi/2,\pi/2) \to \R$ is a bijection. 
    \item By the bijection principle, $\abs{(0,1)} = \abs{(-\pi/2,\pi/2)} = \abs{\R}$.
    \item Thus $\abs{\N_2} < \abs{\R}$ where $\abs{\N_2} = \abs{\N}$, so $\abs{\N} < \abs{\R}$.
  % \item[\it No Countable Difference:] $\abs{S}=\abs{S \cup A}$ ~whenever~ $\abs{\N} \leq \abs{S}$ and $\abs{A} \leq \abs{\N}$.
  %   \begin{itemize}[leftmargin=-.5in]
  %     \item[\bf Proof:] 
  %   \end{itemize}
\end{enumerate}



\section*{Cantor's Theorem}

\begin{enumerate}
  % \item[\it Power Set:] $\wp(A)=\set{X : X\subseteq A}$.
  \item[\it Theorem:] $\abs{A} < \abs{\wp(A)}$ for any set $A$ where $\wp(A)=\set{X : X\subseteq A}$. 
    \item $\abs{A} \leq \abs{\wp(A)}$ since $f(a) = \set{a}$ is an injection. 
    \item Assume there is a bijection $f: A \to \wp(A)$.
    \item Let $D=\set{a \in A: a \notin f(a)}$.
    \item Since $D \subseteq A$, we know that $D \in \wp(A)$.
    \item Since $f$ is surjective, $f(d) = D$ for some $d \in A$. 
    \item But $d \in f(d)$ \textit{iff} $d \in D$ \textit{iff} $d \notin f(d)$.
    \item This has the form $P \leftrightarrow \neg P$ which is equivalent to $P \wedge \neg P$. 
    \item Thus there is no bijection $f: A \to \wp(A)$, and so $\abs{A} \neq \abs{\wp(A)}$.
    \item Given the above, $\abs{A} < \abs{\wp(A)}$.
\end{enumerate}




\section*{Corollary}

\begin{enumerate}
  \item[\it Universal Set:] There is no set of all sets.
    \item Suppose there were a set $U$ of all sets. 
    \item Since every $X \in \wp(U)$ is a set, $\wp(U) \subseteq U$. 
    \item So $f(x) = x$ is an injection $f: \wp(U) \to U$. 
    \item Thus $\abs{\wp(U)} \leq \abs{U}$.
    \item Moreover, $g(x) = \set{x}$ is an injection $g: U \to \wp(U)$.
    \item So $\abs{U} \leq \abs{\wp(U)}$.
    \item Thus $\abs{U} = \abs{\wp(U)}$.
    \item By Cantor's Theorem, $\abs{U} < \abs{\wp(U)}$, so $\abs{U} \neq \abs{\wp(U)}$.
    \item Hence there is no set $U$ of all sets, so no set of everything!
\end{enumerate}



\section*{Axioms and Intuitions}

\begin{itemize}
  \item[\it Continuum Hypothesis:] There is no set $A$ where $\abs{\N} < \abs{A} < \abs{\R}$.
  \item[\it Independent:] Adding CH or its negation to ZFC is consistent if ZFC is consistent.
  \item Is it up to us to choose?
  \item Neither intuition nor mathematical practice seems to decide the issue.
  \item[\it Compare:] G\"{o}del showed that ZFC is consistent if ZF is consistent.
  \item[\it Axiom of Choice:] Every set of sets $X$ has a function $f$ where $f(Y)\in Y$ for all $Y\in X$. 
  \item[\it Well-Ordering Theorem:] Every set $X$ can be well-ordered (its subsets all have least elements). 
  \item AC and WOT are equivalent, intuitive, and extremely useful.
  \item Not so for CH!
\end{itemize}


\end{document}


