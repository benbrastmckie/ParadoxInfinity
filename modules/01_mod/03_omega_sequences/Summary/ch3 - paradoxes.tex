
\documentclass[12pt]{extarticle}

\usepackage{summary-intro}


\begin{document}



\sumintro{Omega-Sequence Paradoxes}{Spring 2023}



\section{What is a Paradox?}

A \textbf{paradox} is an argument that appears to be valid, and goes from seemingly true premises to a seemingly false conclusion. So we must:


\begin{itemize}
\item learn to live with the conclusion;

\item learn to live without one of the premises; or

\item show that the reasoning is invalid.

\end{itemize}
An \textbf{omega-sequence paradox} is a paradox based on an $\omega$-sequence ($| | | \dots$) or a reverse $\omega$-sequence ($\dots | | | $).

\section{Zeno's Paradox\footnote{This is a variant of one of several paradoxes attributed to ancient philosopher Zeno of Elea, who lived in the 5th Century BCE.
} {\normalsize \normalfont [Paradox Grade: 2]}}

\begin{quote}
You wish to walk from point $A$ to point $B$. In order to do so, you must carry out an \(\omega\)-sequence of tasks:
\[
\begin{array}{cc}
\text{\scriptsize Task  1:} & \text{\scriptsize reach $\frac{1}{2}$ mark}\\
\text{\scriptsize Task  2:} & \text{\scriptsize reach $\frac{3}{4}$ mark}\\
\text{\scriptsize Task  3:} & \text{\scriptsize reach $\frac{7}{8}$ mark}\\
\text{\tiny\vdots} & \text{\tiny\vdots}\\
\text{\scriptsize Task  $n$:} & \text{\scriptsize reach $\frac{2^n-1}{2^n}$ mark}\\
\text{\tiny\vdots} & \text{\tiny\vdots}\\
\end{array}
\]
\begin{center}
\begin{tikzpicture}
\draw (0,0) -- (8,0); % line
\draw (4,0) -- (4,-0.2); %1/2 way mark
\node at (4,-0.5) {\scriptsize $\frac{1}{2}$};
\draw (6,0) -- (6,-0.2); %1/4 way mark
\node at (6,-0.5) {\scriptsize $\frac{3}{4}$};
\draw (7,0) -- (7,-0.2); %1/8 way mark
\node at (7,-0.5) {\scriptsize $\frac{7}{8}$};
\node at  (7.65,-0.5) {\dots};

\node at (-.3,0)  {$A$}; %  A
\draw[ultra thick] (0,0) circle [radius=.03];

\node at (8.3,0)  {$B$}; %  B
\draw[ultra thick] (8,0) circle [radius=.03];

\end{tikzpicture}
\end{center}
But it's impossible to complete infinitely many tasks in a finite amount of time. So movement is impossible. 
\end{quote}




\section{Thomson's Lamp\footnote{Thomson's Lamp was devised by the late James Thomson, who was a professor of philosophy at MIT (and was married to (and then divorced from) the great philosopher Judith Jarvis Thomson).} {\normalsize \normalfont [Paradox Grade: 3]}}

You have a lamp with a toggle button: press the button once and the lamp goes on, press it again and the lamp goes off. Here's what happens:
\[
\begin{array}{cc}
\text{\footnotesize Time to midnight} & \text{\footnotesize Status of lamp shortly thereafter} \\
\text{\scriptsize 60$s$} & \text{\scriptsize off} \\
\text{\scriptsize 30$s$} & \text{\scriptsize on}\\ %n = 0, so 2^{2*0+1} = 2^1
\text{\scriptsize 15$s$} & \text{\scriptsize off}\\% n =1
\text{\scriptsize 7.5$s$} & \text{\scriptsize on}\\ %n =1 

\text{\tiny\vdots} & \text{\tiny\vdots}\\
\text{\scriptsize $\dfrac{60}{2^{2n}}s$} & \text{\scriptsize off}\\ %for evens
\text{\scriptsize $\dfrac{60}{2^{2n+1}}s$} & \text{\scriptsize on}\\%for odds; use same n here
\text{\tiny\vdots} & \text{\tiny\vdots}\\
\end{array}
\]
Is the lamp on or off at midnight?
\begin{itemize}
\item For every time the lamp gets turned off before midnight, there is a later time before midnight when it gets turned on. \textbf{So the lamp can't be off at midnight}.

\item For every time the lamp gets turned on before midnight, there is a later time before midnight when it gets turned off. \textbf{So the lamp can't be on at midnight}.

\end{itemize}


\section{The Demon's Game\footnote{Rayo learned about this paradox from philosophers Frank Arntzenius, Adam Elga, and John Hawthorne.} {\normalsize \normalfont [Paradox Grade: 4]}}

\(P_1,P_2,P_3,\ldots\) take turns answering  \emph{aye} or \emph{nay}:

\begin{itemize}
\item If exactly $n$ people say \emph{aye} ($n \in \mathbb{N}$), each person gets $\$n$. 

\item If infinitely many people say \emph{aye}, they all get nothing.

\end{itemize}
It seems rational for $P_k$ to say \emph{aye}: she can't hurt anyone and might help everyone. But if it's rational for $P_k$ it's rational for everyone. So nobody gets anything.




\section{The Bomber's Paradox\footnote{This paradox is due to Josh Parsons, who was a fellow at Oxford until shortly before his untimely death in 2017. (It is a version of Benardete's Paradox.)
} {\normalsize \normalfont [Paradox Grade: 6]}}


There are infinitely many bombs:
\[
\begin{array}{cc}
\text{\footnotesize Bomb} & \text{\footnotesize When bomb is set to go off} \\
\text{\scriptsize $B_0$} & \text{\scriptsize 12:00pm} \\
\text{\scriptsize $B_1$} & \text{\scriptsize 11:30am} \\
\text{\scriptsize $B_2$} & \text{\scriptsize 11:15am} \\
\text{\tiny\vdots} & \text{\tiny\vdots}\\
\text{\scriptsize $B_k$} & \text{\scriptsize \(\dfrac{1}{2^k}\) hours after 11:00am} \\
\text{\tiny\vdots} & \text{\tiny\vdots}\\
\end{array}
\]
Should one of the bombs go off, it will instantaneously disable all other bombs. So a bomb goes off if and only if no bombs have gone off before it:
\begin{description}

\item[(0)] \(B_0\) goes off $\leftrightarrow$ \(B_n\)  fails to go off ($n > 0$).
\item[(1)] \(B_1\) goes off $\leftrightarrow$ \(B_n\)  fails to go off ($n > 1$).
\item[(2)] \(B_2\) goes off $\leftrightarrow$ \(B_n\) fails to go off ($n > 2$).



\hspace{5mm }\vdots

\item[$\bm{(k)}$] \(B_k\) goes off $\leftrightarrow$ \(B_n\) fails to go off ($n > k$).
\item[$\bm{(k+1)}$] \(B_{k+1}\) goes off $\leftrightarrow$ \(B_n\) fails to go off ($n > k+1$).

\hspace{5mm }\vdots

\end{description}
Will any bombs go off?



\section{Yablo's Paradox\footnote{This paradox was discovered by Steve Yablo, a philosophy professor at MIT. 
} {\normalsize \normalfont [Paradox Grade: 8]}}


There are infinitely many sentences:

\[
\begin{array}{cc}
\text{\footnotesize Label} & \text{\footnotesize Sentence} \\
 \text{\scriptsize $S_0$} & \text{\scriptsize ``For each \(i>0\), sentence \(S_i\) is false''} \\
\text{\scriptsize $S_1$} & \text{\scriptsize ``For each \(i>1\), sentence \(S_i\) is false''} \\
\text{\scriptsize $S_2$} & \text{\scriptsize ``For each \(i>2\), sentence \(S_i\) is false''} \\
\text{\tiny\vdots} & \text{\tiny\vdots}\\
\text{\scriptsize $S_k$} & \text{\scriptsize ``For each \(i>k\), sentence \(S_i\) is false''} \\
\text{\tiny\vdots} & \text{\tiny\vdots}\\
\end{array}
\]
The meanings of our sentences guarantee that each of the following must be true:

\begin{description}

\item[(0)] \(S_0\) is true $\leftrightarrow$ \(S_n\) is false (\(n>0\)).

\item[(1)] \(S_1\) is true $\leftrightarrow$ \(S_n\)  is false (\(n>1\)).

\item[(2)] \(S_2\) is true $\leftrightarrow$ \(S_n\) is false (\(n>2\)).

\hspace{5mm }\vdots

\item[$\bm{(k)}$] \(S_k\) is true $\leftrightarrow$ \(S_n\)  is false (\(n>k\)).
\item[$\bm{(k+1)}$] \(S_{k+1}\) is true $\leftrightarrow$ \(S_n\)  is false (\(n>k+1\)).

\hspace{5mm }\vdots


\end{description}
Which sentences are true and which ones are false?


\section{Bacon's Problem\footnote{This paradox is due to USC philosopher Andrew Bacon. 
} {\normalsize \normalfont [Paradox Grade: 7]}}


\begin{itemize}

\item An omega sequence of prisoners: \(P_1,P_2,P_3,\dots\). (\(P_1\) is at the end of the line, in front of her is \(P_2\), in front of him is \(P_3\), and so forth.) 

\item Each person is assigned a red or blue hat, based on the outcome of a coin toss. 

\item Everyone can see the hats of the people in front of her, but cannot see her own hat (or the hat of anyone behind her). 

\item At a set time, everyone has to guess the color of their own hat by crying out ``Red!" or ``Blue!".

\item  People who correctly call out the color of their own hats will be freed. Everyone else will remain captive. %be shot.

\end{itemize}

\noindent
\emph{Problem}: Find a strategy that \(P_1, P_2, P_3, \ldots\) could agree upon in advance that would guarantee that at most finitely many people remain captive. %are shot.



\section{The Three Prisoners\footnote{Rayo doesn't know who invented this paradox, but he learned about it thanks to philosopher and computer scientist Rohit Parikh, from the City University of New York. } {\normalsize \normalfont [Paradox Grade: 2]}}




\begin{itemize} 

\item Three prisoners. Each of them is assigned a red or blue hat, based on the outcome of a coin toss.
\item Each of them can see the colors of the others' hats but has no idea about the color of his own hat.
\item The prisoners are then taken into separate cells and asked about the color of their hat. They are free to offer an answer or remain silent. 

\begin{itemize}
\item If all three prisoners remain silent, all three will remain captive %be killed.
\item If one of them answers incorrectly, all three will remain captive %be killed.
\item If at least one prisoner offers an answer, and everyone who offers an answer answers correctly, then all three prisoners will be freed.
\end{itemize}
\end{itemize}

\noindent
\emph{Problem}: Find a strategy that the prisoners could agree upon ahead of time that would guarantee that their chance of survival is above 50\%.



\begin{figure}
\[
\begin{array}{cccc}
\text{\footnotesize Prisoner $A$} & \text{\footnotesize Prisoner $B$} & \text{\footnotesize Prisoner $C$} &\text{\footnotesize Result of following Strategy} \\
 \text{\scriptsize red} & \text{\scriptsize red} & \text{\scriptsize red} & \text{\scriptsize Everyone answers incorrectly} \\
  \text{\scriptsize red} & \text{\scriptsize red} & \text{\scriptsize blue} & \text{\scriptsize $C$ answers correctly} \\
   \text{\scriptsize red} & \text{\scriptsize blue} & \text{\scriptsize red} & \text{\scriptsize $B$ answers correctly} \\
    \text{\scriptsize red} & \text{\scriptsize blue} & \text{\scriptsize blue} & \text{\scriptsize $A$ answers correctly} \\
     \text{\scriptsize blue} & \text{\scriptsize red} & \text{\scriptsize red} & \text{\scriptsize $A$ answers correctly} \\
      \text{\scriptsize blue} & \text{\scriptsize red} & \text{\scriptsize blue} & \text{\scriptsize $B$ answers correctly} \\
       \text{\scriptsize blue} & \text{\scriptsize blue} & \text{\scriptsize red} & \text{\scriptsize $C$ answers correctly} \\
        \text{\scriptsize blue} & \text{\scriptsize blue} & \text{\scriptsize blue} & \text{\scriptsize Everyone answers incorrectly} \\
\end{array}
\]
\caption{The eight possible hat distributions, along with the result of applying the suggested strategy.}
\label{fig:prisoners}
\end{figure}






\end{document}




