\documentclass[a4paper, 11pt]{article} % Font size (can be 10pt, 11pt or 12pt) and paper size (remove a4paper for US letter paper)

\usepackage[protrusion=true,expansion=true]{microtype} % Better typography
\usepackage{graphicx} % Required for including pictures
\usepackage{wrapfig} % Allows in-line images
\usepackage{enumitem} %%Enables control over enumerate and itemize environments
\usepackage{setspace}
\usepackage{amssymb, amsmath, mathrsfs,mathabx} %%Math packages
\usepackage{stmaryrd}
\usepackage{mathtools}
\usepackage{multicol} 
\usepackage{mathpazo} % Use the Palatino font
\usepackage[T1]{fontenc} % Required for accented characters
\usepackage{array}
\usepackage{bibentry}
\usepackage{prooftrees} 
\usepackage[round]{natbib} %%Or change 'round' to 'square' for square backers
\setcitestyle{aysep=}
% \usepackage{fitchproof} 

% \linespread{1.05} % Change line spacing here, Palatino benefits from a slight increase by default

\newcommand{\tuple}[1]{\langle#1\rangle} %%Angle brackets
\newcommand{\set}[1]{\lbrace#1\rbrace} %%Set brackets
\newcommand{\abs}[1]{|#1|} %%Set brackets
\newcommand{\interpret}[1]{\llbracket#1\rrbracket} %%Double brackets
\newcommand{\N}{\mathbb{N}}
\newcommand{\D}{\mathbb{D}}
\newcommand{\Z}{\mathbb{Z}}
\renewcommand{\Pr}{\mathbb{P}}
\newcommand{\Q}{\mathbb{Q}}
\newcommand{\R}{\mathbb{R}}
\newcommand{\B}{\mathfrak{B}}

\makeatletter
\renewcommand\@biblabel[1]{\textbf{#1.}} % Change the square brackets for each bibliography item from '[1]' to '1.'
\renewcommand{\@listI}{\itemsep=0pt} % Reduce the space between items in the itemize and enumerate environments and the bibliography

\renewcommand{\maketitle}{ % Customize the title - do not edit title and author name here, see the TITLE block below
\begin{flushright} % Right align
{\LARGE\@title} % Increase the font size of the title

\vspace{10pt} % Some vertical space between the title and author name

{\@author} % Author name
\\\@date % Date

\vspace{-10pt} % Some vertical space between the author block and abstract
\end{flushright}
}

%----------------------------------------------------------------------------------------
%	TITLE
%----------------------------------------------------------------------------------------

\title{\textbf{The Higher Infinite}} % Subtitle

\author{\textsc{Paradox and Infinity}\\ \em Benjamin Brast-McKie} % Institution

\date{\today} % Date

%----------------------------------------------------------------------------------------

\begin{document}

\maketitle % Print the title section

\thispagestyle{empty}

%----------------------------------------------------------------------------------------




\section*{The Ordinals}

\begin{itemize}
  \item[\it Cardinalities:] Observe that $\abs{\omega}=\abs{\omega + 1}=\abs{\omega+2}=\ldots=\abs{\omega+\omega}=\ldots$
    \item This is just Hilbert's hotel all over again.
  \item[\it Structure:] But $\omega \neq \omega + 1 \neq \omega + 2 \neq \ldots \neq \omega + \omega \neq \ldots$ 
    \item Ordinals have more structure than is encoded by their cardinalities.
    \item $2$ represents the class of two things where one is \textit{after} the other. 
  \item[\it Order:] The ordinals have order structure--- they are \textit{well-ordered}.
    \item $\alpha$ and $\beta$ have the same order-type \textit{iff} $\tuple{\alpha,<_o} \cong \tuple{\beta,<_o}$.
    \item Even though $\omega \simeq \omega + 1$, we don't get that $\tuple{\omega,<_o} \cong \tuple{\omega+1,<_o}$.
    \item Moreover, $1 + \omega = \omega$ even though $\omega +1 \neq \omega$.
  \item[\it Sequences:] If we care about numbers, why introduce the ordinals at all?
    \item Isn't this a bate and switch?
    \item Why do we need transfinite order structure?

\end{itemize}




\section*{Ordinals and Algorithms}

\begin{itemize}
  \item[\it Algorithms:] Programs/instructions to add $1$ some (ordinal) number of times. 
    \item Roughly: `$3$' says `add $1$ three times'.
    \item Note that we use `three' to say what `$3$' means.
  \item[\it First Pass:] $3 + 2 = 0''' + 0'' = 0'''' + 0' = 0''''' + 0 = 0''''' = 5$.
    \item $\alpha + \beta' = \alpha' + \beta$.
    \item Doesn't generalize since $\omega + \omega = \omega' +\ ?$ 
  \item[\it Adding Limit Ordinals:] How should we think about adding $1$ $\omega + \omega$ times?
    \item What if we could parallel process (add)?
    \item Then we would get $\omega + \omega = \omega$.
    \item Need to preserve order and ``parallel processing'' ignores order.
  \item[\it Simpler Case:] $\omega + 1 \neq 1 + \omega$.
    \item Think of the simplest case you can to exhibit some feature.
    \item Not always the first case you might think of.
  \item[\it Conclusion:] If we ignore order, we are back to cardinality.
\end{itemize}





\section*{Ordinal Arithmetic}

\begin{itemize}
  \item[\it Addition:] 
    $\alpha + 0 = \alpha$\\ 
    $\alpha + \beta' = (\alpha + \beta)'$\\ 
    $\alpha + \lambda = \bigcup \set{ \alpha + \beta : \beta <_o \lambda }$ where $\lambda\neq 0$ is a limit ordinal.
  \item[\it Examples:]
    $3 + 2 = 3 + 1' = (3 + 1)' = (3 + 0')' = (3 + 0)'' = 3'' = 5$.\\
    $\omega + \omega = \bigcup \set{ \omega + 0, \omega + 1, \omega +2, \ldots } = \set{ 0, 1, 2, \ldots, \omega, \omega +1, \omega + 2, \ldots }$.\\
    $1 + \omega = \bigcup \set{ 1 + 0, 1 + 1, 1 + 2, \ldots } = \set{ 0, 1, 2, \ldots } = \omega$.
  \item[\it Multiplication:] 
    $\alpha \times 0 = 0$\\ 
    $\alpha \times \beta' = (\alpha \times \beta) + \alpha$\\ 
    $\alpha \times \lambda = \bigcup \set{ \alpha \times \beta : \beta <_o \lambda }$ where $\lambda\neq 0$ is a limit ordinal.
  \item[\it Examples:] 
    $\omega \times 2 = (\omega \times 1) + \omega = ((\omega \times 0) + \omega) + \omega = (0 + \omega) + \omega = \omega + \omega$.\\
    $\omega \times \omega = \bigcup \set{ \omega \times 0, \omega \times 1, \omega \times 2, \ldots }$\\
    \strut\hspace{.42in} $= 
      \begin{cases}
        0,1,\ldots\\
        \omega, \omega + 1,\ldots\\
        \omega + \omega, (\omega + \omega) +1, \ldots\\
        \strut\quad\vdots\quad, \quad\vdots\quad, \quad\vdots\quad, 
      \end{cases}$
    % $3 \times 2 = (3 \times 1) + 3 = ((3 \times 0) + 3 ) + 3 = (0 + 3) + 3 = 3 + 3$.\\
    % $2 \times 3 = (2 \times 2) + 3 = $
\end{itemize}




\section*{``Small'' Cardinals}

\begin{itemize}
  \item[\it Recall:] $\abs{\N} < \abs{\wp(\N)} < \abs{\wp^2(\N)} < \ldots < \abs{U}$. 
    \item Remember that `$<$' means `injection but no bijection' here.
    \item We can use $\omega$ to define $U$.
  \item[\it Definition:] Let $\B_\alpha = 
    \begin{cases}
      \N                                            & \text{if } \alpha = 0\\
      \wp(\B_\beta)                                 & \text{if } \alpha = \beta'\\
      \bigcup \set{ \B_\gamma : \gamma <_o \alpha } & \text{otherwise.} 
    \end{cases}$
  \item[\it Example:] $\B_\omega = \bigcup \set{\N,\wp(\N),\wp^2(\N),\ldots} = U$.
\end{itemize}



\section*{Uncountable Ordinals}

\begin{itemize}
  \item[\it Countable Ordinals:] We have only seen countable ordinals so far. 
  \item[\it Beth:] Let $\beth_\alpha = \beta$ \textit{iff} $\abs{\beta} = \abs{\B_\alpha}$ and $\beta <_o \gamma$ for all $\gamma$ where $\abs{\gamma} = \abs{\B_\alpha}$. 
    \item $\beth_0 = \omega$ since $\abs{\omega} = \abs{\B_0} = \abs{\N}$.
    \item We can then make big ordinals to make even bigger cardinals, etc.
  \item[\it Motivations:] Why care about all of this?
    \item Because math (is awesome).
    \item We are probing the limits of what is thinkable, not useful.
\end{itemize}


\end{document}


