\documentclass[12pt,letterpaper]{article}
\usepackage{../../assets/styles/rset_2024}
\usepackage{bibentry}
\usepackage[round]{natbib} %%Or change 'round' to 'square' for square backers

%%% CITATIONS %%%
% \usepackage{bibentry} %%Replace \bibliography{} with \nobibliography{} for no bib
\usepackage[round]{natbib} %%Or change 'round' to 'square' for square backers
\setcitestyle{aysep={}}
 %   \citet{key} ==>>                Jones et al. (1990)
 %   \citet*{key} ==>>               Jones, Baker, and Smith (1990)
 %   \citep{key} ==>>                (Jones et al., 1990)
 %   \citep*{key} ==>>               (Jones, Baker, and Smith, 1990)
 %   \citep[chap. 2]{key} ==>>       (Jones et al., 1990, chap. 2)
 %   \citep[e.g.][]{key} ==>>        (e.g. Jones et al., 1990)
 %   \citep[e.g.][p. 32]{key} ==>>   (e.g. Jones et al., p. 32)
 %   \citeauthor{key} ==>>           Jones et al.
 %   \citeauthor*{key} ==>>          Jones, Baker, and Smith
 %   \citeyear{key} ==>>             1990
\usepackage{etoolbox} %%For \citepos
\usepackage{xstring} %%For \citepos

\makeatletter %definition of \citepos
% \patchcmd{\NAT@test}{\else \NAT@nm}{\else \NAT@nmfmt{\NAT@nm}}{}{} %turn on for numeric citations
\DeclareRobustCommand\citepos% define \citepos
  {\begingroup
   \let\NAT@nmfmt\NAT@posfmt% same as for citet except with a different name format
   \NAT@swafalse\let\NAT@ctype\z@\NAT@partrue
   \@ifstar{\NAT@fulltrue\NAT@citetp}{\NAT@fullfalse\NAT@citetp}
  }
   
\let\NAT@orig@nmfmt\NAT@nmfmt %makes adapt to last names ending with an 's'.
\def\NAT@posfmt#1{%
  \StrRemoveBraces{#1}[\NAT@temp]%
  \IfEndWith{\NAT@temp}{s}
    {\NAT@orig@nmfmt{#1'}}
    {\NAT@orig@nmfmt{#1's}}}
\makeatother

%Questions and Answers
\qa{q} % a="answers only"; q ="questions only"; b="both"
\usepackage{../../assets/styles/qa}
\usepackage{setspace}


    % Given the set of integers $\Z$, suppose one were to identify the \textit{rational numbers} with the ordered pairs $\tuple{x,y}$ for $x,y \in \Z$ where $\tuple{x,y}$ aims to represent the fraction $\frac{x}{y}$. 
    % This definition has two problems:
    % (1) we cannot divide by $0$; and
    % (2) $\frac{1}{2} = \frac{2}{4}$ even though $\tuple{1,2} \neq \tuple{2,4}$.
    % In order to avoid these problems we may first 

\begin{document}

\psintro{Response Set 5: Newcomb's Problem and the Prisoner's Dilemma}

\newpage

\vspace{.2in}


\begin{enumerate}

  \item 
    \question{
      Provide a detailed outline of \citepos{Lewis1979a} argument that Newcomb's problem is a prisoner's dilemma.
      Do you find Lewis' argument convincing?
      If so, describe what you find convincing, and if not, say why you think the argument fails.
    }
    \answer{
    \item[\tt (A)] 
      answer
    }

\item 
  \question{
    Explain what Lewis' reasons are for making the following claims.
    \begin{quote}\singlespacing\small
      As Newcomb's Problem is usually told, the predictive process involved is extremely reliable.
      But that is inessential.
      The disagreement between conceptions of rationality that gives the problem its interest arises even when the reliability of the process, as estimated by the agent, is quite poor-indeed, even when the agent judges that the predictive process will do little better than chance.
      \hfill(p.~238)
    \end{quote}
    Present Lewis' reasoning as clearly and succinctly, commenting on why Lewis makes this point (what role it plays in his argument overall).
  }
  \answer{
    answer
  }


\item
  \question{
    In response to \citepos{Lewis1979a} paper, \citet{Bermudez2013} writes the following:
    \begin{quote}\singlespacing\small
      What matters in Newcomb’s problem is not simply that there be a two-way dependence between my receiving \$1,000,000 and it being predicted that I not take the \$1,000.
      That two-way dependence only generates a problem because I know that the contingency holds.
      \hfill(p.~427)
    \end{quote}
    Explain the point that Berm{\'u}dez is making here (and goes on to further elaborates in his paper).
    Do you find this point convincing?
  }

\item
  \question{
    Berm{\'u}dez concludes by writing:
    \begin{quote}\singlespacing\small
      [T]he considerations that Lewis brings to bear to show that the game he starts with is an NP equally show that the game is not a PD.
      \hfill(p.~428-9)
    \end{quote}
    Do you agree?
    Say why or why not, presenting Berm{\'u}dez's argument in support of your evaluation of his conclusion.
  }

\end{enumerate}

\vfill
\bibliographystyle{Phil_Review} %%bib style found in bst folder, in bibtex folder, in texmf folder.
\bibliography{24.118_rset_05} %%bib database found in bib folder, in bibtex folder
\end{document}

