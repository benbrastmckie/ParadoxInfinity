\documentclass[12pt,letterpaper]{article}
\usepackage{../../assets/styles/rset_2024}
\usepackage{bibentry}
\usepackage[round]{natbib} %%Or change 'round' to 'square' for square backers

%Josh: add back in the commented out questions 
%note that I have left out the Team A question involving a causal explanation...seems too fraught philosohically perhaps! 


%Questions and Answers
\qa{q} % a="answers only"; q ="questions only"; b="both"
\usepackage{../../assets/styles/qa}


    % Given the set of integers $\Z$, suppose one were to identify the \textit{rational numbers} with the ordered pairs $\tuple{x,y}$ for $x,y \in \Z$ where $\tuple{x,y}$ aims to represent the fraction $\frac{x}{y}$. 
    % This definition has two problems:
    % (1) we cannot divide by $0$; and
    % (2) $\frac{1}{2} = \frac{2}{4}$ even though $\tuple{1,2} \neq \tuple{2,4}$.
    % In order to avoid these problems we may first 

\begin{document}

\psintro{Response Set 4: The Metaphysics of Time}

\newpage

\vspace{.2in}


\begin{enumerate}

  \item 
    \question{
      McTaggart presents two separate conceptions of the positions in time which go by the names `A-series' and `B-series'.
      Describe each conception, using an example to help explain how each series orders positions in time.
    }
    \answer{
    \item[\tt (A)] 
      answer
    }

\item 
  \question{
    McTaggart takes both the A-series and B-series to be essential to the nature of time, claiming that the A-series might be regarded as more fundamental than the B-series.
    Explain why McTaggart thinks that the B-series alone does not provide an adequate account of the nature of time.
    Do you agree that the A-series is essential?
  }
  \answer{
    answer
  }


\item
  \question{
    \citet[pp.~27-8]{Mctaggart1908} contrasts the case of the poker being hot at one time and cold at another to the meridian of Greenwich.
    Present these examples, explaining what McTaggart takes them to show.
    Do you find these examples to be convincing?
    Say why or why not.
  }

\item
  \question{
    \citet[p.~32]{Mctaggart1908} asserts that: (1) past, present, and future are incompatible determinations; and yet also claims that (2) every event has them all.
    He goes on to consider a revision of (2) which denies that any event is past, present, and future, claiming instead that these, ``characteristics are only incompatible when they are simultaneous.''
    What problem does McTaggart take the revised version of (2) to have?
    Do you find McTaggart's argument convincing?
    Say why or why not.
    % From this he concludes that the reality of the A-series leads to a contradiction, and so not only the A-series but time must not be real.
  }

\end{enumerate}

\vfill
\bibliographystyle{Phil_Review} %%bib style found in bst folder, in bibtex folder, in texmf folder.
\bibliography{Zotero} %%bib database found in bib folder, in bibtex folder
\end{document}

