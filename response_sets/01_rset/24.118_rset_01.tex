\documentclass[12pt,letterpaper]{article}
\usepackage{../../assets/styles/rset_2024}

%Josh: add back in the commented out questions 
%note that I have left out the Team A question involving a causal explanation...seems too fraught philosohically perhaps! 


%Questions and Answers
\qa{q} % a="answers only"; q ="questions only"; b="both"
\usepackage{../../assets/styles/qa}



\begin{document}

\psintro{Response Set 1: Self-Reference}

\newpage

\question{
  \fbox{\parbox{150mm}{
    \textbf{Note:} 
    All references this week will be to Russell's (1908) paper ``Mathematical Logic as based on the Theory of Types.''
  }}
}

\vspace{.2in}

\begin{enumerate}

\item 
  \question{On p.~225, Russell argues that there, ``is no such thing as the class of all classes.'' Provide a reconstruction of this argument, carefully justifying each step. Do you find Russell's argument persuasive?}
  \answer{thing}

\item 
  \question{
    On p.~236, Russell claims that what is essential for legitimately speaking of ``all of a collections'' is not finitude but what he calls \textit{logical homogeneity}, admitting that it is not always obvious whether a collection possesses this property given the, ``concealed ambiguity in common logical terms such as \textit{true} and \textit{false}\ldots'' 
    Explain what Russell means by `logical homogeneity' and why he thinks `true' and `false' are ambiguous.
    Do you find this suggestion plausible? Say why or why not.
  }
  \answer{thing}

\item 
  \question{
    On p.~237, Russell presents one version of his \textit{vicious circle principle} (\textsc{vcp}). 
    Provide a reconstruction of \textsc{vcp} along with an example that fails to conform to \textsc{vcp}.
    Are there innocent (i.e., non-paradoxical) examples that are banned by \textsc{vcp} that are worth trying to save?
    What is the underlying reason one might take \textsc{vcp} to be true and do you find these reasons compelling?
  }
  \answer{thing}

\item 
  \question{On p.~238, Russell solves the liar paradox.
    Provide a reconstruction of his solution and comment on whether you find this solution compelling or not (say why).
    What are the downsides to adopting Russell's solution, and are these costs worth paying?
  }
  \answer{thing}



\end{enumerate}


\end{document}

