\documentclass[12pt,letterpaper]{article}
\usepackage{../../assets/styles/rset_2024}
\usepackage{bibentry}
\usepackage[round]{natbib} %%Or change 'round' to 'square' for square backers

%Josh: add back in the commented out questions 
%note that I have left out the Team A question involving a causal explanation...seems too fraught philosohically perhaps! 


%Questions and Answers
\qa{q} % a="answers only"; q ="questions only"; b="both"
\usepackage{../../assets/styles/qa}


    % Given the set of integers $\Z$, suppose one were to identify the \textit{rational numbers} with the ordered pairs $\tuple{x,y}$ for $x,y \in \Z$ where $\tuple{x,y}$ aims to represent the fraction $\frac{x}{y}$. 
    % This definition has two problems:
    % (1) we cannot divide by $0$; and
    % (2) $\frac{1}{2} = \frac{2}{4}$ even though $\tuple{1,2} \neq \tuple{2,4}$.
    % In order to avoid these problems we may first 

\begin{document}

\psintro{Response Set 3: Set Theory}

\newpage

\vspace{.2in}


\begin{enumerate}

  \question{
  \item 
    Here is a logic-to-mathematical-English dictionary where $\varphi$ and $\psi$ are to be replaced with sentences (possibly including variables) and $\corner{\alpha}$ names the expression $\alpha$:\footnote{You do not need to know how $\corner{\alpha}$ differs from `$\alpha$' to succeed at this translation exercise but $\corner{\cdot}$ is a function from expressions to names for expressions, and `$\alpha$' is a variable for expressions. By contrast, quotes name whatever is inside them, e.g., `$\alpha$' names the first letter of the Greek alphabet, not the value of the variable $\alpha$. You are also encouraged to adapt your translation to be natural sounding in English, rephrasing `is such that', `it is not the case that', etc., while preserving the same meaning.}
      \begin{itemize}
        \item[-] `$(\exists x)\underline{~~~}$' reads `there is some $x$ where\underline{~~~}'.
        \item[-] `$(x)\underline{~~~}$' reads `every $x$ is such that\underline{~~~}'.
        \item[-] `$Sx$' reads `$x$ is a set'.
        \item[-] `$x \in y$' reads `$x$ is a member of $y$'.
        \item[-] `$x = y$' reads `$x$ is identical to $y$'.
        % \item[-] `$x \notin y$' reads `$x$ is not a member of $y$' (also written `$\sim x \in y$').
        \item[-] $\corner{\sim\varphi}$ reads $\corner{\text{it is not the case that } \psi}$.
        \item[-] $\corner{\varphi\ \&\ \psi}$ reads $\corner{\varphi \text{ and } \psi}$.
        \item[-] $\corner{\varphi \vee \psi}$ reads $\corner{\varphi \text{ or } \psi}$.
        \item[-] $\corner{\varphi \rightarrow \psi}$ reads $\corner{\text{if } \varphi \text{ then } \psi}$.
        \item[-] $\corner{\varphi \leftrightarrow \psi}$ reads $\corner{\varphi \text{ if and only if } \psi}$.
      \end{itemize}
    Given the readings above, provide a translation of the following formal sentence after first replacing $\varphi$ with the sentence `$x$ is a natural number':
      \begin{enumerate}
        \item $(\exists y)(Sy\ \&\ (x)(x \in y \leftrightarrow \varphi))$.
      \end{enumerate}
    Even the naive conception of set restricts the sentences that we may substitute for `$\varphi$' by forbidding the variable `$y$' to occur free.\footnote{For instance, `$y$' is \textit{free} in `$y$ admires Cantor' but \textit{bound} and so not free in `every $y$ is such that $y$ admires Cantor'. For this example, it is fine to restrict consideration to simple sentences which do not include quantifiers like `every $y$'. Every variable in a sentence without quantifiers is free.}
    What could go wrong if `$y$' were permitted to occur free in the sentence $\varphi$ that we substitute in (a)? 
    Note that $\varphi$ is permitted to include the variable `$x$'.
    What sort of trouble does this create for the naive theory given that it asserts all instances of (a) where `$y$' is not free in $\varphi$?
  }
  \answer{
  \item[\tt (A)] 
    answer
  }

% \end{enumerate}
%
% \fbox{\parbox{150mm}{
%   The naive conception of set may claim to be among the most intuitive.
%   However, \citet[p.~218]{Boolos1971} makes a case for the \textit{iterative conception of set}, claiming that  ``It is, perhaps, no more natural a conception than the naive conception, and certainly not quite so simple to describe.''
%   What makes 
% }}
%
% \begin{enumerate}
%
% \setcounter{enumi}{3}

\item 
  \question{
    In considering whether there is a set of all sets, \citet{Boolos1971} writes:
    \begin{quote}\small
      Of course a set can and must include itself (as a subset).
      But \textit{contain} itself?
      Whatever tenuous hold on the concepts of \textit{set} and \textit{member} were give none by Cantor's definitions of `set' and one's ordinary understanding of `element', `set', `collection', etc. is altogether lost if one is to suppose that some sets are members of themselves.
    % The idea is paradoxical not in the sense that it is contradictory to suppose that some set is
      \hfill(p.~119)
    \end{quote}
    What does Boolos say in defense of this claim?
    Why does he also believe that there can't be sets that belong to each other?
    Do you find his arguments convincing?
    Which is harder to believe: (a) some sets are members of themselves; or (b) there is no set of all sets?
    Briefly explain what supports your conclusion.
  }
  \answer{
    answer
  }

\newpage

\item
  \question{
    \citet{Boolos1971} describes the iterative conception of set in three parts, writing:
      \begin{quote}\small
        The first is a rough statement of the idea.
        It contains such expressions as `stage', `is formed at', `earlier than', `keep on going', which must be exorcised from any formal theory of sets.
        From the rough description it sounds as if sets were continually being created, which is not the case
        \hfill(p.~229)
      \end{quote}
    Setting these convenient ways of speaking to one side, in what sense are some sets ``earlier than'' others if they are not earlier in time?
  }

\item
  \question{
    \citet{Boolos1971} claims towards the end of his paper:
      \begin{quote}\small
        The axiom of extensionality enjoys a special epistemological status shared by none of the other axioms of ZF.
        \hfill(p.~229)
      \end{quote}
    Do you share in his finding.
    If so, what convinces you that the axiom of extensionality enjoys this special status?
    If not, explain what gives rise to your doubts.
  }

\end{enumerate}

\vfill
\bibliographystyle{Phil_Review} %%bib style found in bst folder, in bibtex folder, in texmf folder.
\bibliography{Zotero} %%bib database found in bib folder, in bibtex folder
\end{document}

