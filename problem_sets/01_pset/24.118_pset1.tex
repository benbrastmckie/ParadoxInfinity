
\documentclass[12pt,letterpaper]{article}
\usepackage{../../assets/styles/pset_2024}
%\usepackage{wasysym}
%\usepackage{marvosym}
\usepackage{mathabx}

%points currently add to 86, which seems dumb. part I on canvas is 31 points. so need to add 14 points to problems below. 

%Question idea analog for Question 9: prove that the cardinality of the algebraic numbers is countable. Note that Cantor did this (for the real algebraic numbers) in 1874. So of historical interest. Argument is very similar to the one that works for the infinite tree: define the height of a polynomial, and argue (from fundamental theorem of algebra) that for each height, there are a finite number of zeros. Associate each root with the polynomial of least height it arises for. Then we have an infinite rung of heights where on each rung there are finitely-many roots. 
% See this video: https://www.youtube.com/watch?v=pSSsZLTMDq0

%Questions and Answers
\qa{q} % q ="questions only"; a="answers only";  b="both"
\usepackage{../../assets/styles/qa}

\begin{document}

\psintro{Problem Set 1: Infinite Cardinalities}

\newpage

% Idea for next time: ask them to show that Cantor's Theorem shows that the powerset axiom is incompatible with the existence of a universal set.



\subsection*{Part I (Quiz on Canvas: 31 points)} 


\begin{enumerate}

\com{
\item 
\question{Let \(A\) be a set and let \(R\) be a relation that holds amongst members of that set. Then: 
  \begin{itemize}
\item   
   \(R\) is \textbf{reflexive} on \(A\) if and only if: for any \(a\in A\), \(aRa\). 
      \item  \(R\) is \textbf{symmetric} on \(A\) if and only if: for any \(a,b\in A\), if \(aRb\) then \(bRa\). 
  \item  \(R\) is \textbf{transitive} on \(A\) if and only if: for any \(a,b,c\in A\), if \(aRb\) and \(bRc\) then \(aRc\). 
\end{itemize}
(\emph{Example:} Let \(A\) be the set of people, and let \(R\) be the relation ``having the same birthday as". \(R\) is reflexive, since everyone has the same birthday as herself. \(R\) is symmetric, since whenever \(a\) has the same birthday as \(b\), \(b\) will have the same birthday as \(a\). And \(R\) is transitive: for any three people \(a\), \(b\), and \(c\), if \(a\) has the same birthday as \(b\), and \(b\) has the same birthday as \(c\), then \(a\) has the same birthday as \(c\).)}

\begin{enumerate}
\item  \question{Consider the relation ``is less than or equal to". Is this relation: $(i)$ reflexive, $(ii)$ symmetric, and $(iii)$ transitive on the set of natural numbers? (5 points.)}

 \item    \question{There are ten guests at a dinner party, sitting around a large table. Consider the relation \(R\) such that guest \(a\) bears \(R\) to guest \(b\) if and only if \(a\)'s seat is immediately adjacent to \(b\)'s seat. Is \(R\): $(i)$ reflexive, $(ii)$ symmetric, and $(iii)$ transitive on the set of guests?  (Keep in mind that no seat is adjacent to itself.) (5 points.)}

\end{enumerate}
}



\item \question{
  A \textbf{relation} with \textbf{domain} $A$ and \textbf{range} $B$ is any set of ordered pairs $\tuple{a,b}$ where $a\in A$ and $b\in B$, i.e., any subset of all such pairs $A\times B=\set{\tuple{a,b}:a\in A, b\in B}$.

  A relation is a \textbf{function} $f : A \to B$ just in case every element in $A$ is assigned to exactly one element of $B$, i.e., if $f(x)=f(y)$ whenever $x=y$.

  A function $f: A \to B$ is an \textbf{injection} just in case no two elements in $A$ are assigned to the same element in $B$, i.e., if $x=y$ whenever $f(x) = f(y)$.

  A function $f: A \to B$ is a \textbf{surjection} just in case for every element $b\in B$ there is some $a\in A$ where $f(a)=b$. 

  Determine whether the following functions are injective or surjective.}


\begin{enumerate}

% Removed for variety
\com{
\item\question{
(3 points)
\[
\begin{array}{ccl}
f(x) &= &2x\\
A &= &\mathbb{N}\\
B &= &\mathbb{N}
\end{array}
\] }

}

\com{ %used in 2022 
\item\question{
(2 points)
\[
\begin{array}{ccl}
f(x) &= &x+1\\
A &= &\mathbb{N}\\
B &= &\mathbb{N}
\end{array}
\] }

}


\item\question{
(2 points)
\[
\begin{array}{ccl}
f(x) &= &x+2\\
A &= &\mathbb{N}\\
B &= &\mathbb{N}-\set{0, 1}
\end{array}
\] }



\item\question{
(2 points)
\[
\begin{array}{ccl}
f(x) &= &2x+3\\
A &= &\mathbb{Z}\\
B &= &\mathbb{Z}
\end{array}
\] }

\item\question{
(2 points)
\[
\begin{array}{ccl}
f(x) &= &x^2\\
A &= &\mathbb{R}\\
B &= &\mathbb{R}
\end{array}
\] }



\com{ %used in 2022 
\item\question{
(2 points)
\[
\begin{array}{ccl}
f(x) &= &x+1\\
A &= &\mathbb{Z}\\
B &= &\mathbb{Z}
\end{array}
\] }

} %end com 



\com{ %used in 2022
\item\question{
(2 points)
\[
\begin{array}{ccl}
f(x) &= &x^2\\
A &= &\mathbb{Z}\\
B &= &\mathbb{Z}
\end{array}
\] }

} %end com 




\item   \question{
(2 points)
\[
\begin{array}{ccl}
f(x) &= &x+\sqrt{2}\\
A &= &\mathbb{R}\\
B &= &\mathbb{R}
\end{array}
\] }








% Removed for variety
\com{
\item   \question{
(3 points)
\[
\begin{array}{ccl}
f(x) &= &x-2\\
A &= &\mathbb{Z}\\
B &= &\mathbb{Z}
\end{array}
\] }


}


\end{enumerate}








\item \question{A function $f:A\to B$ is a \textbf{bijection} just in case $f$ is both injective and surjective. 
  % The sets $A$ and $B$ are said to have the same \textbf{cardinality}--- i.e., $|A|=|B|$--- just in case there is a bijection from $A$ to $B$. 
  For which of the following pairs of sets is there a bijection between them?}

\begin{enumerate}

\item \question{The set of negative integers and the set of non-negative integers excluding finitely-many natural numbers (2~points.)}


\com{ %used in 2022
\item \question{The set of negative integers and the set of non-negative integers. (2~points.)}

}%end com 

\com{%used in 2022

\item \question{The set of prime numbers and the set of real numbers. (2~points.)}

}

\item \question{The set of prime numbers and the set of real numbers between 0 and 0.0001. (2~points.)}



\com{%used in 2022
\item \question{The set of prime numbers and the set of real numbers between 0 and 1. (2~points.)}

} %end com 

\item \question{The rational numbers and the set of rational numbers between 2023 and 2024. (2~points.)}


\item \question{The irrational numbers and the rational numbers (2~points.)}


\end{enumerate}



\item \question{The following principles give conflicting answers to cardinality questions:


\begin{description}
\item[The Proper Subset Principle]
Suppose $A$ is a {proper subset} of $B$. Then $A$ and $B$ are \emph{not} of the same size: $B$ has more members than $A$.
\end{description}
\begin{description}
\item[The Bijection Principle]
Set $A$ has the same size as set $B$ if and only if there is a \emph{bijection} from $A$ to $B$.
\end{description}
For each of the questions below determine which of the following answers is correct: ``yes'', ``no'', or ``not determined''.
}




\begin{enumerate}

\item \question{Are $\set{1992, 1993, 2019}$ and $\set{1992, 1993, 2019, 2024}$ of the same size, according to the Proper Subset Principle? Are they of the same size according to the Bijection Principle? (3~points.)
}




\item \question{Are $\set{0,1,2}$ and $\set{1,2,3,4}$ of the same size, according to the Proper Subset Principle? Are they of the same size according to the Bijection Principle? (3~points.)
}




\item \question{Are the set of prime numbers and the set of natural numbers of the same size, according to the Proper Subset Principle? Are they of the same size according to the Bijection Principle? (3~points.)
}


\end{enumerate}



\question{\fbox{\parbox{150mm}{\emph{Reminder:} Although you'll need to think about the Proper Subset Principle for the purposes of this question, it won't be relevant for the rest of the PSet. At least in our PSets, we follow Cantor---and current mathematical practice---in rejecting the Proper Subset Principle and assessing cardinality questions on the basis of the Bijection Principle. Food for thought: \textit{Is this merely a convention?}}}}








\item \question{
Every room in Hilbert's Hotel is occupied. New guests show up.
}

\begin{enumerate}

\com{ %used in 2022
\item There is one new guest for each rational number. Can all of them be accommodated without asking anyone to share a room? (3~points.)

} %end com 

\com{ %used in 2022
\item There is one new guest for each real number. Can all of them be accommodated without asking anyone to share a room? (3~points.)
 
} %end com 

\item You're in charge of room assignments, and initially you believe there'll be one new guest for each rational number. But then just before arrival, each new guest invites their parents! Understandably, each pair of parents would prefer to stay in a room adjacent to their child. Can all of these new guests be accommodated without asking anyone to share a room? (3~points.)


\item Initially, you think there'll be one new guest for each real number (and you start feeling a bit overwhelmed). Then you hear what seems like good news! The new guests for the real numbers from $2023$ to $2024$ have decided they have to bail. Should this change how you feel, i.e., can all of these new guests be accommodated without asking anyone to share a room? (3~points.)
 



\end{enumerate}



\com{
\item \question{
Say that a sequence of objects is \textbf{dense} if and only if between any two members of the sequence there is a third. Must a sequence have more elements than there are natural numbers in order to be dense? Consider only sequences with at least two elements. (3~points.)
}


}





\end{enumerate}





%%%%%%%%%%%%%%%
%PART II
%%%%%%%%%%%%%%%%
\subsection*{Part II (Submit PDF on Canvas: 69 points)} 
\begin{enumerate}
  \setcounter{enumi}{4}
%%%%%%%%%%%%%%%

\item \question{Describe a set that contains no integers but has the same cardinality as the set of integers. (For this one, no need to provide a justification) (3~points.)}


% Left out for variety
%\com{
\item \question{Construct a bijection between the set of integers \{\ldots -2, -1, 0, 1, 2, \ldots\} and the set of squares of integers $\{0, 1, 4, 9, 16, \dots\}$. (9 points)}


\com{
\item \question{Show that there is a bijection from $\mathbb{Z}$ to the set of powers of seven $\set{7^1,7^2,7^3,\dots}$. (4~points)}

} %end com 


\item  \question{The Cantor-Schroeder-Bernstein Theorem states that if there is an injection from $A$ to (a subset of) $B$ and an injection from $B$ to (a subset of) $A$, then there is a bijection from $A$ to $B$.}
\begin{enumerate}

\item \question{
  Construct an injection from $\mathbb{Z}$ to the set of prime numbers  $\set{2,3,5,7,\dots}$. (10~points) 
  (You may assume that whenever $p_1,\dots,p_n$ are primes, there is a smallest prime greater than each of $p_1,\dots,p_n$.)
}


\item \question{Use the Cantor-Schroeder-Bernstein Theorem to prove that there is a bijection from $\mathbb{Z}$ to the set of prime numbers. (5~points)}

\com{ %easier version used in 2022

\item \question{Show that there is an injection from $\mathbb{N}$ to the set of prime numbers  $\set{2,3,5,7,\dots}$. (5~points) 

(You may help yourself to the observation that whenever $p_1,\dots,p_n$ are primes, there is a smallest prime greater than each of $p_1,\dots,p_n$.) }


\item \question{Use the Cantor-Schroeder-Bernstein Theorem to show that there is a bijection from $\mathbb{N}$ to the set of prime numbers. (5~points)}


} %end com 

\end{enumerate}



%Left out for variety; 2023: using in lecture as practice problem 
\com{

\item \question{Show that there is a bijection from $\mathbb{N}$ to $\set{\seq{n,m} : n,m \in \mathbb{N}}$, which is the set of pairs of natural numbers. No need to explicitly construct the bijection. (9~points)}

}




%%% QUESTION LEFT OUT TO ADD VARIETY
\com{
\item \question{Is there a bijection between the natural numbers and the set of functions from natural numbers to natural numbers? (15 points)}

} %end com 



\item \question{
Let $S=\set{f : f \text{ is a function from the natural numbers to $\set{\Earth,\Sun,\Moon}$}}$
where $\Earth$ is the Earth, $\Sun$ is the Sun, and $\Moon$ is the Moon.
An example of a member of $S$ is the function $g: \mathbb{N} \rightarrow \set{\Earth,\Sun,\Moon}$ such that: 
  \[
    g(n) = 
      \begin{cases}
        \Earth \text{ if $n$ is a power of seven}\\ 
        \Sun \text{ otherwise}
      \end{cases}
  \]
Prove that there cannot be a bijection from the set of natural numbers to $S$. (10~points)
}

%Earlier answer but evil twin function seems less clear: Now consider the `diagonal' sequence $<g_0(0),g_1(1),\ldots>$, and construct its evil twin $<t(g_0(0)),t(g_1(1)),\ldots>$, where $t(x) = \Sun$ if $x \neq \Sun$, and $t(x) = \Earth$ if $x = \Earth$. The evil twin function $h(n)=t(g_n(n))$ is a function from natural numbers to $S$. But $h \neq g_m$ for every $m$, since $g_m(m) \neq t(g_m(m)) = h(m)$. This contradicts our earlier claim that  $S = \set{g_n : n \in \mathbb{N}}$.}


%\item

%\question{The unit cube $C$ is the set of triples $\seq{r,p,q}$, for $r,p,q \in [0,1]$. A point $\seq{r,p,q}$ in $C$ is said to be ``rational'' if each of $r,p,$ and $q$ is rational.}

%\begin{enumerate}

% For next time: people find part (a) confusing (they assume that C consists of only rational points), and the question is pretty close to something explicitly used in lecture. Best to replace (or eliminate) part (a) for next time.

%\item \question{Is there a bijection from the set of points in $C$ to the set of real numbers? (8~points; don't forget to justify your answer)}



%\item \question{Is there a bijection from the set of rational points in $C$ to the set of rational numbers? (8~points; don't forget to justify your answer)}


%From a problem above, we know that there is a bijection $f$ from $\mathbb{N}\times\mathbb{N}$ to  $\mathbb{N}$. We can then define a bijection $g$ from $\mathbb{N}\times\mathbb{N} \times \mathbb{N}$ to  $\mathbb{N}$ as follows:
%$$g(\seq{n,m,l}) = f(f(n,m),l)$$
%Suitable variations of the diagonal construction can then be used to show that there is a bijection $h$ from $[0,1] \cap \mathbb{Q}$ to $\mathbb{N}$, and a bijection $i$ from $\mathbb{N}$ to $\mathbb{Q}$.

%We can then define a conjunction $j$ from the set of rational points in $C$ to $\mathbb{Q}$. For $\seq{r,p,q}$ a rational point in $C$, define $j$ as follows:
%$$
%j(\seq{r,p,q}) = i(g(\seq{h(r),h(p),h(q)}))
%$$




%}




%\end{enumerate}



\item \question{
Consider the following infinite tree:

\Tree [.  [.0 [.0 [.0 0 1 ] [.1 0 1 ] ] [.1 [.0 0 1 ] [.1 0 1 ] ] ] [.1 [.0 [.0 0 1 ] [.1 0 1 ] ] [.1 [.0 0 1 ] [.1 0 1 ] ] ] ]

\begin{center} \ \vdots \end{center}

(When fully spelled out, the tree contains one row for each natural number. The zero-th row contains one node, the first row contains two nodes, the second row contains four nodes, and, in general, the $n$th row contains $2^n$ nodes.)}

\begin{enumerate}

\item \question {Is there a bijection between the \emph{nodes} of this tree and the natural numbers? Don't forget to justify your answer! (10~points)}


\item \question{Is there a bijection between the \emph{paths} of this tree and the natural numbers? A path is an infinite sequence of nodes which starts at the top of the tree and contains a node at every row, with each node connected to its successor by an edge. (Paths can be represented as infinite sequences of zeroes and ones.)   (10~points; don't forget to justify your answers!)}

\end{enumerate}

\item \question{Let $\mathscr{P}^\star(A)$ be the set of \textbf{non-empty} subsets of $A$.}

\begin{enumerate}

\item \question{Show that the following analogue of Cantor's Theorem is false: for any set $A$, $|A| < |\mathscr{P}^\star(A)|$. (2~points)}


\item \question{Show that the following analogue of Cantor's Theorem is true: for any set $A$ with two or more members, $|A| < |\mathscr{P}^\star(A)|$. (10~points)}

\question{\emph{Notation:} $|A|$ is the cardinality of $A$ and $|A| < |B|$ means that there is an injection from $A$ to $B$ but no bijection from $A$ to $B$.}


\end{enumerate}


\end{enumerate}



%Endnotes
%\newpage \begingroup \parindent 0pt \parskip 2ex \def\enotesize{\normalsize} \theendnotes \endgroup


%\newpage\bibliographystyle{linquiry}\bibliography{agustin}

%\end{doublespace}
\end{document}

