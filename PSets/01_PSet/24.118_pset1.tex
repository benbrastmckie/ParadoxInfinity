
\documentclass[12pt,letterpaper]{article}
\usepackage{../pset_2024}
%\usepackage{wasysym}
%\usepackage{marvosym}
\usepackage{mathabx}

%points currently add to 86, which seems dumb. part I on canvas is 31 points. so need to add 14 points to problems below. 

%Question idea analog for Question 9: prove that the cardinality of the algebraic numbers is countable. Note that Cantor did this (for the real algebraic numbers) in 1874. So of historical interest. Argument is very similar to the one that works for the infinite tree: define the height of a polynomial, and argue (from fundamental theorem of algebra) that for each height, there are a finite number of zeros. Associate each root with the polynomial of least height it arises for. Then we have an infinite rung of heights where on each rung there are finitely-many roots. 
% See this video: https://www.youtube.com/watch?v=pSSsZLTMDq0

%Questions and Answers
\qa{q} % q ="questions only"; a="answers only";  b="both"
\usepackage{../qa}


\begin{document}

\psintro{Problem Set 1: Infinite Cardinalities}

\newpage

% Idea for next time: ask them to show that Cantor's Theorem shows that the powerset axiom is incompatible with the existence of a universal set.



\subsection*{Part I (Quiz on Canvas: 31 points)} 


\begin{enumerate}

\com{
\item 
\question{Let \(A\) be a set and let \(R\) be a relation that holds amongst members of that set. Then: 
  \begin{itemize}
\item   
   \(R\) is \textbf{reflexive} on \(A\) if and only if: for any \(a\in A\), \(aRa\). 
      \item  \(R\) is \textbf{symmetric} on \(A\) if and only if: for any \(a,b\in A\), if \(aRb\) then \(bRa\). 
  \item  \(R\) is \textbf{transitive} on \(A\) if and only if: for any \(a,b,c\in A\), if \(aRb\) and \(bRc\) then \(aRc\). 
\end{itemize}
(\emph{Example:} Let \(A\) be the set of people, and let \(R\) be the relation ``having the same birthday as". \(R\) is reflexive, since everyone has the same birthday as herself. \(R\) is symmetric, since whenever \(a\) has the same birthday as \(b\), \(b\) will have the same birthday as \(a\). And \(R\) is transitive: for any three people \(a\), \(b\), and \(c\), if \(a\) has the same birthday as \(b\), and \(b\) has the same birthday as \(c\), then \(a\) has the same birthday as \(c\).)}

\begin{enumerate}
\item  \question{Consider the relation ``is less than or equal to". Is this relation: $(i)$ reflexive, $(ii)$ symmetric, and $(iii)$ transitive on the set of natural numbers? (5 points.)}

\answer{ ``Less than or equal to" is reflexive on the set of natural numbers, since every number is less than or equal to itself. It is also transitive on the natural numbers, since if \(x\) is less than or equal to \(y\), and \(y\) is less than or equal to \(z\), \(x\) is less than or equal to \(z\). It is not, however, symmetric on the natural numbers, since if \(x\) is less than or equal to \(y\), there is no guarantee that \(y\) is less than or equal to \(x\). That is true only in the special case where \(x\) is equal to \(y\).}
  
 \item    \question{There are ten guests at a dinner party, sitting around a large table. Consider the relation \(R\) such that guest \(a\) bears \(R\) to guest \(b\) if and only if \(a\)'s seat is immediately adjacent to \(b\)'s seat. Is \(R\): $(i)$ reflexive, $(ii)$ symmetric, and $(iii)$ transitive on the set of guests?  (Keep in mind that no seat is adjacent to itself.) (5 points.)}
 
 \answer{
 \(R\) fails to be reflexive because nobody is in a seat immediately adjacent to her own seat (your seat is only immediately adjacent to your neighbor's seats).

\(R\) also fails to be transitive; to see this, suppose that \(a\) is seated immediately to the right of \(b\) and that \(b\) is seated immediately to the right of \(c\); then \(a\) bears \(R\) to \(b\) and \(b\) bears \(R\) to \(c\), but \(a\) doesn't bear \(R\) to \(c\).

On the other hand, \(R\) is symmetric, since the only way for \(a\) to be in a seat immediately adjacent to \(b\) is for \(b\) to be in a seat immediately adjacent to \(a\). 
 }


\end{enumerate}
}



\item \question{
  A \textbf{relation} with \textbf{domain} $A$ and \textbf{range} $B$ is any set of ordered pairs $\tuple{a,b}$ where $a\in A$ and $b\in B$, i.e., any subset of all such pairs $A\times B=\set{\tuple{a,b}:a\in A, b\in B}$.

  A relation is a \textbf{function} $f : A \to B$ just in case every element in $A$ is assigned to exactly one element of $B$, i.e., if $f(x)=f(y)$ whenever $x=y$.

  A function $f: A \to B$ is an \textbf{injection} just in case no two elements in $A$ are assigned to the same element in $B$, i.e., if $x=y$ whenever $f(x) = f(y)$.

  A function $f: A \to B$ is a \textbf{surjection} just in case for every element $b\in B$ there is some $a\in A$ where $f(a)=b$. 

  Determine whether the following functions are injective or surjective.}


\begin{enumerate}

% Removed for variety
\com{
\item\question{
(3 points)
\[
\begin{array}{ccl}
f(x) &= &2x\\
A &= &\mathbb{N}\\
B &= &\mathbb{N}
\end{array}
\] }

  \answer{
    The function is injective, since no two elements of \(A\) are mapped to the same element of $B$.
    It is not surjective, since not every element of \(B\) is hit by an element of $A$.
    For instance, there is no number in set $A$ whose square is 69 but 69 is an element of set \(B\).
  }
}

\com{ %used in 2022 
\item\question{
(2 points)
\[
\begin{array}{ccl}
f(x) &= &x+1\\
A &= &\mathbb{N}\\
B &= &\mathbb{N}
\end{array}
\] }

\answer{The function is injective, since no two elements of \(A\) are mapped to the same element of $B$. It is not surjective, since zero is in \(B\) but is not hit by an element of $A$.}
}


\item\question{
(2 points)
\[
\begin{array}{ccl}
f(x) &= &x+2\\
A &= &\mathbb{N}\\
B &= &\mathbb{N}-\set{0, 1}
\end{array}
\] }

\answer{The function is injective, since no two elements of \(A\) are mapped to the same element of $B$. It is also surjective, since every element of \(B\) is hit by an element of $A$ (if we didn't exclude 0 and 1 in \(B\), then 0 and 1 would not be the image of any member of \(A\))}


\item\question{
(2 points)
\[
\begin{array}{ccl}
f(x) &= &2x+3\\
A &= &\mathbb{Z}\\
B &= &\mathbb{Z}
\end{array}
\] }

\answer{The function is injective, since no two elements of \(A\) are mapped to the same element of $B$. But it is NOT surjective: note that $b = 2 (b/2 - 3/2) +3 = f (b/2 -3/2)$. But for an arbitrary b in $\mathbb{Z}$, we don't have $b/2 -3/2 \in \mathbb{Z}$. For example, let b be an even number, e.g. $b=2$. 
} 

\item\question{
(2 points)
\[
\begin{array}{ccl}
f(x) &= &x^2\\
A &= &\mathbb{R}\\
B &= &\mathbb{R}
\end{array}
\] }

\answer{The function is not injective, since, e.g.~$-1$ and $1$ are both mapped to $1$ (the function is symmetric on the y-axis). It is not surjective, since $-1$ is in \(B\) but is not hit by an element of $A$.}

%It IS surjective, since for arbitrary $b \in B$, we have $\sqrt{b} \in A = \mathbb{R}$ is mapped to $b$

\com{ %used in 2022 
\item\question{
(2 points)
\[
\begin{array}{ccl}
f(x) &= &x+1\\
A &= &\mathbb{Z}\\
B &= &\mathbb{Z}
\end{array}
\] }

\answer{The function is injective, since no two elements of \(A\) are mapped to the same element of $B$. It is also surjective, since every element of \(B\) is hit by an element of $A$.}
} %end com 



\com{ %used in 2022
\item\question{
(2 points)
\[
\begin{array}{ccl}
f(x) &= &x^2\\
A &= &\mathbb{Z}\\
B &= &\mathbb{Z}
\end{array}
\] }

\answer{The function is not injective, since, e.g.~$-1$ and $1$ are both mapped to $1$. It is not surjective, since $-1$ is in \(B\) but is not hit by an element of $A$.}
} %end com 




\item   \question{
(2 points)
\[
\begin{array}{ccl}
f(x) &= &x+\sqrt{2}\\
A &= &\mathbb{R}\\
B &= &\mathbb{R}
\end{array}
\] }


\answer{ The function is injective and surjective. Assume that $f(x) = f(y)$, then $x + \sqrt{2} = y + \sqrt{2}$, which entails that $x = y$, showing that the function is injective. To show it is surjective, consider an arbitrary $b \in  B$ and note that $b= (b - \sqrt{2}) + \sqrt{2}$. So $(b - \sqrt{2}) \in A$ is mapped to $b$.}






% Removed for variety
\com{
\item   \question{
(3 points)
\[
\begin{array}{ccl}
f(x) &= &x-2\\
A &= &\mathbb{Z}\\
B &= &\mathbb{Z}
\end{array}
\] }


\answer{ The function is injective and surjective. Every element of set \(A\) is paired with exactly one element of set \(B\), and every element of set \(B\) is paired with exactly one element of set \(A\).}
}


\end{enumerate}


% \question{\emph{Notation:} $\mathbb{N} = \{0,1,2,\dots\}$ is the set of natural numbers, $\mathbb{Z} = \{\dots -2,-1,0,1,2,\dots\}$ is the set of integers, $\mathbb{Q}$ is the set of  and $\mathbb{R}$ is the set of real numbers.}






\item \question{A function $f:A\to B$ is a \textbf{bijection} just in case $f$ is both injective and surjective. 
  % The sets $A$ and $B$ are said to have the same \textbf{cardinality}--- i.e., $|A|=|B|$--- just in case there is a bijection from $A$ to $B$. 
  For which of the following pairs of sets is there a bijection between them?}

\begin{enumerate}

\item \question{The set of negative integers and the set of non-negative integers excluding finitely-many natural numbers (2~points.)}

\answer{Yes: removing finitely-many natural numbers from $\mathbb{N}$ results in a countably-infinite set.}

\com{ %used in 2022
\item \question{The set of negative integers and the set of non-negative integers. (2~points.)}

\answer{Yes.}
}%end com 

\com{%used in 2022

\item \question{The set of prime numbers and the set of real numbers. (2~points.)}

\answer{No.}
}

\item \question{The set of prime numbers and the set of real numbers between 0 and 0.0001. (2~points.)}

\answer{No.}


\com{%used in 2022
\item \question{The set of prime numbers and the set of real numbers between 0 and 1. (2~points.)}

\answer{No.}
} %end com 

\item \question{The rational numbers and the set of rational numbers between 2023 and 2024. (2~points.)}

\answer{Yes.}

\item \question{The irrational numbers and the rational numbers (2~points.)}

\answer{No: note that the reals are the union of rationals and irrationals. Since the rationals are countable, the irrationals must be uncountable.}

\end{enumerate}



\item \question{The following principles give conflicting answers to cardinality questions:


\begin{description}
\item[The Proper Subset Principle]
Suppose $A$ is a {proper subset} of $B$. Then $A$ and $B$ are \emph{not} of the same size: $B$ has more members than $A$.
\end{description}
\begin{description}
\item[The Bijection Principle]
Set $A$ has the same size as set $B$ if and only if there is a \emph{bijection} from $A$ to $B$.
\end{description}
For each of the questions below determine which of the following answers is correct: ``yes'', ``no'', or ``not determined''.
}




\begin{enumerate}

\item \question{Are $\set{1992, 1993, 2019}$ and $\set{1992, 1993, 2019, 2024}$ of the same size, according to the Proper Subset Principle? Are they of the same size according to the Bijection Principle? (3~points.)
}

\answer{No: $\set{1992, 1993, 2019} \subsetneq \set{1992, 1993, 2019, 2024}$

No: there is no bijection.}



\item \question{Are $\set{0,1,2}$ and $\set{1,2,3,4}$ of the same size, according to the Proper Subset Principle? Are they of the same size according to the Bijection Principle? (3~points.)
}

\answer{Answer not determined by the Proper Subset Principle, since neither set is a proper subset of the other.

No: there is no bijection.}



\item \question{Are the set of prime numbers and the set of natural numbers of the same size, according to the Proper Subset Principle? Are they of the same size according to the Bijection Principle? (3~points.)
}

\answer{No: $\set{n : n \text{ is prime}} \subsetneq \mathbb{N}$

Yes: there is a bijection.}

\end{enumerate}



\question{\fbox{\parbox{150mm}{\emph{Reminder:} Although you'll need to think about the Proper Subset Principle for the purposes of this question, it won't be relevant for the rest of the PSet. At least in our PSets, we follow Cantor---and current mathematical practice---in rejecting the Proper Subset Principle and assessing cardinality questions on the basis of the Bijection Principle. Food for thought: \textit{Is this merely a convention?}}}}








\item \question{
Every room in Hilbert's Hotel is occupied. New guests show up.
}

\begin{enumerate}

\com{ %used in 2022
\item There is one new guest for each rational number. Can all of them be accommodated without asking anyone to share a room? (3~points.)

\answer{Yes. There are countably many rational numbers, and countably many spaces can be freed up in Hilbert's Hotel by asking guest $n$ to move to room $2n$.}
} %end com 

\com{ %used in 2022
\item There is one new guest for each real number. Can all of them be accommodated without asking anyone to share a room? (3~points.)
 
\answer{No. There are only countably many hotel rooms. If we could fit uncountably many guests in the hotel, then there would be a bijection between a countable set (the set of hotel rooms) and an uncountable set (the set of guests). But there is no such bijection. So an uncountable number of guests will not fit.}
} %end com 

\item You're in charge of room assignments, and initially you believe there'll be one new guest for each rational number. But then just before arrival, each new guest invites their parents! Understandably, each pair of parents would prefer to stay in a room adjacent to their child. Can all of these new guests be accommodated without asking anyone to share a room? (3~points.)

\answer{Yes. There are countably many rational numbers, and the union of two countable sets is countable. Moreover, countably many spaces can be freed up in Hilbert's Hotel by asking guest $n$ to move to room $3n$, so that we can put each new guest who is a child in a room adjacent to their parents.}

\item Initially, you think there'll be one new guest for each real number (and you start feeling a bit overwhelmed). Then you hear what seems like good news! The new guests for the real numbers from $2023$ to $2024$ have decided they have to bail. Should this change how you feel, i.e., can all of these new guests be accommodated without asking anyone to share a room? (3~points.)
 

\answer{No. $\mathbb{R} - [2023, 2024]$ remains an uncountable set. There are only countably many hotel rooms. If we could fit uncountably many guests in the hotel, then there would be a bijection between a countable set (the set of hotel rooms) and an uncountable set (the set of guests). But there is no such bijection. So an uncountable number of guests will not fit.}


\end{enumerate}



\com{
\item \question{
Say that a sequence of objects is \textbf{dense} if and only if between any two members of the sequence there is a third. Must a sequence have more elements than there are natural numbers in order to be dense? Consider only sequences with at least two elements. (3~points.)
}

\answer{No. The sequence of rational numbers is dense, but there are as many rational numbers as natural numbers, since a bijection can be constructed between the rational numbers and the natural numbers.}

}





\end{enumerate}





%%%%%%%%%%%%%%%
%PART II
%%%%%%%%%%%%%%%%
\subsection*{Part II (Submit PDF on Canvas: 69 points)} 
\begin{enumerate}
  \setcounter{enumi}{4}
%%%%%%%%%%%%%%%

\item \question{Describe a set that contains no integers but has the same cardinality as the set of integers. (For this one, no need to provide a justification) (3~points.)}

\answer{There are many such sets. Here is one example: $\{x+\pi : x \in \mathbb{N} \}$. The set of `half-intgers' provides another:\\ $\set{ \dots -5/2, -3/2, -1/2, 1/2, 3/2, 5/2, \dots } = \{ x + 1/2 : x \mathbb{Z} \}$. }

% Left out for variety
%\com{
\item \question{Construct a bijection between the set of integers \{\ldots -2, -1, 0, 1, 2, \ldots\} and the set of squares of integers $\{0, 1, 4, 9, 16, \dots\}$. (9 points)}

\answer{
Recall that $f(n) = n^2$ is a bijection from $\mathbb{N}$ to their squares. Hence, if we can construct a bijection from the integers to the naturals, then we can compose these two bijections and we are done. 

One such bijection $d$ maps each non-negative integer $n$ to an even number $2n$, and each negative integer $z$ to an odd number $(-2z -1)$. Note that $d$ is well-defined since if $n\geq 0$, then $2n \in \mathbb{N}$ and likewise if $z < 0$, then $(-2z -1)$ is a positive odd number so in $\mathbb{N}$ as well. 

\textit{Injective}: assume $d(x) = d(y)$. Then $2x= 2y$ or $(-2x -1) = (-2y -1)$ (since we can't have an even number equal an odd). In either case, simple algebra shows that $x=y$, so the $d$ is injective. 

\textit{Surjective}: consider an arbitrary $b \in \mathbb{N}$. Note that $b$ is either even or odd. If $b$ is even, then $b = 2(b/2) = d(b/2)$ where $b/2 \in \mathbb{Z}$. If $b$ is odd, then $b = -2(-b/2 - 1/2) - 1 = d(-b/2 - 1/2)$ where $-b/2 - 1/2 \in \mathbb{Z}$.

Hence, the function $f(d(z))$ is a bijection from $\mathbb{Z}$ to the squares. 

Alternatively, we could rely on a bijection $g$ from the natural numbers to the integers:

\begin{itemize}

\item[] $g(n) = n/2$ (if $n$ is even)

\item[] $g(n) = -(n+1)/2$ (if $n$ is odd)

\end{itemize}
So $f(g^{-1}(n))$ is a bijection from the integers to their squares.
%$f(n) = n^2$ is a bijection from the natural numbers to their squares.
}


\com{
\item \question{Show that there is a bijection from $\mathbb{Z}$ to the set of powers of seven $\set{7^1,7^2,7^3,\dots}$. (4~points)}

\answer{$f(n) = 7^{n+1}$ is a bijection from the natural numbers to $\set{7^1,7^2,7^3,\dots}$. And $g$ is a bijection from the natural numbers to the integers:

\begin{itemize}

\item[] $g(n) = n/2$ (if $n$ is even)

\item[] $g(n) = -(n+1)/2$ (if $n$ is odd)

\end{itemize}
So $f(g^{-1}(n))$ is a bijection from the integers to $\set{7^1,7^2,7^3,\dots}$.
}
} %end com 


\item  \question{The Cantor-Schroeder-Bernstein Theorem states that if there is an injection from $A$ to (a subset of) $B$ and an injection from $B$ to (a subset of) $A$, then there is a bijection from $A$ to $B$.}
\begin{enumerate}

\item \question{
  Construct an injection from $\mathbb{Z}$ to the set of prime numbers  $\set{2,3,5,7,\dots}$. (10~points) 
  (You may assume that whenever $p_1,\dots,p_n$ are primes, there is a smallest prime greater than each of $p_1,\dots,p_n$.)
}

\answer{Solution going through the naturals as an intermediary: in lecture, we constructed both (i) an injection from the integers to the naturals and (ii) an injection from the naturals to the primes. Just compose these injections to have an injection from the integers to the primes (one could of course proceed directly). 

\textit{Injection from} $\mathbb{Z}$ to $\mathbb{N}$: rely on the map $d$ defined in the previous problem, sending negative integers to odd naturals and non-negative integers to even naturals.  

\textit{Injection from} $\mathbb{N}$ to the primes: by the given fact, we can index the primes using natural numbers for indices: $2_0, 3_1, 5_2, 7_3, \dots$. Define $f(n) = p_n$ for $n\in  \mathbb{N}$. 

We hopefully proved this is an injection in lecture: assume $f(n) = f(m)$, then $p_n = p_m$, so we must have $n = m$ since each prime has a unique index (e.g. by induction below). 

Hence, $f(d(z))$ is an injection from $\mathbb{Z}$ to the primes. \\[1ex]


\textit{Stuff on induction they don't need to mention; may discuss in lecture}: Let $f(0) = 2$ and, for any $n \in \mathbb{N}$, let $f(n+1)$ be the smallest prime greater than $f(0), \dots, f(n)$. We can use mathematical induction to verify that $f$ is well-defined for each $n \in \mathbb{N}$. (The base case is trivial; the inductive case follows from the observation that whenever $f(0),\dots,f(n)$ are primes, there is a smallest prime greater than $f(0),\dots,f(n)$.) 

To verify that $f$ is an injection from $\mathbb{N}$ to the set of prime numbers, we need to show that $f(n)$ is always a prime number and that $f(n) \neq f(m)$ whenever $n \neq m$. The former is trivial; the latter follows immediately from the observation that the definition of  $f$ guarantees that $f(n) < f(m)$ whenever $n < m$.
}

\item \question{Use the Cantor-Schroeder-Bernstein Theorem to prove that there is a bijection from $\mathbb{Z}$ to the set of prime numbers. (5~points)}

\answer{
The identity function is an injection from the prime numbers to the integers. And it follows from part $(a)$ that there is an injection from the integers to the prime numbers. So, by the CSB Theorem, there is a bijection from the integers to the prime numbers.
}


\com{ %easier version used in 2022

\item \question{Show that there is an injection from $\mathbb{N}$ to the set of prime numbers  $\set{2,3,5,7,\dots}$. (5~points) 

(You may help yourself to the observation that whenever $p_1,\dots,p_n$ are primes, there is a smallest prime greater than each of $p_1,\dots,p_n$.) }

\answer{Let $f(0) = 2$ and, for any $n \in \mathbb{N}$, let $f(n+1)$ be the smallest primer greater than $f(0), \dots, f(n)$. We can use mathematical induction to verify that $f$ is well-defined for each $n \in \mathbb{N}$. (The base case is trivial; the inductive case follows from the observation that whenever $f(0),\dots,f(n)$ are primes, there is a smallest prime greater than $f(0),\dots,f(n)$.) 

To verify that $f$ is an injection from $\mathbb{N}$ to the set of prime numbers, we need to show that $f(n)$ is always a prime number and that $f(n) \neq f(m)$ whenever $n \neq m$. The former is trivial; the latter follows immediately from the observation that the definition of  $f$ guarantees that $f(n) < f(m)$ whenever $n < m$.
}

\item \question{Use the Cantor-Schroeder-Bernstein Theorem to show that there is a bijection from $\mathbb{N}$ to the set of prime numbers. (5~points)}

\answer{
The identity function is an injection from the prime numbers to the natural numbers. And it follows from part $(a)$ that there is an injection from the natural numbers to the prime numbers. So, by the Cantor-Schroeder-Bernstein Theorem, there is a bijection from the natural numbers to the prime numbers.
}

} %end com 

\end{enumerate}



%Left out for variety; 2023: using in lecture as practice problem 
\com{

\item \question{Show that there is a bijection from $\mathbb{N}$ to $\set{\seq{n,m} : n,m \in \mathbb{N}}$, which is the set of pairs of natural numbers. No need to explicitly construct the bijection. (9~points)}

\answer{Use the same construction that was used to show that the natural numbers are in one-one correspondence with the rational numbers. In other words: arrange the pairs $\seq{n,m}$ in a table like that for the rationals, where $n$ increases rightward on the horizontal axis and $m$ increases downward on the vertical axis. Then count diagonally across the table just as we do to count the rationals. This gives us a bijection from the natural numbers to the set of pairs of natural numbers.}

}




%%% QUESTION LEFT OUT TO ADD VARIETY
\com{
\item \question{Is there a bijection between the natural numbers and the set of functions from natural numbers to natural numbers? (15 points)}

\answer{No. This is just a version of Cantor's proof that there is no bijection between the natural numbers and [0,1]. Ideally, the student would reproduce some version of that proof. }
} %end com 



\item \question{
Let $S=\set{f : f \text{ is a function from the natural numbers to $\set{\Earth,\Sun,\Moon}$}}$
where $\Earth$ is the Earth, $\Sun$ is the Sun, and $\Moon$ is the Moon.
An example of a member of $S$ is the function $g: \mathbb{N} \rightarrow \set{\Earth,\Sun,\Moon}$ such that: 
  \[
    g(n) = 
      \begin{cases}
        \Earth \text{ if $n$ is a power of seven}\\ 
        \Sun \text{ otherwise}
      \end{cases}
  \]
Prove that there cannot be a bijection from the set of natural numbers to $S$. (10~points)


}

\answer{One can use a version of the construction that we used to show that there are more real numbers than natural numbers.

Assume for \emph{reductio} that there is a bijection $f$ from the natural numbers to $S$, and let $f(n)$ be the function $g_n: \mathbb{N} \rightarrow \set{\Earth,\Sun,\Moon}$. (Since $f$ is a bijection, this means that $S = \set{g_n : n \in \mathbb{N}}$.)

Now consider the `diagonal' sequence $<g_0(0),g_1(1),\ldots>$, and construct its evil twin $<t(g_0(0)),t(g_1(1)),\ldots>$, where $t(x) = \Sun$ if $x = \Earth$,  $t(x) = \Moon$ if $x = \Sun$, and $t(x) = \Earth$ if $x = \Moon$. The evil twin function $h(n)=t(g_n(n))$ is a function from natural numbers to $S$. But $h \neq g_m$ for every $m$, since $g_m(m) \neq t(g_m(m)) = h(m)$. This contradicts our earlier claim that  $S = \set{g_n : n \in \mathbb{N}}$.}

%Earlier answer but evil twin function seems less clear: Now consider the `diagonal' sequence $<g_0(0),g_1(1),\ldots>$, and construct its evil twin $<t(g_0(0)),t(g_1(1)),\ldots>$, where $t(x) = \Sun$ if $x \neq \Sun$, and $t(x) = \Earth$ if $x = \Earth$. The evil twin function $h(n)=t(g_n(n))$ is a function from natural numbers to $S$. But $h \neq g_m$ for every $m$, since $g_m(m) \neq t(g_m(m)) = h(m)$. This contradicts our earlier claim that  $S = \set{g_n : n \in \mathbb{N}}$.}


%\item

%\question{The unit cube $C$ is the set of triples $\seq{r,p,q}$, for $r,p,q \in [0,1]$. A point $\seq{r,p,q}$ in $C$ is said to be ``rational'' if each of $r,p,$ and $q$ is rational.}

%\begin{enumerate}

% For next time: people find part (a) confusing (they assume that C consists of only rational points), and the question is pretty close to something explicitly used in lecture. Best to replace (or eliminate) part (a) for next time.

%\item \question{Is there a bijection from the set of points in $C$ to the set of real numbers? (8~points; don't forget to justify your answer)}

%\answer{Yes. By exercise 3 of section 1.6.3, there is a bijection from $\mathbb{R}$ to $[0,1]$ and by exercise 6 of section 1.6.3, there is a bijection from $[0,1]$ to $C$. The result follows from transitivity.}


%\item \question{Is there a bijection from the set of rational points in $C$ to the set of rational numbers? (8~points; don't forget to justify your answer)}

%\answer{Yes. 

%From a problem above, we know that there is a bijection $f$ from $\mathbb{N}\times\mathbb{N}$ to  $\mathbb{N}$. We can then define a bijection $g$ from $\mathbb{N}\times\mathbb{N} \times \mathbb{N}$ to  $\mathbb{N}$ as follows:
%$$g(\seq{n,m,l}) = f(f(n,m),l)$$
%Suitable variations of the diagonal construction can then be used to show that there is a bijection $h$ from $[0,1] \cap \mathbb{Q}$ to $\mathbb{N}$, and a bijection $i$ from $\mathbb{N}$ to $\mathbb{Q}$.

%We can then define a conjunction $j$ from the set of rational points in $C$ to $\mathbb{Q}$. For $\seq{r,p,q}$ a rational point in $C$, define $j$ as follows:
%$$
%j(\seq{r,p,q}) = i(g(\seq{h(r),h(p),h(q)}))
%$$




%}




%\end{enumerate}



\item \question{
Consider the following infinite tree:

\Tree [.  [.0 [.0 [.0 0 1 ] [.1 0 1 ] ] [.1 [.0 0 1 ] [.1 0 1 ] ] ] [.1 [.0 [.0 0 1 ] [.1 0 1 ] ] [.1 [.0 0 1 ] [.1 0 1 ] ] ] ]

\begin{center} \ \vdots \end{center}

(When fully spelled out, the tree contains one row for each natural number. The zero-th row contains one node, the first row contains two nodes, the second row contains four nodes, and, in general, the $n$th row contains $2^n$ nodes.)}

\begin{enumerate}

\item \question {Is there a bijection between the \emph{nodes} of this tree and the natural numbers? Don't forget to justify your answer! (10~points)}

\answer{There are as many nodes as natural numbers. One way to see this is to note that each node can be arrived at as a result of taking a finite number of 0-1 decisions as one travels down on the tree. Accordingly, each node corresponds to a unique finite sequence of zeroes and ones. 

So, we need to verify that there is a bijection between the set $B = \set{s : \text{$s$ is a finite binary sequence}}$ and $\mathbb{N}$. But one can generate such a bijection by enumerating the members of $B$:
$$
\seq{}, \seq{0}, \seq{1}, \seq{0,0}, \seq{0,1}, \seq{1,0}, \seq{1,1}, \seq{0,0,0},\dots
$$
and assigning the number $n$ to the $n$-th member of the enumeration. (Think of the enumeration as having a zero-th member.)
}

\item \question{Is there a bijection between the \emph{paths} of this tree and the natural numbers? A path is an infinite sequence of nodes which starts at the top of the tree and contains a node at every row, with each node connected to its successor by an edge. (Paths can be represented as infinite sequences of zeroes and ones.)   (10~points; don't forget to justify your answers!)}

\answer{There are more paths than natural numbers. One way to see this is by noting that each path can be arrived at as a the result of taking infinitely many 0-1 decisions as one travels down the tree. So each node corresponds to an infinite sequence of zeroes and ones, or, more precisely, to a function from the natural numbers to the set $\{0,1\}$. But there is a bijection between the set of such functions and $\powerset(\mathbb{N})$, which we showed in the lecture notes to have the same cardinality as the real numbers.
}
\end{enumerate}

\item \question{Let $\mathscr{P}^\star(A)$ be the set of \textbf{non-empty} subsets of $A$.}

\begin{enumerate}

\item \question{Show that the following analogue of Cantor's Theorem is false: for any set $A$, $|A| < |\mathscr{P}^\star(A)|$. (2~points)}

\answer{$\set{0}$ has exactly one member and exactly one non-empty subset.}

\item \question{Show that the following analogue of Cantor's Theorem is true: for any set $A$ with two or more members, $|A| < |\mathscr{P}^\star(A)|$. (10~points)}

\question{\emph{Notation:} $|A|$ is the cardinality of $A$ and $|A| < |B|$ means that there is an injection from $A$ to $B$ but no bijection from $A$ to $B$.}

\answer{
Consider the map $f$ that takes each member $a \in A$ to the one-element subset $\{a \} \in \mathscr{P}^\star(A)$, i.e. $f(a) = \{a \}$. Clearly, $f$ defines an injection between $A$ and $\mathscr{P}^\star(A)$. Hence, by the Injection Principle, we know that $|A| \leq |\mathscr{P}^\star(A)|$. 

To show that $|A|$ is strictly less than $|\mathscr{P}^\star(A)|$, assume for contradiction that there is an injection $h$ from $\mathscr{P}^\star(A)$ to $A$. 

Next, consider the set $D$ of elements of $A$ in the image of $h$ that are mapped from subsets that do not contain them, e.g. elements $a$ where $h(\{b, c \}) = a$. We can write this set $D$ as follows: \\
$D := \{ a \in A : \exists C \in \mathscr{P}^\star(A)$ such that $h(C) = a$ and $a \not\in C) \}$. We'll use $D$ to derive a contradiction. 

First, note that $D$ is not empty. Suppose otherwise. Then $D$ is empty, so, in particular, $h(\set{a}) = a$ for each $a \in A$. But $A$ has at least two members, $a$ and $b$ ($a \neq b$). And, since $h$ is injective, the fact that $h(\set{a}) = a$ and $h(\set{b}) = b$ entails that we can have neither $ h(\set{a,b}) = a $ nor $h(\set{a,b}) = b$. So $h(\set{a,b}) \in D$, contradicting our hypothesis.

Second, since $D$ is itself a non-empty subset of $A$, $D \in \mathscr{P}^\star(A)$. So we can consider where $h$ maps $D$ in $A$. In particular, either $h(D) \in D$ or $h(D) \notin D$. In either case, we'll derive a contradiction. 

Case 1: Assume that $h(D) \in D$. Then by definition of $D$, there exists some $C \in \mathscr{P}^\star(A)$ such that $h(C) = h(D)$ where $h(D) \notin C$. Since $h$ is an injection, $h(C) = h(D)$ entails that $C = D$, so that $h(D) \notin C$ entails  $h(D) \notin D$. This contradicts our assumption at the start of this case. 

Case 2: Assume that $h(D) \notin D$. Then by the definition of the set $D$, there must not exist \textbf{any} $C \in \mathscr{P}^\star(A)$ such that $h(C) = h(D)$ where $h(D) \notin C$. Equivalently, for all $C \in \mathscr{P}^\star(A)$, either $h(C) \neq h(D)$ or $h(D) \in C$. But since $D$ is a non-empty subset of $A$, $D \in  \mathscr{P}^\star(A)$, so $D$ is `one of those sets $C$'. Since we clearly have $h(D) = h(D)$, we must require that for $C = D$, $h(D) \in C$, i.e. $h(D) \in D$, but that contradicts our assumption at the start of this case. \\ [2ex]

%Assume $h(D) \notin C$: then we must have $h(C) \neq h(D)$. 

%such that the pre-image elements do not contain those $A$-elements


\textit{OG version of this proof}: Suppose otherwise. Then there is some set $A$ such that $|A| \not< |\mathscr{P}^\star(A)|$, which means that either there is no injection from $A$ to $\mathscr{P}^\star(A)$ or there is a bijection from $A$  to $\mathscr{P}^\star(A)$. But $f(x) = \set{x}$ an injection from $A$ to $\mathscr{P}^\star(A)$. So it must be the case that there is a bijection $h'$ from $A$ to $\mathscr{P}^\star(A)$, and therefore a bijection $h$ from $\mathscr{P}^\star(A)$ to $A$.

Let $D = \set{a \in A : \exists C \in \mathscr{P}^\star(A) (h(C) = a \wedge a \not\in C)}$. We start by verifying that $D$ is non-empty. Suppose otherwise. Then $D$ is empty, so, in particular, $h(\set{a}) = a$ for each $a \in A$. But $A$ has at least two members, $a$ and $b$ ($a \neq b$). And, since $h$ is injective, the fact that $h(\set{a}) = a$ and $h(\set{b}) = b$ entails that we can have neither $ h(\set{a,b}) = a $ nor $h(\set{a,b}) = b$. So $h(\set{a,b}) \in D$, contradicting our hypothesis.

Now consider the question whether $h(D) \in D$ is true or false. If it's true, we have $h(C) = h(D) \wedge h(D) \not\in C$ for some $C \in \mathscr{}P^\star(A)$. Since $h$ is an injection, $h(C) = h(D)$ entails $C = D$, so we must have $h(D) \not\in D$, contradicting our hypothesis. If, on the other hand, $h(D) \in D$ is false, $h(C) = h(D) \wedge h(D) \not\in C$ must be false for every $C \in \mathscr{}P^\star(A)$. Since $D$ is non-empty, it follows that $h(C) = h(C) \wedge h(C) \not\in C$ must be false, which entails $h(C) \in C$, contradicting our hypothesis.
}

\end{enumerate}


\end{enumerate}



%Endnotes
%\newpage \begingroup \parindent 0pt \parskip 2ex \def\enotesize{\normalsize} \theendnotes \endgroup


%\newpage\bibliographystyle{linquiry}\bibliography{agustin}

%\end{doublespace}
\end{document}

