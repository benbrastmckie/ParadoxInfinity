

\documentclass[12pt,letterpaper]{article}
\usepackage{../pset_2024}


%Questions and Answers
\qa{b} % a="answers only"; q ="questions only"; b="both"
\usepackage{../qa}


\begin{document}

\psintro{Problem Set 2: The Higher Infinite}

%%%%%%%%%%%%%%%%%%%%%%%%%%%

\newpage



\subsection*{Part I (Quiz on Canvas: 46 points)} 
%currently 46 = 10 +12 +12 +4 +8 points
%part II: 54 = 10+10+5+21+8 



\begin{enumerate}




%left out for variety
%\com{

\item[] \question{\emph{Notation:} \(\emptyset\) is the empty set. If \(A\) and \(B\) are sets, $\powerset(A)$  is the set of $A$'s subsets and \(A-B\) is the set whose members are the elements of \(A\) that are not also elements of \(B\) (so, for instance, \(\{1,2\}-\{1\}=\{2\}\)).}

\item \question{Answer the following questions. (2 points each)}

\begin{enumerate}

\item \question{Is $\powerset(\mathbb{Z})-\powerset(\mathbb{N})$  well-ordered by $\subseteq$?}


\item \question{Is $\powerset(\powerset(\emptyset))$ well-ordered by $\in$?}


\com{ %Used in 2022; seems better discussed in lecture 
\item \question{Is $\powerset(\emptyset)$ well-ordered by $\in$?}

}%end com

\item  \question{Is \(\powerset^4(\emptyset)=\powerset(\powerset(\powerset( \powerset (\emptyset))))\) well-ordered by \(\in\)?} 
  

\com{ %Used in 2022
\item  \question{Is \(\powerset(\powerset( \powerset (\emptyset)))\) well-ordered by \(\in\)?} 
  
}%end com


\item  \question{Is \(\powerset^3(\emptyset)- \{ \{ \{ \emptyset\}\}\} =\powerset (\powerset (\powerset (\emptyset))) - \{ \{ \{ \emptyset\}\}\}\) well-ordered by \(\in\)?} 


\item \question{Is \(\powerset^3(\emptyset)- \{ \{ \{ \emptyset\}\}\} \) an ordinal number?}






\com{ %Used in 2022 
\item \question{Is $\powerset(\mathbb{N})$  well-ordered by $\subseteq$?}

}%end com 



 
 \end{enumerate}

%}



%
\item \question{Recall the following definitions:

  \begin{itemize}
    \item $\mathbb{N}$ is the set of natural numbers. 

    \item $\powerset^n(A) = \underbrace{\powerset(\powerset(\dots\powerset(}_{\text{\tiny $n$ times}}A\underbrace{)\dots))}_{\text{\tiny $n$ times}}$ ($n \in \mathbb{N}$).

    \item $\bigcup A = \set{x : x \in B \text{ for } B \in A}$. 
  \end{itemize}

Determine whether each of the following statements is true or false. (2~points each)}


\begin{enumerate} %start of 2023 version; see below for an earlier version with the naturals 

\item  \question{$\powerset^n(\mathbb{Z}) \subseteq \powerset^m(\mathbb{Z})$, for $n < m$ and $n,m \in \mathbb{N}$.}



\item \question{$|\powerset^n(\mathbb{Z})| < |\powerset^m(\mathbb{Z})|$, for $n < m$ and $n,m \in \mathbb{N}$.}


\item \question{For arbitrary $n \in \mathbb{N}$, $\powerset^n(\mathbb{Z}) \subseteq \set{\powerset^m(\mathbb{Z}): m \in \mathbb{N}}$.}


\item \question{For arbitrary $n \in \mathbb{N}$, $\powerset^n(\mathbb{Z}) \subseteq \bigcup \set{\powerset^m(\mathbb{Z}): m \in \mathbb{N}}$.}


\item \question{$|\powerset^n(\mathbb{Q})| < | \set{\powerset^m(\mathbb{Q}): m \in \mathbb{N}}|$, $n \in \mathbb{N}$}


\item \question{$|\powerset^n(\mathbb{R})| < |\bigcup \set{\powerset^m(\mathbb{R}): m \in \mathbb{N}}|$, $n \in \mathbb{N}$}


%\item \question{$|\powerset^n(\mathbb{Q})| < |\bigcup \set{\powerset^m(\mathbb{Q}): m \in \mathbb{N}}|$, $n \in \mathbb{N}$}


\end{enumerate}


\com{ %2022 version; modifying for 2023 to be based on integers! 

Determine whether each of the following statements is true or false. You may assume that the natural numbers are not sets. (2~points each)

\begin{enumerate}

\item  \question{$\powerset^n(\mathbb{N}) \subseteq \powerset^m(\mathbb{N})$, for $n < m$ and $n,m \in \mathbb{N}$.}



\item \question{$|\powerset^n(\mathbb{N})| < |\powerset^m(\mathbb{N})|$, for $n < m$ and $n,m \in \mathbb{N}$.}


\item \question{$\powerset^n(\mathbb{N}) \subseteq \set{\powerset^m(\mathbb{N}): m \in \mathbb{N}}$, $n \in \mathbb{N}$.}


\item \question{$\powerset^n(\mathbb{N}) \subseteq \bigcup \set{\powerset^m(\mathbb{N}): m \in \mathbb{N}}$, $n \in \mathbb{N}$.}

\item \question{$|\powerset^n(\mathbb{N})| < | \set{\powerset^m(\mathbb{N}): m \in \mathbb{N}}|$, $n \in \mathbb{N}$}


\item \question{$|\powerset^n(\mathbb{N})| < |\bigcup \set{\powerset^m(\mathbb{N}): m \in \mathbb{N}}|$, $n \in \mathbb{N}$}


\end{enumerate}

}%end of long com for this problem 






% Left out for variety %%Hunt using as practice problems in 2023 lecture!; also included in the hidden HW questions for Topic 2, MITx, Problem 2. 
\com{

\item \question{Determine whether each of the following statements is true or false. (2 points each)}

\begin{enumerate}

\item \question{\(0' + 0''' = 0''' + 0'\)}

\item  \question{\(0' \times 0''' = 0''' \times 0'\)}

\item  \question{\(0' + \omega = \omega' + 0\)}
   
\item  \question{\(0' + \omega = 0 + \omega'\)}
   


    
      
%left out for variety
\com{
\item  \question{\(0' + \omega = \omega + 0'\)}
   
}

%left out for variety
\com{
\item \question{$0'' + 0' <_o 0 + 0''' $}

}


%left out for variety
\com{
\item  \question{\((\omega + 0'') + \omega <_o (\omega + \omega) + 0''\)} 

}


%left out for variety
\com{
\item \question{$\omega \times 0'''$ = $0''' \times \omega$}

}


\item \question{\(0''' \times \omega = (\omega + \omega) + \omega\)}
       


\item  \question{\(\omega \times 0''' = \omega + (\omega + \omega)\)}



\item  \question{\((\omega \times 0'') + \omega <_o (\omega \times \omega) + 0''\)} 



 \item  \question{\(\omega \times \omega <_o \omega \times (0'' \times \omega)\)} 



 \item   \question{\(\omega \times (\omega + \omega) = (\omega \times \omega) + (\omega \times \omega)\)}


\item  \question{\(\alpha + 0' = \alpha \cup \{\alpha\}\) ($\alpha$ an ordinal)}




\end{enumerate}

}







\item \question{Recall the following definitions:

\begin{itemize}

\item $\alpha <_o \beta \leftrightarrow_{\text{\emph{df}}} \alpha \in \beta$ ($\alpha, \beta$ ordinals)

\item $|A| < |B| \leftrightarrow_{\text{\emph{df}}}$ there is an injection from $A$ to $B$ but no bijection


 \item \(
\mathfrak{B}_\alpha=
\begin{cases}
\mathbb{N}, \text{ if $\alpha = 0$}\\
\powerset(\mathfrak{B}_\beta), \text{ if $\alpha = \beta'$}\\
\bigcup \{\mathfrak{B}_\gamma : \gamma <_o \alpha\} \text{ if $\alpha$ is a limit ordinal greater than $0$}
\end{cases}
\)

\item $\beth_\alpha \text{ is the $<_o$-smallest ordinal of cardinality } |\mathfrak{B}_\alpha|$


%\item $\beth_\alpha < \beth_\beta \leftrightarrow_{\text{\emph{df}}} |\beth_\alpha | < |\beth_\beta|$

\end{itemize}
Which of the following are true? (2~points each)
}

\begin{enumerate}

\item \question{$\omega <_o \omega +\omega$}


\item \question{$|\omega| < |\omega \times \omega|$}



\item \question{$\omega \times \omega <_o \beth_{0}$}


\item \question{$|\omega \times \omega| < |\beth_{0}|$}


\item \question{Let `$118^o$' denote the ordinal named by $0$ followed by 118 prime symbols. Claim: $\beth_{118^o} <_o \beth_{\omega}$}


\item \question{$|\beth_{118^o}| < |\beth_{\omega}|$}


\com{ %modifying slightly for 2023 version

\item \question{$\beth_0 <_o \beth_{\omega}$}


\item \question{$|\beth_0| < |\beth_{\omega}|$}

} %end com 

\end{enumerate}

\item \question{Let $\mathscr{U} = \bigcup \set{\powerset^m(\mathbb{N}): m \in \mathbb{N}}$. Answer the following questions (2 points each).}

\begin{enumerate}

\item \question{Does $\mathscr{U}$ also contain a set $\underbrace{\{\{\ldots\{\{}_{\mbox{\scriptsize $\infty$ times}}118\underbrace{\}\}\ldots\}\}}_{\mbox{\scriptsize $\infty$ times}}$? }



\item \question{Is the cardinality of the following set greater than the cardinality of $\powerset^n(\mathscr{U})$ for each $n \in \mathbb{N}$? \[
\bigcup \{\mathbb{N}, \powerset(\mathbb{N}), \powerset(\powerset(\mathbb{N})), \dots, \mathscr{U}, \powerset(\mathscr{U}), \powerset(\powerset(\mathscr{U})), \dots\}
\] }



\end{enumerate}


\com{ 
\item \question{Where $\mathscr{U} = \bigcup \set{\powerset^m(\mathbb{N}): m \in \mathbb{N}}$, answer the following questions. (5~points each; don't forget to justify your answers)
}

\begin{enumerate}

\item \question{Show that $\mathscr{U}$ contains the set $\underbrace{\{\{\ldots\{\{}_{\mbox{\scriptsize $n$ times}}17\underbrace{\}\}\ldots\}\}}_{\mbox{\scriptsize $n$ times}}$ for each $n>0$.}


\item \question{Does $\mathscr{U}$ also contain a set $\underbrace{\{\{\ldots\{\{}_{\mbox{\scriptsize $\infty$ times}}17\underbrace{\}\}\ldots\}\}}_{\mbox{\scriptsize $\infty$ times}}$? }



\item \question{
Give an example of a set whose cardinality is greater than the cardinality of  $\powerset^n(\mathscr{U})$ for each $n \in \mathbb{N}$.}



\end{enumerate}
} %end long com 

%Used to be on Part II, but easily made true/false question for Part 1
\item \question{Determine whether the following are true or false (2 points each)}

%To justify your answer, you may use a diagram (or use prose) to give an informal characterization of the respective well-order types. (3 points each.)


\begin{enumerate}
\item \question{$(\omega+0'')+(\omega+0')=(\omega+\omega)+(\omega+0')$ } 


\item \question{$(0'+\omega)\times 0'''=(0'\times 0''')+(\omega\times 0''')$}


\item \question{$(\omega+0'')\times 0''''=(\omega\times 0'')+(0''\times 0'''')$}


\item \question{$(\omega + 0')+0''=(0'+\omega)+0''$}

\end{enumerate}




\end{enumerate}



%%%%%%%%%%%
%PART II
%%%%%%%%%%%%
\subsection*{Part II (Submit PDF on Canvas: 54 points)} 

\begin{enumerate}
  \setcounter{enumi}{5} % use this to set the counter for what number appears first

%used to be on Part 1 but can't code into Canvas Quiz. 
\item \question{Draw a diagram (or use prose) to give an informal characterization of the well-ordering types represented by each of the following ordinals. (2 points each; no need to justify answer but feel free to show work) \\ [1ex]
What does it mean to use prose to give an informal characterization of a well-order type? Suppose, for example, that the well-order type in question corresponded to $\omega$. Then you might say something like ``A countably infinite sequence of items which is ordered like the natural numbers, with an $n$th member for each $n \in \mathbb{N}$)---but no last member." Drawing diagrams is probably easier than using your words!}


\begin{enumerate}

\item \question{$(\omega + 0'') \times 0'''$ }




\item \question{$(\omega \times 0''') \times 0''$ }




\item \question{$(0'''''' \times \omega) \times 0''$ }



\item \question{$(\omega \times \omega) + \omega$}



\item \question{$(\omega \times \omega) \times \omega$}


\end{enumerate}
%\question{What does it mean to use prose to give an informal characterization of a well-order type? Suppose, for example, that the well-order type in question corresponded to $\omega$. Then you might say something like ``A countably infinite sequence of items which is ordered like the natural numbers, with an $n$th member for each $n \in \mathbb{N}$)---but no last member."}



%\com{ 
  \item \question{Recall that a relation constitutes a strict total order on a set just in case it is \\ (i) asymmetric (so irreflexive as well and hence `strict'), (ii) transitive, and (iii) total. \\ Using an example not found in the course material/lecture:}
  
  \begin{enumerate}
  
  \item \question{Specify (i) a set whose members are not numbers and (ii) an ordering on that set that is not a strict total ordering. (5 points)}
  
  
  

  \item \question{Specify (i) a set whose members are not numbers and (ii) a strict total ordering on that set that is not a well-ordering. You do not need to rigorously prove that the relation constitutes a strict total ordering. (5 points)}
  
  
   
  
  \end{enumerate}
  
%  }



%Left out for variety
%\com{
\item \question{Recall that  $\alpha <_o \beta$ is defined as $\alpha \in \beta$, for $\alpha$ and $\beta$  ordinals. Does $\alpha <_o \beta$ entail $|\alpha| < |\beta|$? If so explain, why. If not, give a counterexample. (5 points)}
%}


\item \question{Recall that we think of the ordinals as introduced in stages,  in accordance with the following principles:

\begin{description}
\item[Open-Endedness Principle] However many stages have occurred, there is always a ``next'' stage: a first stage after every stage considered so far.


\item[Construction Principle] 
At each stage, we introduce a new\footnote{A new ordinal is an ordinal that has not been introduced at previous stages.} ordinal, namely: the set of all ordinals that have been introduced at previous stages. 

\end{description}
Use these principles to give an informal justification of each of the following propositions. (7~points each; don't forget to justify your answers)}

\begin{enumerate}


\item \question{No ordinal is a member of itself.}


\item \question{For any ordinal $\alpha$, either $\alpha = \set{}$ or $\set{} <_o \alpha$. }



\item \question{If $\alpha$ is an ordinal with infinitely many members, then either $\alpha = \omega$ or  $\omega <_o \alpha$.}




\end{enumerate}


\item \question{Give an example of a set whose cardinality is ``much greater'' than $|\mathfrak{B}_{\omega \times \omega^\omega}|$, in the sense that there are infinitely many sizes of infinity between the set you identify and $|\mathfrak{B}_{\omega \times \omega^\omega}|$.   (8~points; don't forget to justify your answer.)
}

\com{ %2019 version; also on MITx homework:
\item \question{Give an example of a set whose cardinality is ``much greater'' than $|\mathfrak{B}_{\omega \times \omega}|$, in the sense that there are infinitely many sizes of infinity between the set you identify and $|\mathfrak{B}_{\omega \times \omega}|$.   (2~points; don't forget to justify your answer.)
}

}%end com 

\com{ %Used in 2022; will discuss in 2023 lecture 
\item \question{Give an example of a set whose cardinality is ``much greater'' than $|\mathfrak{B}_{\omega^\omega}|$, in the sense that there are infinitely many sizes of infinity between the set you identify and $|\mathfrak{B}_{\omega^\omega}|$.   (3~points; don't forget to justify your answer.)


}


}%end com 



\com{ %start of long com for Russell's paradox question 
\item \question{Russell's Paradox is the observation that the following principle leads to contradiction when $F$ is the predicate ``is not a member of itself'' (and therefore cannot be true in general):

\begin{description}
\item[Naive Comprehension Principle]
There is a set $\set{x : F(x)}$, which consists of all and only objects that are $F$.
\end{description}

Consider an alternative principle:

\begin{description}
\item[Iterative Comprehension Principle]
For each ordinal $\alpha$, there is a set $\set{x \in V_\alpha: F(x)}$, which consists of all and only objects in $V_\alpha$ that are $F$.
\end{description}
where:
$$
V_\alpha=
\begin{cases}
\set{}, \text{ if $\alpha = 0$}\\
\powerset(V_\beta), \text{ if $\alpha = \beta'$}\\
\bigcup \{V_\gamma : \gamma <_o \alpha\} \text{ if $\alpha$ is a limit ordinal greater than $0$}
\end{cases}
$$


}

\begin{enumerate}

\item \question{Does the Iterative Comprehension Principle lead to contradiction when $F$ is the predicate ``is not a member of itself''? If it does, provide a proof; if it doesn't---or if you're not sure whether it does or doesn't---explain how it blocks the reasoning that leads to Russell's Paradox. (5~points)}



\item \question{Does the Iterative Comprehension Principle entail that there is a set of all sets? (5~points; don't forget to justify your answer.)} 


\end{enumerate}
} %end com 

\com{ %start of long com for a difficult series of questions 
\item \question{Recall the following definitions:

\[\begin{array}{rclcl}
& & 0 &= &\set{}\\ 
& & \alpha' &= &\alpha \cup \set{\alpha}\\ 
& & \omega &= &\{0^{\overbrace{'\dots'}^{\text{$n$-times}}} : n \in \mathbb{N}\} = \set{0, 0', 0'', \dots}\\ \\
 \alpha &+ &0 &= &\alpha \\ \alpha &+ &\beta' &= &(\alpha + \beta)'\\ \alpha &+ &\lambda &= & \bigcup \{\alpha + \beta : \beta < \lambda\} \text{ ($\lambda$ a limit ordinal)}\\ \\ 
%
\alpha &\times &0 &= &0 \\ 
\alpha &\times &\beta' &= &(\alpha \times \beta) + \alpha\\ 
\alpha &\times &\lambda &= & \bigcup \{\alpha \times \beta : \beta < \lambda\} \text{ ($\lambda$ a limit ordinal)} \\ \\ 
%
& & \alpha^0 & = & 0' \\ 
& & \alpha^{\beta'} &= &(\alpha^\beta) \times \alpha \\ 
& & \alpha^{\lambda}  &= & \bigcup \{\alpha^\beta : \beta < \lambda\} \text{ ($\lambda$ a limit ordinal)} \end{array}\]

Use the above definitions to give a \underline{fully rigorous} proof of each of the following identities. In doing so, you may \underline{not} make use of results mentioned in the lecture notes, e.g.~Associativity. You'll have to prove everything from scratch!\footnote{You may use mathematical induction.}  (8 points each)
}

\begin{enumerate}

\item \label{mult-left} \question{$0^{\overbrace{'\dots'}^{\text{$n$-times}}} \times 0' = 0^{\overbrace{'\dots'}^{\text{$n$-times}}}$, for arbitrary $n \in \mathbb{N}$ }


\item \label{mult-left} \question{$0' \times 0^{\overbrace{'\dots'}^{\text{$n$-times}}} = 0^{\overbrace{'\dots'}^{\text{$n$-times}}}$, for arbitrary $n \in \mathbb{N}$ }


\item \question{$w^{0''} = \omega \times \omega$}


\end{enumerate}
} %end of com 


\end{enumerate}


\end{document}





