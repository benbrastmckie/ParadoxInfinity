\documentclass[a4paper, 11pt]{article} % Font size (can be 10pt, 11pt or 12pt) and paper size (remove a4paper for US letter paper)

\usepackage[protrusion=true,expansion=true]{microtype} % Better typography
\usepackage{graphicx} % Required for including pictures
\usepackage{wrapfig} % Allows in-line images
\usepackage{enumitem} %%Enables control over enumerate and itemize environments
\usepackage{setspace}
\usepackage{amssymb, amsmath, mathrsfs,mathabx} %%Math packages
\usepackage{stmaryrd}
\usepackage{mathtools}
\usepackage{multicol} 
\usepackage{mathpazo} % Use the Palatino font
\usepackage[T1]{fontenc} % Required for accented characters
\usepackage{array}
\usepackage{bibentry}
\usepackage{prooftrees} 
\usepackage[round]{natbib} %%Or change 'round' to 'square' for square backers
\setcitestyle{aysep=}
% \usepackage{fitchproof} 

% \linespread{1.05} % Change line spacing here, Palatino benefits from a slight increase by default

\newcommand{\tuple}[1]{\langle#1\rangle} %%Angle brackets
\newcommand{\corner}[1]{\ulcorner#1\urcorner} %%Angle brackets
\newcommand{\set}[1]{\lbrace#1\rbrace} %%Set brackets
\newcommand{\abs}[1]{|#1|} %%Set brackets
\newcommand{\interpret}[1]{\llbracket#1\rrbracket} %%Double brackets
\newcommand{\N}{\mathbb{N}}
\renewcommand{\L}{\mathcal{L}}
\newcommand{\D}{\mathbb{D}}
\newcommand{\Z}{\mathbb{Z}}
\renewcommand{\Pr}{\mathbb{P}}
\newcommand{\Q}{\mathbb{Q}}
\newcommand{\R}{\mathbb{R}}
\newcommand{\B}{\mathfrak{B}}
\renewcommand{\max}[1]{\texttt{max}\set{#1}}

\makeatletter
\renewcommand\@biblabel[1]{\textbf{#1.}} % Change the square brackets for each bibliography item from '[1]' to '1.'
\renewcommand{\@listI}{\itemsep=0pt} % Reduce the space between items in the itemize and enumerate environments and the bibliography

\renewcommand{\maketitle}{ % Customize the title - do not edit title and author name here, see the TITLE block below
\begin{flushright} % Right align
{\LARGE\@title} % Increase the font size of the title

\vspace{10pt} % Some vertical space between the title and author name

{\@author} % Author name
\\\@date % Date

\vspace{00pt} % Some vertical space between the author block and abstract
\end{flushright}
}

%----------------------------------------------------------------------------------------
%	TITLE
%----------------------------------------------------------------------------------------

\title{\textbf{Absolute Generality}} % Subtitle

\author{\textsc{Paradox and Infinity}\\ \em Benjamin Brast-McKie} % Institution

\date{\today} % Date

%----------------------------------------------------------------------------------------

\begin{document}

\maketitle % Print the title section

\thispagestyle{empty}

%----------------------------------------------------------------------------------------

\section*{Restricted Quantifiers}

\begin{itemize}
  \item[\it Coffee:] There is no more coffee.
    \item Of course there is some coffee somewhere.
  \item[\it Bags:] I have packed everything.
    \item But not everything in existence.
  \item[\it Restriction:] The domain of quantification is, somehow, restricted by context.
    \item Context does not change the meaning of `there is' or `everything'.
    \item The quantifiers have the same semantics, just the domain shifts.
\end{itemize}





\section*{Unrestricted Quantifiers}

\begin{itemize}
  \item[\it Biology:] All humans are mortal.
  \item[\it Identity:] Everything is self-identical.
  \item[\it Sets:] No set is a member of itself.
    \item It appears that we can quantify over \textit{everything}.
    \item If not, quantified claims are not as informative as they could be.
  \item[\bf Question:] Are we able to quantify over everything, at least sometimes?
\end{itemize}




\section*{Russellian Doubts}

\begin{enumerate}
  \item[\it Reductio:] Consider the following argument against unrestricted quantification:
    \item Assume that we can quantify over all sets.
    \item The domain of quantification must include all sets.
    \item Quantifier domains are sets.
    \item So there is a universal set of all sets.
    \item We can derive naive comprehension from separation.
    \item A contradiction follows by Russell's paradox.
    \item Hence we cannot quantify over all sets.
    \item Sets are things.
    \item Thus we cannot quantify over everything.
\end{enumerate}




\section*{Domain Free Quantification}

\begin{itemize}
  \item[\it All-in-One:] Cartwright rejects (3) above.
    \item Can we quantify over everything though there is no set of everything?
    \item ``There is no set that has as members all and only those things that are not members of themselves.
      But the things that are not members of themselves can simultaneously be the values of the variables of a first-order language; so at any rate I claim.'' ---Cartwright (1994, p.~3)
    % \item The \textit{universe of discourse} can contain all sets without forming a set.
  \item[\it Absolute Generality:] Why can't $x$ be instantiated by $w$ in $\forall x(x \in w \leftrightarrow x \notin x)$? 
    \item Because there is no set $w$ according to ZFC. 
    \item Rather, by \textit{Separation}, we get $\forall x(x \in w' \leftrightarrow (x \in z \wedge x \notin x))$ for some $z$.
    \item But there is no universal set, and so at most $w' = z$.
    \item But if we can quantify over everything, what justifies \textit{Separation}?
\end{itemize}




\section*{Relatively Unrestricted Quantification}

\begin{itemize}
  \item[\it Indefinite Extensibility:] In quantifying unrestrictedly, new entities can always be defined.
    \item In particular, $w$ falls outside the domain of the quantifier.  
    \item Naive comprehension may be preserved, but $w$ cannot instantiate $x$.
  \item[\it Self-Defeating:] It is not possible to quantify over everything. 
    \item Hence I am not quantifying over everything.
    \item So there is something that I am not quantifying over.
    \item But this is self-defeating, and so false.
  \item[\it Context Domains:] Not everything is quantified over in context $c$.
    \item For the reasons above, this theses is false in $c$.
  \item[\it Context Principle:] For any $c$, there is some $c'$ where something quantified over in $c'$ is not also quantified over in $c$.
    \item So not everything is quantified over in $c$, which is false in $c$.
    \item But $c$ was an arbitrary context, so the \textit{Context Principle} is false in any $c$. 
  \item[\it Show Don't Tell:] The relativist might claim only to be able to always shift the context.
    \item Start in $c$, Russell's paradox moves us to $c'$ with broader quantifiers. 
    \item Should we trust a theory that we can't state?
  \item[\it Absolutism:] Claiming that we can quantify over everything is not self-defeating.
    \item We may quantify over everything but are still beholden to \textit{Separation}.
    \item Not as many sets exist as we might think we are able to naively define.
\end{itemize}





\end{document}



