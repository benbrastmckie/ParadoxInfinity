\documentclass[a4paper, 11pt]{article} % Font size (can be 10pt, 11pt or 12pt) and paper size (remove a4paper for US letter paper)

\usepackage[protrusion=true,expansion=true]{microtype} % Better typography
\usepackage{graphicx} % Required for including pictures
\usepackage{wrapfig} % Allows in-line images
\usepackage{enumitem} %%Enables control over enumerate and itemize environments
\usepackage{setspace}
\usepackage{amssymb, amsmath, mathrsfs,mathabx} %%Math packages
\usepackage{stmaryrd}
\usepackage{mathtools}
\usepackage{multicol} 
\usepackage{mathpazo} % Use the Palatino font
\usepackage[T1]{fontenc} % Required for accented characters
\usepackage{array}
\usepackage{bibentry}
\usepackage{prooftrees} 
\usepackage[round]{natbib} %%Or change 'round' to 'square' for square backers
\setcitestyle{aysep=}
% \usepackage{fitchproof} 

% \linespread{1.05} % Change line spacing here, Palatino benefits from a slight increase by default

\newcommand{\tuple}[1]{\langle#1\rangle} %%Angle brackets
\newcommand{\corner}[1]{\ulcorner#1\urcorner} %%Angle brackets
\newcommand{\set}[1]{\lbrace#1\rbrace} %%Set brackets
\newcommand{\abs}[1]{|#1|} %%Set brackets
\newcommand{\interpret}[1]{\llbracket#1\rrbracket} %%Double brackets
\newcommand{\N}{\mathbb{N}}
\renewcommand{\L}{\mathcal{L}}
\newcommand{\D}{\mathbb{D}}
\newcommand{\Z}{\mathbb{Z}}
\renewcommand{\Pr}{\mathbb{P}}
\newcommand{\Q}{\mathbb{Q}}
\newcommand{\R}{\mathbb{R}}
\newcommand{\B}{\mathfrak{B}}
\renewcommand{\max}[1]{\texttt{max}\set{#1}}

\makeatletter
\renewcommand\@biblabel[1]{\textbf{#1.}} % Change the square brackets for each bibliography item from '[1]' to '1.'
\renewcommand{\@listI}{\itemsep=0pt} % Reduce the space between items in the itemize and enumerate environments and the bibliography

\renewcommand{\maketitle}{ % Customize the title - do not edit title and author name here, see the TITLE block below
\begin{flushright} % Right align
{\LARGE\@title} % Increase the font size of the title

\vspace{10pt} % Some vertical space between the title and author name

{\@author} % Author name
\\\@date % Date

\vspace{-20pt} % Some vertical space between the author block and abstract
\end{flushright}
}

%----------------------------------------------------------------------------------------
%	TITLE
%----------------------------------------------------------------------------------------

\title{\textbf{Absolute Generality}} % Subtitle

\author{\textsc{Paradox and Infinity}\\ \em Benjamin Brast-McKie} % Institution

\date{\today} % Date

%----------------------------------------------------------------------------------------

\begin{document}

\maketitle % Print the title section

\thispagestyle{empty}

%----------------------------------------------------------------------------------------

\section*{Absolutism}

\begin{itemize}
  \item[\it Gloss:] It is possible to quantify over absolutely everything.
    \item By `possible' we mean `there is an interpretation $I,{\hat{a}}$ of the language'.
  \item[\it Minimal Language:] Consider a language $\L$ with just `$\forall$', `$x$', and `$F$' as primitive symbols.
    \item The extension $I(F) \subseteq D_I$ interprets the predicate `$F$'.
    \item $\hat{a}$ is a variable assignment for $\L$ \textit{iff} $\hat{a}(x) \in D_I$.
    % \item $\corner{\forall \alpha \varphi}$ is true in $I,{\hat{a}}$ \textit{iff} $\corner{\varphi}$ is true in $I,{\hat{b}}$ for every $\alpha$-variant $\hat{b}$ of $\hat{a}$. 
    \item `$Fx$' is true in $I, \hat{a}$ \textit{iff} $\hat{a}(x) \in I(F)$. 
    \item `$\forall xFx$' is true in $I,{\hat{a}}$ \textit{iff} `$Fx$' is true in $I,{\hat{b}}$ for every $x$-variant $\hat{b}$ of $\hat{a}$.
  \item[\it Restatement:] For anything $y$ there is an interpretation $I$ of $\L$ where $y \in D_I$. 
    \item Everything belongs to the domain of quantification $D_I$.
    \item $\forall y\exists D(y \in D)$.
    \item But this is just the claim that there is a universal set.
  \item[\it Direct Method:] We must learn to understand absolute quantification directly.
    \item Model theory does not provide an adequate account.
    \item At most we can say: $\forall x \exists y(x =y)$.
    \item But this is trivial, failing to communicate the substance of absolutism.
\end{itemize}




\section*{No Domain Theory}

\begin{itemize}
  \item[\it Absolute Plurality:] There is a plurality of everything, i.e., $\exists xx \forall y ( y \prec xx )$.
    % \item Example: ``The seashells are scattered across the sand.''
    \item ``They lifted the piano,'' ``They won the championship,'' etc.
    \item We read `$y \prec xx$' as `$y$ is \textit{one of} the $xx$s'.
  \item[\it Quinian Doubts:] Some have thought that $\exists xx \varphi$ is just shorthand for $\exists x(S(x) \wedge \varphi)$.
    \item Similarly, `$y \prec xx$' is just shorthand for `$y \in x$'.
    \item But this flattens the absolutists current attempt to state their thesis.
    \item Instead the absolutist must take plural quantifiers to be primitive.
    \item We do seem to have plural quantifiers in English.
  \item[\it Plural Separation:] For any\textit{things}, there is a set of those that are such that $\varphi$.
    \item Formally: $\forall xx \exists y \forall x ( x \in y \leftrightarrow x \prec xx \wedge \varphi )$ where `$y$' doesn't occur in $\varphi$.
    \item Naive comprehension follows: $\exists y \forall x ( x \in y \leftrightarrow \varphi)$.
  \item[\it Plural Comprehension:] $\exists yy \forall x ( x \prec yy \leftrightarrow \varphi )$ where `$yy$' does not occur in $\varphi$.
    % \item Better: $\forall xx \exists yy \forall x ( x \prec yy \leftrightarrow x \prec xx \wedge \varphi )$ where `$yy$' doesn't occur in $\varphi$.
\end{itemize}




\section*{Indefinitely Extensible?}

\begin{itemize}
  \item[\it Extensions:] Predicates are interpreted by assigning them to sets.
    \item But how are we to interpret `\texttt{set}'?
    \item Suppose $I(\texttt{set}) \subseteq D$.
    % \item Assume $I(\texttt{set}) \in D$.
    \item But since $I(\texttt{set}) \notin I(\texttt{set})$, there is a set not in the extension of `\texttt{set}`.
  \item[\it Relativism:] Claims that we can always extend any extension of the predicate `\texttt{set}'.
    \item $I \subseteq J$ \textit{iff} $I(\kappa) \subseteq J(\kappa)$ for any predicate $\kappa$.
    \item $I \subset J$ \textit{iff} $I \subseteq J$ and $J \nsubseteq I$. 
    \item For any $I$ of $\L$, there is some $J$ of $\L$ where $I \subset J$.
    \item Every extension of `\texttt{set}' has a broader extension.
  % \item[\it Modals:] We may now make sense of `can always extend and extension'.
  %   \item Let `$\Diamond A$' read `There is an interpretation in which $A$ is true'.
  %   \item Let `$\Box A$' read `In every interpretation $A$ is true'.
  \item[\it Absolutism:] Instead of sets, suppose extensions are taken to be pluralities.
    \item The intended extension of `\texttt{set}' is the plurality of all sets.
    \item $\exists yy \forall x ( x \prec yy \leftrightarrow \texttt{set}(x) )$.
    \item The relativist may claim that this fails to capture indefinite extensibility.
    \item The absolutist is happy to avoid indefinite extensibility.
\end{itemize}


\end{document}



