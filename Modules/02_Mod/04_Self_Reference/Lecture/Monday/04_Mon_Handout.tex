\documentclass[a4paper, 11pt]{article} % Font size (can be 10pt, 11pt or 12pt) and paper size (remove a4paper for US letter paper)

\usepackage[protrusion=true,expansion=true]{microtype} % Better typography
\usepackage{graphicx} % Required for including pictures
\usepackage{wrapfig} % Allows in-line images
\usepackage{enumitem} %%Enables control over enumerate and itemize environments
\usepackage{setspace}
\usepackage{amssymb, amsmath, mathrsfs,mathabx} %%Math packages
\usepackage{stmaryrd}
\usepackage{mathtools}
\usepackage{multicol} 
\usepackage{mathpazo} % Use the Palatino font
\usepackage[T1]{fontenc} % Required for accented characters
\usepackage{array}
\usepackage{bibentry}
\usepackage{prooftrees} 
\usepackage[round]{natbib} %%Or change 'round' to 'square' for square backers
\setcitestyle{aysep=}
% \usepackage{fitchproof} 

% \linespread{1.05} % Change line spacing here, Palatino benefits from a slight increase by default

\newcommand{\tuple}[1]{\langle#1\rangle} %%Angle brackets
\newcommand{\corner}[1]{\ulcorner#1\urcorner} %%Angle brackets
\newcommand{\set}[1]{\lbrace#1\rbrace} %%Set brackets
\newcommand{\abs}[1]{|#1|} %%Set brackets
\newcommand{\interpret}[1]{\llbracket#1\rrbracket} %%Double brackets
\newcommand{\N}{\mathbb{N}}
\renewcommand{\L}{\mathcal{L}}
\newcommand{\D}{\mathbb{D}}
\newcommand{\Z}{\mathbb{Z}}
\renewcommand{\Pr}{\mathbb{P}}
\newcommand{\Q}{\mathbb{Q}}
\newcommand{\R}{\mathbb{R}}
\newcommand{\B}{\mathfrak{B}}

\makeatletter
\renewcommand\@biblabel[1]{\textbf{#1.}} % Change the square brackets for each bibliography item from '[1]' to '1.'
\renewcommand{\@listI}{\itemsep=0pt} % Reduce the space between items in the itemize and enumerate environments and the bibliography

\renewcommand{\maketitle}{ % Customize the title - do not edit title and author name here, see the TITLE block below
\begin{flushright} % Right align
{\LARGE\@title} % Increase the font size of the title

\vspace{10pt} % Some vertical space between the title and author name

{\@author} % Author name
\\\@date % Date

\vspace{10pt} % Some vertical space between the author block and abstract
\end{flushright}
}

%----------------------------------------------------------------------------------------
%	TITLE
%----------------------------------------------------------------------------------------

\title{\textbf{Self Reference}} % Subtitle

\author{\textsc{Paradox and Infinity}\\ \em Benjamin Brast-McKie} % Institution

\date{\today} % Date

%----------------------------------------------------------------------------------------

\begin{document}

\maketitle % Print the title section

\thispagestyle{empty}

%----------------------------------------------------------------------------------------


\section*{From Cantor to Russell}

  \begin{itemize}
    \item[\it Cantor's Theorem:] Recall the proof that $\abs{A} \neq \abs{\wp(A)}$.
      \item Assume there is a bijection $f: A \to \wp(A)$.
      \item Let $D=\set{a \in A: a \notin f(a)}$.
      \item Since $D \subseteq A$, we know that $D \in \wp(A)$.
      \item Since $f$ is surjective, $f(d) = D$ for some $d \in A$. 
      \item But $d \in f(d)$ \textit{iff} $d \in D$ \textit{iff} $d \notin f(d)$.
      \item This has the form $P \leftrightarrow \neg P$ which is equivalent to $P \wedge \neg P$. 
      \item Thus there is no bijection $f: A \to \wp(A)$, and so $\abs{A} \neq \abs{\wp(A)}$.
    \item[\it Universal Set:] There is no set of all sets.
      \item Suppose there were a set $U$ of all sets. 
      \item Consider the identity map $f: U \to U$.
      \item Let $R = \set{a \in U: a \notin f(a)}$.
      \item Since $R \in U$, we may ask whether $R \in R$.
      \item But $R \in R$ \textit{iff} $R \notin f(R)$ \textit{iff} $R \not \in R$. 
      \item Hence there is no set $U$ of all sets.
  \end{itemize}




\section*{Burali-Forti Paradox}

  \begin{itemize}
    \item[\it Ordinals:] There is no set of all ordinals.
      \item Suppose there were a set $\Omega$ of all ordinals. 
      \item $\Omega$ is set-transitive: if $x \in \Omega$ and $y \in x$, then $y \in \Omega$. 
      \item $\Omega$ is well-ordered: if $X \subseteq \Omega$, then some $y <_o x$ for all $x\in X$. 
        \begin{itemize}
          \item If $x$ and $y$ are ordinals, then $x <_o y$ or $y <_o x$. 
          \item Ordinals contain all of their predecessors.
        \end{itemize}
      \item So $\Omega$ is an ordinal, and hence $\Omega \in \Omega$, and so $\Omega <_o \Omega$.
      \item But $x \nless_o x$ for any ordinal $x$. 
      \item Or, observe that $\Omega <_o \Omega'$ where $\Omega' = \Omega \cup \set{\Omega}$.
      \item Hence $\Omega$ does not include all ordinals. 
  \end{itemize}




\section*{Properties Paradox}

  \begin{itemize}
    \item[\it Horse:] The property \textit{being a horse} is not a horse, i.e., does not instantiate itself. 
    \item[\it Property:] The property \textit{being a property} is a property, i.e., instantiates itself.
    \item[\it Paradox:] Let $P$ be the property of not instantiate itself, i.e., $P(X) \coloneq \neg X(X)$.
      \item But then $P(P)$ \textit{iff} $\neg P(P)$.
      \item $\exists Y[\forall Z(Z = Y \leftrightarrow \forall X[Z(X) \leftrightarrow \neg X(X)]) \wedge Y = P]$. 
      % \item Hence there is no property $P$.
  \end{itemize}





\section*{Universal Liar}

  \begin{itemize}
    \item[\it Liar:] The proposition that \textit{Liar} expresses is false.
      \item If the \textit{Liar} is true, then by its own lights it is false.
      \item If the \textit{Liar} is false, then by its own lights it is true.
    \item[\it Analysis:] $\exists \varphi(\forall \psi[\texttt{Expresses}(\textit{Liar},\psi) \leftrightarrow \varphi = \psi] \wedge \neg \varphi)$.
  \end{itemize}






\section*{Nonexistence?}

  \begin{itemize}
    \item[\it Response:] Isn't the most natural response to just deny that there is a set $R$, or property $P$, or proposition expressed by \textit{Liar}.
    \item[\it Ad Hoc:] Need to explain why there is no such set, property, or proposition.
    \item[\it Proposition:] Why doesn't \textit{Liar} express a proposition?
      \item Can't simply appeal to paradox to explain its nonexistence.
    \item[\it Properties:] Why isn't there such a property as $P$?
      \item Seems like most properties have this property, e.g., \textit{being a horse}.
    \item[\it Sets:] Why isn't there a Russell set $R$?
      \item All sets do not belong to themselves, and there is no set of all sets.
  \end{itemize}




\section*{Vicious Circle Principle}

  \begin{itemize}
    \item[\it Diagnosis:] ``No totality can contain members defined in terms of itself.'' 
      \item Want something that explains all of the ``reflexive paradoxes.''
    \item[\it Take Two:] ``Whatever contains an apparent variable must not be a possible value of that variable.''
      \item $R \coloneq \set{x : x \notin x}$ i.e., $\exists X(R = Y \wedge \forall Y[Y = X \leftrightarrow \forall z(z\in Y \leftrightarrow z \notin z)])$.
    \item[\it Types:] ``Whatever contains an apparent variable must be of a different type from the possible values of that variable\ldots'' 
  \end{itemize}

\end{document}


