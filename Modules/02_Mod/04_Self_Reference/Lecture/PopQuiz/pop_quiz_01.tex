\documentclass[a4paper,10pt,oneside,headsepline]{scrartcl}
\usepackage[ngerman]{babel}
\usepackage[utf8]{inputenc}
\usepackage{wasysym}% provides \ocircle and \Box
\usepackage{enumitem}% easy control of topsep and leftmargin for lists
\usepackage{color}% used for background color
\usepackage{forloop}% used for \Qrating and \Qlines
\usepackage{ifthen}% used for \Qitem and \QItem
\usepackage{typearea}
\areaset{17cm}{26cm}
\setlength{\topmargin}{-1cm}
% \usepackage{scrpage2}
% \pagestyle{scrheadings}
% \ihead{Example questionnaire created with \LaTeX}
% \ohead{\pagemark}
% \chead{}
% \cfoot{}

%% \Qq = Questionaire question. Oh, this is just too simple. It helps
%% making it easy to globally change the appearance of questions.
\newcommand{\Qq}[1]{\textbf{#1}}

%% \QO = Circle or box to be ticked. Used both by direct call and by
%% \Qrating and \Qlist.
% \newcommand{\QO}{$\Box$}% or: $\ocircle$
\newcommand{\QO}{$\ocircle$}% Use circles now instead of boxes.

%% \Qrating = Automatically create a rating scale with NUM steps, like
%% this: 0--0--0--0--0.
\newcounter{qr}
\newcommand{\Qrating}[1]{\QO\forloop{qr}{1}{\value{qr} < #1}{---\QO}}

%% \Qline = Again, this is very simple. It helps setting the line
%% thickness globally. Used both by direct call and by \Qlines.
\newcommand{\Qline}[1]{\noindent\rule{#1}{0.6pt}}

%% \Qlines = Insert NUM lines with width=\linewith. You can change the
%% \vskip value to adjust the spacing.
\newcounter{ql}
\newcommand{\Qlines}[1]{\forloop{ql}{0}{\value{ql}<#1}{\vskip1em\Qline{\linewidth}}}

%% \Qlist = This is an environment very similar to itemize but with
%% \QO in front of each list item. Useful for classical multiple
%% choice. Change leftmargin and topsep accourding to your taste.
\newenvironment{Qlist}{%
\renewcommand{\labelitemi}{\QO}
\begin{itemize}[leftmargin=1.5em,topsep=-.5em]
}{%
\end{itemize}
}

%% \Qtab = A "tabulator simulation". The first argument is the
%% distance from the left margin. The second argument is content which
%% is indented within the current row.
\newlength{\qt}
\newcommand{\Qtab}[2]{
\setlength{\qt}{\linewidth}
\addtolength{\qt}{-#1}
\hfill\parbox[t]{\qt}{\raggedright #2}
}

%% \Qitem = Item with automatic numbering. The first optional argument
%% can be used to create sub-items like 2a, 2b, 2c, ... The item
%% number is increased if the first argument is omitted or equals 'a'.
%% You will have to adjust this if you prefer a different numbering
%% scheme. Adjust topsep and leftmargin as needed.
\newcounter{itemnummer}
\newcommand{\Qitem}[2][]{% #1 optional, #2 notwendig
\ifthenelse{\equal{#1}{}}{\stepcounter{itemnummer}}{}
\ifthenelse{\equal{#1}{a}}{\stepcounter{itemnummer}}{}
\begin{enumerate}[topsep=2pt,leftmargin=2.8em]
\item[\textbf{\arabic{itemnummer}#1.}] #2
\end{enumerate}
}

%% \QItem = Like \Qitem but with alternating background color. This
%% might be error prone as I hard-coded some lengths (-5.25pt and
%% -3pt)! I do not yet understand why I need them.
\definecolor{bgodd}{rgb}{0.8,0.8,0.8}
\definecolor{bgeven}{rgb}{0.9,0.9,0.9}
\newcounter{itemoddeven}
\newlength{\gb}
\newcommand{\QItem}[2][]{% #1 optional, #2 notwendig
\setlength{\gb}{\linewidth}
\addtolength{\gb}{-5.25pt}
\ifthenelse{\equal{\value{itemoddeven}}{0}}{%
\noindent\colorbox{bgeven}{\hskip-3pt\begin{minipage}{\gb}\Qitem[#1]{#2}\end{minipage}}%
\stepcounter{itemoddeven}%
}{%
\noindent\colorbox{bgodd}{\hskip-3pt\begin{minipage}{\gb}\Qitem[#1]{#2}\end{minipage}}%
\setcounter{itemoddeven}{0}%
}
}

%%%%%%%%%%%%%%%%%%%%%%%%%%%%%%%%%%%%%%%%%%%%%%%%%%%%%%%%%%%%
%% End of questionnaire command definitions %%
%%%%%%%%%%%%%%%%%%%%%%%%%%%%%%%%%%%%%%%%%%%%%%%%%%%%%%%%%%%%

\begin{document}

\begin{center}
\textbf{\huge Pop Quiz \#1}
\end{center}\vskip1em

\textit{Please answer the following and return to your TA.}

\bigskip

\Qq{Your Name}: \Qline{8cm}

\bigskip

\Qq{Student Number}: \Qline{8cm}

\bigskip

\Qitem{ \Qq{Question:} Is it possible to consistently assign a single truth-value to the following $\omega$-liar sequence which includes the existential quantifier `some' in place of `every'?
  \hskip0.5cm \QO{} Yes \hskip0.5cm \QO{} No
  \begin{Qlist}
    \item[(1)] $s_1$: $s_n$ is false for some $n>1$. 
    \item[(2)] $s_2$: $s_n$ is false for some $n>2$. 
    \item[(3)] $s_3$: $s_n$ is false for some $n>3$. 
    \item[] $\vdots$
  \end{Qlist}
}

\Qitem{ \Qq{Question:} If you answered `Yes', provide an assignment of truth-values. If you answered `No', briefly explain why you think there is no consistent assignment of truth-values.}

\Qlines{4}\\


\bigskip

\textit{Please evaluate the following ancient paradoxes (no justification needed).}

\bigskip

\Qitem{ In your estimation, how does \Qq{Zeno's Paradox} fair by the following rubric?\\
  \begin{itemize}[leftmargin=1.75in]
    \item[Pointless] \Qrating{10} Informative
    \item[Hard to feel the force of] \Qrating{10} Vivid
    \item[Easy to analyze] \Qrating{10} Difficult
    \item[Obvious solution] \Qrating{10} Controversial
    \item[Silly] \Qrating{10} Interesting
  \end{itemize}
}

\bigskip

\Qitem{ In your estimation, how does \Qq{The Liar Paradox} fair by the following rubric?\\
  \begin{itemize}[leftmargin=1.75in]
    \item[Pointless] \Qrating{10} Informative
    \item[Hard to feel the force of] \Qrating{10} Vivid
    \item[Easy to analyze] \Qrating{10} Difficult
    \item[Obvious solution] \Qrating{10} Controversial
    \item[Silly] \Qrating{10} Interesting
  \end{itemize}
}

\bigskip

\Qitem{ \Qq{Question:} What is your favorite paradox that we have covered so far?}



\end{document}
