\documentclass[a4paper, 11pt]{article} % Font size (can be 10pt, 11pt or 12pt) and paper size (remove a4paper for US letter paper)

\usepackage[protrusion=true,expansion=true]{microtype} % Better typography
\usepackage{graphicx} % Required for including pictures
\usepackage{wrapfig} % Allows in-line images
\usepackage{enumitem} %%Enables control over enumerate and itemize environments
\usepackage{setspace}
\usepackage{amssymb, amsmath, mathrsfs,mathabx} %%Math packages
\usepackage{stmaryrd}
\usepackage{mathtools}
\usepackage{multicol} 
\usepackage{mathpazo} % Use the Palatino font
\usepackage[T1]{fontenc} % Required for accented characters
\usepackage{array}
\usepackage{bibentry}
\usepackage{prooftrees} 
\usepackage[round]{natbib} %%Or change 'round' to 'square' for square backers
\setcitestyle{aysep=}
% \usepackage{fitchproof} 

% \linespread{1.05} % Change line spacing here, Palatino benefits from a slight increase by default

\newcommand{\tuple}[1]{\langle#1\rangle} %%Angle brackets
\newcommand{\corner}[1]{\ulcorner#1\urcorner} %%Angle brackets
\newcommand{\set}[1]{\lbrace#1\rbrace} %%Set brackets
\newcommand{\abs}[1]{|#1|} %%Set brackets
\newcommand{\interpret}[1]{\llbracket#1\rrbracket} %%Double brackets
\newcommand{\N}{\mathbb{N}}
\renewcommand{\L}{\mathcal{L}}
\newcommand{\D}{\mathbb{D}}
\newcommand{\Z}{\mathbb{Z}}
\renewcommand{\Pr}{\mathbb{P}}
\newcommand{\Q}{\mathbb{Q}}
\newcommand{\R}{\mathbb{R}}
\newcommand{\B}{\mathfrak{B}}
\renewcommand{\max}[1]{\texttt{max}\set{#1}}

\makeatletter
\renewcommand\@biblabel[1]{\textbf{#1.}} % Change the square brackets for each bibliography item from '[1]' to '1.'
\renewcommand{\@listI}{\itemsep=0pt} % Reduce the space between items in the itemize and enumerate environments and the bibliography

\renewcommand{\maketitle}{ % Customize the title - do not edit title and author name here, see the TITLE block below
\begin{flushright} % Right align
{\LARGE\@title} % Increase the font size of the title

\vspace{10pt} % Some vertical space between the title and author name

{\@author} % Author name
\\\@date % Date

\vspace{-20pt} % Some vertical space between the author block and abstract
\end{flushright}
}

%----------------------------------------------------------------------------------------
%	TITLE
%----------------------------------------------------------------------------------------

\title{\textbf{Type Theory}} % Subtitle

\author{\textsc{Paradox and Infinity}\\ \em Benjamin Brast-McKie} % Institution

\date{\today} % Date

%----------------------------------------------------------------------------------------

\begin{document}

\maketitle % Print the title section

\thispagestyle{empty}

%----------------------------------------------------------------------------------------


\section*{Against Ramification}

  \begin{itemize}
    % \item[\it Typed Expresions:] Recall the recursive clause for quantification from Russell's RTT:
    %   \item If $\varphi : (t_1^{o_1},\ldots,t_n^{o_n})^a$, then $\forall x : t_i^{o_i} \varphi : (t_1^{o_1},t_{i-1}^{o_{i-1}},\ldots,t_{i+1}^{o_{i+1}},t_n^{o_n})^a$.
    %   \item Observe that $\forall x : t_i^{o_i} \varphi$ is of \textit{higher order}. 
    \item[\it Orders:] Ramsey rejects Russell's divisions of the types into orders.
    \item[\it Autological:] A predicate is \textit{autological iff} it expresses a property it has.
    \item[\it Heterological:] A predicate is \textit{heterological iff} it is not autological.
      \item Whereas `short' is autological, `long' is heterological.
      \item Is `heterological' heterological?
      \item Use/mention distinction.
    \item[\it Solution:] In the style of \textit{Principia Mathematica} we have:
      \item $H(w)$ \textit{iff} there is a property $P$ where $w$ expresses $P$ and $\neg P(w)$.
      \item The variable `$P$' ranges over first-order properties which $H$ is not. 
      \item ``This theory of a hierarchy of orders of functions of individuals escapes the contradictions; but it lands us in an almost equally serious difficulty, for it invalidates many important mathematical arguments\ldots''
  \end{itemize}




\section*{Simple Type Theory}

  \begin{itemize}
    \item[\it Logical:] Ramsey accepts Russell's solution to the logical paradoxes.
    \item[\it Simple Types:] Recall the function application clause from the simple theory of types:
      \item If $\varphi_1 : t_1$, \ldots, $\varphi_n : t_n$, then $z(\varphi_1,\ldots,\varphi_n) : ((t_1,\ldots,t_n),t_1,\ldots,t_n)$.
      \item Observe that $z$ is of \textit{higher type} than its arguments. 
    \item[\it Self-Application:] A predicate, ``cannot significantly take itself as argument\ldots''
      \item Neither $P(P)$ nor $\neg P(P)$ can be simply typed, and so meaningless. 
      \item This case differs from \textit{heterological} since no semantic terms occur.
      % \item What about Russell's paradox?
    \item[\bf Question:] Is there any self application in $R \coloneq \set{ x : \neg \texttt{In}(x,x) }$?
    \item[\it Sets:] Russell identifies classes (sets) with functions (properties). 
      \item `$x \in y$' is notational variant of `$y(x)$'.
      \item Thus `$x \in x$' abbreviates `$x(x)$' which cannot be simply typed.
    \item[\bf Question:] Why not respond by faulting the identification of sets with properties?
      \item Blocking this identification further undoes Russell's solution.
      \item But it does not solve the paradox.
      \item Ramsey thought we should leave this part of Russell's solution.
  \end{itemize}




\section*{Dividing the Paradoxes}

  \begin{itemize}
    \item[\it Russell:] Solves all the paradoxes by accepting RTT + AR.
    \item[\it Ramsey:] Rejects AR, and also rejects RTT to preserve mathematics.
      \item Ramsey divides the paradoxes into the logical and semantic.
      \item The semantic paradoxes include naming, expressing, meaning, etc.
      \item The semantic paradoxes concern the interpretation of language.
      \item Ramsey thought they deserved their own solution.
    \item[\it Tarski:] Recall Tarski's levels of languages.
      \item Could distinguish between true$_1$, true$_2$, etc.
      \item The semantic terms for a language cannot belong to that language.
      \item The object/metalanguage distinction is widely accepted.
    \item[\it Truth:] What of a theory of truth, meaning, reference, etc.?
      \item The object/metalanguage distinction doesn't provide a theory of truth.
      \item Nor does it provide a theory of meaning, or reference, etc.
      \item Lots remains to be done to understand how language works.
  \end{itemize}






\section*{Higher-Order Logic}

  \begin{itemize}
    \item[\it Mathematics:] Truth theories remain controversial but mathematics is preserved.
      \item The logical paradoxes are solved.
      \item We also get the theory of simple types as a result.
    \item[\it Higher-Order:] Since talk of orders has been banished, we may reclaim the term.
      \item Today, `order' refers to the type of the variable bound by a quantifier. 
      \item The quantifiers of \textit{first-order logic} bind variables in name position.
      \item The quantifiers of \textit{higher-order logics} bind variables of higher type.
    \item[\it Quine:] Thought higher-order logic was ``set theory in sheep's clothing.''
      \item Instead of recasting sets as properties, \textit{first-orderists} go the other way.
      \item But then how are we to escape Russell's paradox (this is for next week).
    \item[\it Types:] Consider the following defense of higher-order logic:
      \item Properties, relations, etc., are not first-order objects.
      \item If we recognize type distinctions, why only first-order quantifiers?
      \item If there are some higher-order things, why not quantify over them?
    \item[\it Controvercy:] Quine's shadow remains long in philosophy but not computer science.
  \end{itemize}








\end{document}



