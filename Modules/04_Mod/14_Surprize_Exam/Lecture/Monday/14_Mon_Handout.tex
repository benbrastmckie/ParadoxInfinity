\documentclass[a4paper, 11pt]{article} % Font size (can be 10pt, 11pt or 12pt) and paper size (remove a4paper for US letter paper)

\usepackage[protrusion=true,expansion=true]{microtype} % Better typography
\usepackage{graphicx} % for adding extra padding to rows
\usepackage{wrapfig} % Allows in-line images
\usepackage{enumitem} %%Enables control over enumerate and itemize environments
\usepackage{setspace}
\usepackage{amssymb, amsmath, mathrsfs,mathabx} %%Math packages
\usepackage{stmaryrd}
\usepackage{mathtools}
\usepackage{multicol} 
\usepackage{mathpazo} % Use the Palatino font
\usepackage[T1]{fontenc} % Required for accented characters
\usepackage{array}
\usepackage{bibentry}
\usepackage{prooftrees} 
\usepackage[round]{natbib} %%Or change 'round' to 'square' for square backers
\setcitestyle{aysep=}
% \usepackage{fitchproof} 

% \linespread{1.05} % Change line spacing here, Palatino benefits from a slight increase by default

\DeclareSymbolFont{symbolsC}{U}{txsyc}{m}{n}
\SetSymbolFont{symbolsC}{bold}{U}{txsyc}{bx}{n}
\DeclareFontSubstitution{U}{txsyc}{m}{n}
\DeclareMathSymbol{\boxright}{\mathrel}{symbolsC}{"80}
\DeclareMathSymbol{\circleright}{\mathrel}{symbolsC}{"91}
\DeclareMathSymbol{\diamondright}{\mathrel}{symbolsC}{"84}
\DeclareMathSymbol{\medcirc}{\mathrel}{symbolsC}{"07}

\newcommand{\tuple}[1]{\langle#1\rangle} %%Angle brackets
\newcommand{\corner}[1]{\ulcorner#1\urcorner} %%Angle brackets
\newcommand{\set}[1]{\lbrace#1\rbrace} %%Set brackets
\newcommand{\abs}[1]{|#1|} %%Set brackets
\newcommand{\interpret}[1]{\llbracket#1\rrbracket} %%Double brackets
\newcommand{\N}{\mathbb{N}}
\renewcommand{\L}{\mathcal{L}}
\renewcommand{\O}{\mathcal{O}}
\newcommand{\A}{\mathcal{A}}
\newcommand{\B}{\mathcal{B}}
\newcommand{\D}{\mathbb{D}}
\newcommand{\Z}{\mathbb{Z}}
\renewcommand{\Pr}{\mathbb{P}}
\newcommand{\Q}{\mathbb{Q}}
\newcommand{\R}{\mathbb{R}}
\newcommand{\B}{\mathfrak{B}}
\renewcommand{\max}[1]{\texttt{max}\set{#1}}

\makeatletter
\newcommand{\superimpose}[2]{%
  {\ooalign{$#1\@firstoftwo#2$\cr\hfil$#1\@secondoftwo#2$\hfil\cr}}}
\makeatother

\newcommand{\past}{\mathpalette\superimpose{{\Diamond}{\raisebox{1.5pt}{\tiny \hspace{.4pt}\textsc{p}}}}}

\newcommand{\Past}{\mathpalette\superimpose{{\Box}{\raisebox{1.2pt}{\tiny \textsc{p}}}}}

\newcommand{\future}{\mathpalette\superimpose{{\Diamond}{\raisebox{1.5pt}{\tiny \textsc{f}}}}}

\newcommand{\Future}{\mathpalette\superimpose{{\Box}{\raisebox{1.2pt}{\tiny \textsc{f}}}}}

\newcommand{\always}{\ensuremath \raisebox{1.3pt}{\rotatebox[origin=c]{180}{$\triangle$}}}

\newcommand{\sometimes}{\ensuremath \raisebox{-1.3pt}{$\triangle$}}

\makeatletter
\renewcommand\@biblabel[1]{\textbf{#1.}} % Change the square brackets for each bibliography item from '[1]' to '1.'
\renewcommand{\@listI}{\itemsep=0pt} % Reduce the space between items in the itemize and enumerate environments and the bibliography

\renewcommand{\maketitle}{ % Customize the title - do not edit title and author name here, see the TITLE block below
\begin{flushright} % Right align
{\LARGE\@title} % Increase the font size of the title

\vspace{10pt} % Some vertical space between the title and author name

{\@author} % Author name
\\\@date % Date

\vspace{-30pt} % Some vertical space between the author block and abstract
\end{flushright}
}

%----------------------------------------------------------------------------------------
%	TITLE
%----------------------------------------------------------------------------------------

\title{\textbf{Surprise Exam Paradox}} % Subtitle

\author{\textsc{Paradox and Infinity}\\ \em Benjamin Brast-McKie} % Institution

\date{\today} % Date

%----------------------------------------------------------------------------------------

\begin{document}

\maketitle % Print the title section

\thispagestyle{empty}

%----------------------------------------------------------------------------------------

\section*{The Exam}

\begin{itemize}
  \item[\it Setup:] A single surprise exam is announced for next week ($9$am $m, w$, or $f$).
    \item Let `$E_i$' read `The exam occurs on $i$' where $i \in \set{m, w, f}$.
    \item Let `$\B_i(A)$' read `The students believe $A$ at $9$am on $i$'.
    \item $E_i$ is a \textit{surprise iff} $E_i \wedge \neg \B_i(E_i)$.
    \item Let $S = (E_m \wedge \neg \B_m(E_m)) \vee (E_w \wedge \neg \B_w(E_w)) \vee (E_f \wedge \neg \B_f(E_f))$.
    \item The students believe this announcement $S$ throughout the week.
  \item[\it Closure:] If $\B_i(A)$ for all $A \in \Gamma$ and $\Gamma \vdash B$, then $\B_i(B)$.
    \item We only need limited instances of \textit{Closure} to hold.
  \item[\it Informed:] The students learn each day if there is an exam, forming a true belief.
  \item[\it Memory:] The students maintain their beliefs from the previous days.
  % \item[\it Introspection:] Both $\B_i(A) \vdash \B_i(\B_i (A))$ and $\neg \B_i(A) \vdash \B_i(\neg \B_i (A))$.
    % \item These principles needn't hold universally, just for the cases here.
  \item[\it Friday:] On Monday $8$am, the students reason as follows:
    \item If $E_f$, then $\neg E_m$ and $\neg E_w$, so $\B_m(\neg E_m)$ and $\B_w(\neg E_w)$ by \textit{Informed}.
    \item So $\B_f(\neg E_m)$ and $\B_f(\neg E_w)$ by \textit{Memory}, where $\B_f(S)$ is a premise. 
    \item But $S, \neg E_m, \neg E_w \vdash E_f$, and so $\B_f(E_f)$ by \textit{Closure}. 
    % \item Also $S, \neg E_m, \neg E_w \vdash \neg \B_f(E_f)$, and so $\B_f(\neg \B_f(E_f))$ by \textit{Closure}. 
    % \item Since $\B_f(\B_f(E_f))$ by \textit{Introspection}, $\B_f(\bot)$.
    \item Thus $E_f$ is not a surprise, i.e., $\neg \neg \B_f(E_f)$, and so $\neg (E_f \wedge \neg \B_f(E_f))$.
    \item In this way, the students come to $\B_m(\neg (E_f \wedge \neg \B_f(E_f)))$.
    \item However, $S, \neg (E_f \wedge \neg \B_f(E_f)) \vdash (E_m \wedge \neg \B_m(E_m)) \vee (E_w \wedge \neg \B_w(E_w))$.
    \item By \textit{Closure}, $\B_m(S')$ where $S' = (E_m \wedge \neg \B_m(E_m)) \vee (E_w \wedge \neg \B_w(E_w))$.
  \item[\it Wednesday:] The students (on Monday 8:05am) turn to reason about Wednesday:
    \item If $E_w$, then $\neg E_m$, so $\B_m(\neg E_m)$ by \textit{Informed} and $\B_w(\neg E_m)$ by \textit{Memory}.
    \item However, $\B_m(S')$ by \textit{Friday}, and so $\B_w(S')$ by \textit{Memory}.
    \item Since $S', \neg E_m \vdash E_w$, it follows by \textit{Closure} that $\B_w(E_w)$.
    % \item By repeating the \textit{Friday} argument on Wednesday morning, $\B_w(\neg E_f)$.
    % \item However, $\B_w(E_m \vee E_w \vee E_f)$, and so $\B_w(E_w)$ by \textit{Closure}.
    \item Thus if $E_w$, then $E_w$ is not a surprise, and so  $\neg (E_w \wedge \neg \B_w(E_w))$.
    \item In this way, the students come to $\B_m(\neg (E_w \wedge \neg \B_w(E_w)))$.
    \item However, $S', \neg (E_w \wedge \neg \B_w(E_w)) \vdash E_m \wedge \neg \B_m(E_m)$.
    \item By \textit{Closure}, $\B_m(S'')$ where $S'' = E_m \wedge \neg \B_m(E_m)$. 
  \item[\it Monday] The students now turn to consider Monday (still on Monday 8:10am):
    % \item However, they come to realize that their belief $\B_m (S'')$ is very strange.
    \item $\B_m(S'')$ entails $\B_m(E_m)$ and $\B_m(\neg \B_m(E_m))$. %, so $\B_m(E_m) \nvdash \B_m(\B_m(E_m))$.
    % \item But it would seem that the students can introspect about $\B_m(E_m)$.
    \item $\B_m(E_m) \vdash \B_m(\B_m(E_m))$ leads to believing a contradiction.
    \item So the students would seem to have reason to reject $\B_m(S'')$.
\end{itemize}






\section*{Moore's Problem}

\begin{itemize}
  \item[\it Rain:] It is raining $(R)$ but I do not believe that it is raining $\neg \B (R)$.
    \item Can be true, but can't be asserted (normally).
    \item Can \textit{assert} either $R$ or $\neg \B(R)$, but not both. 
    \item OK to assert: It is raining but \textit{you} do not believe that it is raining.
  \item[\it Belief Norm:] Don't assert what you don't yourself believe (in normal circumstances).
    \item One could appeal to this norm to infer $\B(R)$ from an assertion of $R$.
    \item Similarly, $\B(\neg \B(R))$ can be inferred from an assertion of $\neg \B(R)$.
    \item Can \textit{believe} $R$ or $\neg \B(R)$, but not both?
  \item[\it Introspection:] The following introspection principles have many true instances.
    \vspace{-.05in}
    % \begin{itemize}
      \begin{multicols}{2}
        \item[] (\textit{Positive})~~ $\B (A) \vdash \B(\B (A))$.
        \item[] (\textit{Negative})~~ $\neg \B (A) \vdash \B(\neg \B (A))$.
      \end{multicols}
    % \end{itemize}
    \vspace{-.1in}
    \item Nothing seems to block introspection for $A = E_m$.
    \item As above, $\B_m(S)$ entails $\B_m(E_m)$ and $\B_m(\neg \B_m(E_m))$. 
    \item So $\B_m(\B_m(E_m))$ follows by \textit{Positive Introspection}. 
    \item Moreover $\B_m(E_m), \neg \B_m(E_m) \vdash \B_m(E_m) \wedge \neg \B_m(E_m)$.
    \item Hence $\B_m(\B_m(E_m) \wedge \neg \B_m(E_m))$ follows by \textit{Closure}. 
    \item But $\B_m(E_m) \wedge \neg \B_m(E_m)$ is a contradiction. 
  \item[\it Contradiction:] Don't believe contradictions (revise your beliefs accordingly).
    \item Since $\B_m(S) \vdash \B_m(\B_m(E_m) \wedge \neg \B_m(E_m))$, we get $\neg \B_m(S)$.
    \item But the students are able to believe that there will be a surprise exam.
  \item[\it Blindspot:] Is the paradox solved by claiming that it is impossible for $\B_m(S)$?
    \item Is it still possible for $S$ to be true? 
\end{itemize}







% \section*{A Comparison}
%
% \begin{itemize}
%   \item[\it Prisoners:] The prisoner's from before play three rounds $P_m$, $P_w$, and $P_f$.
%     \item Assume it is rational not to cooperate in a single round.
%     \item Then it is rational not to cooperate in the final round $P_f$.
%     \item Both prisoners assume as much about each other on $P_f$.
%     \item Given that $P_f$ is settled, shouldn't the same be true for $P_w$ and $P_m$? 
%   \item[\it Cooperation:] Is there a better strategy?
%     \item Begin by baiting the other prisoner into cooperating by not taking.
%     \item If they also don't take, continue the strategy.
%     \item If they do take, try again?
%     \item Even if they don't take, what do you do on the last round?
%     \item But then if both take on the last, why not also on the second to last?
%   \item[\it Rational:] Can the cooperation strategy be rational at any round?
%     \item How similar is this case to the surprise exam paradox?
% \end{itemize}







% \bibliographystyle{Phil_Review} %%bib style found in bst folder, in bibtex folder, in texmf folder.
% \nobibliography{Zotero} %%bib database found in bib folder, in bibtex folder
\end{document}
