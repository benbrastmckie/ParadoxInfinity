\documentclass[11pt]{article}

\usepackage{amsmath,amsthm,amsfonts,amssymb,amscd}

\usepackage[margin=1.5in,headsep=.5in]{geometry}

\usepackage{fancyhdr}

\setlength{\headheight}{20pt}

\usepackage[colorlinks]{hyperref} 
\usepackage{cleveref}


\theoremstyle{definition}
\newtheorem{defn}{Definition}
\newtheorem{reg}{Rule}
\newtheorem{exer}{Exercise}
\newtheorem{note}{Note}
\newtheorem*{theorem*}{Theorem}
\newtheorem{theorem}{Theorem}[section]
\newtheorem{corollary}{Corollary}[theorem]
\newtheorem{thm}{Theorem}
\newtheorem{prop}[thm]{Proposition}
\newtheorem{lem}[thm]{Lemma}
\newtheorem{conj}[theorem]{Conjecture}

\pagestyle{fancy}

\begin{document}

\pagenumbering{gobble}

\lhead{Xinhe Wu (xinhewu@mit.edu)}
\rhead{$24.118$ Paradox and Infinity $|$ Recitation $1$}



\begin{center}
{\LARGE \bf Cantor's Theorem and More}
\end{center}

\smallskip

\section{Cantor's Theroem}

\begin{theorem}
Let $A$ be a set. $|A| < |\wp(A)|$.
\end{theorem}

\begin{corollary} \label{Corollary 1.1.1}
There is no set of all sets.
\end{corollary}
\begin{proof}
Suppose there is a set of all sets $U$. Consider $\wp(U)$. Since every member of $\wp(U)$ is a set, $\wp(U) \subseteq U $. Hence $|\wp(U)| \leqslant |U|$. Hence we have
$$ |\wp(U)| \leqslant |U| \leqslant |\wp(U)| $$
Applying the Schr\"{o}der–Bernstein Theorem, we get $|U|=|\wp(U)|$, against Cantor's Theorem.
\end{proof}

\section{Cardinality Exercises}

\begin{exer}
Let $S$ be the set of all finite sequences of natural numbers; that is, $S = \{ <a_1, a_2, ..., a_n> : a_1, a_2, ..., a_n \in \mathbb{N} \}$. Show that $|S|=|\mathbb{N}|$.
\end{exer}

\begin{exer}
Show that $|\mathbb{R}|=|(a,b)|$ for any $a, b\in \mathbb{R}$.
\end{exer}

\begin{exer}
The Cartesian product $X \times Y$ for two sets $X$ and $Y$ is the set of all ordered pairs $<x,y>$ such that $x \in X$ and $y \in Y$.
Show that $|\mathbb{R}|=|(a,b)\times(a,b)|$ for any $a, b\in \mathbb{R}$.
\end{exer}

\section{Russell's Paradox}

\begin{conj} \label{Conjecture 3.1}
Let $\phi (x)$ be a formula containing $x$ as the only free variable. For any $\phi(x)$, there is a set $\{ x: \phi(x) \}$.
\end{conj}

\begin{theorem}
\hyperref [Conjecture 3.1] {Conjecture 3.1} leads to a contradiction.
\end{theorem}

\begin{proof}
Let $\phi(x) = x \notin x$. Let $R=\{x:x \notin x\}$. Then $R \in R \leftrightarrow R \notin R$.
\end{proof}

\begin{exer}
Show that \hyperref [Conjecture 3.1] {Conjecture 3.1} is also in conflict with \hyperref [Corollary 1.1.1] {Corollary 1.1.1}.
\end{exer}

\noindent
Discussion: Facing Russell's paradox, how should we think of the concept of set?

\end{document}