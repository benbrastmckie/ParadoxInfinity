\documentclass[justified]{tufte-handout} 
\usepackage{amsfonts, amssymb, stmaryrd, natbib, qtree, amsxtra}
\usepackage{linguex, color, setspace, graphicx}
\usepackage{enumitem}
\usepackage{bussproofs}
\usepackage{turnstile}
\usepackage{phaistos}
\usepackage{protosem}
\usepackage{txfonts}
\usepackage{pxfonts}
\usepackage[super]{nth}
\thispagestyle{plain}
\definecolor{darkred}{rgb}{0.7,0,0.2}
\bibpunct{(}{)}{,}{a}{}{,}

\input xy
 \xyoption{all}

%New Symbols
\DeclareSymbolFont{symbolsC}{U}{txsyc}{m}{n}
\DeclareMathSymbol{\strictif}{\mathrel}{symbolsC}{74}
\DeclareMathSymbol{\boxright}{\mathrel}{symbolsC}{128}
\DeclareMathSymbol{\Diamondright}{\mathrel}{symbolsC}{132}
\DeclareMathSymbol{\Diamonddotright}{\mathrel}{symbolsC}{134}
\DeclareMathSymbol{\Diamonddot}{\mathord}{symbolsC}{144}
\renewcommand{\labelitemi}{$\triangleright$}
\renewcommand{\labelitemii}{$\circ$}
\renewcommand{\labelitemiii}{$\triangleright$}

%New commands
\newcommand{\bitem}{\begin{itemize}}
\newcommand{\eitem}{\end{itemize}}
\newcommand{\lang}{$\langle$}
\newcommand{\rang}{$\rangle$}
\newcommand{\back}{$\setminus$}
\newcommand{\HRule}{\rule{\linewidth}{0.1mm}}
\newcommand{\llm}[2][]{$\llbracket${#2}$\rrbracket^{#1}$}
\newcommand{\ul}{$\ulcorner$}
\newcommand{\ur}{$\urcorner\ $}
\newcommand{\urn}{$\urcorner$}
\newcommand{\sub}[1]{\textsubscript{#1}}
\newcommand{\sups}[1]{\textsuperscript{#1}}
\newtheorem{proposition}{\textbfb{Proposition}}[section]
\newtheorem{definition}[proposition]{\textbf{Definition}}
\newcommand{\bfw}{\begin{fullwidth}}
\newcommand{\efw}{\end{fullwidth}}

\begin{document}

\frenchspacing

\begin{fullwidth}
\noindent\Large Section 1, Infinity \large \\[.3cm]
\noindent  David Boylan \hfill{11-12, 66-154}

\noindent\HRule
\end{fullwidth}

\section{Administrative Whatnots}

\begin{itemize}


\item My office hours will be 12-1 on Wednesdays. My office is 32-D912. I'm also more than happy to schedule meetings outside of office hours.

\item All requests for extensions should go through $S^3$. Please speak to them first if you feel your circumstances require an extension.


\item Late problem sets will not be accepted. 


\item All grading will be anonymous.\marginnote{If for whatever reason you're not comfortable giving me your student number, feel free to make up some other nine digit number that I can use to keep track of you.} Please include only your MIT ID on your problem sets and do not write your name. I'll shortly send around an email asking for your student numbers.



\end{itemize}



\section{Size and Cardinalities}


\begin{itemize}

\item Let's go over some definitions: 



\begin{itemize}

\item When is a relation reflexive? 


\item Transitive? 


\item Antisymmetric? 


\item Give an example of each kind of relation.

\end{itemize}




\item Let $A =\{Alice, Bob, Carol\}$ and $B=\{Daniel, Ellie, Fred\}$


\begin{itemize}

\item When is a $f: A\rightarrow B$ injective? 

\item Surjective?

\item Bijective?

\item Give an example of such a function in every case. (And give different examples each time!)

\end{itemize}


\item Let's verify a few properties of bijections: 


\begin{itemize}

\item For any $A$, there's a bijection from $A$ to $A$. Why?

\item If there is a bijection from $A$ to $B$ then there is a bijection from $B$ to $A$. Why?

\item If there is a bijection from $A$ to $B$ and a bijection from $B$ to $C$, then there's a bijection from $A$ to $C$. Why?


\end{itemize}


\end{itemize}
\section{The Cardinality of the Reals}

\begin{itemize}

\item In lecture it was stated that $|N| \neq |R|$. Let's prove it now. 

\item In fact, what we will actually prove is that $|N| \neq |[0,1)|$. (Why does this suffice to show $|N| \neq |R|$?)

\item The proof is by \emph{reductio}: we suppose there were a bijection and we show that we can derive a contradiction.

Suppose we did have such a bijection. Then we would be able to draw a two-by-two grid (see the board).

\item But we can use this table to show that there must be some number that was left off the list, so to speak. 

By laying things out on a grid, we can see that there will be a \emph{diagonal} number.

\item We can now do something to that number to ensure that the result will \emph{not} be on our list

Any ideas?


\item But then contradiction! We supposed there were a bijection and then showed there was a number left off the list, in which case it is not a bijection after all.


\end{itemize}

\section{Some Useful Results}

\begin{itemize}


\item Where $A$ and $B$ are countable, is $A\cup B$ countable? Why or why not?


\item In fact, we can prove that a countable union of countable sets is itself countable.


\item We can also prove that where $A$ is not countable but $B$ is, then $|A|=|A\cup B|$.

Let's walk through this:

\begin{itemize}

\item Let $A_N$ be a countable subset of $A$. Why is $|A_N\cup B|$ countable?



\item So we've shown $|A_N\cup B| = |A_N|$. We know that $|A_N| < |A|$. (Why?)

What follows from this?


\end{itemize}




\end{itemize}


\section{Binary Notation}

\begin{itemize}

\item We can represent real numbers using binary notation. We'll just focus on representing the numbers in $[0,1]$ with binary numbers.

\item A ($B_0$) binary sequence is an infinite sequence of the form $0.d_1d_2d_3...$ where, for all $n$, $d_n$ is 1 or 0.


In general, the sequence $0.d_1d_2d_3...$ represents the number $\frac{d_1}{2^1}+\frac{d_2}{2^2} + \frac{d_3}{2^3} + \Sigma^{\infty}_{n=4}(\frac{d_n}{2^n})$
 


\item \emph{Fact:} The binary expansions of rational numbers are always periodic: they end end with an infinitely repeating (i.e. finite) string of digits.





\item \emph{Further fact:} The only real numbers with multiple binary expansions are ones that have a binary expansion ending in an infinite sequence of 1s. Such numbers always have exactly two names: one ending in an infnite sequence of 1s and one ending in an infinite sequence of 0s.



\item $|B_0| = |[0,1]|.$ Can you show why this might be?

\end{itemize}




\end{document}







