\documentclass[11pt]{article}

\usepackage{amsmath,amsthm,amsfonts,amssymb,amscd}

\usepackage[margin=1.5in,headsep=.5in]{geometry}

\usepackage{fancyhdr}

\setlength{\headheight}{20pt}

\usepackage[colorlinks]{hyperref} 
\usepackage{cleveref}


\theoremstyle{definition}
\newtheorem{defn}{Definition}
\newtheorem{reg}{Rule}
\newtheorem{exer}{Exercise}
\newtheorem{note}{Note}
\newtheorem*{theorem*}{Theorem}
\newtheorem{theorem}{Theorem}[section]
\newtheorem{corollary}{Corollary}[theorem]
\newtheorem{thm}{Theorem}
\newtheorem{prop}[thm]{Proposition}
\newtheorem{lem}[thm]{Lemma}
\newtheorem{conj}[theorem]{Conjecture}

\pagestyle{fancy}

\begin{document}

\pagenumbering{gobble}

\lhead{$24.118$ Paradox and Infinity }
\rhead{Recitation $1$: Infinite Cardinalities}



\begin{center}
{\LARGE \bf Cantor's Theorem and Russell's Paradox}
\end{center}

\smallskip

\section{Corollaries to Cantor's Theorem}

\begin{theorem} [Cantor's Theorem]
Let $A$ be a set. $|A| < |\wp(A)|$.
\end{theorem}

\begin{corollary} \label{Corollary 1.1.1}
There is no set of all sets.
\end{corollary}
\begin{proof}
Suppose there is a set of all sets $U$. Consider $\wp(U)$. Since every member of $\wp(U)$ is a set, $\wp(U) \subseteq U $. Hence $|\wp(U)| \leqslant |U|$. Hence we have
$$ |\wp(U)| \leqslant |U| \leqslant |\wp(U)| $$
Applying the Cantor-Schr\"{o}der–Bernstein Theorem, we get $|U|=|\wp(U)|$, contradicting Cantor's Theorem.
\end{proof}

\begin{corollary}
There is no set of everything.
\end{corollary}

\begin{corollary}
There is no set of all infinite cardinalities, i.e. there is no set $U$ such that for any infinite set $y$, there is some $x \in U$ such that $|x| = |y|$.
\end{corollary}

\begin{proof}
Suppose there is a set of all infinite cardinalities $U$. Let $T = \bigcup U$, that is, $z \in T$ just in case for some $x \in U$, $z \in x$. Then for any $x \in U$, $x \subseteq T$ and therefore $|x| \leqslant |T|$.

Consider $\wp(T)$. Since it is infinite, there is some $y \in U$ such that $|\wp(T)| = |y|$. Hence $|\wp(T)| = |y| \leqslant |T|$, contradicting Cantor's Theorem.
\end{proof}

\section{Russell's Paradox}

\subsection{The Naive Conception of Set}

\begin{conj} [The Naive Conception of Set] \label{Naive}
Let $\phi (x)$ be a property/condition (for example, the property of being a person, or the property of being a number). For any $\phi(x)$, there is a set $\{ x: \phi(x) \}$, the set that consists of and only of all things that satisfy $\phi$.
\end{conj}

\begin{theorem}
\hyperref [Naive] {The Naive Conception of Set} leads to a contradiction.
\end{theorem}

\begin{proof}
Let $\phi(x) = x \notin x$. Let $R=\{x:x \notin x\}$. Then $R \in R \leftrightarrow R \notin R$.
\end{proof}

Can you Show that \hyperref [Naive] {The Naive Conception of Set} is also in conflict with the three corollaries to Cantor's Theorem?

\subsection{A Folklore Version of Russell's Paradox}

\begin{quote}
That contradiction [Russell's paradox] is extremely interesting. You can modify its form; some forms of modification are valid and some are not. I once had a form suggested to me which was not valid, namely the question whether the barber shaves himself or not. You can define the barber as "one who shaves all those, and those only, who do not shave themselves". The question is, does the barber shave himself? In this form the contradiction is not very difficult to solve. But in our previous form I think it is clear that you can only get around it by observing that the whole question whether a class is or is not a member of itself is nonsense, i.e. that no class either is or is not a member of itself, and that it is not even true to say that, because the whole form of words is just noise without meaning. \\

\hfill (Bertrand Russell, \textit{The Philosophy of Logical Atomism})
\end{quote}

\hfill

Facing Russell's paradox, how should we think of the concept of set?

\end{document}