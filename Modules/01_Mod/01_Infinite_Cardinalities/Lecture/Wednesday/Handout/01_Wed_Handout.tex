\documentclass[a4paper, 11pt]{article} % Font size (can be 10pt, 11pt or 12pt) and paper size (remove a4paper for US letter paper)

\usepackage[protrusion=true,expansion=true]{microtype} % Better typography
\usepackage{graphicx} % Required for including pictures
\usepackage{wrapfig} % Allows in-line images
\usepackage{enumitem} %%Enables control over enumerate and itemize environments
\usepackage{setspace}
\usepackage{amssymb, amsmath, mathrsfs,mathabx} %%Math packages
\usepackage{stmaryrd}
\usepackage{mathtools}
\usepackage{multicol} 
\usepackage{mathpazo} % Use the Palatino font
\usepackage[T1]{fontenc} % Required for accented characters
\usepackage{array}
\usepackage{bibentry}
\usepackage{prooftrees} 
\usepackage[round]{natbib} %%Or change 'round' to 'square' for square backers
\setcitestyle{aysep=}
% \usepackage{fitchproof} 

% \linespread{1} % Change line spacing here, Palatino benefits from a slight increase by default

\newcommand{\tuple}[1]{\langle#1\rangle} %%Angle brackets
\newcommand{\set}[1]{\lbrace#1\rbrace} %%Set brackets
\newcommand{\abs}[1]{|#1|} %%Set brackets
\newcommand{\interpret}[1]{\llbracket#1\rrbracket} %%Double brackets
\newcommand{\N}{\mathbb{N}}
\newcommand{\D}{\mathbb{D}}
\newcommand{\Z}{\mathbb{Z}}
\newcommand{\Q}{\mathbb{Q}}
\newcommand{\R}{\mathbb{R}}

\makeatletter
\renewcommand\@biblabel[1]{\textbf{#1.}} % Change the square brackets for each bibliography item from '[1]' to '1.'
\renewcommand{\@listI}{\itemsep=0pt} % Reduce the space between items in the itemize and enumerate environments and the bibliography

\renewcommand{\maketitle}{ % Customize the title - do not edit title and author name here, see the TITLE block below
\begin{flushright} % Right align
{\LARGE\@title} % Increase the font size of the title

\vspace{10pt} % Some vertical space between the title and author name

{\@author} % Author name
\\\@date % Date

\vspace{-10pt} % Some vertical space between the author block and abstract
\end{flushright}
}

%----------------------------------------------------------------------------------------
%	TITLE
%----------------------------------------------------------------------------------------

\title{\textbf{Infinite Cardinalities}} % Subtitle

\author{\textsc{Paradox and Infinity}\\ \em Benjamin Brast-McKie} % Institution

\date{\today} % Date

%----------------------------------------------------------------------------------------

\begin{document}

\maketitle % Print the title section

\thispagestyle{empty}

%----------------------------------------------------------------------------------------


\section*{Cardinality Principles}

\begin{enumerate}
  \item[\it Bijection Principle:] $\abs{A}=\abs{B}$ \textit{iff} $A\simeq B$. 
    \begin{itemize}
      \item[\tt Reflexive:] $A \simeq A$.
      \item[\tt Symmetric:] if $A \simeq B$, then $B \simeq A$.
      \item[\tt Transitive:] if $A \simeq B$ and $B \simeq C$, then $A \simeq C$.
    \end{itemize}
  \item[\it Injection Principle:] $\abs{A} \leq \abs{B}$ \textit{iff} $A \simeq C$ for some $C\subseteq B$. 
    \begin{itemize}
      \item[\tt Reflexive:] $|A| \leq |A|$.
      \item[\tt Transitive:] if $|A| \leq |B|$ and $|B| \leq |C|$, then $|A| \leq |C|$.
      \item[\tt Anti-Symmetric:] if $|A| \leq |B|$ and $|B| \leq |A|$, then $|A| = |B|$.
      \item[\tt Total:] $|A| \leq |B|$ or $|B| \leq |A|$.\\
        (Equivalent to the Axiom of Choice)
    \end{itemize}
\end{enumerate}



\section*{Cantor-Schroeder-Bernstein}

\begin{enumerate}
  \item[\it Theorem:] If there are injective functions $f: A \to B$ and $g: B \to A$, then there is a bijective function $h: A \to B$.
\end{enumerate}


\section*{Countable Infinity}

\begin{enumerate}
  \item[\it Identities:] Bijection Principle makes many sets countably infinite.
    \begin{itemize}[leftmargin=-.2in]
      \item $\abs{\N} = \abs{\N_m} = \abs{\N_{(m)}}$.  
      \item $\abs{\N} = \abs{\Z} = \abs{\Q}$.
      \item[\bf (?)] What about $\abs{\N} = \abs{\R}$?
    \end{itemize}
  \item[\it Next Time:] We will show that there are different sizes of infinity.
\end{enumerate}










\end{document}


