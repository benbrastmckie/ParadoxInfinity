\documentclass[a4paper, 11pt]{article} % Font size (can be 10pt, 11pt or 12pt) and paper size (remove a4paper for US letter paper)
\usepackage[protrusion=true,expansion=true]{microtype} % Better typography
\usepackage{graphicx} % Required for including pictures
\usepackage{wrapfig} % Allows in-line images
\usepackage{enumitem} %%Enables control over enumerate and itemize environments
\usepackage{setspace}
\usepackage{amssymb, amsmath, mathrsfs,mathabx} %%Math packages
\usepackage{stmaryrd}
\usepackage{mathtools}
\usepackage{multicol} 
\usepackage{mathpazo} % Use the Palatino font
\usepackage[T1]{fontenc} % Required for accented characters
\usepackage{array}
\usepackage{bibentry}
\usepackage{prooftrees} 
\usepackage[round]{natbib} %%Or change 'round' to 'square' for square backers
\setcitestyle{aysep=}
% \usepackage{fitchproof} 

% \linespread{1} % Change line spacing here, Palatino benefits from a slight increase by default

\newcommand{\tuple}[1]{\langle#1\rangle} %%Angle brackets
\newcommand{\set}[1]{\lbrace#1\rbrace} %%Set brackets
\newcommand{\abs}[1]{|#1|} %%Set brackets
\newcommand{\interpret}[1]{\llbracket#1\rrbracket} %%Double brackets
\newcommand{\N}{\mathbb{N}}
\newcommand{\D}{\mathbb{D}}
\newcommand{\Z}{\mathbb{Z}}
\newcommand{\Q}{\mathbb{Q}}
\newcommand{\R}{\mathbb{R}}

\makeatletter
\renewcommand\@biblabel[1]{\textbf{#1.}} % Change the square brackets for each bibliography item from '[1]' to '1.'
\renewcommand{\@listI}{\itemsep=0pt} % Reduce the space between items in the itemize and enumerate environments and the bibliography

\renewcommand{\maketitle}{ % Customize the title - do not edit title and author name here, see the TITLE block below
\begin{flushright} % Right align
{\LARGE\@title} % Increase the font size of the title

\vspace{10pt} % Some vertical space between the title and author name

{\@author} % Author name
\\\@date % Date

\vspace{-10pt} % Some vertical space between the author block and abstract
\end{flushright}
}

%----------------------------------------------------------------------------------------
%	TITLE
%----------------------------------------------------------------------------------------

\title{\textbf{Infinite Cardinalities}} % Subtitle

\author{\textsc{Paradox and Infinity}\\ \em Benjamin Brast-McKie} % Institution

\date{\today} % Date

%----------------------------------------------------------------------------------------

\begin{document}

\maketitle % Print the title section

\thispagestyle{empty}

%----------------------------------------------------------------------------------------

\section*{Where to Begin\ldots}

\begin{enumerate}
  \item[\it Assumptions:] Theories have to begin somewhere. 
    \begin{itemize}[leftmargin=-.2in]
      \item A theory that is neutral on everything is no theory at all. 
      \item But a conclusion cannot come from nothing.
    \end{itemize}
  \item[\it Concepts:] Can't define everything.
    \begin{itemize}[leftmargin=-.2in]
      \item Must take some concepts as primitive.
      \item Intuitions from patterns of use.
      \item Principles encode theoretical function.
    \end{itemize}
  \item[\it Example:] Consider the concept of \textit{number}. 
    \begin{itemize}[leftmargin=-.2in]
      \item Numbers are answers to `How many?'-Questions.
      \item What principles might we affirm?
    \end{itemize}
\end{enumerate}





\section*{Finite Intuitions}

\begin{enumerate}
  \item[\it Proper Subset Principle:] $A\subset B \rightarrow \abs{A}<\abs{B}$.
    \begin{itemize}[leftmargin=-.2in]
      \item There are more mammals than lamas.
      \item There are more reals than rationals.
    \end{itemize}
  \item[\it Bijection Principle:] $A\simeq B \leftrightarrow \abs{A}=\abs{B}$.
    \begin{itemize}[leftmargin=-.2in]
      \item $A\simeq B$ means the $A$s and $B$s can be paired one-to-one with no remainders.
      \item We can define this without recourse to the concept of number.
      \item[\it Ordered Pair:] $\tuple{a,b}\coloneq \set{\set{a},\set{a,b}}$.
      \item[\it Relation:] $A\times B\coloneq \set{\tuple{a,b}:a\in A,\ b\in B}$.
      \item[\it Function:] A \textit{(total) function} $f: A\to B$ is any relation $f\subseteq A\times B$ where for every $a\in A$: (1) there is some $b\in B$ where $\tuple{a,b}\in f$; and (2) if there are some $b,c\in B$ where both $\tuple{a,b},\tuple{a,c}\in f$, then $b=c$. 
      \item If $f$ is a function, then we may take `$f(a)=b$' to abbreviate `$\tuple{a,b}\in f$'. 
      \item[\it Injective:] $f: A\to B$ is \textit{injective iff} for any $a,b\in A$, if $f(a)=f(b)$, then $a=b$. 
      \item[\it Surjective:] $f: A\to B$ is \textit{surjective iff} for all $b\in B$ there is some $a\in A$ where $f(a)=b$.
      \item[\it Bijection:] $f: A\to B$ is \textit{bijective iff} $f$ is an injective and surjective function. 
      \item[\it Equinumerous:] $A\simeq B$ \textit{iff} there is a bijection $f: A \to B$.
    \end{itemize}
\end{enumerate}




\section*{Infinity and Paradox?}

\begin{enumerate}
  \item[\it Hilbert's Hotel:] Always room for (countably many) more guests.
    \begin{itemize}[leftmargin=-.2in]
      \item Given a domain $\D$, $f:x\mapsto t(x)$ defines the function $\set{\tuple{x,t(x)}:x\in\D}$ where `$t(x)$' is a term that may include `$x$', e.g., `$x+1$`.
      \item $f_m:n\mapsto n+m$ is a bijection $f_m : \N \to \N_m$ where $\N_m=\set{k \in \N : k\geq m}$.
      \item $g_m:n\mapsto n\times m$ is a bijection $g_m : \N \to \N_{(m)}=\set{k\times m: k \in \N}$.
      \item[\bf (?)] What about countably many countable groups of new guests?
    \end{itemize}
  \item[\it Galileo's Roots:] ``Every square has its own root and every root has its own square, while no square has more than one root and no root has more than one square.'' 
  \item[\it Paradox:] There are many equinumerous proper subsets of infinite sets.
    \begin{itemize}[leftmargin=-.2in]
      \item By the principles above, both $\abs{\N_2}<\abs{\N}$ and $\abs{\N_2}=\abs{\N}$.
      \item But $x < y$ \textit{iff} $x \leq y$ and $x \neq y$.
      \item Thus $\abs{\N_2}<\abs{\N}$ entails $\abs{\N_2} \neq \abs{\N}$: contradiction.
      \item[\bf (?)] Which principle should we give up? 
    \end{itemize}
\end{enumerate}



\section*{The Abductive Method}

\begin{enumerate}
  \item[\it Deductively Closed:] Good theories include all of their implications.
  \item[\it Consistency:] Good theories exclude contradictions.
  \item[\it Simplicity:] Good theories are easy to understand, e.g., are finitely axiomatizable in terms of intuitively compelling and conceptually elegant concepts.
  \item[\it Strength:] Good theories say more rather than less.
  \item[\it Utility:] Good theories serve our aims, e.g., have useful applications.
\end{enumerate}





\section*{A Metaphysical Aside}

\begin{enumerate}
  \item[\it Subjectivity:] Is theory choice by abduction a reflection of human psychology?
    \begin{itemize}[leftmargin=-.2in]
      \item The abductive method describes how we typically choose theories.
      \item Are we right to use the abductive method and why?
    \end{itemize}
  \item[\it Realism:] Is the abductive method well suited to the task of describing reality?
    \begin{itemize}[leftmargin=-.2in]
      \item We don't need to decide this before using the abductive method.
      \item It will help to put the method to work.
    \end{itemize}
  \item[\it Example:] Which of the principles above should we give up?
\end{enumerate}





\section*{Towards a Theory of Number}

\begin{enumerate}
  \item[\it Hypothesis:] Suppose we were to retain the \textit{Proper Subset Principle} (PSP).
    \begin{itemize}[leftmargin=-.2in]
      \item Then $\abs{\N}>\abs{\N_{2}}>\abs{\N_{3}}>\ldots$ and $\abs{\N}>\abs{\N_{(2)}}>\abs{\N_{(4)}}>\ldots$ etc. 
      \item How are we to compare $\abs{\N_{(2)}}$ and $\abs{\N_{(3)}}$?
    \end{itemize}
  \item[\it Linear Ordering:] Numbers are linearly ordered by $\leq$ and so must satisfy the following.
    \begin{itemize}[leftmargin=-.2in]
      \item[\tt Reflexive:] $x \leq x$ for any number $x$.
      \item[\tt Transitive:] If $x \leq y$ and $y \leq z$, then $x \leq z$.
      \item[\tt Anti-Symmetric:] If $x \leq y$ and $y \leq x$, then $x = y$.
      \item[\tt Total:] Either $x \leq y$ or $y \leq x$ for any numbers $x$ and $y$.
      \item Compare giving a theory of identity where symmetry fails.
      \item We wouldn't really be talking about identity.
    \end{itemize}
  \item[\it Incomplete:] Converse of PSP is false and so PSP does not define $<$.
    \begin{itemize}[leftmargin=-.2in]
      \item[\bf (?)] Could PSP be supplemented in some way?
      \item[\bf (?)] How would we compare the cardinality of two bags of stones?
      \item By trying to line them up one-to-one.
    \end{itemize}
  \item[\it Injection Principle:] $\abs{A} \leq \abs{B}$ \textit{iff} $A \simeq C$ for some $C\subseteq B$. 
    \begin{itemize}[leftmargin=-.2in]
      \item This assumes \textit{Bijection Principle} (BP) which is in tension with PSP.
      \item Restricting the \textit{Injection Principle} (IP) and BP to finite sets is \textit{ad hoc}.
      \item Better to take Hilbert's Hotel to be a counterexample to PSP.
    \end{itemize}
\end{enumerate}




\section*{Countable Infinity}

\begin{enumerate}
  \item[\it Principles:] Given IP and BP, we may show that numbers are linearly ordered.
    \begin{itemize}[leftmargin=-.2in]
      \item \texttt{Anti-Symmetric} is proven by the Cantor-Schroeder-Bernstein theorem.
      \item \texttt{Total} is equivalent to the Axiom of Choice.
    \end{itemize}
  \item[\it Countable Sets:] A set $A$ is \textit{countably infinite iff} $\abs{A}=\abs{\N}$.
  \item[\it Identities:] Bijection Principle makes many sets countably infinite.
    \begin{itemize}[leftmargin=-.2in]
      \item $\abs{\N} = \abs{\N_m} = \abs{\N_{(m)}}$.  
      \item $\abs{\N} = \abs{\Z} = \abs{\Q}$.
      \item[\bf (?)] What about $\abs{\N} = \abs{\R}$?
    \end{itemize}
  \item[\it Next Time:] We will show that there are different sizes of infinity.
\end{enumerate}










\end{document}


