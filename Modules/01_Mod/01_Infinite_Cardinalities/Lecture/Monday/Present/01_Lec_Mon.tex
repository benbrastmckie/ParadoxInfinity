%%%%%%%%%%%%%%%%%%%%%%%%%%%%%%%%%%%%%%%%%
% Beamer Presentation
% LaTeX Template
% Version 1.0 (10/11/12)
%
% This template has been downloaded from:
% http://www.LaTeXTemplates.com
%
% License:
% CC BY-NC-SA 3.0 (http://creativecommons.org/licenses/by-nc-sa/3.0/)
%
%%%%%%%%%%%%%%%%%%%%%%%%%%%%%%%%%%%%%%%%%

%----------------------------------------------------------------------------------------
%	PACKAGES AND THEMES
%----------------------------------------------------------------------------------------

% \documentclass{beamer}
\documentclass[handout]{beamer}

\mode<presentation> {

% The Beamer class comes with a number of default slide themes
% which change the colors and layouts of slides. Below this is a list
% of all the themes, uncomment each in turn to see what they look like.

\usetheme{default}
%\usetheme{AnnArbor}
%\usetheme{Antibes}
%\usetheme{Bergen}
%\usetheme{Berkeley}
%\usetheme{Berlin}
%\usetheme{Boadilla}
%\usetheme{CambridgeUS}
%\usetheme{Copenhagen}
%\usetheme{Darmstadt} %Nice
%\usetheme{Dresden} %Nice
% \usetheme{Frankfurt} %Nice
%\usetheme{Goettingen} %Sidebar
%\usetheme{Hannover}
%\usetheme{Ilmenau}
%\usetheme{JuanLesPins}
%\usetheme{Luebeck}
%\usetheme{Madrid}
%\usetheme{Malmoe}
%\usetheme{Marburg}
%\usetheme{Montpellier}
%\usetheme{PaloAlto}
%\usetheme{Pittsburgh}
%\usetheme{Rochester}
%\usetheme{Singapore}
%\usetheme{Szeged}
%\usetheme{Warsaw}

% As well as themes, the Beamer class has a number of color themes
% for any slide theme. Uncomment each of these in turn to see how it
% changes the colors of your current slide theme.

% \usecolortheme{albatross}
% \usecolortheme{beaver}
% \usecolortheme{beetle}
% \usecolortheme{crane}
% \usecolortheme{dolphin}
%\usecolortheme{dove}
%\usecolortheme{fly}
%\usecolortheme{lily}
%\usecolortheme{orchid}
%\usecolortheme{rose}
%\usecolortheme{seagull}
%\usecolortheme{seahorse}
%\usecolortheme{whale}
%\usecolortheme{wolverine}

% Custom
\setbeamercolor{normal text}{fg=white,bg=black!90}
\setbeamercolor{structure}{fg=white}
\setbeamercolor{alerted text}{fg=red!85!black}
\setbeamercolor{item projected}{use=item,fg=black,bg=item.fg!35}
\setbeamercolor*{palette primary}{use=structure,fg=structure.fg}
\setbeamercolor*{palette secondary}{use=structure,fg=structure.fg!95!black}
\setbeamercolor*{palette tertiary}{use=structure,fg=structure.fg!90!black}
\setbeamercolor*{palette quaternary}{use=structure,fg=structure.fg!95!black,bg=black!80}
\setbeamercolor*{framesubtitle}{fg=white}
\setbeamercolor*{block title}{parent=structure,bg=black!60}
\setbeamercolor*{block body}{fg=black,bg=black!10}
\setbeamercolor*{block title alerted}{parent=alerted text,bg=black!15}
\setbeamercolor*{block title example}{parent=example text,bg=black!15}

%\setbeamertemplate{footline} % To remove the footer line in all slides uncomment this line
%\setbeamertemplate{footline}[page number] % To replace the footer line in all slides with a simple slide count uncomment this line

%\setbeamertemplate{navigation symbols}{} % To remove the navigation symbols from the bottom of all slides uncomment this line
}

%%% SYMBOLS AND STYLES %%%

\DeclareSymbolFont{symbolsC}{U}{txsyc}{m}{n} 
\DeclareMathSymbol{\strictif}{\mathrel}{symbolsC}{74}
\usepackage{multicol}
\newcommand{\tuple}[1]{\langle#1\rangle} %%Angle brackets
\setbeamercovered{transparent}
\usepackage{graphicx}
\usepackage{mathabx}

\renewcommand{\tuple}[1]{\langle#1\rangle} %%Angle brackets
\newcommand{\set}[1]{\lbrace#1\rbrace} %%Set brackets
\newcommand{\abs}[1]{|#1|} %%Set brackets
\newcommand{\interpret}[1]{\llbracket#1\rrbracket} %%Double brackets
\newcommand{\N}{\mathbb{N}}
\newcommand{\D}{\mathbb{D}}
\newcommand{\Z}{\mathbb{Z}}
\newcommand{\Q}{\mathbb{Q}}
\newcommand{\R}{\mathbb{R}}

	
%%% CITATIONS %%%
\usepackage[round]{natbib} %%Or change 'round' to 'square' for square backers
\setcitestyle{aysep={}}
\newcommand\citeapl[2][]{\citeauthor{#2}'s#1} %%Use \citeapl is for possessive author name only.
\newcommand\citea[2][]{\citeauthor{#2}#1} %%Use \citea is for author name only, with optional page numbers.
\newcommand\citepl[2][]{\citeauthor{#2}'s (\citeyear[#1]{#2})}%%The command \citepl is for possessive citations.
\usepackage{bibentry}

\usepackage{graphicx} % Allows including images
\usepackage{booktabs} % Allows the use of \toprule, \midrule and \bottomrule in tables

%----------------------------------------------------------------------------------------
%	TITLE PAGE
%----------------------------------------------------------------------------------------

\title[Short title]{\LARGE Paradox and Infinity} % The short title appears at the bottom of every slide, the full title is only on the title page

\author{\it Today's Topic: Infinite Cardinalities} % Your name
\date{} % Date, can be changed to a custom date

\begin{document}

\begin{frame}
\titlepage % Print the title page as the first slide
\end{frame}


%----------------------------------------------------------------------------------------
%	PRESENTATION SLIDES
%----------------------------------------------------------------------------------------

%%% NOTES %%%

%Recall: \pause for 

%Recall: ITEMIZE

%\begin{itemize}  
%\item<1-> 
%\end{itemize}

%\onslide<1->{ SLIDE }

%\begin{itemize}[<+(1)->]
%\begin{itemize}[<+->]

\begin{frame}

  \textbf{Proper Subset Principle:} $A\subset B \Rightarrow \abs{A}<\abs{B}$.
  \vspace{.2in}
  \pause

  \textbf{Bijection Principle:} $A\simeq B \Leftrightarrow \abs{A}=\abs{B}$.
  \vspace{.2in}
  \pause

  \textbf{Bijection:} $A\simeq B$ \textit{iff} there is a bijection $f: A \to B$.
  \vspace{.2in}
  \pause

  ``There is a one-to-one pairing of $A$s and $B$s with no remainder.''

\end{frame}


%%% -------------------------------------------------------


\begin{frame}

  \textbf{Ordered Pair:} $\tuple{a,b}\coloneq \set{\set{a},\set{a,b}}$.
  \vspace{.2in}
  \pause

  \textbf{Relation:} $A\times B\coloneq \set{\tuple{a,b}:a\in A,\ b\in B}$.
  \vspace{.2in}
  \pause

  \textbf{Function:} $f\subseteq A\times B$ is a \textit{function} $f: A\to B$ \textit{iff} every $a\in A$:\\ 
    \begin{enumerate}
      \item There is some $b\in B$ where $\tuple{a,b}\in f$; and
      \item For any $b,c\in B$, if both $\tuple{a,b},\tuple{a,c}\in f$, then $b=c$. 
    \end{enumerate}
  \vspace{.2in}
  \pause

  \textbf{Abbreviation:} `$f(a)=b$' abbreviates `$\tuple{a,b}\in f$' if $f$ is a function. 
  \vspace{.2in}

\end{frame}


%%% -------------------------------------------------------
  

\begin{frame}

  \textbf{Injective:} $f: A\to B$ is \textit{injective iff} for any $a,b\in A$, if $f(a)=f(b)$, then $a=b$. 
  \vspace{.2in}
  \pause

  \textbf{Surjective:} $f: A\to B$ is \textit{surjective iff} for all $b\in B$ there is some $a\in A$ where $f(a)=b$.
  \vspace{.2in}
  \pause

  \textbf{Bijection:} $f: A\to B$ is \textit{bijective iff} $f$ is injective and surjective. 
  \vspace{.2in}
  \pause

  \textbf{Equinumerous:} $A\simeq B$ \textit{iff} there is a bijection $f: A \to B$.
  \vspace{.2in}

\end{frame}

%%% -------------------------------------------------------

\begin{frame}

  \textbf{Paradox:} There are equinumerous proper subests of infinite sets.
  \pause

  \begin{itemize}
    \item $\N_1 \subset \N$ where $\N_i\coloneq\set{x \in \N : i \leq x}$.
    \item The successor function $x'= x + 1$ is a bijection from $\N$ to $\N_1$. 
  \end{itemize}
  \vspace{.2in}
  \pause


  \textbf{Hilbert's Hotel:} Always room for (countably many) more guests.
  \pause

  \begin{itemize}
    \item $f_m(n) = n + m$ is a bijection from $\N$ to $\N_m$ .
    \item $g_m(n) = n \times m$ is a bijection where $\N_{(m)}=\set{k\times m: k \in \N}$.
  \end{itemize}
  \vspace{.2in}
  \pause

  \textbf{Question:} Can Hilbert's Hotel accommodate an infinite number of groups of infinitely many new guests?

\end{frame}

%%% -------------------------------------------------------

\begin{frame}

  \textbf{Contradiction:}
  \pause

    \begin{itemize}
      \item Both $\abs{\N_1}<\abs{\N}$ and $\abs{\N_1}=\abs{\N}$ by the principles.
      % \item But $x < y$ \textit{iff} $x \leq y$ and $x \neq y$.
      \item Totality: $x < y$, $x = y$, or $x > y$ (exclusive).
      \item Thus $\abs{\N_1}<\abs{\N}$ entails $\abs{\N_1} \neq \abs{\N}$: contradiction.
    \end{itemize}
    \vspace{.2in}
    \pause

  \textbf{Options:}
  \pause

    \begin{itemize}
      \item Accept contradictions?
      \item Give up totality?
      \item Give up one of the principles above?
    \end{itemize}

\end{frame}

%%% -------------------------------------------------------

\begin{frame}

  \textbf{Abductive Method:}
  \pause

  \begin{itemize}
    \item Deductively Closed: $\Gamma \vdash \varphi \Rightarrow \varphi \in \Gamma$.
    \item Consistent: $\Gamma\nvdash \varphi\wedge\neg\varphi$.
    \item Simple: finitely axiomatizable in intuitive terms.
    \item Strong: says more rather than less.
    \item Practical: has useful applications.
  \end{itemize}
  \vspace{.2in}
  \pause

  \textbf{Metaphysical Aside:} Is theory choice subjective?

\end{frame}

%%% -------------------------------------------------------

\begin{frame}

  \textbf{Proper Subset Principle:} $A\subset B \Rightarrow \abs{A}<\abs{B}$.
  \pause

  \begin{enumerate}
    \item $\abs{\N}>\abs{\N_{2}}>\abs{\N_{3}}>\ldots$ and $\abs{\N}>\abs{\N_{(2)}}>\abs{\N_{(4)}}>\ldots$
    \item How are we to compare $\abs{\N_{(2)}}$ and $\abs{\N_{(3)}}$ for size?
    \item Could PSP be supplemented?
    \item Finite case: set of people vs. set of seats.
  \end{enumerate}
  \vspace{.2in}
  \pause

  \textbf{Injection Principle:} $\abs{A} \leq \abs{B}$ \textit{iff} $A \simeq C$ for some $C\subseteq B$. 
  \pause

  \begin{itemize}
    \item 
    \item What kind of theory does this give us?
  \end{itemize}

  % \begin{itemize}
  %   \item Reflexive: $x \leq x$ for any number $x$.
  %   \item Transitive: If $x \leq y$ and $y \leq z$, then $x \leq z$.
  %   \item Anti-Symmetric: If $x \leq y$ and $y \leq x$, then $x = y$.
  %   \item Total: Either $x \leq y$ or $y \leq x$ for any numbers $x$ and $y$.
  % \end{itemize}

\end{frame}

%%% -------------------------------------------------------





%     \end{itemize}
%   \item[\it Linear Ordering:] Numbers are linearly ordered by $\leq$ and so must satisfy the following.
%     \begin{itemize}[leftmargin=-.2in]
%       \item[\tt Reflexive:] $x \leq x$ for any number $x$.
%       \item[\tt Transitive:] If $x \leq y$ and $y \leq z$, then $x \leq z$.
%       \item[\tt Anti-Symmetric:] If $x \leq y$ and $y \leq x$, then $x = y$.
%       \item[\tt Total:] Either $x \leq y$ or $y \leq x$ for any numbers $x$ and $y$.
%       \item Compare giving a theory of identity where symmetry fails.
%       \item We wouldn't really be talking about identity.
%     \end{itemize}
%   \item[\it Incomplete:] Converse of PSP is false and so PSP does not define $<$.
%     \begin{itemize}[leftmargin=-.2in]
%       \item[\bf (?)] Could PSP be supplemented in some way?
%       \item[\bf (?)] How would we compare the cardinality of two bags of stones?
%       \item By trying to line them up one-to-one.
%     \end{itemize}
%   \item[\it Injection Principle:] $\abs{A} \leq \abs{B}$ \textit{iff} $A \simeq C$ for some $C\subseteq B$. 
%     \begin{itemize}[leftmargin=-.2in]
%       \item This assumes \textit{Bijection Principle} (BP) which is in tension with PSP.
%       \item Restricting the \textit{Injection Principle} (IP) and BP to finite sets is \textit{ad hoc}.
%       \item Better to take Hilbert's Hotel to be a counterexample to PSP.
%     \end{itemize}
% \end{enumerate}
%
%
%
%
% \section*{Countable Infinity}
%
% \begin{enumerate}
%   \item[\it Principles:] Given IP and BP, we may show that numbers are linearly ordered.
%     \begin{itemize}[leftmargin=-.2in]
%       \item \texttt{Anti-Symmetric} is proven by the Cantor-Schroeder-Bernstein theorem.
%       \item \texttt{Total} is equivalent to the Axiom of Choice.
%     \end{itemize}
%   \item[\it Countable Sets:] A set $A$ is \textit{countably infinite iff} $\abs{A}=\abs{\N}$.
%   \item[\it Identities:] Bijection Principle makes many sets countably infinite.
%     \begin{itemize}[leftmargin=-.2in]
%       \item $\abs{\N} = \abs{\N_m} = \abs{\N_{(m)}}$.  
%       \item $\abs{\N} = \abs{\Z} = \abs{\Q}$.
%       \item[\bf (?)] What about $\abs{\N} = \abs{\R}$?
%     \end{itemize}
%   \item[\it Next Time:] We will show that there are different sizes of infinity.
% \end{enumerate}













%----------------------------------------------------------------------------------------
\nobibliography{Zotero}
\bibliographystyle{Phil_Review}

\end{document} 
