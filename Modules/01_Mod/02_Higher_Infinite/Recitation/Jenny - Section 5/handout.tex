\documentclass[justified]{tufte-handout} 
\usepackage{amsfonts, amssymb, stmaryrd, fitch, natbib, qtree}
\usepackage{linguex, color, setspace, graphicx}
\usepackage{enumitem}
\usepackage{bussproofs}
\usepackage{turnstile}
\usepackage[super]{nth}
\thispagestyle{plain}
\definecolor{darkred}{rgb}{0.7,0,0.2}
\bibpunct{(}{)}{,}{a}{}{,}

\input xy
 \xyoption{all}

%New Symbols
\DeclareSymbolFont{symbolsC}{U}{txsyc}{m}{n}
\DeclareMathSymbol{\strictif}{\mathrel}{symbolsC}{74}
\DeclareMathSymbol{\boxright}{\mathrel}{symbolsC}{128}
\DeclareMathSymbol{\Diamondright}{\mathrel}{symbolsC}{132}
\DeclareMathSymbol{\Diamonddotright}{\mathrel}{symbolsC}{134}
\DeclareMathSymbol{\Diamonddot}{\mathord}{symbolsC}{144}
\renewcommand{\labelitemi}{$\triangleright$}
\renewcommand{\labelitemii}{$\circ$}
\renewcommand{\labelitemiii}{$\triangleright$}

%New commands
\newcommand{\bitem}{\begin{itemize}}
\newcommand{\eitem}{\end{itemize}}
\newcommand{\lang}{$\langle$}
\newcommand{\rang}{$\rangle$}
\newcommand{\back}{$\setminus$}
\newcommand{\HRule}{\rule{\linewidth}{0.1mm}}
\newcommand{\llm}[2][]{$\llbracket${#2}$\rrbracket^{#1}$}
\newcommand{\ul}{$\ulcorner$}
\newcommand{\ur}{$\urcorner\ $}
\newcommand{\urn}{$\urcorner$}
\newcommand{\sub}[1]{\textsubscript{#1}}
\newcommand{\sups}[1]{\textsuperscript{#1}}
\newtheorem{proposition}{\textbfb{Proposition}}[section]
\newtheorem{definition}[proposition]{\textbf{Definition}}
\newcommand{\bfw}{\begin{fullwidth}}
\newcommand{\efw}{\end{fullwidth}}

\begin{document}

\begin{fullwidth}
\noindent\LARGE More on Ordinals and Cardinals  \normalsize \\[.3cm]
\noindent  \textsc{24.118 Recitation Section $\bullet$ Matthias Jenny\\  {\texttt{\href{mailto:mjenny@mit.edu}{mjenny@mit.edu}}} $\bullet$ Office:  32-D927 $\bullet$ Hours: Thu 11:30-12:30} \hfill{October 17, 2014}
\noindent\HRule
\end{fullwidth}


\section{Pset 5, problem 2}

\noindent A relation $R$ on a set $A$ is a \emph{linear order} (or \emph{total order}) iff: \underline{\hspace{8.5cm}}\\\\\underline{\hspace{16.88cm}}\\\\\underline{\hspace{16.43cm}}\\


\begin{enumerate}[label=\alph*.]
\item Is $\mathcal{P}(\mathcal{P}(\mathcal{P}(\emptyset)))$ linearly ordered by $\in$?\marginnote{{\large$\square$} Yes  {\large$\square$} No}\\

\noindent \emph{Notes:}  \underline{\hspace{15.4cm}}\\

\item Is $\mathcal{P}(\mathcal{P}(\mathcal{P}(\emptyset)))-\{\{\{\emptyset\}\}\}$ linearly ordered by $\in$?\marginnote{{\large$\square$} Yes  {\large$\square$} No}\\

\noindent \emph{Notes:}  \underline{\hspace{15.4cm}}\\


\end{enumerate}


\section{More on the Continuum Hypothesis}

Recall this table from the online notes:

\begin{table}[htdp]
\begin{center}
\begin{tabular}{|c|c|c|}
Ordinal & 	Corresponding set & 	Name for that set  \\ \hline\hline
$0_O$ & $\mathbb{N}_O$ & $\beth_0$\\
$1_O$ & $\mathcal{P}(\mathbb{N}_O)$ & $\beth_1$\\
$2_O$ & $\mathcal{P}^2(\mathbb{N}_O)$ & $\beth_2$\\
\dots & \dots & \dots\\
$\omega$ & $\bigcup\limits_{n\in\mathbb{N}_0}\mathcal{P}^n(\mathbb{N}_O)$ & $\beth_\omega$\\
$\omega+1_O$ & $\mathcal{P}(\bigcup\limits_{n\in\mathbb{N}_0}\mathcal{P}^n(\mathbb{N}_O))$ & $\beth_{\omega+1}$\\
$\omega+2_O$ & $\mathcal{P}^n(\bigcup\limits_{n\in\mathbb{N}_0}\mathcal{P}^n(\mathbb{N}_O))$ & $\beth_{\omega+2}$\\
\dots & \dots & \dots\\
$\omega+\omega$ & $\bigcup\limits_{m\in\mathbb{N}_0}\mathcal{P}^m(\bigcup\limits_{n\in\mathbb{N}_0}\mathcal{P}^n(\mathbb{N}_O))$ & $\beth_{\omega+\omega}$\\
\dots & \dots & \dots
\end{tabular}
\end{center}
\label{default}
\end{table}%
\noindent And recall that we've defined $\aleph_0$ as the cardinality of $\mathbb{N}$ and $\aleph_{n+1}$ as the least cardinal number larger than $\aleph_{n}$.

\begin{enumerate}[label=\roman*.]
\item True or false? $\beth_0=\aleph_0$.\marginnote{{\large$\square$} True  {\large$\square$} False}\\

\noindent \emph{Notes:}  \underline{\hspace{15.4cm}}\\\\\underline{\hspace{16.43cm}}\\

\item True or false? There's no cardinality strictly between $\aleph_0$ and $\aleph_1$.\marginnote{{\large$\square$} True  {\large$\square$} False}\\

\noindent \emph{Notes:}  \underline{\hspace{15.4cm}}\\\\\underline{\hspace{16.43cm}}\\

\item True or false? The cardinality of $\mathbb{Q}$ is $\aleph_1$.\marginnote{{\large$\square$} True  {\large$\square$} False}\\

\noindent \emph{Notes:}  \underline{\hspace{15.4cm}}\\\\\underline{\hspace{16.43cm}}\\

\item True or false? The cardinality of $\mathbb{R}$ is $\beth_1$.\marginnote{{\large$\square$} True  {\large$\square$} False}\\

\noindent \emph{Notes:}  \underline{\hspace{15.4cm}}\\\\\underline{\hspace{16.43cm}}\\

\item True or false? $\beth_1=2^{\beth_0}$.\marginnote{{\large$\square$} True  {\large$\square$} False}\\

\noindent \emph{Notes:}  \underline{\hspace{15.4cm}}\\\\\underline{\hspace{16.43cm}}\\

\item True or false? $\aleph_1=2^{\aleph_1}$.\marginnote{{\large$\square$} True  {\large$\square$} False}\\

\noindent \emph{Notes:}  \underline{\hspace{15.4cm}}\\\\\underline{\hspace{16.43cm}}\\

\item True or false? For any ordinal $\alpha$, $\aleph_\alpha=\beth_\alpha$.\marginnote{{\large$\square$} True  {\large$\square$} False}\\

\noindent \emph{Notes:}  \underline{\hspace{15.4cm}}\\\\\underline{\hspace{16.43cm}}\\

\end{enumerate}

\end{document}
