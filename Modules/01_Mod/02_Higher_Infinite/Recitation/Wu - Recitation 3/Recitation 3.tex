\documentclass[12pt]{article}

\usepackage{amsmath,amsfonts,amssymb,amscd}

\usepackage{amsthm}

\usepackage[margin=1.5in,headsep=.5in]{geometry}

\usepackage{fancyhdr}

\setlength{\headheight}{20pt}

\usepackage[colorlinks]{hyperref} 
\usepackage{cleveref}

\usepackage{setspace}
\usepackage{enumitem,linegoal}


\newtheorem{theo}{Theorem}[section] 

\theoremstyle{definition}
\newtheorem{defin}[theo]{Definition} 
\newtheorem{lema}[theo]{Lemma} 
\newtheorem{cor}[theo]{Corollar}
\newtheorem{prop}[theo]{Proposition}

\pagestyle{fancy}

\begin{document}

\pagenumbering{gobble}

\lhead{Xinhe Wu (xinhewu@mit.edu)}
\rhead{$24.118$ Paradox and Infinity $|$ Recitation $3$}



\begin{center}
{\Large \bf Ordinals, Burali-Forti and Reductionism}
\end{center}

\smallskip

\section{Ordinals: The Official Definition}

\begin{defin}
A set $x$ is \textit{transitive} if and only if for any $y, z$, if $z \in y$ and $y \in x$, then $z \in x$.
\end{defin}

\begin{defin}
A set $x$ is an \textit{ordinal} if and only if $x$ is transitive and is well-ordered by $\in$, i.e. for any $a, b, c \in x$, 
\begin{enumerate} [leftmargin = 2cm]
\item[(1)] If $a \in b$, then $\neg (b \in a)$.
\item[(2)] If $a \in b$ and $b \in c$, then $a \in c$.
\item[(3)] If $a \neq b$, then either $a \in b$ or $b \in a$.
\item[(4)] If $y \subseteq x$ and $y \neq \varnothing$, then there exists $z \in y$ such that for any $v \in y$, $v \notin z$.
\end{enumerate}
\end{defin}

\begin{prop}
\begin{enumerate}
\item[(i)]The ordinals themselves are well-ordered by $\in$.
\item[(ii)] If $x$ is an ordinal and $y \in x$, then $y$ is an ordinal.
\item[(iii)] If $x$ is an ordinal, then $x \cup \{x\}$ is an ordinal.
\item[(iv)] If $S$ is a set of ordinals, then $\bigcup S$ is an ordinal. 
\end{enumerate}
\end{prop}

\section{The Burali-Forti Paradox}

\begin{theo}
There is not set of all ordinals.
\end{theo}

\begin{proof}
Let $\Omega$ be the set of all ordinals. By 1.3(ii), $\Omega$ is transitive. By 1.3(i), $\Omega$ is well-ordered by $\in$. By 1.2, $\Omega$ is an ordinal.
Therefore $\Omega \in \Omega$. Hence $\{\Omega\} \subseteq \Omega$. By 1.2(4), $\{ \Omega \}$ has a $\in$-smallest element. But it does not. Contradiction.
\end{proof}

\section{Set-theoretic Reductionism}


\begin{theo}
The structure $<\omega, \varnothing, S>$ is a Peano system, where for any set $x$, $S(x) = x \cup \{x\}$.
\end{theo}

\noindent
\textbf{(Strong Reductionism)} Natural numbers are reducible to sets, in the sense that every natural number is \textit{identical} to a set.

\end{document}